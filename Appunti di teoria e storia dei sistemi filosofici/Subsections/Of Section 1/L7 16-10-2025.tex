% !TEX root = ..\..\main.tex
\documentclass[../../main.tex]{subfiles}%
\ifSubfilesClassLoaded{\addbibresource{../../bibliography.bib}}{}

\begin{document}

% Diderot, Pensieri sull'interpretazione della natura.

% Il 'tramonto dell'occidente' da dentro
%
% 1. La questione dell'assolutezza della conoscenza: l'esempio della
% mappa, e la critica a Leibniz.
%
% 2. La successione delle discipline, le matematiche non rappresentano
% il fenomeno della vita.
%
% 3. Il modello morfologico e le metafore biologiche.
%
% 4. Il rapporto con gli antichi.
%
% 5. Non solo un universo infinito, ma un tempo infinito.

La questione dell'apollineo e il faustiano.

pp. 267-268 capitolo anima apollinea, anima faustiana, anima magica.

Riferimento all'esteso, chiaramente in polemica col tema
dell'estensione cartesiana. Anima faustiana il cui simbolo è lo spazio
puro e illimitato, e il cui corpo è la civiltà occidentale.

Un problema, un punto importante è appunto quello del rapporto con
Cartesio. Qui entra in gioco -- e non solo per Spengler -- la
questione delle matematiche e delle scienze esatte, e quindi con la
tecnica e la tecnologia, e quindi con la capacità degli uomini di
manipolare la natura, che è il problema di Nietzsche. In che
situazione ci si trova come moderni? O qual'è la rottura del moderno,
e che si trova formulato in modo come Cartesio? L'idea di concepire la
conoscenza come un rapporto tra soggetto e oggetto. Spengler raccoglie
questo, e noi tendiamo a concepire come eterno questo rapporto, e così
com'è stato concepito è un'invenzione del moderno: i greci non ce
l'avevano, e sia Heidegger che Wittgenstein avevano ragione su questo.

Il problema è esattamente quello del discorso sul metodo: io sono un
soggetto che metto un discussione tutto, applico il metodo scettico,
cosa mi resta? Qual'è il resto di mettere in dubbio tutto? Il cogito,
che serve quello che diverrà l'argomento tipico di Heidegger,
l'affermazione specifica dell'essere -- tema greco che però qui si
sta riformulando in chiave moderna: non è l'essere ma un \textquote{io
sono}, attraverso cioè l'idea dell'io che diventa soggetto e
fondamento di tutta la conoscenza. È l'individuo che in qualche modo
si libera. Poi, dopo che Cartesio ha buttato al mare tutto ed ha posto
questo problema tra il cogitare e l'essere riformulandolo nella
soggettività, pone un problema: che cosa conosco io, e come faccio a
conoscere qualcosa che è al di fuori di me se sono io il soggetto che
conosco? Come faccio a garantire l'oggettività del discorso (valido
per tutti i soggetti possibili)? Lì viene fuori il grande tema, e lui
lo dice chiaramente che lo fa attraverso due mosse, la prima è quella
di prendere il verosimile, ciò che somiglia al vero ma è una finzione,
e metterlo nel campo del falso, cioè toglierlo dal vero: il mito, per
esempio, entra nel campo del falso a questo punto. Tutto ciò che
somiglia al vero senza esserlo entra nel campo del falso. Il vero
allora che cos'è? Quindi qui si pone un problema, e qui introduce la
seconda pedina, ed è la matematica: è vero ciò che corrisponde
all'esattezza della descrizione dell'esteso. Ed introduce un elemento
fondamentale, il chiaro ed evidente, che è proprio il rapporto che si
viene a creare fra verità ed esattezza, è vero ciò che è esatto.

L'\textit{Epoca della fine del mondo} (?) Heidegger, distingue il
rigore e l'esattezza, rimette in discussione questo. Il criterio di
esattezza a cosa ci ha portato storicamente? Concepiva il corpo come
una macchina calcolabile matematicamente secondo le leggi della
meccanica. Il corpo, infatti, per Cartesio è un pseudomeccanismo della
mente. Quando nelle discussioni contemporanee si recupera il tema del
corpo, il grande problema è di uscire da questo tipo di situazione: il
corpo è gesto, la mente è corpo, oggi si parla di
\textquote{embodiment}, però è vero che c'è questa tematica, e uno dei
filosofi contemporanei che hanno sollevato questo è Ponty. C'è un dato
di ricerca e di discussione, e poi un dato inerziale che è politico:
il paradigma di Cartesio non è morto, e nella realtà effettiva le
cliniche e i carceri funzionano su quel modello, i teatri, ecc., per
cui non basta superare il paradigma in dei seminari, e risolto il
problema. Sorge un \emph{problema tra la filosofia e il senso comune},
e ci si pone il problema di un modello aristocratico-elitario, o un
cambiamento del senso comune. Di fondo ci sono questi aspetti, e si
viene a costruire il rapporto soggetto-oggetto da questo punto. Il
problema è la necessità di trovare una corrispondenza ontologica tra
soggetto e oggetto, ed è la matematica a trovare questo collegamento.

Da dove nasce questo tipo di problema? Chiaramente sorge da una realtà
che in qualche modo esplode nel Rinascimento, e che Galileo e Cartesio
fanno proprie costruendo una rivoluzione in questo senso, che le
immagini si possono duplicare in modo preciso, e lo fanno attraverso
l'invenzione della prospettiva lineare. Applicazione della geometria
in chiave prospettica per dare un'illusione di terza dimensione
essendo in due. Il modello della prospettiva però è conosciuto dai
Greci e dai Romani, sicuramente il libro X della \emph{Repubblica}, il
personaggio è un pittore prospettico, quello che rifiuta Platone è
identificare l'imitazione con la copia, cioè percettivamente la cosa
funziona se non si ha consapevolezza che si ha qualcosa di falso.
Libro 35 Plinio il vecchio, gara di due pittori i quali fanno la gara
a chi riesce ad ingannare l'occhio più dell'altro. Ma cosa succede per
quanto riguarda la prospettiva lineare moderna? Massaccio, Alberti,
ecc., riscoprono il mondo antico. Cosa succede nella Firenze del
Quattrocento? Si riscopre il Greco che si era perso, e la prima cosa
che viene fuori è che viene visto Tolomeo e così via. Il nuovo del
moderno è l'\emph{applicazione della geometria e della matematica},
nasce tutt'una tecnologia per fare questa cosa, tecnologia la cui
metafora forte è la finestra, che Alberti usa nel \textit{De pictura}
come modello per riprodurre un oggetto nel miglior modo possibile da
parte dell soggetto che lo vuole riprodurre. È costruita con un
reticolo, gli assi cartesiani, ed è grazie a quello che diventa
precisa. Il modello finestra è esattamente il modello della nostra
vita, e questo è il modello moderno: con esso viene inventato il
quadro, il cui vuol dire che può essere trasportabile; dopo il quadro
il teatro, che si costruisce nel modo Shakespeariano, stare attorno, e
l'altro, che è quello di una finestra, ed è quello che è prevalso.
Andiamo ancora avanti: il cinema funziona come una finestra, le
fotografie, la televisione, i computer e poi gli smartphone, sono
finestre. Dovrebbe essere avvertenza cosciente della percezione perché
c'è la finestra, si entra o si esce e c'è il di fuori. L'universo del
telefonino è diverso da quello di quest'aula. L'operazione è così
forte che ci si dimentica, e questo deriva dalla finestra albertiana,
una finestra di conoscneza che ha un suo aspetto tecnico, è un
dispositivo.

% La finestra di alberi e l'Aleph di Borges

A questo punto si ha una situazione in cui ci sono il soggetto e
l'oggetto mediati da una finestra, e tutta la conoscenza, quella che
poi viene prodotta da Cartesio -- che consce il modello prospettico e
anche quello dell'anamorfosi -- \textit{Gli ambasciatori} quadro, c'è
una striscia grigia sul pavimento, modello prospettico. Se uno si
sposta lateralmente, quella striscia diventa qualcos'altro, la
possibilità di vedere una figura solo da un punto di vista. Quando
Galileo e Cartesio operano su queste cose, il disegnatore di Galileo è
un prospettico di professione, Niceron, allievo del grande amico di
Cartesio Mersenne. Quello interessante, dal punto di vista teorico è
che questo discorso soggetto-oggetto, ha l'idea che si ha un oggetto
che sta di fronte, e quindi si presuppone una conoscenza frontale che
è già una posizione, noi non stiamo solo di fronte, ma abbiamo un
dintorno, e lo dice ponta. L'elemento frontale è un elemento
aggressivo, importante, porta il soggetto, il cogito ergo sum, ad
essere contemporaneamente un contemplatore distaccato, e nello stesso
tempo un manipolatore dell'oggetto, perché sorge l'esperimento, ecc. È
ambivalente il soggetto moderno, l'oggetto lo possiamo modificare,
basta conoscerlo matematicamente.

Due incisioni di Albert Durer (?) dal punto di vista della tecnologia
della rappresentazione. Ci sono quadri che si intitolano prospettiva,
interessanti dal punto di vista filosofico, descrivono la tecnica per
farlo, presuppone la vista di un solo occhio, quindi non è realistica
per niente, e l'oggetto che dev'essere descritto sono due, una donna e
un (?) non c'è differenza, sono corpi descrivibili, ecc. La corporeità
non distingue tra oggetto vivente e non vivente, la tecnica di
riproduzione è la stessa. Salta l'idea della inter-soggettività (r). Si
pensi al rapporto fenomeno, noumeno da questo punto di vista, Nietzsche
dice che il soggetto non serve a niente. La sintesi a priori comporta
la costruzione della capacità di conoscenza del soggetto, con qualcosa
che si presenta come universale, e quindi costruisce tutto sulla
geometria euclidea, ma non c'è una sola geometria, e questo fa capire
la posizione di Nietzsche rispetto a Kant. Si capisce il libro di
Kuhn, dove sposta tutto rispetto alla posizione della comunità
scientifica rispetto al paradigma, e lui si rifà a Ludwig Fleck. Ma
tutte queste cose mettono in discussione questi rapporti di
verità-assolutezza, che funziona l'idea che, liberato il soggetto, si
trova la corrispondenza della verità con ciò che è oggettivo. Da lì
nasce il problema secondo cui i modelli della scienza, i sistemi
scientifici si devono ridurre a un solo modello, quello dell'esattezza
e della matematica. Quando si dice che la quantità, le domande a
risposta chiusa, sono tutte legate a quel modello. All'ansia di
garantire l'oggettività, ed è una cosa che nonostante sia saltata, ha
una funzionalità estrema e perversa, cioè il problema appunto del
quantitativo. Perché questa cosa? Perché Spengler è fra quelli che
capiscono che l'uomo faustiano nasce col modello prospettico.

[A essa oppongo l'anima faustiana, lo spazio puro e illimitato], è il
modello prospettico, e che cos'è e perché è faustiano. È quello che
cosa fa rispetto all'infinito? Quello in cui due rette parallela si
incontrano in un punto, e proprio nel chiudersi danno un senso
all'infinito, ed è lì che nasce l'estensione in uso moderno.

% Gli esempi geometrici di Cusano.

Chi è Fausto? Quello che fa il patto con Mefistone? La conoscenza e la
giovinezza, in questa ridicola situazione, che è però il lato ridicolo
di questo modello, di combattere il senso della morte, di recuperare
l'immortalità in una chiave materialistica, essere eternamente giovane
e fare il patto col diavolo. Ecco perché Spengler parla di uomo
faustino, la cultura moderna, oggi in crisi -- quella che è descritta
è quella nascente. Fausto rappresenta questo, fondamentalmente.
Ritorna a essere giovane, ha una storia con Margarita, conosce. È il
mito borghese della modernità, conquistare e gestire la natura ed
essere giovane.

% Bacone NO, I, 3. Il fausto e il Quijote, Robinson. 

Questi sono i grandi miti, e certo lo è anche Hamlet, però
quest'ultimo è molto moderno fondamentalmente per un motivo, che
anticipa il cogito ergo sum, perché il vero punto è che il fatto che
quando dialoga con sé stesso, alla fine diventa un'altro, cresce,
l'autocoscienza a livello forte. Non è la figura del dubitoso che non
sa decidere, ma l'esatto contrario. Lui ha il dubbio che però non
ferma l'azione. Dubbio è come doppio, ti confronti con te stesso, ed è
fondamentale come passaggio, mi sdoppio, e arrivo alla conclusione
perché mi sdoppio, perché il pensiero diventa oggetto di me stesso,
una cosa che un greco non farebbe, e non avrebbe senso per lui una
cosa di questo genere. Don Quijote è in un certo senso la fine di
un'epoca e l'annuncio di una nuova, e poi Robinson siamo noi, il
self-made man. Robinson è colui il quale riesce a costruire un mondo
da solo, ma è veramente da solo? No, per costruire il mondo
nell'isola, deve recuperare gli strumenti, e cosa sono? Lavoro
sociale: usa ciò che è stato fatto socialmente, e anche la Bibbia,
carta, e senza di questo non lo costruisce, e appena ha un rapporto
con altro lo sottomette, la prima cosa che fa è la scoperta di
un'orma, e come reagisce? L'altro gli fa paura, e passa i giorni a
costruire mura, si costruiscono mura perché quell'orma è un nemico.
Ibn Tufayl, quando scopre un altro si mette a gioire. È un punto di
vista... Dopo fa operazione coloniale: my name is master, e lo dice in
inglese. Qual'è l'unica cosa che non gli insegna? A caricare il
fucile, la tecnologia no. Il modello Robinson fu criticato da Marx su
una base, porta a una società che serviva ai fini privati di un
individuo. L'uso delle relazioni sociali per i propri fini privati.

Quando Spengler si pone questo problema perché analizza la questione
prospettica come modello, dice che la matematica [nasce e somiglia
nella sua nascita a un mito. Il primitivo trasforma nell'espressione
l'estraneo in divinità... Analogamente i numeri circoscrive.
intelletto umano si conquista il potere sul mondo, hanno in fondo la
stessa scrittura, la logica è matematica e viceversa... Inferenza...
esistono per l'uomo desto -- termine id eraclito -- ha oggetti
propreità relazione unità molteplicità, struttura immagine del mondo
che chiama natura, e che come natura conosce in funzione dell'uomo.
Natura è ciò che si può contare, storia ciò che no... La metafora di
Galileo]. È evidente che Heidegger ha letto Spengler su queste cose,
e la distingue concettualmente dal greco physys: mondo che si offre,
non mondo su cui si interviene (r).

Dentro questo modello nell'anima faustiana interviene un'idea
importante di profondità. Il modello prospettivo identifica la
profondità con la distanza. È esattamente questo modello ciò che viene
messo in discussione da parte di Nietzsche e dalla filosofia e l'arte
contemporanea, e questo tema di profondità-distanza, prospettico, non
è nel mondo greco. Il rapporto soggetto-oggetto è lavorato in modo
importante nei saggi sul rinascimento di Cassirer. Il primo che
capisce che la prospettiva pittorica e matematica non è un aspetto
naturale, ma ha una forma simbolica, è un allievo di Cassirer e lo
cita. È questo tentativo di togliere l'aspetto della simbolicità, con
tutte le problematiche che comporta, e l'unico filosofo che ha
anticipato queste cose è Vico, e non è un caso che rivaluta il greco 
(r) e anche l'assomigliare il verosimile al falso. Rivalutazione
ottocentesca del mito, parte del Novecento ha già cercato da uscire
dal modello classico del positivismo, e siamo cominciati a muoverci in
quel modello per cui il mythos e il logos sono due modalità diverse di
arrivare al vero. Analisi del rapporto tra argomentazione e
narrazione, e la cosa non funziona nemmeno per Platone. Hegel diceva
che Platone che i miti non sono la vera filosofia, ma il Parmenide, il
Timeo, ma sbagliava, perché interpretava il mito secondo il passaggio
del mythos al logos. In realtà non è così perché oggi nessuno studioso
di Platone l'ameterebbe mai, i miti producono senso, e si raggiungono
profnondità attraverso il mythos che non si raggiungono col logos e
viceversa. Ogni modo di affrontare il vero comporta sempre uno scarto.

\end{document}
