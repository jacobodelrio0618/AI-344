% !TEX root = ..\..\main.tex
\documentclass[../../main.tex]{subfiles}%
\ifSubfilesClassLoaded{\addbibresource{../../bibliography.bib}}{}

\begin{document}

L12 13-11-2025

Anders

Totalitarismo atomico, la minaccia atomica non è mezzo legittimo per
contrastare il totalitarismo, ma è il possedere la bomba ciò che
caratterizza un governo totalitario. \textit{Holocaust}. Solo
attraverso la finzione l'accaduto può essere reso visibile, il 1978 è
il 1945. La Germania post-Norimberga non aveva guardato il suo
passato. Non solo non ci sono ricordi, ma nemmeno traumi.

Individualità degli eventi, va collegata anche con la questione di
Foucault e Nietzsche, se massifichi, annienti. Nessuno si ricorda dei
mori in mare, nessuno sa chi sono, e l'operazione che si fa è di
questo tipo. Funerali del bambino annegato sono stati fatti soltanto
con l'autorità, non si devono vedere queste cose, ma semmai le cifre,
che non dicono niente. Questo è una cosa che ci dovrebbe fare
riflettere alla questione di cui si parla tanto, le emozioni. Questo è
un grande problema, ed è una nostra cultura, della cifra, del numero,
dell'esattezza. Fino alla guerra del Vietnam i giornalisti potevano
andare in prima linea, e ci sono fotografie che appunto fecero epoche,
e che hanno avuto un significato enorme proprio perché erano singole,
ma si è capito che la questione è personalizzare. L'idea, che è
fortissima della nostra civiltà, della guerra come sterminio. I grandi
problemi nascono dopo la prima guerra mondiale, perché si usa la
tecnologia per massacrare, ed è qualcosa di moderno, come agli
aztechi, è uno sterminio non dichiarato, e non si deve sapere. Nella
tradizione antica non esisteva il massacro, ma il sacrificio. La
società che vanta l'individualismo è l'epoca dove, di più, l'umanità
si è massificata.

Problema, rapporto che c'è tra finzione e verità. L'operazione che fa
Anders è che, se riesce a provare qualcosa di individuale, è meglio
dei numeri, e rompe il modello del moderno. Deve recuperare il
concetto di finzione nel senso letterario, come dare forma e figura,
immaginare, e in modo esplicito, non finto. Male-bene, vero-falso, la
cosa è più complessa, e se non ci fosse non avrebbero senso nè la
poesia e il teatro, che pure mo sappiamo che possono dare del vero. È
un problema che si è riproposto in maniera tragica proprio con i
lager. L'immagine, malgrado tutto, queste due fotografie (dei lager)
se si trovano fuori contesto, non si sa che sono dei lager, ma di
gente che va in gitta, ed è un rapporto tra l'evidenza e il fatto, non
si sa cosa sta succedendo, ha a che fare con il rapporto teoria-fatti. 

Il film e le statistiche si correggono a vincenda (bambino con il
pijama a righe, errori storici). Il grande problema è il rapporto tra
i documenti e l'elemento immaginario, e ciò che è importante è sapere
ciò che è \bsq{falso}. Il problema, naturalmente, è che rapporto si va
a istituire tra i documenti e la loro interpretazione, che ha sempre
bisogno dell'immaginario, e lì è un confronto continuo, ma i documenti
devono esserci, altrimenti non c'è differenza tra storia e romanzo.
Negazione e dimensione apollinea, si trovano le cenere nell'acqua del
campo accanto, e si tenta di strappare il bambino da questo fiume e si
torna in una realtà apollinea. È stato un momento generazionale in
Germania.

--- Hoizinga

\textit{Nelle ombre di domani}, viviamo in un mondo demente, e lo
sappiamo pure. È uno storico olandese che ha insegnato a Leiden,
Liberale, Cristiano. Le sue opere sono \textit{L'autunno del
medioevo}, classico della storiografia. Parla del ruolo della cultura
nella società, periodo prebellico, tra le due guerre, vigilia della
guerra civile spagnola, crisi delle democrazie, crisi del 29, che è
importante. La crisi inizia come la prima guerra mondiale, ma la crisi
del 29 è la presa di coscienza di questo. Il libro parla prima del
concetto di crisi di civiltà, poi entra nel tema della crisi attuale,
dal punto di vista intellettuale, scientifico e artifiiciale, ed è
fiducioso comunque della capacità dell'occidente di superare la crisi.
Rispetto alla crisi dell'impero romano abbiamo come analogia il fatto
che siamo alla fine di un'epoca, di una dissoluzione di costumi. La
crisi dell'occidente è velocissima, quella dell'antico dura tre
secoli. Poi c'è quella protestante, ma a differenza della crisi
attuale è un conflitto di valori, mentre quelle dell'occidente è
\emph{fra valori e negazione di valori in assoluto}. Rivoluzione
francese, faceva ancora sperare un futuro migliore. Quella dopo la
prima guerra mondiale è una \emph{crisi del concetto di progresso} che
porta una sfiducia nella ragione, ed è caratterizzata da una
regressione e irreversibilità -- mentre si pensa che gli antichi
potessero sempre restaurare i vecchi valori. Dopo questo Hoizinga si
chiede, quand'è che si dà civiltà? È caratterizzata da equilibrio,
armonia spirituale-materiale, ricerca dle bello, buono, ecc., e
dominio sulla natura -- non solo quella esterna, ma anche del dominio
interiore, sulle passioni, le emozioni, ecc. Oggi si ha una crisi
perché appunto c'è uno squilibrio in cui la tecnica domina sullo
spirito, ci si trova a una molteplicità di valori, e si distrugge la
natura senza tensione al dominio di sé. Per quanto riguarda la crisi
della scienza occidentale è caratterizzata del trionfo della volontà
di potenza sulla volontà di sapere, ed è l'esempio delle pseudoscienze
razziste.

Volgarissima superstizione del fatto tecninco, una delle teorie che
vengono utilizzate come strumento di dominio dell'uomo sull'uomo,
negando il principio antropologico della scienza
dell'\textlatin{\textit{humanitas}}. Questa tecnicizzazione, inoltre,
consente all'uomo alla possibilità di distruzione della natura, nel
senso che la guerra non è più un mezzo dell'ottenimento di beni, ma
porta alla distruzione. Lui non ha così presente il problema della
guerra nucleare anche se la esprime bene.

\emph{Cultura di massa}, non processo di appropriazione della cultura,
ma essa diviene un prodotto, e si porta alla perdita della capacità di
giudizio e critica. La causa di questo è la scuola pubblica,
l'abbassamento degli standard critici, il giudizio di valore. Ora,
questa crisi della cultura popolare ha portato al fenomeno del
\emph{puerilismo}, atteggiamento più immaturo di quel che le facoltà
permetterebbero. Abbiamo li strumenti per capire i problemi, e
tuttavia siamo disarmati di fronte a quest problemi. Si torna alla
propaganda. L'uomo-bambino tratta in maniera seria cose che non lo
sono e viceversa, come la politica. Mosse, \textit{La nazionalizzazione
delle masse} e l'estetizzazione della guerra con il futurismo. Dal
punto di vista filosofico, la crisi dell'occidente è caratterizzato
della cosiddetta filosofia della vita, l'istinto, l'azione immediata,
Nietzsche. La scienza è una volontà dell'uomo di affermare sé stesso,
eliminare gli ostacoli per i suoi desideri, rinuncia alla parte
intellettuale. Abbiamo accoppiato a questo la figura del superuomo,
colui che si pone al di sopra o al di là della morale comune, che non
è responsabile nei giudizi degli altri, e questo ci porta direttamente
all'aspetto etico-morale della crisi, dove si vede il crollo del
razionalismo etico, che viene sostituito da un nichilismo dove è
un'opinione e uno strumento, e l'esempio è il marxismo e Freud. Tutte
queste filosofie, deresponsabilizzano l'uomo moderno.

Ritorno all'internazionalismo, cultura intereuropea, ritorno a un
realismo morale e scientifico forte. Sembra interessante mettere a
cofnronto la figura del puerilismo con l'uomo faustiano di Spengler,
l'archetipo della modernità, che incarna la tensione verso superarer
il limite della conoscneza umana, disposto a tutto per ottenere questo
sapere. Ma non è l'archetipo della modernità, ma della crisi,
rappresenta le caratteristiche del puerilismo. Volontà di potere in
Fausto, rovesciamento della volontà di sapere, e il sogno dell'eterna
giovinezza, l'eterno bambino che non si assume mai le responsabilità,
la puerilità del giovane eterno. Ora, si potrebbe obiettare che tutte
queste cose in Fausto sono in qualche modo per la volontà di sapere, e
poi si potrebbe obiettare anche il fatto che, se l'occidente è
arrivato alla crisi, un motivo ci sarà. Hoizinga ha un'idea di
modernità cristiana, è un conservatore, e non ha l'idea atea alla
Spengler o Anders.

Critica a Schmidt e a l'autonomia del politico dalla morale. Il
concetto fondamentale del politico e il concetto amico-nemico; sono
concetti primitivi (in senso tecnico), dunque la politica è autonoma.
Secondo Hoizinga è una petizione di principio e non la dimostra. Si
deve dimostrare che il nemico è il concetto primitivo del politico, ma
pure amettendo la validità dell'argomento, si riduce il diritto a una
decisione del sovrano, il cui significa che, nel caso in cui si ha una
situazione democratica, un'equivalenza di forze sociali, si ha
automaticamente una guerra civile, oppure una dittatura. In politica
esterna si ha la guerra di tutti contro tutti fra stati. E non si ha
più una definizione giuridica di stato.
 
Cavalieri medievali, utopia, regole del gioco nei conflitti. Schmidt
ha ragione quando dice che contrappone la teorie a quelle che si fanno
in nome dell'universalità dell'umanità e del giusto, il realismo
politico è la critica dell'universalismo ipocrita, fortissimo nella
nostra cultura, Stati Uniti, democrazia, ecc. Schmidt vede l'ipocrisia
del liberalismo. Il nemico, per Schmidt è rispettabile nella sua
umanità. L'amico-nemico come gioco. Hoizinga, nonostante sia
cristiano, non accetterebbe la guerra in nome dell'umanità, e sono
temi portati avanti in modo \textquote{progressisti}, da movimenti,
direbbe Foucault, di liberazione, promanazioni della teoria del
progresso per cui la teoria rivoluzionaria si innesta nel processo di
liberazione che libera dell'umanità.

Storicamente c'è stato un mutamento che nasce nell'illuminiso, e lo
teorizza Kant e lo riprende Foucault. Riflettiamo nella rivoluzione
francese, dice Kant. È chiaro che chi ha vissuto il terrore non ci
vuole tornare, però se la valutiamo storicamente ci sono certamente
cose che si adattano: autodeterminazione dei popoli e guerra solo
difensiva -- tralasciando le ipocrisie.

Piano mondano e oltremondano, speranza di rifiutare il primato
dell'azione sulla verità e rifiutare la materialità per una metafisica
che tocca l'ideale, pur riconoscendo l'ipocrisia di questo. E comunque
sulla base di questi valori si creavano massacri e stermini. Rifiutare
le due cose contemporaneamente, il primato all'agire e il primato di
sapere, dialetticamente un superamento di questo conflitto tra uno
Schmidt e uno Hoizinga, riconoscere il primato dell'agire, e
rifiutando l'inevitabilità del conflitto. Liberare, poi il politico,
del resto è problematico.

Il problema è quello della disuguaglianza, e lì c'è il problema del
prendere parte, e Foucault questa cosa l'aveva capita; c'è un problema
del prendere parte, che ha un significato. L'idea del conflitto
assoluto poi diventa una metafisica e metastorica.

\end{document}
