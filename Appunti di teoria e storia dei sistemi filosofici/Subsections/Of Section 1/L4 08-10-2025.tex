% !TEX root = ..\..\main.tex
\documentclass[../../main.tex]{subfiles}%
\ifSubfilesClassLoaded{\addbibresource{../../bibliography.bib}}{}

\begin{document}

Questi testi sono soprattutto fondamentali se, in qualche modo, la
filosofia si vuole confrontare con la storia -- e non è lo stesso della
storia della filosofia, cosa che a giorno di oggi si è perso. C'è un
conformismo che è l'esatto opposto della retorica della ricerca come
qualcosa di innovativo, e c'è un rischio enorme, specialmente nella
fase storica attuale, dove si pensa che la storia è finita, e invece
siamo davanti ad avvenimenti del tutto \bsq{storici}, che solitamente
sono tragici. La storia in quanto tale significa mutamento, e il
mutamento è sempre inquietante; la tendenza è sempre a negarlo, e non
c'è ragione da credere che l'era moderna è quella del mutamento, è
invece costruita su mutamenti deboli. La moda è una novità continua
per confermare ciclicamente o stagionalmente quello che c'era prima.
Lo stesso mito del progresso ha anche problemi di questo tipo. E il
progresso è la parte che si presenta come non conservativa, e ciò non
è un caso. Il problema è che oggi ci si trova di fronte al
capovolgimento delle parole, e, citando Focault -- l'ultimo filosofo
che non ha mai abbandonato la questione della storia. L'ultimo
Focault, s'era posto un problema che è ancora un problema di oggi: che
cosa è successo e che cosa succede quando le forze di liberazione o
rivoluzionarie vanno al potere? Perché posseggono la parola
rivoluzione e liberazione, e in nome di quelle parole usano il potere
in maniera tirannica, ed è una questione storicamente complicata. 

Lui sta riflettendo sulle liberazioni nei paesi colonizzati in Africa
come la rivoluzione in Algeria, tutti i movimenti di liberazione
anticoloniale, che però magari alla fine costituiscono regimi
dispotici. Oggi non c'è nessuno che rinneghi la libertà e la
democrazia, in filosofia si è discusso molto dei paesi orientali
dispotici, e invece un occidente faro della democrazia; sono vecchi
discorsi con una storia. E si deve tener conto in primo luogo che già
la parola \textquote{oriente} è un'invenzione occidentale. Si cita
\textit{L'Orientalismo} di Said, e il maestro suo era un grande
critico di Vico, su cui torneremo.

Detto questo, se l'ultimo che si interessa nella storia nel secondo
Novecento è Focault, l'avvio di Husserl e il problema della
fenomenologia in chiave diversa diventa un punto centrale per la
discussione in Stati Uniti e in Europa. Non è un caso che Derrida
andava a insegnare in Stati Uniti non nei dipartimenti di filosofia,
ma di letteratura, e lo stesso vale per l'ultimo Focault, ed è
significativo tutto questo.

La ragione per cui è interessante affrontare queste cose nel
Novecento, è perché ci sono spaventose analogia -- seppure con
profonde differenza, naturalmente -- ci ritroviamo in una situazione
che ha delle forti analogie con il momento decisivo, quello che poi in
qualche modo prepara il 68 che è la guerra del Vietnam, che fu un
momento decisivo per la pressa di coscienza della gioventù. Con
moltissime differenze, che il mondo era diviso in due in primo luogo,
ma la cosa importante è che riguardava gli Stati Uniti, più che
l'Europa. I problemi sono rispetto alla storia, cosa sta succedendo e
cosa può succedere. E si noti il dibattito a livello anche televisivo,
dove tutto nella norma della libertà si è innescato un racconto
amico-nemico, quel meccanismo della semplificazione che è sempre stato
considerato pericoloso da tutti i pensatori. La semplificazione è
chiaro che uccide il dialogo, che è la vera democrazia, e siamo in
quella temperie in questo momento. Nella prima guerra mondiale c'è
proprio un salto rispetto a quelle precedenti, le trincee, le navi, i
l'inizio dei carri armati. Si arriva alla guerra con un'esaltazione
estrema della tecnica e del progresso, le città diventano metropoli, i
collegamenti diventano le ferrovie, ci sono le grandi esposizioni
universali nelle città europee, è il clima in cui Marx scrive
\textit{Il capitale}, e anche ciò non va dimenticato, il fatto che non
si percepiscano i rapporti di sfruttamento. Il contesto è quello di
grandi mutamenti che si portano dietro delle ferite enormi.
Contestualmente l'Ottocento si porta avanti il colonialismo, e si
pensi alla Gran Bretagna.

Ed è sul piano della seconda guerra mondiale che succedono delle
riflessioni, si pensi che per esempio che Anderson, allievo di
Heidegger, marito di Arendt, dopo Hiroshima e Nagasaki si pone
proprio il problema della guerra atomica come problema filosofico
centrale. Non si torna indietro nella storia, c'è un dato di
irreversibilità, e si giocano molte questioni, una delle quali non
viene discussa oggi, ed è il futura: che futuro ci sarà? Lo stesso di
oggi? Si può cambiare? È questa la domanda di Focault sui movimenti di
libertà, ha posto il problema sulla ferita, e nasce sia per i
movimenti di liberazione, sia per i movimenti rivoluzionari che
falliscono come la rivoluzione russa e il \bsq{socialismo reale}: ci
si ritrova senza libertà. Sono macro-problemi storici. Qual'è
l'alternativa? Inquinamento, o \bsq{capitalismo ambientalista}? Si è
capito che non è possibile una cosa del genere per la stessa struttura
del \bsq{capitalismo reale}. È tipico dell'arroganza occidentale, il
massimo del colonialismo, che la democrazia si può esportare. Nessun
paese che è stato sottomesso è stato costruito sulla democrazia, e
l'unico al mondo che l'ha fatto è l'Israele, che però è anche l'unico
stato etnico.

Come si fa a farsi queste domande rispetto a fatti storici così
potenti che non si vedono, o meglio, che si vedono senza vedere in
senso freudiano. Rendere visibile, nell'arte, significa che l'arte fa
vedere in modo diverso quello che si vede già, e questo è il punto,
ed è esattamente il problema dell'arte contemporanea. La ragione è che
il visibile, ciò che l'arte fa vedere è ciò che si vede già; non ciò
che si vede nel senso che è nascosto, ma che si vede e si nega. Questa
è la vera rivoluzione di (?), non a caso il pittore di Ponty, perché è
il primo che capisce in modo esplicito questa cosa. La pittura è
un'affermazione sul mondo, non è una copia, non si deve porre il
problema tipico della modernità di dare una terza dimensione. Anche
Nietzsche fa un passo sulla profondità, \textquote{i greci erano
superficiali per la profondità}, il profondo non è quello che è
nascosto dietro, ma è nella superficie, sebbene nascosto da essa. 
% Paul Valery
Lo stesso Spengler entra in questo contesto. Lui è un matematico e si
pone il problema delle epoche storiche e delle civiltà ponendosi
proprio questo tipo di questione. Quando parla dell'uomo faustiano, è
esattamente colui che pensa l'infinito nell'astratto come se fosse
nella profondità, questo è l'uomo dell'azione, l'uomo che agisce e che
usa la tecnica, che si muove per trasformare il mondo, ed è il
contrassegno della profondità e quindi anche della prospettiva. L'uomo
che utilizza la matematica per riprodurre \emph{esattamente} il mondo,
e l'esattezza comincia a confondersi con la verità; è il modello di
Cartesio, che vuole assimilare il verosimile al falso e così via: non
è che nega il verosimile, ma lo collega al falso facendo una
fortissima operazione che ci trasciniamo dietro, e una delle
conseguenza di questo è che il mythos è falso e il logos è vero, ed è
la stessa cosa che smentirà Nietzsche e che smentirà anche Vico. Non
c'è questo contrasto che invece è potentissimo nell'Ottocento, in un
ripensamento del pensiero greco in chiave moderna. Per Aristotele e
Platone il \textit{philosophos} è parente stretto del
\textit{mythophilos}. Questa è la ragione per la quale Max Weber dice
che c'è la delusione del concetto dopo la prima guerra mondiale,
ritorniamo a credere ai miti, la potenza del logos che fa del bene, le
persone cominciano a non crederci più, ed è per questo che Nietzsche
viene letto, che mette in questione l'astrattezza del concetto.
Spengler coglie nella modernità questo aspetto, l'uomo faustiano o
l'uomo della tecnica.

La faccia parallela di tutto ciò è Heidegger, i greci, i problemi
della tecnica, e lui da questo punto di vista si presenta come
critico della modernità. Frase di Protagora che l'uomo è misura di
tutte le cose, e l'essere non è a misura dell'uomo, e questa è
paradossalmente la frase di Alberti.

% L'uomo e la tecnica di Spengler, va messo in collegamento con
% Heidegger e col discorso suo sulla tecnica. Coglie uno dei grandi
% problemi, tipico della tradizione tedesca in cui si mettono insiemei
% la rivalutazione del mythos ma in chiave romantica e nostalgica.
% Questo tema tedesco dell'originario, ritorno indietro, mondo scisso,
% si torna indietro a questo mondo dove pensare ed essere si fondono,
% oppure siamo costretti a pensare alla dannazione della scissione?
% Modello sulla scorta di Schelling, tipico della tradizione tedesca,
% il mondo va in peggio, decade, cade, che è il problema di Spengler,
% il tramonto dell'occidente. L'atteggiamento della cultura tedesca
% dentro questo, c'è Jung, Schmidt (?). Questo tema non porta a un
% futuro, ma alla fine, viene tolto un futuro e la storia diventa solo
% il luogo della lotta dove prevale il più forte, diventa lotta e
% potere, una società diversa non è immaginabile. In Spengler è più
% complessa perché non ragiona in termini di tempo lineare, e il
% tramonto dell'occidente convive con i cicli di altre civiltà e così
% via. Non c'è la veduta del tempo-freccia.


% La crisi della civiltà. Un conservatore antinazista, internato per
% due anni, morto a causa della prigionia. È famoso per essere
% l'autore di un libro discutibilissimo che è l'autunno del medioevo,
% e subito dopo la crisi della civiltà scrisse (?) che è un
% capolavoro. Lui sì ha l'uomo del mito, nostalgico, per lui il
% medioevo è il mondo della cavalleria, ecc., lui analizza quello che
% si è perso del medioevo nell'affermazione della modernità. Quasi gli
% unici interessati al gioco nella filosofia, a parte Platone sono
% Schiller, (?), [D'Alembert e Diderot?].

% Origine e senso della storia, del 59, importante perché in questo
% tentativo di ridiscutere la storia universale dopo la seconda guerra
% mondiale, e ovviamente quando si vuole discutere la storia
% universale si pensa a Spengler, ma il riferimento è anche ovviamnete
% Hegel. Quando Spengler parla di tramonto dell'occidente va contro
% Hegel, per cui il punto più alto della storia è l'occidente. Per
% Hegel il passaggio da oriente a occidente significa che alcuni stati
% o nazioni incorporano la storia del mondo, e si dice che l'impero
% romano è il modello della storia universale. Nel settecento
% Montesquieu studia la decadenza dell'impero romano, e l'illuminismo
% è molto interessato alla decadenza, oltre il progresso. Storico
% scozzese amico di Hume, Gilbert o Hilbert, studia la decadenza
% dell'impero romano, del medioevo e del rapporto tra cristianesimo e
% modernità, che è quello che fa Hegel, ma conclude che il futuro è
% l'America, e questo suona molto attuale, visti i fatti recenti in
% Stati Uniti. Teoria dei quattro imperi universali, trasformata con
% la fine della storia, che è la tradizione apocalittica, che è nella
% tradizione testamentaria, che ad un certo momento la storia finisce
% e nasce un nuovo mondo, e la cosa ritorna nella filosofia fino a
% Melantone, e poi si arriva Hegel che non crede alla visiona
% apocalittica, ma che il mondo si muove verso occidente. Spengler
% rovescia la questione e si fa la domanda, dopo la prima guerra
% mondiale, se la civiltà occidentale non sia alla fine. In un certo
% senso è anche la risposta a Hegel.

% Perioso assiale, periodo di vari secoli, in grosso modo dall'ottavo
% secolo ac al secondo dopo cristo dove in tutto il mondo c'è un
% grande cambiamento. A un certo momento l'intera umanità ha un
% mutamento, e questo è interessante, molti l'hanno ripreso. È
% interessante perché in questa svolta culturale nascono versioni
% diverse del periodo assiale, Buddha viene posto da Jaspers allo
% stesso livello di Cristo, e questo dal punto di vista dell'occidente
% è ovviamente problematico dal punto di vista del Dio unico.

% Significato e fine della storia, è un analisi il cui sottotitolo è:
% i presupposti teologici della filosofia della storia. Analisi delle
% teorie della storia, moderne e non moderne. Comincia da una figura
% importante che è Burckhart (?), il grande punto di riferimento di
% Nietzsche, e ha scritto questo libro del rinascimento che ha fatto
% epoca. Poi analizza il materialismo storico di Marx, Vico, Voltaire,
% Bossuet, Gioacchino da Fiore e l'idea apocalittica, e il rapporto
% che Agostino ha con la storia, e uno degli allievi di Agostino è uno
% degli ultimi teorici della teoria dei quattro imperi, e gli
% agostiniani sostentavano questa tesi della successione degli
% imperi, e la grande battagli era qual'era l'ultimo impero, e i
% tedeschi si sono mossi molto in questa chiave, idea di Melantone che
% Hegel riprende e si vede Carlo Magno come continuazione dell'impero
% romano, che oggi sembrano operazioni fantasmagoriche, ma al tempo
% erano battaglie filosofiche di legittimazione. Jean Bodin attacca la
% teoria di Melantone, non erano cose accademiche ma avevano un senso
% politico molto forte.

% Il mito dell'eterno ritorno, filosofo antropologo, storico delle
% religioni molto importante, ed è un testo interessante e
% significativo perché ci sono capitoli interessanti. Contrappone il
% mito dell'eterno ritorno al tempo-freccia, e l'ultimo --
% importantissimo -- il terrore nella storia. Per lui il fatto che
% l'apocalittica non c'è più e quindi viene a meno la rigenerazione
% del mondo, rende povero il tempo-freccia. Tu non puoi più
% giustificare il male, ma accettarlo come realtà. Ecco perché lui
% parla dell'eterno ritorno, della semplicità. E dopo la seconda
% guerra mondiale, uno dei grandi problemi è quello della
% reversibilità: i morti sono morti, e non c'è più in Dio che se li
% porta dietro, sono morti laici. Siamo nell'epoca
% dell'esistenzialismo e del nichilismo, tanto che Hans Jonas (?)
% sostiene che l'esistenzialismo contemporaneo è il nichilismo più
% imponente. Il terrore della storia è legato a questo non c'è senso
% della storia, che è precisamente quello che interessa Jaspers. Non
% riesci più a spiegare che cosa significa la morte di bambini
% innocenti. Il concetto di Dio dopo Auschwitz, problema del rapporto
% tra l'ebreo e Dio, e dopo quello non può essere onnipotente, perché
% se lo ammettiamo così vuole il male, e quindi l'unica spiegazione è
% che non è onnipotente. Lo stesso problema sollevato da Voltaire e di
% Rousseau dopo il terremoto a Lisbona. Quindi c'è questo tratto
% drammatico davanti alla storia che si pone la filosofia, e allora
% come ci rapportiamo con la storia? Che i bambini innocenti vengano
% uccisi? Tutto nasce dal fatto che le guerre hanno sempre effetti
% collaterali, fa parte della guerra che uccidano civili. Lo devi fare
% anche se poi scrivi che è stato un errore, per vincere la guerra.

% Non esiste tempo lineare e tempo ciclico da solo, si intrecciano.

% L'ultimo testo è Benjamin, dove recupera lui da ebreo la tradizione
% apocalittica in una chiave d'interpretazione legata al materialismo
% storico.

\end{document}
