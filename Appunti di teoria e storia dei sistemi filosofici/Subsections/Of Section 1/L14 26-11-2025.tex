% !TEX root = ..\..\main.tex
\documentclass[../../main.tex]{subfiles}%
\ifSubfilesClassLoaded{\addbibresource{../../bibliography.bib}}{}

\begin{document}

Benjamin, questione storia e progresso. Storiografia materialistica.

Prima tesi, mette insieme due aspetti, materialismo storico e
messianismo. Esempio dell'automa che gioca agli scacchi. Vince ogni
partita perché ha un nano che lo gestisce. Si intravvedono i temi di
materialismo e teologia (l'istinto messianico). Osservare in maniera
staccata la dominazione, come fa Marx nel \textit{Manifesto}.
Nell'idea della società senza classe, c'è qualcosa di messianico, e la
social-democrazia tentò di arrivare a quest'idea, ha elevato questa
storia materiale, priva di una teologia, a un infinito, a qualcosa di
inevitabile. Quest'interpretazione ha un corrispettivo che si è visto
nell'avvento dei fascismi. Il fascismo, che viene visto come uno stato
di eccezione per i fautori del progresso, in realtà è la norma. Stato
di eccezione di Schmidt, per Benjamin lo stato di eccezione diventa la
norma. La teologia, in tutto ciò, intesa come spirito messianico serve
come atemporalità riguardo alla storia, e la secolarizzazione della
teologia può portare [...]. Il messianismo è composto da vari aspetti. 

La storiografia, resoconto del passato, è possibile solo con una
redenzione, in termini politici, una \emph{rivoluzione}. Qual'è
l'oggettivo critico di Benjamin, e come guarda al progresso? Il vedere
la storia come successione [...]. Lo storico è portato ad approcciare
la storia dalla parte dei \bsq{vincitori} e porta alla
\emph{malinconia}. Questa successione di fatti storici si lega con
l'idea di un continuo progresso, presente nelle social-democrazie e
nel marxismo. Il conformismo è una causa del crollo della democrazia
(Weimar). Il pensiero che il progresso tecnico porti a un
miglioramento politico, e la consecuzione di questo stato tecnologico
è il fascismo, che non è eccezione. È una storia, quella del
progresso, è visto come universale, valido a prescindere, illimitato,
neanche limiti della natura, e il fascismo è una variante, non
un'interruzione del progresso. Temporalità omogenea e vuota, non c'è
niente se non il progresso, non ci sono soggetti storici, possibilità,
ecc. Pretesa dogmatica. L'idea di un progresso del genere umano nella
storia è inseparabile di quella di un tempo continuo e vuoto.

Da tutto ciò, cosa dev'essere la storiografia materialistica?
Innanzitutto ci vuole un'idea non oggettivante della storia, delle
strutture, ma stanno in un rapporto dialettico con la storia materiale
passata. Le lotte spirituali sono presenti nella lotta di classe [?].
Da qui viene fuori un rapporto diverso del passato che rifiuta
d'accettarlo così com'è stato, o com'è stato scritto (dai vincitori).
Immagine del passato che mostra le possibilità rivoluzionarie che si
trovavano lì dentro.

Non significa conoscere com'è stato davvero il passato, ma è
l'appropriarsi di un ricordo. Bisogna ovviamente allontanarsi dallo
storicismo. Questo rapporto con il passato crea un legame, che non è
col futuro [?]. Si evita la trappola del progresso. Rapporto
dialettico col passato, le possibilità rivoluzionarie del passato sono
possibilità adesso. L'\emph{adesso} mette in relazione le temporalità,
e permette di unire i due aspetti del messianismo che stanno insieme.
L'adesso come modello del tempo messianico, coincide con la storia di
che cos'è l'umanità. Cos'è questo adesso? L'adesso del materialismo
storico dev'essere una continuità che si arresta in un'unità. Unità di
tensioni, crea, costruisce un'immagine cristallizzata, monade, unità
che è un una relazione che restituisce la totalità del mondo.
Principio costruttivo del pensiero, cristallizzazione in una monade,
chance rivoluzionaria. Questo adesso, questa monade che si trova tra
il passato e il presente, sono disseminate nelle possibilità
rivoluzionaria [?].

Quadro, lettura allegorica di \textit{angelus novus}. 

Tempo lineare e tempo ciclico [...]. È questa che permette l'ideale
messianico del ritorno all'originario, risurrezione dei morti nella
rammemorazione di essi.

---

Commento Iacono.

Come ci si confronta di parte alla storia. Molta parte della filosofia
rissa con la storia, e spesso molti autori vengono affrontati non in
rapporto alla storia, in particolare Benjamin e Foucault. Perché
determinati filosofi hanno bisogno di confrontarsi con la storia? Il
fascismo e nazismo, sono eccezioni della storia oppure no? In termini
d'oggi, il rapporto capitalismo-democrazia è un'eccezione della storia
oppure no? Il modo di produzione capitalistico va verso un ritorno a
una posizione eccezionale che è il fascismo? È il problema di Benjamin
con la socialdemocrazia tedesca, nel concetto di progresso, che è
quello che dona continuità alla storia. Comunque sia, va, pur se ci
sono rallentamenti, e ciò che alla fine diceva Kant, \emph{non si può
andare se non verso il meglio} che si traduce in un \emph{non puoi
fare altrimenti di come fai}. Il problema di Benjamin, che sarà anche
quello di Horkheimer e Adorno è far notare che il nazismo e nel
fascismo illimunano elementi del capitalismo, non sono eccezioni nel
continuo dell'avanzata verso la felicità. Se si legge la dialettica
dell'illuminismo horkheimmer e adorno sostengono fortemente questa
tesi. Si fa una carrellata rispetto all'occidente per arrivare a
questo punto. E si pensi al dibattito superficiale odierno, la
discussione è ancora questa, siamo a livello di quello che Benjamin
chiama conformismo, che per la socialdemocrazia aveva portato alla
prima guerra mondiale -- parte della socialdemocrazia si era schierata
con la guerra --. Mettere insieme materialismo e tradizione della
Kabbala ebraica. Il materialismo non è l'intepretazione migliore del
progresso, ma che comporta non la continuità, ma l'idea della rottura,
e lì entra il motivo teologico, che è epistemologico in Benjamin.
L'idea come arresto, e nello stesso tempo il qui ed ora. Non è il
tempo cronologico ma il tempo-struttura (tempo di guerra, ecc.).
Ripresa del modello teologico, comporta dell'irrompere della storia.
Quando parla di catastrofe cita il sogno di nabuconodosor nel profet
adaniele. La prima formulazione della teoria dell'impero universale,
ma il punto teorico complesso di questa storia. La questione teorica
complessa è che c'è un mazzo, arriva ai piedi di argilla, il gigante
crolla. Tutto il passato, la successione dei quattro imperi,
\emph{crolla di una volta}, la storia è un precipitare. È tutto contro
una filosofia della storia che prende schemi. Passato-lotta,
passato-massacre, passato-lotta di classe, che la storia nasconde,
comunque si va sempre per il meglio [?], il futuro riserva qualcosa di
meglio, anche grazie al sacrificio. Il dato di fondo è quest'elemento
del modello teologico per smuovere l'aspetto della rottura della
storia. Nello stesso tempo che fa queste cose (verso il 1938) scopre
un autore di nome Jochman [??] perché è un filosofo tardo-illuminista
e dice che, se si pensano alle piramidi, nessuno si ricorda delle
migliaie di persone morte per farle.

Ricolloca le catastrofi conformisticamente in un mondo che deve andare
come deve andare. Ecco perché parla del fermarsi della storia. Questo
Jochman, Benjamin si mette in testa che è seguace di Vico. Perché
Benjamin riconosce in Vico colui che comincia --ed è una cosa presente
in marz, nota 89 del capitale -- il ruolo della storia in vico, e
anche quello \emph{non dei vincitori della storia}, ripensamento
rispetto al modello gestito dai vincitori, perché oggi è così, con i
mass media che funzionano a cancellare la storia, e lo si fa
impunemente. Si è al cinismo più spaventoso, ed ecco perché ce l'ha
col progresso. Il punto è che il modello progresso è fondamentalmente
giustificatorio, e il sistema di giustificazione storica non è
ingenuo, può riconoscere i danni ma li riassorbe, riconoscendo che
comunque il risultato finale è per il meglio. Il corrispettivo è
l'idea che, per esempio, i capitalisti, se si arricchiscono con ogni
mezzo, fanno il bene dell'umanità perché anche i poveri si
arricchiscono per il trickle down. Modello che però è stato acquisito
come tradizionale.

Il punto è rompere l'occultamento, Foucalt, follia, non è l'eccezione
che viene riassorbita, si legge il normale nel patologico, ed è quello
che sta sparendo oggi. C'è un rischio cognitivo, si ritorna a un
modello di giustificazione adeguata al tempo: conformismo, sviluppo.
C'è tutta questa visione pulita, i paesi vanno aiutati. Queste guerre,
(Ucraina-Russia), sono la normalità dell'occidente o sono eccezioni [fa appello al riarmo europeo e all''ipocrisia']?

Questa è davvero la filosofia di fronte alla storia. E Benjamin,
soprattutto l'ultimo, è legato a questo tipo di dimensione. Forse è
venuto il tempo di essere meno accademici e affrontare le cose come
sono.

Togliatti e Gramsci. La situazione politica in Italia, crisi della
democrazia, impossibilità di gestire il futuro.

Referenza a Palestina, si è parlato del dibattito accademico sull'uso
della parola \bsq{genocidio}, e invece i bambini muoiono come nulla.

L'informazione come bene privato, nessuno tocchi la proprietà privata,
e nessuno, nemmeno a sinistra, dice niente. Le cose di Mamdani sono le
cose tipiche di una sinistra moderata, quasi di un centro-sinistra, la
patrimoniale? Si disturbano i ricchi!

\end{document}
