% !TEX root = ..\..\main.tex
\documentclass[../../main.tex]{subfiles}%
\ifSubfilesClassLoaded{\addbibresource{../../bibliography.bib}}{}

\begin{document}

Goethe e Nietzsche Spengler. Goethe modello morfologico, sia dal campo
della storia del vivente, sia dal campo della storia, e questo modello
deriva, a sua volta, da una forte tradizione francese di cui Goethe
era molto attento, perché soprattutto a inizio Ottocento, apre il
museo di storia naturale, che era la sede dell'istituto non solo
botanico ma biologico. Fu il teatro di un'enorme discussione teorica
sulle forme di vita. Era una discussione (Parigi 1831) antagonismo tra
l'idea di forma e l'idea di mutamento. Lamarck parla dell'evoluzione,
e in qualche modo va tenuto conto che Parigi è il luogo dove nasce un
primo tipo di sociologia, Auguste Comte, progresso in senso forte, e
anche in Brasile (ordine e progresso), però contemporaneamente a
questa teoria si sono innescate altre, una quella lamarckiana, uno
dell'altro padre della geologia, che sosteneva che la storia della
terra andava per catastrofi. Quindi nella teoria, in qualche modo non
era un evoluzionista perché negasse il mutamento, ma perché era una
rottura totale. Contemporaneamente ci sono biologi importanti che
sostengono che il campo fondamentale della vita è la \emph{forma}: uno
si chiamava Ettiene Geoffroy-St. Hilaire e Cuvier: c'è uno scontro a
Parigi tra la forma e non altro. Sono personaggi importantissimi,
aprono il dibattito. Il figlio del primo è interessante (Focault li
consoceva benissimo) è il primo che inventa quello che allora era un
campo del sapere, lo studio dei mostri, di tutte le varianti
mostruose del campo della vita, e questo poi si troverà nel dibattito
del Novecento, quale significato dare al mostro. La domanda di fondo è
se il cosiddetto mostro è foriero di una nuova specie, oppure una
deviazione destinata a morire, e che quindi va eliminata. Il tema
della follia, Focault. Un biologo degli anni Trenta del Novecento
usava questo termine (?) tradotto mostro speranzino, le deformità che
però aprono a un ramo possibile. I rami di deviazione, vengono
eliminati o aprono il discorso a una nuova interpretazione?

Il problema della rottura sono tutte consapevolmente metafore
politiche. Quando Napoleone si insedia lo fa come rivoluzionario.
Rottura di una storia o aggiustamento? E fin'ora gli studiosi aprono
una discussione se era una rivoluzione di rottura. Diventa un problema
forte, e il parallelo avviene nel campo biologico. Quello che ora si
chiama neodarwinismo è un tentativo di trovare una sintesi tra il
darwinismo e la teoria delle forme. Quando Goethe si pone questo
problema, si pone proprio il problema della forma. È la risposta al
dibattito francese -- e si crea una tradizione tedesca -- cerca la
pianta originaria, cioè l'archetipo, il modello da cui poi tutte le
forme, modificandosi si generano. Era questo il problema di Goethe.
Spengler riprende a pieno questo tema. Per lui apre una tradizione che
non è nemmeno in Goethe, la tradizione romantica tedesca, la nostalgia
per l'originario che arriva fino a Heidegger, lo stesso tema del
rapporto che i tedeschi hanno con i greci antichi nasce all'interno di
questa dimensione nostalgica, e critica del progresso in questa
chiave. L'idea è che c'era un mondo originario, poi una decadenza e
diviene un mondo mitico, ed è in questo contesto che viene rivalutato
Vico. Viene rivalutato grazie al fatto che lo traducono, e viene
considerato anticipatore della teoria dei miti dell'Ottocento. Il
discorso d'analisi di Heidegger, fra Parmenide ed Eraclito sono tutte
legate a questo contesto teoretico-culturale, contesto dove aveva già
operato Nietzsche, che nasce come filologo antico. Spengler è proprio
tra Goethe e Nietsche in questo senso. Spengler applica di Goethe il
modello morfologico, ma per la storia della civiltà, per cui per lui,
qualunque civiltà è come una pianta, ha un momento di fioritura,
cresce e poi muore, indipendentemente che ci siano contatti,
confronti, ecc. È un'analogia insostenibile, ma quello non è il punto.
L'operazione di Spengler \emph{rompe i due grandi schemi della
tradizione europea}, lo schema del \emph{progresso}, la successione,
il modello ehgeliano e il discorso globale, la storia del mondo dove
ogni nazione incorpora l'intero movimento della storia, ritradotta in
chiave del progresso, ancor oggi potentissima. 

Il problema di questa teoria della ciclicità è che non è una posizione
totale rispetto al progresso, la teoria della ciclicità, sia quella di
Machiavelli e Vico. Le teorie della ciclicità amettono la cumulazione,
non è un ritorno indietro totale, ma è un ripetersi nella cumulazione,
cioè nel nuovo. In realtà non esiste nessuna teoria totalmente ciclica
o lineare, ma le due cose hanno un equilibrio diverso. La teoria
ciclica fa capire come si ritorna, ma non è totalmente. E tutte
comportano il fatto che coloro che le adoperano maneggiano benissimo
il tempo lineare -- Machiavelli si rifà a Tito Livio.

La questione è che anche nella ripetizione della storia c'è
un'irreversibilità in quello che è accaduto. Gli unici, invece, che
hanno pensato a un ritorno, sono i romantici attraverso la possibilità
di tornare a un mondo della poesia, e in molti c'è l'idea che questo è
un mito. Il dato di fondo della storia, chiaramente, è che studia il
passato, e il passato è irreversibile. Ci sono delle analogie, dei
ritorni, ma in condizioni nuove. Se nei cicli della politica c'è una
successione dalla democrazia alla tirannia, teoricamente è un ritorno,
ma le modalità in cui degenera è diversa, e persino le persone sono
diverse, sono modelli teorici che servono a capire il gioco delle
analogie. Non negano mai la storia come dato irreversibile e
cumulativo, altrimenti non ci sarebbe storia, e questo va chiarito
perché il rapporto tra ciclico e lineare è molto complesso.

Il punto di riferimento di tutti per il tempo ciclico è Polivio, e
come dicono gli storici antichi, questo è vero, ma lui usa il tempo
lineare per fare storia, e non può non usarlo. La storia, in quanto
storia è un mutamento, se no, non sarebbe storia. Non è un mutamento
che torna indietro mai, ma come modello sociale, politico, i
protagonisti di questo ritorno sono nuovi e con eventi nuovi,
altrimenti non ci sarebbe storia né memoria, e sono tentativi, in
qualche modo di giocare la partita in rapporto al tempo lineare.

La figura importante è Jean Bodin, attacca Melantone, mentore di
Lutero, perché Melantone sostiene la teoria dei quattro imperi
universali, perché è una teoria della successione di imperi.
Melantone, in sostanza, anticipa Hegel, riprende il fatto che se la si
ricostruisce si pensa a Carlo Magno -- fussione tra tradissione
germanica e celtica, e si arriva alla costruzione che la Germania è il
fine. In questo caso il tempo ciclico lo si considera in questa
posizione, e questo va chiarito da questo punto di vista. I due
modelli sono in qualche modo questi. Quello che caratterizza Spengler
è che non c'è un discorso sull'unificazione della storia universale
che si muov eciclicamente o linearmente come un tutt'uno. Lui è un
conservatore reazionario e relativista. Analogia della pianta senza
che ci sia né un passaggio né un ritorno indietro. Per questo usa
Goethe, e per questo Spengler è morfologico.

E invece qui il dato curioso è che ci troviamo di fronte all'idea che
per lui esistono tanti mondi, oltre che tante civiltà. Di che cosa
viene accusato lui? Un autore importante, per esempio, uno dei grandi
padri dell'antropologia contemporanea, Marcel Mauss, scrive fra le
altre cose perché scrisse negli anni Venti il saggio sull'oro (?),
rovesciando la figura dell homo economicus, che nasce primitivamente, la tendenza naturale dello scambio sull'oro.

Marx si richiama, al contrario, la tendenza dell'uomo naturale a
barattare è una cosa mitica, l'uomo isolato non esiste, ma comincia a
isolarsi nei rapporti sociali più sviluppato, e si richiama al modello
aristotelico di animale sociale. È su questa linea che Mauss dice che
lo scambio non è l'elemento determinate della natura umana, ma il
\emph{dono} come strumento di potere. Quindi il modello dell'uomo
economico è appunto una costruzione coloniale inventata. Salis, studiò
con Strauss, studia le società neolotiche, dimostra tecnicamente che
nella società neolotiche le persone lavorano quattro ore al giorno.
Sono proiezioni nostre, l'idea che un selvaggio passa tutta la
giornata a cacciare, è una proiezione del mondo borghese a quello
preistorico. Allora, perché si è evocato Mauss? Perché è allievo di
Durkheim, e rappresentate della posizione morfologica della società, e
in un saggio intitolato \textit{Le civiltà} non può non tener conto di
Spegnler, e lo accusa di essere troppo generico, e forse ha ragione,
però non può non tener conto.

Hozinga, mito del gioco, come si prende distanza dalla società
capitalistica? Il fatto che crollano questi valori. Marx ammiratore di
James Stuart, da destra, coglieva le contraddizioni borghesi le
vedeva, anche se in chiave conservative, avvertivano la crisi. L'epoca
che secondo Spengler stiamo sperimentando è quello del tracollo, e
nasce soprattutto dopo la prima guerra mondiale, dove l'esplosione
della tecnologia come mutamento diventa mezzo per il massacro di
milioni di uomini. Ecco perché c'è la nostalgia del cavaliere, l'uomo
delle regole, ci si sfidava, ma era come un gioco. La prima guerra
mondiale non ha più regole. 

Questi avvertono -- e riscoprono Nietzsche -- qui viene fuori
l'analisi dell'uomo faustiano, non è soltanto l'uomo borghese,
l'imprenditore, ecc., ma c'è questo carico potentissimo la rottura
delle guerre. Non l'accettano, quello dell'idologia dominante, della mano
invisibile. L'idea che nel fare il proprio interesse economico si fa
l'interesse della società è il tema del neoliberismo contemporaneo, il
tema dei governi attuali in Europa negli Stati Uniti. Colui che fa
moltissimi soldi è un benefattore della società perché opera una forma
di trascinamento. Era un po' il mantra negli anni Novanta che era
Berlusconi, l'imprenditore che va al governo.

Kant fa un errore, riferisce lo spazio a una geometria universale, tra
la geometria intuitiva e la natura, che non può essere dimostrata.
Elemento fondamentale in Wittgenstein, non c'è relazione
linguaggio-mondo, mito che da Cartesio arriva a Kant l'ha fatto
saltare Nietzsche, viene aperto nel Novecento, nessuno pensa che si
sia una corrispondenza linguaggio-mondo. Non si ha questo modello
assoluto universale, e la questione si dà, con grande senso di
smarrimento, ma con grande senso di liberazione anche. Dice Spengler,
che ha studiato matematica: la diretta corrsipondenza non può essere
applicata, nemmeno con strumenti di misura. E soprattutto il secondo
Wittgenstein ha a che fare con Spengler. E il problema è la misura,
che nella trdizione moderna significa esattezza: questa cosa fallisce,
non risciamo ad avere una comprensione esatta del mondo. Non esiste
una mappa che sia la misura esatta del territorio che deve descrivere.
Una mappa che descrive in modo totale il territorio rappresenterebbe
un paradosso, sarebbe la riproduzione del territorio. La mappa non è
il territorio, affermazione semplice ma importantissima. Se si
indentificasse si raggiunge l'assurdo, l'aleph, il golem. L'elemento
della conoscenza è nella dissimmetria, nello scarto, non nella
perfetta corrispondenza ma in ciò che resta. È quello che dice Calvino
citando Wittgenstein, e Calvino è un uomo dell'esattezza, e il
problema è che il linguaggio toglie sempre qualcosa e aggiunge
qualcosa -- sembra Cusano -- ed è ciò che si chiama ermeneutica.
Sapere di non sapere presuppone questo, lo scarto. La mappa non è il
territorio. Dobbiamo essere un po' fuori dalla posizione dello
spettatore della galleria di stampe di Escher. L'arte non può mai
essere una duplicazione di ciò che rappresenta. L'arte non deve
copiare, ma rendere visibile, sichtbar machen. Il tentativo di far
vedere il confine tra forma e materia come passaggio, col paradosso
che è un passaggio fermo, che deve però far vedere il movimento.
\emph{La forma che sta uscendo dalla materia, come se il confine fosse
un punto che si stesse formando}, questo cambia un po' l'idea di
quello che abbiamo avuto, l'arte non è solo il prodotto finito, si può
fare poesia con i frammenti, non è necessariamente sistema. Nessuno
nega la bellezza della \textit{Pietà}, e non è che non l'ha finita
perché è morto, ma perché non è finita, e muta l'idea di bellezza,
cambia tutti i rapporti possibili della questione del dare forma.
Nasce un'idea, che è moderna, di profondità, ed è legata al problema
della misura. L'idea della profondità non è l'idea degli antichi,
quella dell'uomo faustiano, e forse non è quella nostra. Si cita in
questo caso Spengler. Quando c'è quello che chiamiamo paesaggio, la
forma dell'ituizione contraddice fondamentalmente la matematica.
L'esperienza della profondità si sottrae alla cosa numerica. L'idea di
profondità a cui fa riferimento, quello che è l'uomo faustiano è
qualcosa che is è già accennato, l'idea legata all'intenzione della
prospettiva pittorica, che lega in modo forte la profondità alla
distanza, e la domanda è: siamo sicuri che la profondità è la
distanza? I greci non lo pensavano, e nemmeno a livello contemporaneo
si pensa a qualcosa del genere, e si lega alla distanza perché
l'invenzione della distanza prospettica ce l'abbiamo già accennata nel
1300 toscano, in Giotto, ma il modello decisivo lo si trova con
Brunelleschi e Alberti. E si tenga conto che Alberti non è
necessariamente un pittore, ma un architetto, ed è anche un geografo,
e torna in giro Tolomeo, viene tradotto finalmente, e naturalemente
Vitrubio. Il punto è che la prospettiva presuppone il fatto che ciò
che è profondo è ciò che è distante, visione frontale, si ha una
finestra, la famosa finestra di Alberti, che è ancheun dispositivo
tecnico, e dietro la finestra c'è la profondità. Si gioca sul fatto
che il personaggio è più piccolo, ecc. Bisogna aspettare gli anni
venti del Novecento perché si capisca del tutto che in realtà questa
visione calcolata matematicamente, questo modello prospettico è una
forma simbolica, e non il mezzo intutivo fisiologico, e questo lo si
trova in un testo il quale in questo scritto cita e riprende la
filosofia delle fonti simboliche di Cassirer, perché tutti coinciano a
scoprire e capire che il modello prospettico, il modello matematico
della distanza è un modello che contrasta con la visione fisiologica,
quella prospettica presuppone un occhio solo e una precisione
matematica che gli occhi non hanno, e prima ancora già la questione è
in crisi. C'è questo modello che comincia a essere in crisi, perché
gli scienziati pongono questo problema, e lo fa Ernst Mach, il perché
gli uomini hanno due occhi e non uno. Scatta questa cosa per cui,
quello che sembra più preciso, il modello geometrico, è quello che
corrisponde meno alla visione effettiva dell'occhio, e nasce la crisi
della corrispondenza, cosa di cui Spengler tiene conto. Quella che
viene chiamata prospettiva lineare, quella moderna, perché la
prospettiva, in quanto esperienza di profondità matematica contraddice
l'immagine intutiva.

Spazio cosmico infinito, spazio cosmico interiore che include quello
della vista. Molti filosofi si sono resi conto che l'idea di
profondità in Eraclito non è questa qui. Non è basato a una totale
identificazione con la distanza, ma già in Eraclito si propone come
profondità interiore, con un'idea che non riguarda la distanza ma
qualcos'altro, \textgreek{batus} bathus, il modello prospettico non è
così, la partita cambia, a maggior ragione quando non c'è solo la
geometria eclidea, ma molte geometrie. Una cosa di cui si era accorto
molto bene Adorno, sull'idea di profondità. E che cosa vuol dire
questo? Che noi ci troviamo con l'uomo faustiano come un'uomo
dell'astratto, implica un'idea paradossale dell'infinito, tema del
punto di fuga, legato alla distanza. Si pensa a un infinito
calcolabile ma astratto, e non è questa l'idea dei greci, né per
Nietzsche, e la profondità è nella superficie, non nascosta. La
profondità è nascosta nelle trame della superficie, il rapporto
apparenza-essenza non è di distanza, ma di come vedere le cose.
Lettera trovata Edgar Allan Poe, faustiano moderno, la verità è
distante, l'apparenza inganna, ci sono varianti ovviamente, però il
paradigma a cui ci si confronta è che c'è un paradigma, cogito ergo
sum, manipolare e contemplare l'oggetto da una finestra per cui tutto
questo viene dal soggetto singolo con l'oggetto esterno, ma siamo
sicuri che le cose stanno così, e non dell'intersoggettività? Lo si
trasforma, si trasforma la antura, ma fino a che punto. Il Seicento
era un'epoca diversa, ed era bello per lui che sia il corpo che la
natura siano al servizio della mente dell'uomo; oggia abbiamo un
problema con questo. C'è qualcosa che non va in questo modello che è
il modello faustiano il cui principio era l'azione. È la crisi di
questo. È difficile capire Heidegger e Wittgenstein, con quel modello
dominante, gli ospedali i teatri, ecc., sono fatti con quel modello.
La famosa questione della filosofia e del senso comune si gioca su
questo. Ecco perché diviene interessante chi è l'uomo faustiano, e
Goethe l'aveva colto, il principio è l'azione, la modificazione dle
vangelo di Giovanni, il principio era il verbo. L'uomo faustiano,
borghese, scientifico, trasofrma il logos in un livello pratico, e
quindi trasforma la natura e la vuole sottomettere attraverso la
matematica, e il grande problema è perché questo modello non può
esaurire la vita dei modelli, anche se ci sta provando.

\end{document}
