% !TEX root = ..\..\main.tex
\documentclass[../../main.tex]{subfiles}%
\ifSubfilesClassLoaded{\addbibresource{../../bibliography.bib}}{}

\begin{document}

L11 12-11-2025

Focault, questione della storia. Cosa intende per storia, e in che
senso la sua filosofia è un'indagine storica? Per arrivare a questo
nucleo di testi si parte dal testo \textit{Che cos'è l'illuminismo},
conferenza di Focault in America, dà un'interpretazione tutta sua di
Kant, e una delle ultime descrizioni che Focault parla delle storie.
Interesse strettamente politico della ricerca storica di Focault. La
genealogia non guarda solo alle condizioni di possibilità -- non che
le escluda -- ma si riferisce al presente per pensare alla domanda che
Focault attribuisce a Kant. Che differenza c'è fra l'oggi e l'ieri?

Questo è il quadro per collocare il testo di Nietzsche. Il periodo
genealogico -- di Focault -- coincide con la nomina e la lezione
inaugurale, che termina con una frase significativa, \emph{riscoprire
il discorso come ponte}, ponte alla fase archeologica. L'evento è per
Focault il momento della ruttura e la discontinuità della storia, ed è
lì che parte la genealogia, come si ha una possibilità politica di non
essere, non fare, ecc., quello che si è e si pensa. Questo testo,
dedicato a Nietzsche. Il \emph{problema degli origini}, la genealogia
non si oppone alla storia. Perché rifiuta l'origine? ...Sono senza
essenza... critica all'idea dell'origine intesa come essenza degli
oggetti storici, della storia stessa che dà un senso, e dice qualcosa
di più: se vogliamo guardare dove gli oggetti storici sono nati, c'è
una data di nascita ma non un'origine metafisica, che per Focault
sarebbe impossibile trovare, ma il disparato, la discordia. La storia
non ha un'essenza, una logica, ma partendo dal presente, dalle
pratiche, ecc., possiamo confrontarci con quello che è la parte
storica. Termine di \emph{provenienza} e di \emph{emergenza}. Primo
origine di tutti quei caratteri molteplici che hanno fato origine a un
certo oggetto storico. C'è una proliferazione di questi caratteri
formativi. Cose disparate che si incrocciano, e quella è la
provenienza di un fatto storico, non una linea unica e direzionata, ma
più linee che casualmente e contingente mente si incrociano e portano
a un'emergenza, termine che Foucault recupera da Nietzsche. Emergenza,
come un certo oggetto storico viene alla luce. Tesi di Foucault è
proprio che non ci siano oggetti storici immutabili, ma sono cose che
ad un punto contingentemente emergono, il fatto è la parte della
sessualità, di cui non si può parlare nel medioevo, perché è un
oggetto preciso che emerge da qualcosa di disparato. La storia di
Focault è segnata da discontinuità in questo senso, non la
discontinuità come il contraltare della continuità...

\emph{rivoluzioni che modificano la storia, non è così, ma come l'idea
che ad un certo punto, contingentemente, emerge una singolarità che
rivoluziona ciò che era stato, non nel senso di interrompere una
continuità, ma di creare qualcosa di nuovo}.

Passo 5 (dispensa), tema della discontinuità, il sapere non è fatto
per comprendere, ma per prendere posizione. Il rendere evento,
neologismo foucaultiano, non è altro che il momento in cui una
singolarità emerge nella storia e si propone come evento. Qui c'è un
primo interrogativo, che sorge da questi testi, è se questa cosa
avvenga nella storia, e, per così dire il filosofo che fa ontologia
dell'attualità ricerca o individua, o se anche la ricerca storica
stessa, se quelli che sono eventi non sono per così dire
essenzialmente degli eventi, ma se un'altra persona può riconoscere
altre fratture ed altre discontinuità. È la discontinuità univoca? O è
quella di chi fa la storia?

L'altro tema, quello della \emph{problematizzazione}. Nell'altro testo
declina la problematizzazione all'interno della storia, interrogarsi
sugli enventi storici come singolarità, e come le cose storiche
abbiano trovato una risposta. Qualcosa che ha a che fare con
l'indagine. Se è un prendere posizione, il genealogista sceglie cosa
problemattizzare. Che valore ha il fatto che il genealogista Foucault
sceglie questo e non quell'altro oggetto storico?

La genealogia deve procedere per evenezializzazione e
problematizzazione. Scoprire che il nostro presente non va spiegato
alla luce di un'essenza di un'origine, ma si può \bsq{smontare} pezzo
per pezzo gli oggetti storici, e quindi quelli che sono pratiche e
discorsi effettivi che agiscono e che funzionano cercando nella
provenienza.

Tendenza continuistica, anche quando gli storici parlano di
discontinuità. Non movimentare la sotoria cercando continuità,
confrontando testi, ecc., ma con tutti quei dispositivi che ci
determinano, e lui gioca molto soprattutto sul tema del corpo.

Crisi della metafisica e la discontinuità. Si può leggere Heidegger
come la critica della metafisica. Quanto gioca l'osservatore e quanto
gioca la cosa? Quando distingue fra comprendere e prendere la
posizione, bisogna capire che comprendere l'intende probabilmente nel
senso metafisico, l'universalizzazione. Svolta nella filosofia prima
di Nietzsche e dopo lui. E c'è un prima-Nietzsche, per lui è proprio
la rottura del modello metafisico. Lui cosa dice: se io uso la parola
foglia, essa non esiste, ma esiste ogni foglia individuale. Me la sono
inventata perché l'intelletto inventa questa cosa ed è finzione, e
perfino la matematica. Serve parlare del concetto di foglia, ma ogni
singola non si può ridurre all'altra, perché ogni foglia è diversa, la
classificazione. Focault parte di questo per sottolineare la
discontinuità, la disparità. Lui sta contrastando un'idea di storia
che non è solo quella dell'origine, che fra l'altro è anche legato
all'esmepio della foglia, l'archetipo, il modello. Ma dietro c'è anche
il problema teleologico di costruire la storia ad hoc come un fascio
di continuità come tutte le teorie del progresso, Condorcet. Questo è
il contrario del discorso che fa Foucault quando dice che il presente
rompe, mentre il modello condorcet, comte, dominante, ecc, gioca
sull'incontrario, sul fatto che tutto il passato è il compimento
dell'oggi, facendo un rovesciamento sociale, si costruisce a partire
dall'oggi senza dirlo. È la stessa cosa che ha detto Wittgenstein. Fa
apparire la scienza come conseguenza della magia, è un modo di
spiegazione, ma non è la spiegazione, c'è questo continuismo molto
apologetico, e c'è anche l'influsso id Marz in questo, il materialismo
storico e il rifiuto alla teleologia: le idee dominanti sono le idee
della classe dominante.

Il discorso del trascendentale, ha in mente Foucault una critica
all'idea di pensare a una logia rivoluzionaria, che si può tradurre
come una logica della storia, come e quando si può determinare un
evento, è discontinuo il modo in cui avvengono le discontinuità. Non
solo l'evento, ma anche il modo stesso in cui emerge è discontinuo.
Non si può pensare a una logica della storia, a una logica della
rivoluzione. Problematizza il concetti di rivoluzione riflettendo sul
fatto, come si può essere antagonisti e lottare contro forze che in
nome della liberazione opprimono, ed è un problema, effettivamente. La
tragedia di questo periodo, che è di liberazioni, la storia del
Novecento ci fa ritrovare in una situazione in cui in termini di
potere e di dominio si rovesciano. Il bisogno di sottolineare la
discontinuità è in fatto di libertà, ogni evento è singolo.

In fondo la storia è mutamento, altrimenti non è storia, e i mutamenti
non sono sempre come ci si vuole, ad un certo momento si scattano dei
meccanismi che creano mutamenti. Processo di Eichmann, scomporre
l'idea e cerca di trovare l'origine di quel cattivo, chiunque potrebbe
andare incontro a quella cosa lì. La banalità del male e la
burocrazia.

\end{document}
