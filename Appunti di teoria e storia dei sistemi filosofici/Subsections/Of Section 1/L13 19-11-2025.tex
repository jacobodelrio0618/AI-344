% !TEX root = ..\..\main.tex
\documentclass[../../main.tex]{subfiles}%
\ifSubfilesClassLoaded{\addbibresource{../../bibliography.bib}}{}

\begin{document}

L13 19-11-2025

Confronto Hozinga e Schmidt sulle categorie del politico. Si parla
del gioco visto nell'ambito conflittuale e bellico, e come l'intende
Schmidt.

Ci sono tre tempi (non dalla critica) per vedere Schmidt, in questo
caso si parla del primo, lo Schmidt giurista. Sono opere di stampo
molto teorico di carattere giuridico, si concentra sulla costituzione
di Weimar, e poi sposta sul concetto di sovranità. In questa fase lui
è ancora è un teorico. Su che cosa si fonda il concetto di sovranità,
Hans (?) è una legge, una forma che sta sopra l'ordinamento statale,
mentre per Schmidt non risiede nell'applicazione normale della norma,
ma nell'eccezione, quando decide il sovrano; lì si fonda appunto la
sovranità. Art. 49 della costituzione di Weimar e salita di Hitler al
potere.

L'emergenza, questo del caso eccezionale ritorna nelle categorie del
\textit{Politico}. Poi aderisce al nazionalsocialismo e ci sono
opere molto interessanti. Dopo Norimberga soffre un esilio accademico.

Categorie del \textit{Politico}. Edizione Il Molino 1979, edizione
approvata da Schmidt stesso. In generale, Schmidt vuole definire, e
quindi anche rifondare la politica, che era fin'ora sempre stata
definita in negativo, come opposto all'economia o alla morale (passo 1
handout). È stata vista come sovrapposizione all'ordinamento, allo
stato, come se fossero la stessa cosa. Per Schimdt non è lo stesso, il
politico precede lo stato, e se non ci fosse la politica, nemmeno
l'ordinamento avrebbe ragione di esistere.

Secondo Schmidt le discipline si fondono sugli \emph{opposti}, in
questo caso la politica sul nemico-amico, rendendo la politica
autonoma dagli altri campi. Comunque, si definisce nel libro solo il
nemico, e la critica è stata mossa ad esempio da Hozinga. Quindi
l'accusa della fondazione negativa della politica può in qualche modo
essere rivolta anche lui.

La guerra non è quotidiana, tra amico-nemico deve esserci solo una
tensione, ma il conflitto è solo il caso estremo. Scmidt, di nuovo,
fonda la politica sul caso eccezionale, e per Schmidt tutti gli stati
politici, per essere tali, deve crearsi questa tensione, che però non
deve degenerare in guerra. Il nemico, per schmidt, non perde mai
l'identità umanitaria (\textquote{dello stesso genere}). Dice proprio
che non va mai spogliato della sua umanità -- e qui ovviamente si vede
un'incoerenza col nazionalsocialismo.

Distinzione tra il nemico pubblico e il privato: non bisogna per forza
odiare il nemico pubblico. Schmidt è cattolico e reazionario, amate i
vostri nemici: il vangelo si riferisce al nemico privato. Questa cosa,
dice, c'è nella filosofia platonica.

\emph{Teoria del partigiano} del 64, continuità (e discontinuità) del
\emph{Politico} perché mette in dubbio la distinzione amico-nemico. Si
è vista una grande somiglianza tra un certo tipo di partigiano e
quello che fa lo stato in chiave difensiva. Nella teoria del
partigiano va a vedere i guerrilleros spagnoli. Erano irregolari, però
nel voler cacciare gli eserciti napoleonici, era una guerra difensiva,
difensiva della terra, concetto importante per Schmidt. Questi
partigiani erano in continuità con la concezione dello stato. Molto
spesso la figura del partigiano perde la sua concretezza e diventa un
modello astratto come colui che difende la patria, e in questo momento
che diventa astratto, prende le vesti dello stato.

Critica all'ipocrisia delle politiche liberali dell'epoca proprio in
temi di guerra e colonialismo. Si fa espansione liberale in nome
dell'umanità, si parla di guerra giusta, ecc., e invece qui Schmidt
riprende Grozio, secondo cui, nella definizione di guerra la giustizia
non può entrarci. La guerra non potrà mai essere qualcosa di giusto,
semmai necessario perché viviamo in un mondo politico. La guerra non
ha un senso normativo ma solo esistenziale. Riferimento lontano a
Heidegger, l'uomo è gettato in un mondo politico. Per Schmidt è un
fatto reale di quest'uomo gettato in una dimensione politica.

Aggiunge in un'edizione successiva a un altro saggio, parla della
storia propriamente, tracciamento non universale e astratto, ma in
punto concreto della civiltà europea. Si rifà a Spengler, critica alla
tecnica vista come campo neutrale -- la tecnica invece porta
contemporaneamente alla pace e alla guerra -- e il passo finale è da
leggere.

La tecnica è la fine della vita organica, e c'è la fine del conflitto
corpo a corpo, a volte non si vede nemmeno il nemico che si vuole
uccidere.

OSSERVAZIONI IACONO

Tempo di guerra di Scmhidt non è uguale al combattimento. Questo è
Hobbes, che non è guerra con regole, ecc., ma il tutti contro tutti.
Tema importantissimo che cambia le cose dal punto di vista teorico per
cui non è il semplice combattimento, è più complesso e ampio.

Problema della terra, ambiguità di Schmidt, blut und boden. È però
anche un problema militare. Non c'è guerra e non c'è azione senza
territorio, ed è vero anche oggi nell'epoca del virtuale. 

Sull'amico-nemico dice cose molto interessanti, \emph{però} riduce
questo rapporto a una condizione esistenziale, e quindi naturale più o
meno eterna; non c'è cambiamento. Dice quindi cose interessanti, però
quando lui dice che in realtà il nemico è lo straniero, però c'è un
grosso problema. Qual'è il confine fino al quale riconosci l'umanità
al tuo straniero? Non è la guerra fra il tuo nemico interno, in
famiglia, però se è estraneo si pone come altro. E questa è la
tragedia della teoria amico-nemico applicata al nazismo, anche se il
punto non è quello di accostarlo immediatamente ad esso. Se non si
definisce l'amico, comporta che si è amico solo se si ha in comune un
nemico (?). Ti identifichi col ricco perché sei bianco, e il nemico
diventa il nero. Ti identifichi con il maschilismo perché l'altro è
omosessuale. L'amico si definisce negativamente, ed è una debolezza
della teoria di Schmidt, e questo è il problema del ruolo della
politica, e avviene anche nel marxismo contemporaneo, e cioè, è
possibile imamginare una plitica che abbia in sé una vivsione di un
cambiamento? Per Schmidt no, e qui è la parte conservatrice e
reazionaria. È dopo questa debolezza teorica che lui dice le cose
interessantissime che dice: immaginare la guerra per fare la pace,
ecc., la più grande delle ipocrisie. Si fa la lotta in nome
dell'umanità, si salva l'umanità. Questa cosa universalistica e
generica nel nome di cui si fanno le peggiori cose.

E Schmidt dice delle cose importanti, dopodiché quando applica il tema
del decisionismo pone un problema su cui ci vorrebbe un attimo di
riflessione. Il problema della politica come decisione che modifica le
cose, si oppone a Hans (?) teorico del formalismo. Di fatto quasi
tutti i modelli giuridici lavorano sul modello kesseriano (?). Il
punto fondamentale: \emph{che rapporto c'è tra una costituzione e la
politica?}. Esiste una meta-dimensione a cui si fa riferimento
(costituzione)? In Weimar questo si sfonda, c'è stata un'entrata nella
dimensione giuridica per modificarlo, e Schmidt sottovaluta ciò per
una ragione, per lui la politica è questo, e Marx e Gramsci non
accetterebbero mai una cosa di questo genere. Anche nel liberlaimso e
Marx la politica è un'attività sociale, l'uomo è un'animale sociale.
Che rapporto ha la politica col principio di uguaglianza? Per schmidt
non è importante politicamente, in Marx e Gramsci invece sì.

Il rapporto col gioco è importantissimo, però in Hozinga è un modello
ideale, ecc., medioevo, però il problema di collegare e rapportare la
politica al gioco, cosa che può essere interessante, è praticamente
come collegare la politica allo sport, dove c'è un avversario che
spesso si può odiare: siamo sicuri che una cosa del genere possa
reggere? Era il problema di Levi ne \textit{I Sommersi e i Salvati}.
Se noi abbiniamo la storia allo sport c'è chi vince e chi perde, si fa
una cattiva semplificazione, e questo Levi lo faceva perché si
richiama alla complessità, si analizzano i Kapò, ebrei che diventano
abusivi con gli altri ebrei. Come diventa animale quell'altro che fa
parte persino della tua famiglia? Lì le teorie di Scmhidt non reggono. 

SECONDA PRESENTAZIONE

Teoria del partigiano, figura fondamentale rappresenta il passaggio
dal nemico vero al nemico assoluto. Guerrilla spagnola del 1810,
partigiano spagnolo che nasce come figura irregolarr , indidivud ai
pilastri del partigiano, che lo caratterizzano nella sua figura. Il
primo punto è che è \emph{irregolare}. Agisce nell'oscurità e
illegalità. Il secondo punto è l\emph{impegno politico}, centrale
perché è ciò che lo contraddistingue dal resto dei criminali. Rimanda
a un'idea di un gruppo combattente con una determinata carica
politica, apre dimensione che si strutturerà in maniera più assoluta
fino ai nostri giorni. Difende il proprio territorio, combatte
l'ivasore, lo \emph{straniero} con qualsiasi mezzo possibile, anche se
non segue un mezzo \bsq{giusto}. È un aspetto meramente tecnico e
militare, editto prussiano del 1813, il partigiano venne scoperto
dalla filosofia. Dimensione nuova, vera enemicizia, anima di doppio
taglio, si può criminalizzare e trattare il nemico a voglia, non è
tutelato da nessun diritto. Fa l'esempio della guerra
franco-prussiana, dove i prussiani dovevano gestire i partigiani
francesi che si opponevano alla fine della guerra. Il risultato fu che
fucilarono tutti. Editto di 1813 è regio, quando c'è lo scontro tra
l'esercito prussiano e quello napoleonico. Inneggia riferimenti alla
guerrilla spagnola in difesa della patria, però è un editto \emph{dal
re}, viene riconosciuto all'interno della società con una figura con
un ruolo ben definito.

Lenin assolutizza la figura del partigiano, lo mette al centro del
programma di diffusione della rivoluzione nel mondo, diventa la figura
centrale e assoluta, e anche si centralizza un tipo di inimicizia
assoluta, portata avanti dal punto di vista del partigiano. La
rivoluzione di Mao ha un carattere più tellurico della rivoluzione
leninista. Il partigiano è incentrato sul suo proprio suolo, sul suo
proprio paese. Raul Salan, generale francese, parallelismi
interessanti. Il suo problema è quello dell'individuazione del nemico,
che Lenin e Mao avevano fatto, mentre Salan ha questa problematica,
combatte sia contro il governo francese, sia contro i ribelli
algerini, e il risultato è che la Francia ha perso giurisdizione in
Algeria. Lenin e Mao hanno avuto successo, nella teoria di Schmidt,
perché hanno un nemico ben definito, cosa che invece non succede con
Salam.

Affermazione di una regolarità onnipresente nel contemporaneo, e c'è
la minaccia nucleare, quindi il partigiano smette di avere senso?
Secondo Schmidt potrebbe definire un \emph{disordine ideale}, o c'è
l'opzione tabula rasa in cui viene eliminata la vita sulla terra per
il nucleare dove muoiono tutti, al di là dell'amico e il nemico. Nello
stato moderno vince sempre la legalità, quindi qualsiasi modello può
essere giusto nello stato moderno, sfoggia nell'illegalità.
Assolutizzazione del nemico di oggi, e per giustificare l'uso di
queste armi, il nemico non può che essere assoluto.

IACONO

Che differenza c'è tra partigiano e terrorista. Secondo lo schema
ufficiale non ci sarebbe, e nel caso italiano della storia dei
partigiani, no, c'è stata la resistenza. I Viet Com erano terroristi o
partigiani? A prescindere da Schmidt, questo è un problema. Lenin, per
esempio, è contrario ai terroristi, e quindi c'è una distinzione forte
tra il rivoluzionario e il terrorista. Nel caso di Mao, il problema
nasce dal fatto che la Cina è invasa dai giapponesi, e quindi si dà
come lotta nazionale che a inizio non è terrorista. Forme di lotta sul
territorio, IRA, terrorismo in Spagna, in Italia. C'è il terrorismo
che distrugge gli oggetti e poi quello che uccide, e poi ci sono tipo
le brigate rosse e queste cose. È molto complesso, e oggi nessuno
vuole più parlare. In questo caso il terrorismo non è contro uno
invasore, non c'è la disumanizzazione. Poi c'è il terrorismo nero, che
era distruttivo nel senso del massacro, chi moriva moriva. Il
terrorismo è sempre stato usato. Che rapporto c'è con l'essere
partigiani? Dov'è il confine? C'è il confine, ma il problema si pone,
perché sul piano politico spesso vegnono considerati terroristi, come
nel caso della Palestina dove c'erano terroristi, oggi non più (?).
Quello che in parte è variabile e di cui Schmidt non tiene conto è che
il massacro non si basa sul nemico assoluto, ma anche sulla
disumanizzazione, il finto errore tecnico -- il discorso tecnologico
si usa per gisutificare un massacro. Uccidere civili fa parte della
guerra, fare pressione nei confronti del nemico. Il modello
ottocentesco crolla completamente con la prima guerra mondiale, una
volta che la tecnologia porta il massacro tecnico. Tutti hanno
avvertito che la situazione cambia da questo punto di vista. Non è un
caso che i regimi totalitari si scatenano dopo la prima guerra
mondiale.



\end{document}
