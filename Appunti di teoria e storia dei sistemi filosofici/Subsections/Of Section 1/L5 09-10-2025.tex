% !TEX root = ..\..\main.tex
\documentclass[../../main.tex]{subfiles}%
\ifSubfilesClassLoaded{\addbibresource{../../bibliography.bib}}{}

\begin{document}



Spengler gioca a grosso modo su tre dicotomia. (1) natura-storia; (2)
cultura-civilizzazione [nelle rispettive parole tedesche che non si
traducono direttamente]; (3) vita-conoscenza. Perché sono importanti
queste distinzioni? Perché sintetizzano un problema tipico
dell'epistemologia della modernità occidentale: si può ridurre tutto a
una scienza, oppure quelle sociali hanno qualcosa di irriducibile a
quelle naturali? Se si pensa all'\textit{Epoca dell'immagine del
mondo} del secondo Heidegger, ad un certo punto in questo saggio
Heidegger distingue tra rigore ed esattezza, e lo fa perché secondo
lui l'esattezza è un modo di interpretare il mondo -- la verità della
matematica non corrisponde al mondo, e Spengler conosce le
meta-matematiche, le geometrie non euclidee, ecc. -- e dice che le
scienze storico-sociali, per essere rigorose non devono essere esatte.
Significa che il criterio dell'esattezza, cartesiano e moderno,
accostabile alla verità come criterio è una costruzione. Il chiaro ed
evidente si basa sull'esatto, ma nel Novecento questo salta. Tutta la
teoria kantiana della conoscenza è basata sulla geometria euclidea, su
un modello matematico unico, ma questo modello salta; salta la
\emph{corrispondenza tra rappresentazione e cose}, al mondo matematico
non corrisponde un mondo nella natura. Noi costruiamo modelli ma c'è
sempre uno scarto, in qualche modo. In fondo tutto questo non è così
lontano dal discorso della tradizione ermeneutica -- e quindi si
arriva alla tradizione tedesca in questo caso, Schleirmacher -- perché
inevitabilmente serve uno scarto fra l'interpretazione e il testo. Lo
scarto è la ragione per la quale il mondo è fatto da interpretazioni
di interpretazioni. Nietzsche radicalizza questo, cioè che la
conoscenza è un passaggio dal concreto all'astratto per via
dell'esattezza, e sebbene si tolga il criterio di verità legato
all'esattezza, non si toglie l'ultima. Heidegger esprime la tradizione
che conosceva molto bene che è quella di Geist (?). In psicanalisi un
termine tecnico di Freud è perturbante, però decide perché
effettivamente se si guardano le altre lingue l'unheimlich è di
difficile traduzione, e c'è anche sempre uno scarto da una lingua
all'altra. Il concetto di batus (?) di profondo, ma non è lo stesso di
quello moderno, che è un concetto prospettico legato alla lontananza,
la tradizione greca non è così, ma è originalmente ciò che contiene
l'acqua sotto terra, non è di fronte ma sta sotto ed è il contenitore.

% Geometrie non euclidee, a questo punto scatta la questione per cui
% l'osservatore è interno al sistema d'osservazione. Le teorie si
% costruiscono, e non sparisce l'oggettività ma cambia il concetto.
% Implica proprio lo scarto. Complicato vuol dire piego insieme, ed è
% il modello dell'orologio di Cartesio. Non si può spegare l'arte
% astratta. Fisica e filsoofia di heisenberg o scrodinger. Teorema di
% Godel. Popper. 

Non bisogna confondere i mutamenti epistemologici, con i mutamenti o
non mutamenti a livello di senso comune. Il senso comune della ricerca
non è il senso comune del mondo pratico. Un fisico come Kuhn che nel
1962 spesso viene interpretato come se fosse una scoperta al giorno. A
parte che ce ne sono poche, lui sottolinea il fatto che il ricercatore
non abbandona il proprio paradigma, ma vi si lega finché non può non
farne a meno, perché giustamente è abituato a un paradigma, e si
guardi che ciò è un problema politico, perché teorizzi le elite, o se
non lo fai, devi tenere conto di questo scarto, che è il problema di
Gramsci, ed è un problema che ha Focault, perché vede che esiste un
problema di passaggio della concettualità, che non si può cambiare
così come se niente fosse. 

% Waddington. Strumenti per pensare, introvabile. Whitehead e marx ed
% engels.

Un vero problema è quello di dare dignità alle persone, e quindi come
si vede la complessità della natura e la storia si scontrano ancora, e
anzi, il modello che sta riprendendo fede nelle università è che il
metodo delle scienze naturali è quello delle scienze naturali,
specialmente per quelle sociali. In un certo senso, la risposta di
diltai è una risposta al modello positivista, modello ottocentesco.
Gadamer riprende Heidegger però lo sposta molto sul terreno
dell'ermeneutica. Lui eredita questo. In Derrida è più complessa la
questione, il primo sicuramente si muove dentro questa dimensione
molto forte, l'ultimo invece si è scivolato molto dentro il problema
del decostruzionismo. E il fantasma rispetto a cui ci si confronta è
il socialismo reale, e le ultime cose di Deleuze, sembra che si
fermino molto dentro l'idea che l'unica cosa che si può fare è
l'evversione specifica, e il sistema neoliberista ha dimostrato (?).
Anni settanta, tutte le cose che erano antagoniste era riuscito ad
assorbire questo sistema, oggi ci si trova in una situazione
rovesciata.

Quindi ci si rende conto di quanto è complessa la dicotomia
natura-storia, e quando contrappone Spengler questi temi è consapevole
delle ripercussioni di questo.

Poi c'è kultur-zivilization, che è stata affrontata molto in cultura
tedesca. La distinzione \bsq{classica} è quella per cui la kultur si
fa fine a sé stessa, mentre la civilizzazione è l'aspetto materiale di
questa cultura. La città ideale rinascimentale fa parte della kultur,
il palazzo che fai invece all'altra. Nel caso di Spengler, però la
cosa è un po' diversa, perché lui fa la distinzione all'interno di una
civiltà, mentre considera il primo come momento bello, il secondo come
momento di decadenza, e per lui siamo già nell'epoca della
civilizzazione, l'uomo faustiano. È chiaro che per lui come per
moltissimi il vero problema è la tecnologia, e non è che ne siano
nemici, ma fa passare una civiltà dalla kultur alla zibilisation, e
questo dopo la prima guerra mondiale. È finito il mondo fine a sé, e
la calcolabilità è uno dei grandi problemi che ha Heidegger, e anche
Lukacsz. Civilizzazione siamo l'uomo faustiano e così via, e qui pare
che c'è una nuova traduzione del libro, però è incollabile, perché uno
di questi problemi è la traduzione di questa dicotomia.


Vita e conoscenza. Anche questo è un grande tema, ed è quello della
vitalità. È importante perché è un tema... Se una conoscenza è
astratta che rapporto ha con la vita. Si tenga conto -- si è segnalato
il fatto che Webber parla di delusione del concetto -- i movimenti di
massa di destra del Novecento erano vitalisti, rifiutavano la
conoscenza astratta e ponevano il problema della vita. Ecco perché
Spengler fa questo contrasto, perché nella modernità la conoscenza
prende il luogo della vita. Quindi questa è una dicotomia tutto
sommato abbastanza classica, un vecchio tema, il tema di Platone
rovesciato. Se lì si ha l'idea che è il concetto, poi c'è il mondo
sensibile che è la vita e che è movimento e quindi anche morte... È
come se il mondo di Platone dovesse essere rovesciato, in modo anche
feroce ad un certo punto, ma tutto deriva da questa crisi che emerge
sulle capacità di manipolare il mondo attraverso la tecnologia, che --
si ripete -- non è lo schema come si pensa nei manuali, dimensione
mistica antitecnologica; la questione è molto più complessa di questo
tipo di scarto, che si è molto semplificata, ed è legata al ruolo che
ha la tecnologia in un mondo che è dominato dalla calcolabilità, e la
calcolabilità uccide la vita, fa ritornare il corpo a quel modello che
diventò paradigmatico che è l'assimilazione del corpo a una macchina,
e il modello era l'orologio in quel caso. Una macchina, se ne conosci
la struttura la puoi modificare. Questo permise una grande rivoluzione
perché adesso puoi guarire il corpo e così via. Il corpo sparisce,
diventa astratto, e ci sono risposte che non sono solo di sinistra, e
questa è una delle ragioni per cui, quando si afferma il fascismo,
molti teorici che passano a sinistra accolgono il fascismo perché lo
considerano un movimento anti borghese. Si pensi nell'arte al
futurismo, i quali vengono in qualche modo assunti nella tradizione
fascista, mettere la tecnologia nella vita, riprendendo roba
contemporanea (?), l'unico tentativo serio di mettere l'astratto e la
vita insieme, e per questo è importante il futurismo. La storia è
complessa, non è banale e non si può semplificare in questo senso.
Ecco, era solo per dare un senso a queste cose. In Spengler non si
trovano queste cose, ma le dà per scontate, e gioca su questo terreno.

Uno dei problemi legati a questo tipo di tema -- spengler p.94 --
diventa interessante l'idea di Goethe, il suo tentativo, a proposito
dell'argomento, è quella che Goethe chiama \emph{fantasia sensibile
esatta} e li cita perché i due autori che riconosce come fonte diretta
sono Goethe e Nietzsche. 

È il tentativo quasi utopico di mettere
insieme cose che poi diventano divise, la possibilità di costruire una
fantasia attraverso il sensibile e l'esatto. Un esatto paradossale, è
il tentativo di, in qualche modo, determinare superando le dicotomie
la modernità in questa chiave, che non ha più la dicotomia tra
l'immaginario e l'esatto. 

Si parte dal presupposto cartesiano, nel discorso sul metodo, appena
lui scopre che pensare significa essere, la seconda cosa che dice è
che dobbiamo assimilare il verosimile al falso, e quindi
l'immaginazione, mentre la parte concettuale diventa la parte
dell'esattezza. Goethe cerca di ricucire la ferita attraverso
quest'idea, il Werther, e anche Goethe è stato importante per la
biologia e per i colori, e ancora oggi è considerato un punto
importante per la rivoluzione della biologia, in particolare delle
specie, era la cosa di cui si occupava di più. 

Lui muore mentre sta seguendo il dibattito in Francia, il momento
decisivo della questione evolutiva, ed è lui, a differenza di una
tradizione nostalgica, cerca quello che per i tedeschi era originario,
ma nel mondo biologico. Lui non ha un atteggiamento nostalgico , ma
molto scientifico, cerca l'archetipo, la pianta originale, l'archetipo
però sensibile da cui derivano tutte le variazioni. È molto importante
nel campo della storia della biologia, è fra coloro che applicano il
modello così detto morfologico, fra il rapporto che c'è tra la vita e
la forma, la vita è vita se c'è forma, e la forma significa una forma
che non può essere vita se non attraverso confine. 

Mettere insieme l'esattezza dell'immagine e il concreto. Fausto in
fondo è questo, è legato a questo tipo di dimensione, la vita che
ritorna, la possibilità di amare, il miracolo di qualcosa che non è
possibile. Per Spengler quesot è importantissimo, è tramite Goethe che
lui interpreta le civiltà come organismo, ed essendo organismi hanno
elementi della decadenza e la vecchiaia. Questo comporta una cosa,
fondamentalmente, ed è il fatto che lui non usa la scala del
progresso, e non pensa che c'è il progresso, ma il tramonto
dell'occidente, ha posizione conservatrice reazionaria, ma è
relativista, le civiltà sono organismi che nascono e muoiono. C'è un
nichilismo in questo, e si pone in una posizione critica rispetto al
modello progressivo. Wittgenstein dà molto peso a Spengler.

\end{document}
