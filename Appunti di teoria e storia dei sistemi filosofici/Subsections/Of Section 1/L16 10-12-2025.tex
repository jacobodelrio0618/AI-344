% !TEX root = ..\..\main.tex
\documentclass[../../main.tex]{subfiles}%
\ifSubfilesClassLoaded{\addbibresource{../../bibliography.bib}}{}

\begin{document}

L16 10-12-2025

Musica e arti plastiche, Spengler.

Se la musica... L'anima faustiana anima musicale [?]. Le leggi della
morale sono ance quelle dell'arte. Qualsiasi distinzione che faccia
riferimento agli organi di senso nella arti minaccia la vita interna
delle arti [?]. \textquote{Se un arte a dei confini, confini diventati
forma}. Perché una grande arte suole finire? Modello morfologico,
organicità dell'arte.

Corrispondenza di una trascendenza spaziale, Seicento. Cinquecento
contrapposizione tra voci cantati e voci strumentali. Musica tedesca,
la musica si fa assoluta di fronte alla pittura e l'architettura.

[manca un pezzo]

Il carattere di quasi tutta la musica di Bach è organistico,
compresenza armonica, riempire lo spazio. Altra armonia, si suona come
melodia, non può riempire lo spazio, e non ha quello scopo, né deve
\bsq{raggiungere il cuore dello spettatore}. Si libera di una
struttura formale per raggiungere il sentimento, la melodia si libera
da determinati schemi dettati dalla forma. Nel romanticismo si
riscopre Bach, ma si tende a liberare la produzione artistica da
determinati schemi, la melodia si libera dalla sua dimensione
armonica, e acquisisce un ruolo più sentimentale. Passaggio dalla
forma super personale all'espressione personale del maestro. Dopo Bach
entra la soggettività umana nella produzione. Corrispondenza tra lo
sviluppo dell'armonia e la conoscenza dell'anima umana, qualcosa di
profondamente umana. L'anima del compositore si mette nella
composizione. Nascita dell'orchestra, se durante il medievo strumenti
raggruppati in base alle possibilità tecniche, l'orchestra fa sì che
si uniscano in un'unità organica, che cambia anche la proiezione del
suono. 

L'opera d'arte faustiana viene caratterizzata come azione, una
simbolizza un organo, l'altra un processo organico. Non è qualcosa di
operazionale, ha l'biettivo di toccare l'ascoltatore. Dal punto di
vista dell'artista, spersonalizzazione dell'artista, erano i primi
maestri della forma,mentre i secondi realizzano lo stesso ideale
attraverso la forma della bellezza [?].

Jazz iniziale, la batteria diventa per la prima volta uno strumento
non di accompagnamento ma principale. Simonide, pittura come poesia,
Orazio, pittura poiesis. Breve introduzione su Eliade, nasce a
Bucharest, si laurea con una tesi su Giordano Bruno e Ficino.

Secondo seminario
Quest'eterno ritorno tradisce un'ontologia. Mito dell'eterno ritorno
associato con il pensiero greco. Il problema è che la contrapposizione
netta per cui la concezione ciclica del tempo si una concezione greca
mentre quello lineare è propria del pensiero cristiano è falsa.

Mazzarino cita Eliade. Questione del futuro, si può sostenere che per
gli storici greci il futuro non è da determinare a livello
epistemologico. Si può vedere una dicotomia, una contrapposizione. In
realtà questa contrapposizione ha senso e no, perché c'è anche un
punto di incontro rappresentato dal libro di Daniele, col quale si
dota di futuro la teoria ciclica, dei quattro imperi universali,
questa teoria greca acquista un carattere apocalittico. Sole di
Nabucodonosor, immagine dei quattro metalli, quattro imperi, i piedi
sono di ferro e di argilla, ma l'elemento decisivo e di unità è
rappresentato da un masso che colpisce la statua e la fa crollare d'un
solo colpo. La storia quindi si può abolire o mediante il ritorno
all'origine o con la promessa di una fine. Nella concezione messianica
la storia dev'essere sopportata perché ha un fine escatologico.

Eliade si chiede: riesco a sopportare la storia? Vive secondo questi
modelli extra-umani, quindi in questo quadro anche la sofferenza è
giustificata, non può essere messa a caso, ha un suo ruolo, non è
un'esperienza assoluta. Il dolore è sopportabile perché non è privo di
senso, ultimo capitolo, perché l'uomo storico può sopportare la
storia. A differenza dell'arcaico la storia la fa. Si rende conto di
essere totalmente esposto quando vengono a meno tutte le
considerazioni epistemologiche, si ritrova responsabile delle cose, e
anche del tutto solo e abbastanza insicuro anche.

La storia si pensa sia fatta un po' da se, se ne subisce il terrore,
paradossalmente l'uomo arcaico è più creatore dell'uomo moderno,
subisce meno la storia [?]. Disperazione, un testo in cui viene
travolta la storia dell'uomo moderno, e quindi da quando la fede è
stata inventata, l'uomo moderno si può difendere dalla storia senza
cadere nella disperazione totale dall'idea di un dio, altrimenti è
condannato.

\end{document}
