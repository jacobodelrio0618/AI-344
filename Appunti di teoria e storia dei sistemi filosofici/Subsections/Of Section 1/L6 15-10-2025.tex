% !TEX root = ..\..\main.tex
\documentclass[../../main.tex]{subfiles}%
\ifSubfilesClassLoaded{\addbibresource{../../bibliography.bib}}{}

\begin{document}

% Storia della scienza, 27-11, Muller.

Concetto di uomo faustiano.

Si era detto che l'uomo faustiano in Spengler è contrapposto all'uomo
apollineo, ed evidentemente se uno sente una cosa di questo genere, il
punto di confronto di Spengler è Nietzsche. I punti di rifermineto
fondamentali id Spengler sono N. e Goethe, e per quest'ultimo in
particolare il tema della morfologia, che ritorna nell'ambito della
storia della natura, per esempio. Ma è anche un punto di riferimento
importante per la nascita di una parte della sociologia. Non ci si sta
riferendo a Comte, ma a Durkheim e la sua scuola. Levi Strauss, viene
da questo tipo di natura, anche se introduce modelli linguistici,
quello che sarà noto come strutturalismo, diffuso negli anni
Cinquanta. Marcel Mauss. È stato fondamentale proprio per la questione
di Durkheim proprio per la morfologia. Sia Durkheim che la sua scuola
(e quindi anche Strauss) si muove in questo. Lo stesso strutturalismo
è abbastanza legato al concetto di morfologia. Se si prende per
esempio l'interpretazione che un filosofo come Althusser ha dato di
Marx, il modello è di fatto strutturalista. Lo stesso Focault si muove
in quello stesso contesto. L'idea che la storia non è fatta in
successione progressiva, ma che ogni entità è una forma a sé. Ogni
civiltà non è su una scala di progresso, ma ogni civiltà ha una sua
economia e si compara con ogni altra civiltà.

Dietro il concetto di struttura e di forma emergeva più potentemente
il senso di rottura rivoluzionaria. Quindi c'era anche uno scontro
teorico, una vera discussione. Rottura epistemologica, tradizione
francese, la storia va per rotture, non per mutamenti logici, e quindi
è una discussione che si porta dietro e che c'è ancora in
epistemologia, continuismo e mutamento delle forme che implica le
rotture delle forme stesse. Nella storia scientifica, in un certo
senso, è stato trovato un compromesso. Tutto questo per dare un quadro
di perché Spengler si rifà a colui che per primo in epoca moderna si
pone il problema della forma, Goethe, che studiava moltissimo la
biologia, voleva trovare la pianta archetipica, il modello del modello
dal punto di vista della storia biologica. Spengler prende da Goethe
soprattutto il \emph{modello morfologico}, non si trova in Goethe una
cosa della successione delle società, non si muove verso quel criterio
sia sul piano filologico, sia su quello sicura secondo la scala del
progresso, dal primitivo, verso l'avanzato. I primitivi non sono
soltanto quelli dell'epoca antica, ma anche i contemporanei che vivono
come in epoca antica in cui siamo progrediti. Si usa primitivo proprio
in questo testo, uomini primi che però vivono come noi. O non sono
umani, o se sono umani va benissimo, però vivono esattamente come
vivevano i nostri antenati. Sono arretrati e primitivi, e questo è uno
schema dominante tutt'oggi. Sono tutte varianti di questo modello
legato al progresso, e come diceva Durkheim, all'inizio del Novecento
non si metteva in discussione, era una parola tabù. Entra talmente
nella pelle che realizza anche ciò su cui rifletteva Marx. Le idee
dominanti sono quelle della classe dominante, si incorporano e fanno
parte della condizione naturale, le si accetta inconsapevolmente e
diventa una dimensione naturale, quella che verrà chiamata la critica
all'ideologia, che non è l'ideologia come intendiamo oggi -- un punto
di vista scelto consapevolmente -- ma il contrario, l'ideologia che
non si avverte come ideologia che fa problemi, ma che si avverte come
condizione naturale. È quello che succede sempre nelle lotte, sul
lavoro, ecc. Oggi c'è una naturalità del concetto che se sei povero è
per colpa tua, ecc., che viene considerata come un'idea generale e
naturale, ma lo è? O funziona grazie al sistema che funziona così? È
un meccanismo. E lasciamo perdere la parte sociale. Il meccanismo
capitalistico è una cosa di apparente libertà: offrire la forza
lavoro. Il problema di Marx è legato a questo meccanismo, il mercato
del lavoro è giocato su questo terreno. Il meccanismo è giocato sul
problema della libertà: tu sei libero di offrire te stesso, e puoi
anche dare il tuo corpo. Che cosa si intende per libertà, dunque?

Prendiamo l'esempio dei diritti civili. Fino a quarant'anni fa, gli
omosessuali erano considerati malati, già nell'accezione progressiva
-- in quella conservatrice erano peccatori o anti-natura -- li
classificava come malati mentali. Ci sono avute le lotte per questo.
Ma lasciamo perdere; la questione interessante da un punto di vista
storico-filosofico è: come ci siamo persuasi che è una condizione di
all'inazione mentale e malattia? È questo il punto: come si fa ad
accettare questa dimensione, ed è ciò di cui parla Marx quando dice
che le idee dominanti sono quelle della classe dominant , che non è
una cosa semplice, non sono le idee naturali di tutta la società, ma
di una classe. Tutto il sistema di sfruttamento capitalistico avviene
in nome della libertà. È naturale? Questo è il punto. Una grossa
tradizione anche marxista è in questo quadro che per esempio parla del
concetto di egemonia in Gramsci, anzi, due concetti quello di
rivoluzione passiva e quello di egemonia -- che può essere dei
conservatori, come in questo momento -- ed è fortemente legata al
discorso fatto su Marx. Non è una visione del mondo, ma un fatto
culturale che diventa naturale, e la rivoluzione passiva è la capacità
di assorbire qualcosa di antagonistico, farlo proprio, e trasformarlo
in una cosa domesticate -- i Jeans rotti che non hanno niente di
eversivo, sono conformismo puro. Non è un caso che filosofi come Zimel
e Benjamin si siano posti il problema della moda come un problema
fondamentale, perché è un meccanismo che funziona così. La moda è un
sistema di cambiamento debole, si rinnova tutte le stagioni, ma lo fa
esattamente per conservare il sistema. Il meccanismo è legato alle
condizioni del mercato. Ciò per dire quali sono i passaggi.

Il caso della \emph{dimensione morfologica è il caso di una teoria che
si contrappone al modello egemonico, la teoria del progresso, ecc.}, e
questo avviene non perché è di sinistra o destra, ma è un tentativo di
rifiutare la centralità dell'occidente come modello portante per
descrivere la storia, o, più precisamente, quella che noi chiamiamo
storia universale, che è un'esigenza tipicamente occidentale. Nel caso
del cristianesimo quello è una cosa fondamentale, e anche nelle cose.
La necessità di riconsiderare la cosa più ovvia, l'idea di una storia
universale, modello che si sviluppa con la teoria dei quattro imperi,
il cui paradigma massimo è la filosofia della storia di Hegel.
% Termine tedesco.
Operazione grandiosa di dare senso alla storia di dare senso
occidentale alla storia, e persino la parola oriente è a misura del
meridiano di Greenwich. E tutti gli schemi, la Russia, la Cina, il
modello in cui l'occidente ha la libertà, in Cina non c'è storia,
ecc., era un pregiudizio sviluppato nel Settecento, nell'epoca in cui
si comincia a prendere in mano il mondo, con delle contraddizioni,
naturalmente, alcune anche ironiche come l'ordine dei gesuiti. Tutti i
cosiddetti missionari devono fare una relazione che viene raccolta in
una serie di volumi che c'è a Parigi. La cosa paradossale è che ci
sono state esperienze comuniste tipo il Paraguay, e quando andavano in
Cina si rendevano conto che avevano libri probabilmente più antichi
dell'antico testamento. Oggi l'occidente non è più il centro, ma non è
così, e sul piano del modello occidentale, la crescita del PIL, altri
paesi l'hanno già sorpassato, e questo spiega il nervosismo piuttosto
feroce dell'occidente, e lo si dice perché Spengler sposa non il
sistema dall'alto al basso, ma delle civiltà, che sono come degli
organi viventi, nascono e muoiono. È un dato interessante che un
nichilista nietzschiano come Spengler, conservatore, usa un modello
morfologico piuttosto che quello progressivo.


Per quanto riguarda invece Nietzsche la questione è più familiare per
certi aspetti, ed è legata all'esito nichilista della riflessione
nietzschiana. Nietzsche è elemento di rottura con gli elementi
occidentali, ha elementi di ritorno che in parte riprende idealisti
romantici, ma in parte li distacca. Non si può dire che c'è la
nostalgia del mondo greco in chiave romantica. Non c'è la nostalgia di
ritorno ha, ma nella storia che il rapporto con i greci ci portano ad
essere molto critici nei nostri confronti. Per Nietzsche l'attività
intellettuale è una funzione che serve agli umani per sopravvivere,
per lui i concetti sono metafore... Ed è un tema che si trova anche in
Spegnler, per cui la matematica è una costruzione simbolica, però al
di là di questo, questa è la ragione per la quale Spengler, variando
il pensiero nietzschiano -- apollineo e dionisiaco -- modello della
rottura e dell'armonia, in Spengler c'è la variazione nel senso che
mantiene l'apollineo, ma anche l'uomo faustiano. Non c'è l'uomo
dionisiaco, ma distinzione tra l'apollineo e faustiano, e quindi
dobbiamo capire che significa faustiano.

% Citazione di Spengler sull'apollineo.

Tipo ideale dell'esteso i singoli corpi sensibili presenti, come tipo
ideale dell'esteso, e quindi una sorta di Cartesio. Il modello che
considera l'estensione come ciò che determina astrattamente i corpi è
Cartesio. I greci non lo facevano, per loro l'esteso sono i singoli
corpi. Il problema è che si intende in senso astratto e matematico, e
questo è quello che in fondo pensa Nietzsche. È con la modernità che
l'esteso diventa matematico. Dopo Nietzsche una tale espressione è
comprensibile da ogni uomo. Ad essa si oppone quella faustiana, spazio
puro e illimitato il cui corpo è la civiltà occidentale.

Quello che è importante è capire cosa c'è dietro. È lo stesso problema
che avrà Heidegger rispetto a Nietzsche, come uscire dal paradigma
cartesiano, e uno dei modi è tornare ai greci, e la chiave è con
Nietzsche. La modernità nasce già con l'idea che i moderni sono
superiori agli antichi. Se si prende il dibattito della dine del
Seicento che si svolge a Parigi, la disputa fra antichi e moderni, la
grande discussione era proprio legata al fatto se i moderni erano
superiori o no. Fino a quel tempo si diceva che gli antichi erano
superiori dal punto di vista poetico, ecc., che poi i romantici
pensano diversamente in chiave nostalgica e così via. Essi cercano di
rispondere al criterio forte della modernità che nasce con la scuola
cartesiano. Il capo è Fontanelle un cartesiano, un grande personaggio
che scrisse delle cose importantissime. Battaglia contro la
superstizione, nasce l'illuminismo, uomo con grandissimo succesos,
principe dei dibattiti e le dispute. Nel 1688 scrive questa cosa che è
la disputa degli antichi e moderni, dove sostiene la tesi che i
moderni sono superiori, e non è una banalità, ma aiuta a prendere
l'atocoscienza borghese. E perché sono superiori? Perché essendo la
testa dei greci come la nostra la stessa -- qui non c'è progresso --
abbiamo la possibilità di imparare dagli errori degli antichi.
Cresciamo e possiamo ritenerci superiori. Il dato di fondo, che è
all'origine del moderno è il modus, stare qui, la consapevolezza del
tempo in cui stai. Siamo noi che abbiamo inventato come moderni il
medioevo. L'auto consapevolezza del tempo presente della modernità è
moderna, si trova in Fontanelle, Leibniz, Spinoza. Il contuse come
forma che permette dare potenza. La svolta è sicuramente il paradigma
di Cartesio. Il punto di riferimento è Cartesio. Husserl che porta le
\textit{Meditazioni}, quello che praticamente costruisce il paradigma
della modernità, e fra tanti c'è il ripensamento del corpo che
dev'essere descritto come estensione matematicamente, e il corpo
vivente, in particolare, è una macchina, e il modello è l'orologio.
Cartesio rende nota l'idea che si può intervenire sul corpo e
prolungare la vita, comincia con Vesalio, ecc. C'è la dimensione del
corpo su cui si può intervenire, che però aveva un effetto
collaterale, perché il copro si presenta come qualcosa al servizio
della mente, ecco il dualismo cartesiano, la mente è signora e padrona
del corpo. Pensiero ed estensione si giocano comunque sull'astratto.
Sono i temi che ha presente anche Heidegger, sono i temi che stanno
nella questione di Spengler e Heidegger della tecnica. Quella che oggi
si chiama la \emph{condizione complicata di fronte al complesso}, e
vuol dire che si può dispiegare. Se conosco i componenti e la loro
disposizione, la si può aprire e smontare, e si sa che una macchina,
col modello meccanico, si sa che c'è un rapporto tra causa-effetto. Il
complesso è il contrario, insieme, e un organismo vivente non può
essere orologio e meccanico, esattamente perché entrano in gioco
sistemi che producono realtà continuamente, irriducibile ad una
macchina, proprio per motivi teorici. Questo comporta cambiamenti dal
punto di vista della conoscenza. Ci sarà sempre un'incognita più, c'è
uno scarto continuo, cosa che poi è esplosa con le geometrie non
euclidee e così via, fra Ottocento e Novecento, e Spengler tiene
presenza di queste cose. La testa di turco è Cartesio, ecco perché si
critica il detto di Protagora, e il concetto dell'uomo faustiano è
entro questa dimensione. Fino a gran parte dell'Ottocento e del
Novecento c'è l'idea della riduzione all'uno, ridurre tutti i saperi a
uno solo. Oggi nessuno ci crede e non ha senso. Ogni scoperta apre
nuovi orizzonti, non si può chiudere il discorso, c'è un'irrestabilità
della domanda nel momento in cui si dà una risposta.

Questo comporta il fatto che ogni sapere ha uno statuto e un suo
modello. Non si può ridurre tutto a una cosa, non si può ridurre tutto
all'estensione. Prendiamo il concetto di intenzionalità, e mettiamolo
nella scala di esseri. Se si vede a un essere umano che segue un altro
si vede un'intenzione. Se si prende l'ameba non si usa intenzionalità,
ma cose fisiche e chimiche. Quando è legittimo usare l'intenzionalità?
Dipende dal paradigma culturale, dal linguaggio che si deve usare.
Dov'è la scala altrimenti? Una cellula è in relazione con tutto
l'universo se ci si pensa: fin dove si deve arrivare? Racconto della
carta geografica. La conoscenza totale nella duplicazione porta allo
zero. La mappa ha quella condizione paradossale per cui si può
conoscere solo perché non è il territorio, e nello stesso tempo si
pone un limite su ciò che si può fare con le mappe. C'è sempre un
resto, e in questo senso la mappa non ha un territorio. Questo ha a
che fare con gli osservatori. Non si raggiunge mai il livello della
mappa assoluta che si identifica con il territorio e dà conoscenza.
Quando si parla dello scarto fra testo e interpretazione in
Ermeneutica si dice precisamente questo. Questo tipo di riflessione --
nata col micorscopio e poi con le cose fisiche -- è dentro la cosa
osservata. Si è sempre dentro il contesto di osservazione, e quindi ci
sono limiti da questo punto di vista.

Questo non vuol dire che non ci sia oggettività, ma che muta il
concetto di oggettivo come la pubblica riproducibilità. È questo che è
pubblico ed oggettivo, ma non nel fatto che si raggiunge l'assoluto.
Bisogna distinguere l'oggettivo dall'assoluto, e questo ha a che fare
con la verità, e qua ritorna Nietzsche, che vede la conoscenza
intellettuale come una visione. Sono metafore di cui abbiamo
dimenticato l'uso, ma se sono metafore vuol dire che sono spostamenti.
La terra è un'arancia blu, si dice. La definizione più generica, ma
per poterla definire si deve avere un linguaggio metaforico, e
l'identificazione metaforica è indiretta. Il sogno da Cartesio in poi
è dare come definizione perfetta quella matematica, perché a parole
non si può fare, ma siccome la matematica non è una sola, anche questo
diventa impossibile, e perché diventa importante Nietzsche? Perché
mette in discussione questo paradigma, ma non per rendere il sapere
non scientifico, ma per rimodellare il senso assoluto del sapere
scientifico, che non c'è più. C'è uno scarto, non la si può fare. Un
osservatore è dentro il contesto.

\end{document}
