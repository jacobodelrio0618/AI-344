% !TEX root = ..\..\main.tex
\documentclass[../../main.tex]{subfiles}%
\ifSubfilesClassLoaded{\addbibresource{../../bibliography.bib}}{}

\begin{document}

23-10-2025

Quali sono le figure mitologiche della modernità (che naturalmente non
chiamiamo miti)? Uno è Fausto, l'altro è Hamlet, l'altro Don Quijote e
Robinson. Questi sono i miti che hanno fatto l'identità dell'occidente
moderno. Potremmo dire che sono \bsq{personaggi concettuali}. Sono
miti, ma allo stesso tempo personaggi concettuali. Se si parte dal
presupposto secondo cui il mito non è l'antitesi al logos, è chiaro
che vengono personaggi con una forza concettuale imponente. Il fausto
di Goethe, questa figura legata all'azione, al sapere scientifico,
alla capacità di manipolazione del mondo, il mito della giovinezza, il
mito di mercato più potente che ci sia oggi, e l'altro è la
conoscenza. Contemplazione teorica e manipolazione. È la ragione per
la quale filosofi come Heidegger, Spengler e altri si sono posti la
questione della tecnologia come elemento drammatico e persino tragico
di questo rapporto. Forza e potenza, le grandi qualità del mondo
borghese. Gli intrecci del mondo naturale e il mondo materiale nella
trasformazione della natura. Quello che vedono è proprio la questione
della traformazione del mondo con la techne, ed è una cosa
caratteristicamente moderna. Quando Heidegger dice che la physis è
diversa al nostro concetto di natura, dà del concetto di natura ad
esempio in Eraclito e la tradizione greca, il concetto di qualcosa che
si offre, che si apre, mentre nella tradizione moderna è di tipo
diverso, quella di cui tu ti confronti in quanto soggetto rispetto
all'oggetto. La grande caratteristica è il rapporto soggetto-oggetto.
La natura è oggetto che va studiato, scritto e manipolato. Questo è
l'aspetto importante. Il Fausto un po' rappresenta la dimensione
dell'agire in tutto questo. Il grande tema di tutto questo processo
complesso è il \emph{lavoro}. Nella tradizione greca non è così, e fa
filosofia chi è liberato. Nella tradizione moderna invece il lavoro è
l'elemento centrale, che nel caso della tradizione protestante
calvinista edifica. Per questo il lavoro è così centrale in Hegel --
anche se non compare nei manuali -- la dialettica servo padrone è la
dialettica tra quello che lui chiama desiderio tenuto a freno, legato
chiaramente al lavoro, e la crescita di questa dimensione, e nella
\textit{Logica} dice che (r). Lo strumento di lavoro permane, dura di
più della vita di un uomo, si deposita qualche cosa, e ovviamente la
dimensione del lavoro diventa centrale in Marx. Marx gioca sulle
diverse possibilità che la storia può dare, ed è ambivalente, c'è una
liberazione del lavoro, o è il lavoro che dà soddisfazione? Il lavoro
si dà con la tecnica, però è anche vero che descrive il lavoro in
certi aspetti in termini hegeliani, l'attività che dà soddisfazione,
solo se è uno stesso a compierla. Così come si rapporta all'arte, è la
trasformazione delle cose, quella che noi possiamo interpretare
passione -- la passione di Descartes e Spinoza, che cos'è una passione
effettivamente.

E confrontiamo il concetto di passione e quello di emozione. Abbiamo
scoperto che la ragione non è il punto di non-emozione, e sembra
quindi che siamo liberato del pesso della ragione. La differenza è che
l'emozione è \emph{un turbamento che si ripete}, mentre le passioni
sono un desiderio prolungato, che non si soddisfa, legato alla
produzione simbolica, ecc., (r), vivere nella situazione in cui si
rischia di esserer disorientati, ma per questo si trova un
orientamento autonomo, qualcosa di complesso legato alla passione, non
alle emozioni. Quello che fa il capitalismo è confondere (r), danno
l'illusione di soddisfare un desiderio piuttosto che un bisogno e si
confonde passione con emozione. Facciamo due esempi molto semplici:
che differenza c'è fra il mangiare e la cucina? Che differenza c'è fra
il sesso e l'erotismo? Non sono la stessa cosa, il mangiare soddisfa
un bisogno, la cucina è un fatto culturale che soddisfa il bisogno. Il
punto forte è questa dimensione, che la cucina è un fatto culturale,
così come l'erotismo, che ha a che fare col vedere e non vedere,
mentre il sesso è una soddisfazione necessaria. Infatti l'erotismo
cambia storicamente, come la cucina, ed è tutto giocato sul simbolico
del vedere e non vedere, che viene fra l'altro utilizzato per vendere
le merce esattamente per confondere il desiderio e il bisogno. Sono
giochi complessi, sono quelli che Marx descriveva come feticismo della
merce, esattamente perché c'è complessità in tutto questo, la merce è
carica di valore simbolico, e dentro la merce c'è il lavoro. Il punto
è che nel mondo borghese la questione del lavoro diventa un elemento
centrale, perché è vero che c'è un mutamento di relazione tra gli
uomini, in comparazione col mondo greco e il mondo medievale, il mondo
greco e medievale il lavoro appartiene agli schiavi. Qua il lavoro
diventa la trasformazione del mondo, in cui diventa un'abilità
\bsq{nobile} per l'essere umano.

Aristotele si era reso conto dell'impossibilità di cogliere il
problema dell'uguaglianza. Esempio (r) e Marx dice che non lo può
fare, bisogna una società con valore di scambio, con qualcosa in cui
tutto si universalizza. Non è la polis, comunità di appartenenza, e
dall'oikos, la struttura amministrativa di una famiglia, che era una
struttura amministrativa, e non a cambio gli scambi li facevano i
meteci. E anche nella filosofia medievale non c'è questa dimensione,
che piano piano si afferma.\textit{Tempo della Chiesa e tempo del
mercato}, lui segna con questo il passaggio dal campanile della chiesa
al campanile del comune, all'orologio. Un cambiamento dell'idea del
tempo dal mondo agricolo al mondo cittadino. Il tempo della Chiesa non
era omogeneo come il nostro, sui secondi, minuti e ore, era costruito
sulla natura, così come la vita dell'agricoltura. Non c'era questa
dimensione, che nasce piano piano. Cartesio utilizza l'orologio per
l'uomo macchina, e segnava il tempo non a caso. L'uomo faustiano è il
segno di questo passaggio, che sia della tecnologia e così via. Ma
fausto non è il solo, e ci si vuole mettere in guardia, non è il solo
mito di autoconsapevolezza del mondo borghese moderno.

Non è un caso che Hegel si richiama all'antico rispetto a una
contraddizione della storia legata alla modernità, lo scontro tra
diritto di famiglia e diritto dello stato.

Una figura centrale, che in un certo punto espone le contraddizioni
del Fausto (?) è Hamlet. \textit{Discorso sul metodo} cogito, dialogo
con se stesso, e in questo dialogo con sé stesso c'è un cambiamento.
Il punto di arrivo, l'unica cosa è il cogito ergo sum, penso, dunque
sono, sono qualcosa diverso dal pensiero... (r). Questo è un discorso
che deriva dalla tradizione inaugurata da Hamlet. Tradizione di
grandissima importanza perché la caratteristica di Hamlet è che nei
monologhi Hamlet cambia, cioè il monologo è un dialogo dove avviene un
cambiamento. Non si è più la stessa persona, si cambia.

È difficile capire il tema in Hegel dell'autocoscienza, che nasce
perché c'è una dialettica fra contrari. Termine tedesco (?), qualcosa
che dilegua, sparisce, ma è un dileguare che permane, resta, lo si
tiene dietro. Hegel perviene con questa dialettica al tema
dell'autocoscienza assoluta, e il tema della conoscenza nasce con
Hamlet, nel guardarsi allo specchio e vedere una persona diversa da
quella che ci si aspettava. Inizio di \textit{Uno, nessuno e
centomila}. Il problema di guardarsi allo specchio è per certi versi
il problema di Hamlet, si deve vedere se c'è qualcosa di diverso da sé
stessi, e non a caso si appartiene a tutta la tradizione fantastica
del doppio. Il tema del doppio è collegata nella tradizione fantastica
allo specchio (r esempi). Hoffman, tradizione romantica, fu un punto
di riferimento fondamentale per Baudelaire, che ha a che fare con quel
demoniaco che si trova già in Fausto. Il demoniaco è l'altro da te,
che può avere moltissimi livello e valori. È il dramma dell'alterità
che si scopre in se stessi, non negli altri. Questa è una
caratteristica della modernità. I greci e i cristiani non hanno questo
problema se non come vaghe anticipazioni. Le confessioni di Agostino
non si muovono in quest'ottica, non sono Cartesio, e poi c'è Don
Quijote. Don Quijote è la nascita della modernità, ma Don Quijote è
colui che si scontra con la modernità, è l'uomo del passato che c'è
nel presente, la figura nella modernità che si scontra
generazionalmente. C'è questo problema, chi sono loro? Sì, sono la
dialettica del servo-padrone, però non lo sono nei termini di
Robinson, termini che ha in mente Hegel (r) ma un rapporto che già non
c'è un questo mondo cambiato, e in questo mondo cambiato, è folle? O è
diverso? O ha un rapporto diverso? E questo rapporto diverso,
naturalmente, diventa qualcosa che apre un rapporto tra l'immaginario
e il reale, tra fantastico e razionale. Ad un certo punto in una
vincenda comica, divertente, Don Quijote va dal burattinaio che fa da
saraceno. Lo scontro è di solito tra saraceni e cristiani.
Naturalmente cosa fa Quijote? \emph{Scambia i burattini per esseri
reali}, perché non coglie più la differenza che noi cogliamo, o
dovremmo cogliere normalmente quando si guarda un film e si va al
teatro, la differenza tra illusione e inganno (r). Infatti, i registi
di cinema, molto abili, nel film horror giocano su questo, cercano di
mettere paura in cui non c'è bisogno di esibire il visibile, perché
c'è il disbelief. 

L'ultimo mito è Robinson, l'uomo che si costruisce la vita da solo in
un'isola, diventa imprenditore di te stesso, caricatura di un mito
gigantesco. Costruirsi un mondo, farsi una proprietà, e una volta
fatto questo è il rapporto con gli altri. La cosa che viene
trascurata, tipico da Robinson, il fatto che non potrebbe sopravvivere
se non avesse presso gli strumenti, che ha bisogno di strumenti. E lì
è realistico con la roba che raccoglie, ma che cos'è la roba che
raccoglie? Il lavoro sociale cristallizzato. Quindi c'è tutt'una
discussione su quello, e nemmeno Rousseau l'ha immaginato, l'uomo che
per natura, quello che criticava Marx, l'uomo economico che nasce di
natura alla mentalità dello scambio... è vero semmai il contrario, si
nasce in comunità.

Questi sono prolungamenti dell'uomo faustiano, di questa figura, il
segno proprio dell'individuo moderno. Libertà, rapporto con gli altri,
sono tutti costruiti su questa storia, che segnano la modernità
rispetto al passato, nel bene e nel male, e tutto è giocato su questo
rapporto. L'elemento cooperativo è però anche il maggior fattore di
sfruttamento, quello che viene sfruttato, quello che non c'è, chi lo
desidera? Il mito di vita è quello del manager (r). Noi ci troviamo di
fronte a questo tipo di dimensione dell'uomo faustiano, e questo ha
delle passioni, non delle emozioni, e queste sono idee sparite. Le
passioni sono fatte di emozioni, ma non lo sono, ancora meno quelle
cartesiane o spinoziane, qualcosa di più complesso, differito nel
tempo. Non vengono più dichiarate passioni mentre viene acetato il
fluire delle emozioni, e le emozioni senza passione creano il più
grosso problema della contemporaneità che è la dipendenza.

La passione per la ricerca, l'arte, ecc., non è diversa dalla
dipendenza, non c'è differenza di fondo, ma dove si indirizza? Per
l'arte e così via non c'è sempre bisogno di una maggiore dose. La
dipendenza è il desiderio di riprodurre un'emozione. Il desiderio
nasce dai vincoli, anche il gioco, senza regole non c'è gioco, sena
ritualità non c'è sacro, non si distingue il sacro dal gioco, dal
profano (i). Tutti i temi della tradizione psicanalitica sono tutte
produzioni simboliche derivate da vincoli e spostamenti (?), e gli
antropologi hanno parlato di questo. Se noi siamo animali simbolici,
non vediamo perché c'è spostamento, e questo vale anche per la
matematica, e lo dice Spengler che è simbolo, e l'origine dei numeri è
identica a quella del mito, e c'è un elemento di verità in tutto
questo. È chiaro che la matematica è un gigantesco processo di
sostituzione che porta alla concettualità, è un'invenzione
straordinaria. La natura della matematica è evidentemente simbolica, e
non l'ha inventato Spengler, è un tema non molto studiato, ma per
esempio presente in un filosofo che oggi quasi nessuno studia che si
chiamava Hippolyte Taine, e questo è una figura molto importante,
anche se oggi non suscita interesse. Primo, è quello che inaugura la
(r) che è l'inizione del rapporto tra filosofia e psicologia, ed è
corrispondente alla nascita della psicologia clinica, e lo fa insieme
a questo psicologo che si chiama (r), ma a parte quello, lui diventa
famoso soprattutto perché ha scritto dei libri sulla rivoluzione
francese, come storico, è stato anche un grandissimo critico
letterario, e ha scritto un libro che si intitola dell'intelligence
(?), dove lui sostiene -- ed è un sensista molto legato alla
tradizione di Condillac e l'empirismo inglese -- che la nostra
conoscenza è un procedere per sostituzione, ed è una cosa
interessantissima, e si tenga conto del fatto che Taine era
ammiratissimo da Nietzsche, ed era molto amico di un grandissimo
maestro del romanzao relaistico dell'Ottocento (r), ed è uno di questi
che apre in aula il discorso della sostituzione, che si trova via a
via, in parte anche Spengler. Ed è proprio in questa chiave del ruolo
della matematica come simbolo che Spengler apre questo discorso sul
rapporto tra modernità, pittura, matematica e prospettiva. L'uomo
faustiano è un uomo prospettico, attraversato da Brunelleschi, perché
si apre una visione del concetto di spazio e infinito, che fino
all'arrivo di matematiche non euclidee, diventa l'invenzione di un
nuovo moderno. pp. 262-264. Sottolinea un aspetto importante, ripresso
anche da un autore come (?), perché in questo modello si riproduce
nella modernità una nuova idea di modernità. Il concetto di profondo,
che per noi è ovvio, è una cosa che si modifica. Il nostro concetto di
profondità, che appunto Spengler collega alla prospettiva, in quanto
collegato ad essa, si associa alla distanza, e questo è una cosa che
il postmoderno ha cercato di eliminare, in parte a ragione, in parte a
torto. \textlatin{\textit{profundus}} latino, profondità a distanza,
modello prospettico, cogliere l'infinito, che se è profondo è a
distanza, si deve avere il senso della distanza.

Verità, distante, nascosto e profondo. Il fenomeno ti inganna, e
l'operazione di ricerca è guardare in fondo, tema di ricerca che non
hanno i greci, ed è la cosa dell'uomo prospettico che poi diventa uomo
faustiano. Sempre qualcosa di più profondo che si deve scoprire. È
Nietzsche che mette in crisi di togliere quest'idea di profondo,
inconoscibile, e lui dice che ne possiamo fare a meno, perché lui dice
nella gaia scienza che i greci sono profondi per superficialità, ed è
un tema che si trova contemporaneamente e indipendentemente in (r) che
leva la prospettiva. Cerca la profondità, ma non la cerca in fondo, e
infatti inaugura il pensiero astratto e così via, come dirà Paul
Valery, quasi ironizzando, che la profondità si trova nella pelle.
Dov'è nascosta? Nella superficie, non la si trova nascosta da qualche
parte, si pone il problema, quella è la verità granitica, si è
raggiunta la dimensione della certezza assoluta, e quindi c'è qualcosa
che non va in questa ricerca del fondamento.

(?) Operazione nella filosofia postmoderna, tutto avviene in un
piano, risposta molto pacificatrice, non ci sono più sbalzi, tutto è
uguale a sé stesso, e invece proprio a partire da Nietzsche non è che
sparisce la prospettiva della profondità. Ma la profondità non si
cerca più nella distanza, ma la si cerca in primo piano. Profondo non
è distante e non lo è per i greci, ma è un tema che attraversa parte
della filosofia contemporanea che arriva fino a Italo Calvino. Lui si
rende conto che non è possibile una corrispondenza (lezioni
americane), e cita (?) e Wittgenstein. Quel che è nascosto non ci
interessa, siamo da un'altra parte.

% Doppio, specchio, ritratto. Anche lì è Fausto, nel senso della
% giovinezza che riproduce in chiave moderna o quasi postmoderna. Il
% suo paradigma è tradizionale, ma la sua dialettica lo smentisce
% chiaramente, è complesso essendo un filosofo dialettico. Quello che
% salva Hegel è che questi modelli sono quasi tutti centralisti. Hegel
% è più complesso da questo punto di vista (r).

\end{document}
