% !TEX root = ..\..\main.tex
\documentclass[../../main.tex]{subfiles}%
\ifSubfilesClassLoaded{\addbibresource{../../bibliography.bib}}{}

\begin{document}

L15 03-12-2025

Tema della crisi della ragione. È un libro di antropologia, sociologa,e
filosofia della storia. Riflette sui concetti ontologici ed
epistemologici della sua disciplina. Tra otto e Novecento c'è un
cambiamento cateoirale nella fiosofia europea, non c'è più un tempo
unico lineare, non c'è più soltanto lo spazio euclideo, non c'è
soltanto la prevalenza di un solo tipo di prospettiva. La crisi del
pensiero classico è insieme crisi ce investe le discipline umanitisce,
artistica e scientifiche. Su questo È molto preciso e dice ce bisonga
inserire il all'interno di una concezioen ce non è quella dell'uomo
razionale omo faber, alternativa al positivismo. Omo ludens in
opposizione all'uomo faber al punto della raione, e fi fa a
riferimenti ben precisi nietzsche, e burckardt, entrambi in rierimento
a Schopenhauer.

Filosofo, antropologo, filosofo che lavora a un nuovo concetto di
storia. Warburg. Si deve pensare anche per Hozinga la lezione
dell'eterno ritorno Nietzscheano. Principio di non contraddizione,
asse della ragione classica. La storia, disciplina del divenire deve
contenere allo stesso momento e quasi con eternità [...]. Tutta la
riflessione della storia è incardinata in questo gioco fra divenire e
quasi eternità, Ponty, Foucault. Sul fatto che esistano strutture,
invarianti, strutturalismo francese, formalismo russo, modi di
riprendere la questione. Insistere su una storia che non è solo
lineare è anche alla radice del \emph{decadentismo}, che criticare la
linea significa vedere il luogo della decadenza e la crisi nella
storia. 

Sul punto di una nuova concezione del tempo che è anche decadenza, si
pensa sia alla montagna incantata di Mann, nuova concezione del tempo,
i personaggi sono in uno spazio e in un tempo diverso, come in un
gioco, e naturalmente un sanatorio. Questo unisce tutta la grande
letteratura del Novecento, Woolf, Joice, ecc. E infine, il capolavoro
storiografico è il testo che l'ha reso celebre, \emph{L'autunno del
medioevo}, riletto alla luce del decadentismo. Profonda diversità
della mentalità medievale, anche sanguigno, che cerca di vedere anche
nella sua attualità, è un passato che ci accompagna, e dunque diviene
presente. Vico aveva fatto enormi sforzi per immergersi nella mente
primitiva, così Hozinga li fa nella mentalità medievale rispetto
quella umanistica e poi illuministica. Fondamentale per lui è la
questione dell'immagine, è uno degli storici del Novecento che prima
ha insistito sul ruolo capitale dell'immagine per lo storico. Essa, se
pensata in modo adeguato, rappresenta l'aspetto mitologico della
mentalità umana, rimosso che nello stesso tempo è presente
nell'inconscio. In questo modo, prima di passare all'\emph{Uomo
ludens}, rispetto a Burckardt -- autore della \emph{Civiltà del
rinascimento} produce uno spostamento netto, rinascimento letto ancora
con il punto di vista di un modello greco, di una ragione classica.
Invece rispetto al rinascimento Hozinga lo fa nel medioevo morente,
alla fine del quattordicesimo e fine del quindicesimo secolo, civiltà
decadente e condannata, ma in quella civiltà si rispecchia e si vede
la propria decadenza, un nuovo concetto di storia che non è il
progresso, ma la crisi dentro di sé. Il decadentismo di Hozinga non è
un nichilismo, rimane una grande valorizzazione della vita, ma che non
nasconda le parti mortali della vita, che è ance limite, caducit,
frailit del quotidiano.

Ora si vuole cominciare o esporre semplicemente li aspetti principali
del libro. Nietszche, Burkhardt, Hozinga, tre forme diverse di opporsi
al progresso lineare. Il primo col recupero del greco, il secondo
esalta il rinascimento in opposizione alla industrialità, e Hozinga il
decadentismo. L'altra cosa da sottolineare è che \emph{vico è centrale
in quest'idea}, e uno scrittore che risente di Vico esplicitamente è
James Joyce, nel suo ultimo romanzo. 

Nel concetto di crisi della ragione, il riferimento fondamentale è
Wittgenstein.

Entriamo nel meccanismo del libro, e innanzitutto quello di chiarire
la questione del \emph{rapporto tra gioco e cultura}. Lui vuole dire
che il gioco è la matrice del processo di cultura e civiltà. Non si
trattava di domandare quale posto ha nella cultura, ma quanto la
cultura stessa ha il carattere di gioco. Si tratta di integrare il
concetto di gioco in quello di cultura, ed è una proposta anche
provocatoria dare questa centralità al gioco, che è insieme
antropologica, sociologica e filosofica. Questo è un punto
fondamentale, e ci sarebbe da dire che in questo c'è un aspetto
specificamente olandese di Hozinga, poiché il gioco è importante per
la storia olandese... non è un caso che sia un olandese a produrre
questo testo.

Quando giochiamo siamo seri tanto quanto lavoriamo. Distinzione serio
e saggezza-follia. Quando ci si impegna in in gioco si è profondamente
impegnato e impresso in questo gioco. Si è seri e giocosi allo stesso
modo, critica alla società del lavoro che è solo seria. Insiste sul
fatto che il gioco è libertà, tempo libero, non lavoro e necessità.
Non è libertà assoluta, ma uno è libero di mettersi a giocare o meno.
Ancora, il gioco fa entrare in uno spazio e un tempo diverso, affine a
quello del sacro. Grossa parte del libro parla dell'elemento
religioso, e siccome vuole far vedere che cose delle società primitive
sono parti di quella odierna, e tuttavia questa distinzione fra sacro
e profano non c'era nell'antichità ordinaria. L'attività diventa
disinteressata, non ha una finalità di utilità, è fine a sé stesso, è
superfluo. Quando si gioca, quando anche il bambino gioca è nella
dimensione del credere e non-credere. Doppia consapevolezza, di farla
da un lato, e assorbito in essa, seppure non completamente.

Per tornare sulla questione del quotidiano, si pensi anche che
rispetto alla vita quotidiana, questa viene trasvalutata nel gioco,
perché sono una serie di azioni che spesso hanno a che fare con la
vita quotidiana, ma la cosa importante è che il gioco tiene la
'domenica della vita' della festa, di attività che non sono solo
quelle lavorative, ed è una specie di sacralizzazione del mondo
terreno -- tema tipicamente protestante --. Pittura olandese del
Seicento, scopre il quotidiano soprattutto nella pittura. Questo ci
permette di affrontare maggiormente un altro aspetto del gioco, il
fatto che esso è l'apertura di uno spazio e un tempo particolari, è
istituzione di un ordine, lo spazio e il tempo sono limitati.
Introduce un ordine nel caos della vita, e in questo ha una funzione
di civilizzazione. Il gioco comincia, e ad un certo momento
finisce; può essere però ripreso dopo. In quasi tutte le forme
sviluppate si incontrano questi elementi di ripresa, di cambio di
turno, e poi spiega anche la questione dell'ordine negli spazi, domina
un ordine proprio e assoluto, esso crea un ordine, realizza nel mondo
imperfetto e nella vita confusa una perfezione temporanea e limitata.
Forma di sacralizzazione e compromesso per cui si dà un valore che non
è infinito, si è fuori dalla dimensione dell'assoluto, e questo porta
ad approfondire il concetto di libertà. Il gioco è il fatto di esser
libero proprio nel sottomettersi alle regole, che le viole esce dalla
logica del gioco, non è lui che fa civiltà. Questo significa anche che
il gioco è qualcosa di estetico, patto tra libertà e necessità nel
bello -- Schiller -- non si approfondisce, ma è una presenza
importante nel discorso di Hozinga, che lo unisce anche alla scuola di
Francoforte. Proprio per questo si deve pensare che quest'idea è un
discoros profondamente politico, patto reciproco silenzioso per cui si
crea una società libera, patto rousseauiano. Contrapporsi implicito di
Hozinga a Karl Schmidt e il suo concetto di politica, cioè a una
visione per cui la relazione fondamentale è quella amico-nemico.
Invece per Hozinga è quella del gioco, che ci vincola reciprocamente,
e che istituisce un ordine dal basso. Non è qualcosa di autoritario.
Questo è un punto, c'è quest'aria conservatrice, ma c'è dietro questo
messaggio politico, molto radicale. Peraltro, con questa proposta, non
ci sta facendo un discorso ottimistico: se si prosegue nel testo si
vede che quest'idea che noi facciamo un patto e si stabiliscono le
regole, non è qualcosa che eviti i conflitti. Nel gioco c'è
aggressività e persino guerra; è solo che non è il carattere
essenziale della natura umana. L'espressione più profonda nella vita,
che gioca anche nei suoi aspetti più pericolosi... è importante tener
conto che si lega a Mauss, tematica del dono e al Potlach. Il gioco è
istituire le regole, ma è anche agone, è un aspetto non secondario,
costruire società conflittuale. Quindi si rifà appunto a questo
Potlach e quindi c'è quest'idea dell'agone, gara e conflitto in cui
però, questo momento di agone e di conflitto, per Hozinga, è anche un
momento di rappresentazione e teatralizzazione del conflitto, che è
anche una maniera di controllare. Di nuovo, portano in primo piano il
tema di una comunità partecipante. Quello che si rappresenta è un
dramma, sia nella forma di una rappresentazione, sia in una gara. Non
soltanto come rappresentazione, ma come indicazione. La sua funzione
non è un puro e semplice imitare, ma co-comunicare e co-agire,
elementi di teoria di una comunità. Siamo qui davanti a una
presentazione del gioco come momento prelogico, fa pensare sia alla
mente primitiva di Vico, sia al grande antropologo [?]. Passaggio dove
si sente questo accento prelogico, valorizzazione dell'elemento
mitologico come elemento dell'uomo anche moderno. Questa sfera del
gioco sacro è quella dove si ritrovano il bimbo e nel selvaggio. La
sensibilità estetica dell'uomo moderno. Il gioco è la sfera del
primitivo, del bambino, del collega. Concetto del gioco come gara, fa
vedere come tramite il gioco si possono vedere tutte le attività
umane, il mito, la poesia, tutte le forme della cultura sono viste
come radicate nel gioco. Parlando qui di società primitive e antiche,
fa un discorso che ha a che fare con l'oggi, con la società
industriale, può riesplodere, il calcio, fenomeni della società di
massa. Hozinga fa anche una polemica con un certo tipo di marxismo e
socialismo che hanno privilegiato la dimensione utile, l'economia non
è l'essenza dell'uomo. Qui parlava del marxismo della seconda
internazionale, il marxismo volgare.

È attraverso il conflitto che si crea una comunità, Kant, che gli dà
una direzione molto liberale, invece qui è meno borghese, e poi dietro
c'è anche, oltre che Mauss, anche la figura hegeliana del servo e del
patrone. Si valorizza il momento concettuale, si pensa che è un nucleo
del momento sociale, l'antropologia fondamentale, relazione uomo-uomo
e uomo-leviatano.

Ora si arriva alla conclusione, il finale del libro è appunto un
finale che esplicitamente tira fuori le concezioni di Schmidt. Hozinga
non contrappone a Schmidt l'ottimismo, non è banale razionalimso la
visione del gioco, e qui vale la pena citare alcune parti di questa
polemica esplicita e forte. Hozinga dice che Schmidt e questo teorici
sono rozzi, c'è svalutazione che fa molta impressione in positivo,
perché dagli anni Ottanta in poi in Italia c'è una visione
sopravalutata di Schmidt. Proprio nel superare il modello
amico-nemico, l'umanità entra nella sia dignità. La politica è fissata
nel terreno della cultura legata in competizione, e può elevarsi solo
con una morale, la contrapposizione è un gioco che ha dentro di sé una
dimensione \emph{etica}. Alcuni l'hanno interpretato come posizione
religiosa, ma è una cosa molto moderna, dobbiamo scegliere queste
regole contro le guerre, su questo non ci sono garanzie -- non è un
teorico della garanzia metafisica -- ma non è vero che l'essenza del
rapporto umano sia la guerra. Può esserci anche una scelta per la
pace. Il finale è importante anche come attualità. 

Nella prospettiva di Hozinga c'è un fondamento cristiano, e quindi
un'ideale di uguaglianza. Quindi non è l'idea di contrapporre
l'antico al moderno, ma l'idea che nella stessa civiltà moderna ci
sono tutt'una serie di cose che non si sono tirate fuori. Auerbach,
vero realismo quello che tratta le cose più banali, che nella
posizione classica era svalutata, mentre invece il moderno è la
capacità invece di prendere sul serio ciò che è più banale e più
quotidiano. Questo cambia col realismo anche in pittura.

Gioco simile al ruolo della corporazione in Marx, il capitalismo
produce valore grazie al modello cooperativo, e tuttavia è l'elemento
primo dello sfruttamento.

\end{document}
