% !TEX root = ..\..\main.tex
\documentclass[../../main.tex]{subfiles}%
\ifSubfilesClassLoaded{\addbibresource{../../bibliography.bib}}{}

\begin{document}

Bolla \textlatin{\textit{inter caetera}} inizio della globalizzazione.
SI prendono disposizioni che hanno valore per tutto il globo.
Premessa per stabilire doveri che mettevano in moto un meccanismo
di interdipendenza tra questi popoli colonizzatori. Naturalmente
l'autorità del Papa era divina, vicario di Cristo. Papa discusso
spagnolo di Valencia, incline ad accontentare il suo sovrano
Ferdinando, discusso per i suoi costumi privati. Tutto questo faceva
pensare che il pontefice avesse più a cuore gli interessi famigliari
che la salvezza dei fedeli, cosa che porterà a una cattiva immagine
del Papa Alessandro. Egli emette questa bolla e trova l'accomodamento
tra le corone di Spagna e Portogallo stabilendo un confine, che per la
prima volta non era un confine geografico, ma astronomico, e non solo
si poggia sulle stelle, ma soprattutto a questa dimensione
universale, come universale dovrebbe essere l'autorità del pontefice;
che si dovesse estendere al mondo secolare e della politica era
questione di discussione e disputa. Le questioni economiche e
demografiche, ci sono anche tantissimi che credevano assolutamente in
questa missione di salvezza, che credevano fermamente che la
conversione dei popoli selvaggi era portare loro un dono, la vita
eterna. Non è quindi un pretesto.

% Non mangiano carne, sono docili e acconsentiranno di buon grado
% all'appello che i cristiani gli rivolgeranno. Hanno delle credenze
% religiose, un istinto religioso che li rende atti a ricevere la
% fede, qualora fossero istruiti.

% C'è un obbligo, piuttosto che un permesso del re Ferdinando di
% evangelizzare quelle terre. 

% Re di Castiglia e di Leon. Gli aragonesi saranno esclusi dal
% privilegio.

% Questione della scomunica. All'epoca aveva effetti notevoli,
% soprattutto se era data ad un regnato, perché legittimava ai
% sudditi a disobbedire a quel potere.

Anche nell'arbitrarietà dell'atto si cercava di dare auterevolezza
giuridica alla decisione. Nel medioevo c'era già stata una
\emph{disputa sulla giurisdizione marittima} [R]. Fin dove si estende
il potere di un principe secolare nel mare. Si partiva sempre dalla
base delle leggi allora condivise, le leggi romane, che tenevano
insieme tutta l'eredità latina e anche oltre. Tra questi giuristi uno
dei più famosi che appartiene alla scuola dei [R], è il primo che
comincia a interpretare le leggi e se ne discosta con le sue
interpretazioni personali, discostandosi dalle autorità consuete. Lui
commenta, ed insegnava a Pisa, e all'epoca la repubblica pisana aveva
il bisogno di rivendicare il possesso della Corsica, ed estese la
giurisdizione 100 miglia fino ad essa, solo che poi i portoghesi non
si accontentano, e lo spostano a 200 miglie con Tordesillas. 

% Questo è stato il titolo giuridico della conquista. Joseph de Maistre,
% Du Pape, pensiero reazionario dopo la rivoluzione rimpiange l'autorità
% che il papa aveva allora. Diverso il commento di Voltaire nel
% Dizionario filosofico. Commento di Guillame Raynald, nel bel mezzo
% della disputa fra i coloni nordamericani e l'Inghilterra: primo
% testo di condanna e critica del sistema coloniale.

Naturalmente l'esplorazione continua, e da lì a pochi anni c'è la
prima circumnavigazione del globo da Magalellano che venne poi
uccisso. Dopo che fu copiata questa prima circumnavigazione sorge il
problema di spartire anche il resto del mondo. Così le corone del
portogalle e la spagna, nel 1529 con un trattato bilaterale,
completano il dispositivo di Tordesillas e la Bolla col trattato di
Saragozza. Restano fuori le isole filippine, che si chiamano così in
onore del re Filippo II.

La colonizzazione dei portoghesi e dei castellani assumono caratteri
diversi. Il Portogallo non ha la popolazione per stabilite colonie
vere e proprie, è un popolo di marinai e commercianti. Essi si
accontentano di controllare sulla costa dell'Africa così da garantire
il traffico di merce. Un carico di spezie poteva produrre 4 volte il
guadagno rispetto al prezzo di compra. Un viaggio intero da Lisbona
poteva metterci due anni. Il Brasile diventa a poco a poco una colonia
di popolamento; del resto sono avamposti mercantili, di cui i più
importanti è Goa, dove si stabiliscono e trattano con le autorità
locali. Diventa il centro di coordinamento dell'impero portoghese per
tutto l'estremo oriente.

La colonizzazione castigliana, invece comportò la migrazione di tante
persone ed è di sfruttamento agricola. L'istituto che permisse la
messa in cultura dei nuovi territori è la cosiddetta incomienda,
risultato della conquista una volta che militarmente il re riesce a
controllare i territori. I conquistadores vengono remunerati con la
concessione di terre, che rimangono formalmente in proprietà dei re,
sono come una sorta di feudi, la titolarità del possesso è legalmente
dalla corona, ma il diritto di sfruttamento è concesso ai
conquistadores, con i rispettivi schiavi assegnati a coltivarla. Ma
anche ci sono compromessi per i possessori, che devono teoricamente
applicare quello che dice nella bolla \textlatin{\textit{inter
caeteras}} ed evangelizzare gli indigeni. Ovviamente questo non
necessariamente succede. Fondamentalmente il feudo gestisce
un'economia di sussistenza, che è mirata all'autoconsumo. Tutti quelli
che lavorono nel feudo vivono di quello che producono, e c'è veramente
poco avviato al mercato. Nel caso dell'incomienda, invece, non solo
c'è un'integrazione con le dinamiche del mercato, ma è un mercato
globale. Coltivavano merci che potevano rendere commercialmente, lo
zucchero inizialmente, poi il caffé e il cacao, e altri nuovi prodotti
molti dei quali fin'ora erano sconosciuti agli europei. Quindi la
incomienda ha dei tratti si sommiglianza con il feudo, ma è un
meccanismo di sfruttamente all'interno di un'economia che è già
globale. Quello che è prodotto nel messico era consumato a Parigi, a
Madrid e così via.

Quello che cercarono di stabilire sono aree di sfruttamento esclusivo
per produrre merce da vendere in Europa, o anche come mercato per
l'assorbimento di prodotti europei, principalmente il tessile. Bisogna
vestire queste persone che sono nude, come diceva Colombo. Gran parte
della storia della marineria inglese è una storia di violazione di
questo tentativo di creare un monopolio impermeabile. È un'epoca che
in quell'area è caratterizzata da una guerra a bassa intensità, con
tanto contrabbando. Gli sforzi degli inglesi dall'inizio del Seicento
fino alla guerra di successione spagnola furono volti a mettere le loro
merci nei mercati americani.

Ma prima di arrivare a tutto ciò si deve passare per la conquista che
è il dramma del continente americano. La storia delle popolazioni
cambia in maniera radicale; questo proposito di evangelizzazione porta
con se l'idea che tutte le credenze religiose di questi popoli siano
prodotto del demonio, era paganesimo e andava erradicato. BIsognava
fare un'opera ci cancellazione culturale. La conquista è stata un
trauma per i popoli americani, ed è difficile anche raccontare questa
storia dal punto di vista dei conquistati. La maggior parte delle
fonti storiche sono fonti spagnole. Il punto di vista degli indigeni
lo si può solo intuire. C'è stata effettivamente una cancellazione
della cultura indigena. C'è un libro famoso di un etnografo francese
del secolo scorso, Francois... \textit{la visione degli indi}, come fu
vissuta la conquista da parte degli indigeni? Sappiamo quasi tutto
sulle impresse ci Cortez e Pizarro, e anche sul diritto all'esercizio
della violenza e così via, però non c'è più la voce dei vinti, sono
arrivate pochissime fonti scritte, quasi nulle. Ora questo etnografo
immaginò di ricostruire il trauma della conquista con fonti
etnografiche. Con i racconti poplari, le feste, ecc., che ancora negli
anni cinquanta del secolo scorso erano ancora vivi. La tradizione
orale, e solo attraverso di essa si riesce a percepire la drammaticità
dell'affare. Il più famoso conquistador è Hernan Cortez, conquistatore
del messico. Con la conquista delle nuove terre ci furono tanti che
partirono per far fortuna, si favoleggiava del Dorado e così via.
Cortez amava presentarsi come un idalgo, e quindi in qualche modo è un
primo grado di nobiltà rispetto a quelli che neppure sanno chi è il
padre, anche se il padre non gli ha lasciato niente, altrimenti non
sarebbe stato concreto ad affrontare il pericolo eo così via. Arriva a
Cuba, primo ed ultimo posseso stabile della corona spagnola, con
l'intensione di addentrarsi in terra inesplorata. Ma il governatore di
Cuba gli vieta tassativamente di imbarcarsi nella conquista. Ma Cortez
disobbedisce, spalleggiato da un certo numero di volontari, uomini come
lui anche loro desiderosi di arricchirsi, e nel 1519 partono. La cosa
strepitosa è che in brevissimo tempo, questi trecento uomini fanno
crollare l'impero azteco, quello meglio attrezzato dell'America.
Conoscevano i metalli molli, ma non lavorare il ferro. C'è un libro
classico sulla conquista del Messico di Prescott (?). Il punto è che
Cortez non era soltanto un militare, ma anche un gran politico. Lui
aveva capito che un grande impero come quello azteca, e si fece
un'amante indigena, e non solo grazie a questa donna si impadronì
delle lingue locali, grazie a questa donna che gli faceva
d'interpreti, ma poi soprattutto capì che l'impero era un popola che
avevano assoggettato altri popoli che non vedevano l'ora di sottrarsi
al giogo dei loro padroni aztechi, e quindi fu anche soprattutto alla
sua capacità diplomatica di stringere alleanze con le vittime
dell'impero azteco che riuscì a sopraffare gli avversari. Nel 1535 si
conclude l'impresa, e le aspettative dei popoli che aiutarono Cortez
non si compiono naturalmente. Analogamente si possono fare le stesse
osservazioni per la conquista del Peru ad opera di Francisco Pizarro.

Pizarro era ancora più spregiudicato, e l'impero Inca bisogna capire
che era una sorta di monarchia teocratico-solare. L'inca era
responsabile della riproduzione materiale e spirituale della società
che viveva con organizzazione sociale di tipo comunista, c'erano le
statue di Lenin. Non conoscevano gli scambi e le monete. Tutto
dipendeva dal potere dell'inca che era anche di tipo religioso, e da
questo punto di vista era anche una società fragile. Bastava colpire
al centro e il sistema crollava, e Pizarro sequestrò un inca e fu
anche assassinato. È memorabile l'incontro tra Pizarro e Atahualpa,
esemplare dell'intreccio tra la spregiudicatezza e ipocrisia di questi
uomini. E Pizarro aveva nel suo seguito uomini di fede, missionari che
lo consigliavano. Quando si incontrò con Atahualpa il frate disse ad
Atahualpa che doveva capire che c'è un Dio che voleva che lui e il suo
popolo si assoggettassero al re di Spagna.

Ci furono anche altri che si stupissero per la violenza dei
conquistadores, come Anton de Montesinos (1511) che predicava nell'Avana
ammonendo i cristiani che quello che stavano combinando era contro
Cristo.

% In questo passo ci sono i temi fondamentali che si scatenao
% all'origine di quello. Qual'è il diritto sulla conquista? La bolla
% del Alessandro, ma questo pariva la discussione sull'interpretazione
% che si dà all'autorità del Papa. Non sono uomini? Questa è l'altra
% domanda fondamentale. Ci saranno tanti che riterranno che non erano
% uomini.

\end{document}
