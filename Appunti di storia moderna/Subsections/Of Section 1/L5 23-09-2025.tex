% !TEX root = ..\..\main.tex
\documentclass[../../main.tex]{subfiles}%
\ifSubfilesClassLoaded{\addbibresource{../../bibliography.bib}}{}

\begin{document}

% Periodizzazione con la caduta dell'Impero romano d'oriente. Ma allo
% stesso modo, si potrebbe pensare che una dicontinuità molto forte
% tale da poter dire che si apre una tappa nuova potrebbe essere il
% 1300, però tradizionalmente qua si assuma la periodizzazione solita
% che va dalla scoperta dell'America al Congresso di Vienna, che si
% riferiscono a fasi di cambiamento più accelerate che non si
% riducono a questo o quel avvenimento. Ora si cercherà di
% ripercorrere le diverse fasi della storia moderna.

Si è pensato di aprire il capitolo delle esplorazioni geografiche a
partire dalle spezie, che riassumono molte delle motivazioni che
spinsero gli europei alle esplorazioni, a mettere in discussione tutte
le verità ricevute nell'incontro con l'altro, nell'entrare in contatto
con un mondo diverso di cui nessuno in Europa aveva sentito parlare.
C'erano motivazioni politiche legate alla caduta dell'Impero romano
d'oriente, la conquista di Costantinopoli che interrompeva le più
tradizionali vie di comunicazione con l'Oriente. C'erano motivazioni
di tipo economico perché le spezie erano un genere molto ricercato per
vari motivi, e il commercio di esse era molto redditizio. Le spezie
erano un consumo di lusso. Basta pensare ai ricettari, i libri per le
corti rinascimentali, era proprio un'ostentazione, e anche nella
celebrazione del banchetto si ostentava la posizione eminente che si
aveva, così come la seta. È sbagliato pensare che fossero ricercate
per ragioni di conservazione di cibi come certe volte si trova
scritto. I cibi potevano essere conservati principalmente grazie alla
salatura, o roba di fumigazione. Le specie semmai coprivano cattivi
sapori di cibi che non erano freschissimi, ma erano comunque per lo
più simboli di status, legati a certe fantasie anche queste fondate
sul racconto religioso. Si pensava che questi prodotti dovessero
venire dall'altra parte del mondo, quella che gli uomini avevano
abbandonato dopo il peccato originale, incorrotta e incorruttibile. E
questo perché avevano e hanno una caratteristica di mantenere la loro
fragranza molto a lungo, le spezie non si corrompono e quindi
portavano a un'idea dell'eternità, di un posto fantastico. Erano
caricate di significato religioso e anche terapeutico, erano insomma
molto ricercato. Il fatto poi che si mantenessero incorrotte nel tempo
aveva un risvolto pratico molto evidente dal punto di vista del
commercio, che si poteva portare a lunghissime distanze, che a lungo
furono gli unici commerci veramente redditizi, rifornivano un settore
molto alto dei consumatori del mercato. Naturalmente venivano
commerciate sin dall'antiquità. Venivano dalla Cina, India, ecc. La
rotta della seda e delle spezie fu anche l'origine della richezza
delle città che attraversava, furono per secoli i canali di
comunicazione dei commerci, il collegamento fra l'Europa e l'oriente,
i cui prodotti venivano distribuite nel mediterraneo, ad esempio dalle
repubbliche marinaie nel medioevo. Verso il 1300 il commercio era
fatto in precedenza da viaggiatori che si portavano nelle loro spalle
i prodotti. La maggior parte del commercio a partire dal Trecento si
era perfezionato grazie proprio ai commercianti dell'età comunale
italiana, che avevano messo in gioco strumenti per il commercio a
lunga distanza basati sulla fiducia reciproca ed elementi di
trasferimento di denaro [R]. Nasce il commercio a commissione [R].
Strumento fondamentale di questa strutturazione del mercato sono delle
forme contrattuali nuove, la lettera di cambio che permette di
trasferire fondi da una piazza mercantile ad altra senza trasferirli
effettivamente. In questo modo cerano corrispondenze tra commercianti
che avevano in pratica partite doppie per gestire rapporti
moltilaterali da fare. L'altro strumento che permette lo sviluppo, che
poi è la premmessa della monteizzazione dell'economia prima del
Trecento -- gran parte dell'economia è di sussitenza, si sta
sviluppaondo un mercato -- sono le \emph{assicurazioni} che permettono
di distribuire il rischio tra una platea di sottoscrittori ampia,
meccanismo di reciprocità che funziona e vale per tutti, finchè
nasceranno le grandi aziende nel Settecento. Ma dopo il trecento,
nella rivoluzione commerciale quello cambia, il commerciante non è più
il viaggiatore che porta le cose alle spalle, ma è davanti al libro
contabile, e in questo senso c'è anche una cambiamento strutturale
nell'economia che c'è da guardare. TUtte queste innovazioni di
carattere legale, tecnico e contabile sono tutte invenzini degli
italiani che poi si preoccupano di diffondere nel resto d'Europa. In
età medievale i due poli di sviluppo sono l'altra italia e le Fiandre
che già è un polo industriale, e l'asse dell'economia e la valle del
Reno, dove si sviluppa anche la grande finanza. È l'epoca anche in cui
si affermano le grandi famiglie di banchieri come i Medici.

I generi che venivano commercializzati dovevano avere la
caratteristica di non essere deperibili, e dovevano essere di poco
ingombro e alto valore monetario, vale a dire che gli utili su ogni
affare commerciale fossero molto elevati per pagare eventuali
naufragi, banditi, ecc. Non si commerciavano i fagioli a lunga
distanza, e neache il grano, che era un bene strategico per la
sopravvivenza delle popolazioni. Si commerciava la seta, le pietre
preziose, le spezie, tutte questi affari vennero a interrompersi con
l'\emph{espansione turca}.

Essi erano un popolo nomade dell'asia centrale che si era spostato
sulla peninsola analtolica già nel medioevo, da dove furono cacciati
ma poi si consolidarono con la dinastia fondata da Osman I, gli
ottomani discendenti di Osman, con capitale inzialemnte a Edirne. SOno
un popolo di pastori nomadi che stanno a loro agio a cavallo, sono
anche dei combattenti eccezionali. E riescono a conquistare
inizialmente tutta l'area delle pensinsola anatolica. La conquista di
Costantinopoli fa sì che quello turco diventi un vero impero che ha
sotto di se popoli molto diversi. Un potere autocratico quello del
soltano, sono un popolo che con il loro arrivo in medio oriente si è
anche islamizzato, però questo grande impero che nella sua massima
estensione arriva fino a VIenna e il marocco riuniva sotto di sé
popoli di fede molto diversa. IL sultano era signore assoluto, tutti
gli altri erano formalmente i suoi schiavi. Rivendica anche il titolo
di cesare, legittimo discendente della roma imperiale, la terza roma
-- e se ne parlerà dopo del funzionamento dell'impero Ottomano, grande
antagonista dell'Europa cristiana, altro mondo con altre regole, e
sarà anche grande antagonista dei persiani e poi dell'impero russo.

Nel 1451 tutti gli approdi del Mediterraneo orientale, soprattutto
dopo le conquiste di Selim II, figlio di Solimano il Magnifico. Per
sostentare costantinopoli c'era bisogno di tanto grano che veniva
dall'nord africa, ed era la linea di commercio più importante
dell'Impero. Con la caduta i costi del commercio di spezie salgono, e
tutti gli intermediari vogliono la loro parte, e questo abassa le
prospettive di guadagno di questo commercio importante. Questa è una
delle ragioni  -- ci sono altre di ordine demografico -- che spingono
gli europei a saltare gli intermediari, un'altra via. E i primi che si
mettono in cerca di una strada per i mercanti europei sono i
\emph{portoghesi}.

% Conoscenza geografiche scarse, immagine, basata sul racconto biblico
% della dispersione dei tre popoli.

% Forte di un'immagine, trampolino delle esplorazioni [R]. Forte capo
% san vincenzo. Punta nordovest del Portogallo. Giovanni il
% navigatore.

Naturalmente bisognava impadronirsi di tante nuove conoscenza, prima
di tutto le tecniche di navigazione astronomiche, perché naturalmente
bisogna avere punti di riferimento, e in mezzo all'atlantico ci sono
solo le stelle. I marinai sapevano certo che la stella polare indicava
il nord, ma non andavano mai in mare aperto, quindi era un'impressa
nuova, e bsognava disporre di navi abbastanza solide per le tempeste
alantiche. Questo sforzo per il portogallo era del tutto giustificato.

% Strumenti astronomici, arabi, raffigurazioni delle due imbarcazioni.
% Gallera mediterranea e Galleone atlantico. Mare interno regime di
% venti variabili, esposta a bonacce tali da bloccare la nave. La
% forza propulsiva era data dall'equipaggio di rogatori che il più
% delle volte erano prigionieri di guerra, schiavi. L'impego di
% gallere di guerra furono motivi dell'istituzione della schiuavitù
% prosperò fino a quasi la fine dell'età moderna. Naturalmente una
% schiavitù diversa di quella della tratta atlantica. Il galleone
% invece era disegnato per portare meno equipaggio, più manovrabilità
% e disegnate per andare nell'atlantico. GLi olandesi capissero come
% andare contro il vento.

Il portogallo si dirige in tanto verso l'Africa, l'esplorazione più
semplice. Comincia a esplorare la costa nord-occidentale dell'Affrica,
capoverde nel 1456, ecc. 1470 superano l'equatore, e poi
l'esplorazione si fa più complicata una volta superato l'equatore,
perché le correnti sono contrarie, il regime di venti nell'atlantico è
costante; nell'emisfero dell'europa boreale, c'è un moto in senso
orario, mentre nell'emisfero australe... moto antiorario. Queste navi
dunque si trovavano vento e corrente contro. Per arrivare allo scopo
dei portoghesi. Quello a cui mirano sono i ricchi mercati di
approvigionamento dell'estremo oriente, hanno intensione di arrivare
in Cina, e quindi si sono ripromessi di agirare questo continente che
non sanno quanto sia stesso, e lo sapranno solo nel 1486 quando
Bartolomeo Diaz ragigunge il capo di buona speranza. E poi si apre
un'altra fase che è quell dello scontor con i marinai arabi che
controllavano già l'oceano indiano.

All'epoca la peninsola iberica non era divisa com'è adesso e come sarà
di lì a poco il matrimonio tra Ferdinando di aragona e quella di
castilla, ci fu la possibilità perché alla corte si scontravano due
partiti, uno favorevole di un'alleanza con gli aragonesi, gli altri
con la corona del portogallo. ALla fine preval eil partito aragonese,
ed è il motivo anche di uno scontro militare che si conclude con la
pace di Alcacovas nel 1479, scontro che va raccontato perché sono
avvenimnetni che portano al formarsi della monarchia ispanica. In
questa occasione ci si misura per la prima volta con la confinazione
di nuovi acquisti marittimi, ed è im quest'occasione che i portoghesi
portano a porre un limite alle ambizioni spagnole nel mare. IL
trattato stabilisce che il parallelo che va oltre le canarie è vietato
alla esplorazione spagnola. In realtà i portoghesi vogliono
l'esclusiva nell'avventurarsi alla ricerca di nuove terre, richezze e
tesori, vogliono escludere gli altri europei.

Qui inizia la vicenda di Colombo. Prima si rivolge al re del
Portogallo, e poi finalmente viene ascoltato dalla Spagna. La
pirateria fenomeno diffuso e legittimato dalla tradizione. Aristotele
considera la rapina come un metodo di acquisto al pari di... Fino ad
allora la corona era stata impegnata nella guerra di riconquista
contro gli arabi che occupavano il sud della pensandola. La corona
espelle ai musulmani del continente, il fatto sta che tutta la
peninsola tranné il portogalle è sotto il controllo della corona
ispanica, e questo risultato si è spinto moltissimo lo spirito della
crociata, ed è vero che questo spirito anima anche l'esplorazione. Del
resto è uno dei commandamenti della religione cristiana, quella di
diffondere il vangelo, convertire tutti i popoli della terra. I
cristiani sono tenuti a fare testimonianza della fede, anche con la
spada, come diveca il vangelo di Matteo... costringeteli al banchetto.
Questo spirito, naturalmente, riguardava tutti i popoli europei
cristiani. IL primo viaggio di Colombo, che approda nelle isole
caraibiche... Cadi non è stata ancora fondata. Approda a Lisbona. I
portoghesi si arrabbiano.

% Dibattito pseudo-storiografico sulla scoperta dei cinesi.

Anche se fosse, non ha un significato storico, e se l'avesse in
America si parlerebbe mandarino. Dietro Colombo c'è invece una
monarchia che ha l'intensione di avere le ricchezze del nuovo mondo.
Come mai il Portogalle reagisce malamente e con quale argomenti
sostiene che il viaggio di Colombo non è lecito? Si appoggia da alcuni
documenti legali emersi da diversi pontefici sui mandati di
evangelizzazione. Nicola V, Alfonso del Portogallo 1452, concessione
rinnovara da Callisto III, Pio II e Sisto IV. Ecco che si arriva a un
passaggio piuttosto importante che gli storici hanno variamente
commentato. Momento che stabilisce l'inizio della prima
globalizzazione. Un primo atto giuridico volto alla spartizione del
mondo. Come si risolve fra il Portogallo e la Spagna? Vediamo di
trovare una mediazione con il papa regnante, Alessandro VI Borgia,
papa suddito del re fernando, che non è contrario all'idea. IL papa
interviene per fare pace tra questi due regni che sono sull'orlo della
guerra. IL papa come autorità superiore media il conflirrto e lo fa
con la bolla Inter Coetera.

% Bolle pontificie.

Vale la pena leggere questa bolla perché restituisce la mentalità del
tempo. Spartizione, a ovest il compito di evangelizzare sarà della
Spagna. A est sarà il Portogallo ad avere il compito. Per dividere
queste sfere d'influenza si stabilisce un confine. In oceano non ci
sono confini se non astronomici. E il confine è astronomico, 100 mille
ovest del punto di riferimento, meridiano che passa da quel punto
divide il globo in due sfere d'influenza diversa. Era il primo atto
giuridico che riguarda tutto il mondo. Due anni dopo spostato altre
100 mille, trattato di Tordesillas. Questo accordo veniva osservato,
cossiché quando accade che Pedro Cabral, dovendo allargare verso
ovest -- per prendere il capo bisognava andare vero ovest e poi
prendere il vento al sud dell'Africa -- ma andando molto verso
l'ovest si scopre il Brasile. Si fanno due calcoli e si dice, ma qui
si è a est del meridiano del trattato di Tordesillas, quindi questo è
roba nostra. Per quanto limitate -- questa è una delle differenze
radicali tra la colonizzazione spagnola e quella portoghesa, quella
spagonal furono veramente colonie vere e proprie, invece il portogallo
era più interessato a della basi commerciali. Non aveva la potenza
demografica di Castilla.

% Citazione della Bolla Inter Coetera.






\end{document}
