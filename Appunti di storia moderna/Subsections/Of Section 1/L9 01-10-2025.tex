% !TEX root = ..\..\main.tex
\documentclass[../../main.tex]{subfiles}%
\ifSubfilesClassLoaded{\addbibresource{../../bibliography.bib}}{}

\begin{document}

Canali di mobilità sociale tramite la chiesa in contesti dove non c'è
nell'ambito civile. Lo stato clericale permette una certa mobilità
sociale, che però non è trasmissibile perché c'è il voto della
castità, che ha anche questa precisa finalità. L'ordine sociale è
pensato come voluto da Dio, è il migliore ordine possibile,
legittimato dalla religione, e si deve far sì che si riproduca sempre
uguale a sé stesso. Questo era il primo punto da tenere presente.

Non c'è la nozione dell'uguaglianza davanti alla legge, e non è
soltanto una costume, ma c'è scritto nella legge. L'ordine sociale è
legittimato dal dogma religioso e anche politico. I governi hanno una
legittimazione sacra, e si pensi al rito dell'incoronazione del re di
Francia che si teneva alla cattedrale di Reims, l'unzione con l'olio
santo, una \emph{reliquia} e si intendeva che intervenisse lo spirito
santo a sancire l'incoronazione.

Naturalmente il potere secolare è legittimato da Dio e ha il compito
di esercitare il potere per trattenere gli uomini dal compiere
peccato. Ha un mandato religioso e in questo esercita la
giurisdizione, fare giustizia; reato e peccato sono grosso modo la
stessa cosa. Gli uomini sono divisi in categorie definite dalla legge,
e possono essere tante quanto più articolata è la divisione del
laboro, ma fondamentalmente l'immagine ideale è quella tripartita
della società -- ed è quella che sarà convocata a Versailles nella
crisi del antico regime. L'idea fondamentale è che c'erano tre
funzioni e tre ruolo rappresentati dagli stati. I nobili, preposti
alla difesa della comunità contro i nemici esterni e al governo dei
sudditi, il clero, preposto anch'esso alla difesa della comunità
rinnovando l'alleanza col divino, e poi i sudditi, che dovevano
fornire le risorse necessarie alla riproduzione sociale.

Vale la pena soffermarsi sulla parola \emph{stato}, che all'epoca non
esisteva così com'è concepito oggi, ma era appunto la \emph{condizione
giuridica della persona}, \textquote{il mio stato}. Lo stato delle
persone eminenti si articolava in una serie di rapporti di dipendenza,
si parla di governo pastorale, vale a dire che quello che è importante
non è il controllo sul territorio, ma il controllo che il principe
esercita sulle persone, e attraverso le persone sulle cose. In altre
parole, adesso lo stato esercita le sue funzioni attraverso un
appartato burocratico con i funzionari, parola modernissima diffusa
dopo la rivoluzione francese. Questo significa che, chi lavora per
conto del potere politico, esercita una funzione impersonale di questo
ente morale e astratto che è lo stato. Nell'antico regime non esiste
il funzionario, ma il servitore. Questo significa ache il rapporto tra
chi da le ordini e chi le esegue è un rapporto personale, che può
essere anche messo in discussione. Significa che c'è uno scambio di
fondo, quello tra il prestare servizio in questa società incardinata
sul principio della disuguaglianza e quindi del privilegio dove la
legge di ciascuno è una legge di privilegio, che può essere
discriminatorio o lodevole, e anche la legge che impedisce agli ebrei
di possedere beni immobili è un privilegio rispetto alla norma, ma
discriminatorio. Scambio tra servizio e protezione, e anche concezione
di adagi e privilegi.

Le posizioni sociali sono e devono essere trasmissibili, e un altro
paradigma dell'organizzazione sociale dell'antico regime è la
famiglia, perché sulla riproduzione biologica viene strutturata la
riproduzione sociale, e questo giustifica l'importanza a livello
politico delle dinastie. Gli scontri militari nascono tutti da
problemi dinastici, le successioni ad una data posizione. 

1492 si rompe l'equilibrio politico in Italia con le morti di Lorenzo
il magnifico ed Innocenzo VIII. C'era una frammentazione immensa in
Europa, lo stesso regno di Francia frantumato fra i grandi feudi e
possessi del re. Queste entità politiche, però, si basavano appunto su
rapporti di dipendenza sulle persone, non sulla terra.

% Carta geografica frammentata può essere fuorviante.

In Italia non c'è grande unità politica perché è la sede del papa. Gli
strati cittadini però hanno cominciato ad avere un controllo
territoriale. È solo a partire dal Quattrocento... Il ducato di Milano
è sotto la famiglia Sforza, principi nuovi. Il duca di Milano non
discende da una grande famiglia aristocratica, ma d'un condottiere che
si è impadronito del ducato all'inizio del Quattrocento. È un secolo
di rotture veramente importanti. Due famiglie si avviano a dominare la
politica europea dell'età moderna, ossia gli Asburgo e la famiglia
reale di Francia. Massimiliano I eletto imperatore del sacro romano
impero. Esso si ritiene diretto successore dell'impero romano, ma
riesce a tenere insieme principalmente la comunità tedesca, fanno
parte dell'impero tantissimi principi, città indipendenti, stati
regionali, e quello che c'è di singolare è che queste due autorità
universali, il pontificato e l'impero sono le uniche che sfuggono,
almeno formalmente, al principio della successione famigliare. Il
pontefice è eletto da un conclave e l'imperatore da una Dieta
Imperiale, e tra questi membri solo sette hanno il potere di voto e
possono esprimere una volontà riguardo l'elezione del principe. Sono
quattro principi territoriale laici, e tre ecclesiali:
% Elenco.
L'impero sfugge alla legge dinastica, ma con Massimiliano, anche se
formalmente rimane così, anche l'impero si trasforma in un'entità che
viene sempre trasmessa all'interno della famiglia degli Asburgo, che
riusciranno sempre a prevalere con diversi strumenti, come la
possibilità di designare il successore dal re attuale, anche se deve
comunque passare per i sette elettori. Però da Massimiliano I al
crollo dell'impero nel 1806 l'impero sarà sempre controllato dalla
famiglia degli Asburgo, che ha possedimenti ereditari in Austria e nel
Tirolo, quest'ultimi particolarmente importanti perché c'erano le più
importanti miniere d'argento. Quindi gli Asburgo fornivano il mercato
europeo di metalli preziosi, e c'erano legami fra gli Asburgo e gli
altri personaggi importanti d'Europa. Senza fare una guerra e con
alleanze matrimoniali è riuscito a far ciò. Gli Asburgo, grazie ai
matrimoni, arrivarono a controllare mezza Europa.

Carlo V, restaurare l'impero con la forza che aveva all'epoca di Carlo
Magno, l'unità politica della cristianità...

Dall'altra parte la dinastia dei Valois. Carlo VIII, antagonista di
Carlo V, quello di ieri. Carlo ottavo ritiene di aver diritto al
sud'Italia perché un tempo i possedimenti appartenevano alla famiglia
degli... possedimenti pressi dai Valois. Stabilisce subito un'alleanza
con Ludovico il moro, reggente di Milano, e il duca tutolare è
promesso sposo alla figlia del re di Napoli. Lui vuole prendere il
posto del nipote ed impedire l'alleanza con Napoli, e per evitare
questo, apre le porte a Carlo VIII. A Firenze il figlio di Lorenzo il
Magnifico gli apre le porte. Naturalmente c'è la rivoluzione
sabonarola a Firenze.

Il papa tratta con Carlo, gli contenete il passaggio per lo stato
pontificio, prende Napoli, caccia gli aragonesi, se non che si forma
nel frattempo una coalizione antifrancese. Ludovico il moro ci ripensa
e anche gli altri principi. A Napoli poi scoppia la peste, e lui si
ritira per non essere imbottigliato. Scappano e vengono intercettati a
Fornovo. In realtà non ci fu un vero vincitore della battaglia, gli
italiani non riuscirono ad impedire la fuga. 

A Firenze, come detto, esplode la rivolta popolare guidata da un
frate... uno dei più eminenti prediatori del convento di San Marco a
Firenze, predicatori che si lamenta della decadenza della chiesa,
fiero avversario di Alessandro VI e la cultura umanistica che hanno
favorito... sono i primi sintomi di malessere per la condotta delle
alte gerarchie della chiesa. La rivoluzione di San Girolamo, dotto dal
punto di vista religioso, moralizzatore dei costumi. Abbastanza
singolare è che tra i suoi seguaci ci fosse Botticelli. Voleva anche
restaurare gli ordinamenti repubblicani compromessi dalla dinastia
medicea. Poi è sottomesso a processo con gli altri frati ed è finito
sul rogo. L'episodio ha un'importanza notevole perché è un segnale che
l'autorità religiosa è sempre più screditata, autorità religiosa che
nel frattempo pensa di dare un futuro alla sua famiglia. Vuole aiutare
il figlio Cesare voglia a ritagliarsi uno stato territoriale nel
centro della penisola. Luigi XII stringe l'alleanza con Venezia, non
rivendica Napoli ma Milano. Stringe anche un'alleanza con il papa.
Luigi entra a Milano, Ludovico il Moro e la moglie scappano. Nel
frattempo Cesare Borgia occupa Imola a spese di Caterina Sforza,
Forlì, Cessena, si impossessa della Romagna, occupa l'Elba e Piombino,
dica Valentino. Nello stesso tempo c'è da pensare al destino
dell'Italia meridionale. Luigi XII fa un trattato di spartizione col
re di Spagna, il trattato di Granada del 1500. La parte settentrionale
alla Francia, Sicilia, Calabria e Puglia alla Spagna, ma alla fine le
cose crollano, c'è guerra e gli spagnoli hanno la meglio nelle
battaglie di Cerignola e del Garigliano.

In questa guerra anche Venezia prende parte, e si impossessa di alcuni
approdi pugliesi e d'una buona parte della Romagna. Muore Alessandro
VI, quindi Cesare non ha più né la protezione del padre né l'alleanza
con i francesi, quindi i veneziani vanno nella Romagna e prendono
territori che appartenevano al pontefice. Viene eletto Giulio II,
grande nemico di papa Alessandro. È un papa diverso da Alessandro, ma
neppure lui un papa \bsq{evangelico}, è appunto il papa che scende in
battaglia in corazza. Mette insieme la lega di Cambrai, c'è la famosta
battaglia di Agnadello, una disfatta di Venezia che cerca scampo sulla
laguna, perso tutto lo stato di terra ferma. A questo punto però
Giulio si rende conto di aver esagerato, che voleva riprendere la
Romagna e ridimensionare il peso di Venezia, ma non cancellarla, è
utile al mantenimento dell'equilibrio in Italia. Toglie l'interdetto a
Venezia, e siccome Luigi XII occupa Bologna e Mirandola.

% Giulio escluso dal regno dei cieli.

Il re Luigi convoca un concilio a Pisa, il \bsq{conciliabolo},
tentativo di bilanciare la politica conciliare del Quattrocento, tutta
la storia ecclesiastica del Quattrocento è uno scontro tra
conciliarismo e l'ipotesi opposta del potere autocratico del
pontefice. Luigi XII cerca di animare l'anima conciliarista del secolo
precedente ma non ci riesce. Giulio organizza una lega contro i
francesi, scontro papa e Re di Francia, quest'ultimo che cerca in
tutti i modi di ridimensionare la situazione. Lega santa... In
particolare Massimiliano d'Asburgo si rende a cuore di portare al
trono di Milano l'erede Sforza. Ma poi alla morte dell'ultimo Sforza
il ducato rimane in mani asburgici, muore Giulio II, i francesi
tornano all'offensiva e occupano il ducato di Milano. Tutte queste
guerre mirano dal punto di vista strategico ad occupare il centro
della pianura padana, parte strategica. All'epoca l'ultimo Sforza
aveva stretto un'alleanza importante con i cantoni svizzeri che gli
fornivano il grosso dell'esercito. Battaglia di Melegnano 1515.

Leone X tratta con il re di Francia, il concordato di Bologna, in
pratica il pontefice accorda al re autorità sull'organizzazione
interna della Chiesa di Francia. Si è agli origini dell'autonomia
della Chiesa di Francia. Nel frattempo sono morti sia Luigi XII che
Massimiliano, c'era già Carlo, e questa fase si chiude nel 1516 con la
pace di Noyon, che stabilisce il predominio francese sull'Italia, che
però verrà rapidamente capovolto.

\end{document}
