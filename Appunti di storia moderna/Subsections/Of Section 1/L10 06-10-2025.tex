% !TEX root = ..\..\main.tex
\documentclass[../../main.tex]{subfiles}%
\ifSubfilesClassLoaded{\addbibresource{../../bibliography.bib}}{}

\begin{document}

Il rapporto tra l'andamento dei prezzi e la quantità di moneta in
circolazione è vero, però le indagini della storiografia novecentesca
per valutare l'ipotesi non hanno dato una conferma univoca. Per vedere
se c'era una correlazione tra l'andamento di prezzi e la crescita del
metallo bisogna sapere quanto arrivò in Europa. Bastava guardare i
registri dei registri del metallo che entrava in Europa, e quando si è
fatta questa ricerca si è resi conto che l'importazione di metallo
inizia a seconda metà del Cinquecento, mentre invece la crescita di
prezzi è anteriore. Quindi sebbene sia collegata all'importazione del
metallo prezioso, ci sono altri fattori che contribuiscono a questo.
Nel Cinquecento c'è anche una crescita molto cospicua della
popolazione, e quindi il fenomeno inflativo innescato dall'aumento
della domanda, poi si è anche rinforzato con l'importazione dei
metalli.

Questo comporta una trasformazione importante dell'economia,
naturalmente ci sono più mezzi di scambio sul mercato, e tutto questo
favorisce la crescita dell'economia, che è poi un tratto fondamentale
dell'età moderna e oltre.

L'altra cosa che si è saltata è il tema della riforma, che è
naturalmente fondamentale, ma prima di parlare di Lutero e del
malessere della cristianità c'era forse da dire qualcosa della
narrazione che si è fatta delle guerre in Italia.

La penisola italiana perde l'indipendenza politica con la Francia e la
Spagna, ed è scossa anche da profondi rivolgimenti di natura
costituzionale perché. Basta considerare il caso emblematico di
Firenze, dove si passa da una costituzione principalmente repubblicana
a un principato -- anche se è connotato da elementi di repubblica
forti. Fra questi mutamenti erano stati oggetto di riflessione di
diversi autori, primo di tutti Machiavelli, che non è soltanto uno che
osserva e cerca di interpretare i fatti dell'epoca, ma apre una nuova
strada politica. Lui scrive il \textit{Principe} che si iscrive a
prima vista in una tradizione diffusa al tempo, è un genere
(\textlatin{\textit{speculum principia}}) di precettistica destinato a
chi dovrà ricoprire funzioni di governo. Ma queste prescrizioni,
consigli e considerazioni, normalmente erano volte a far sì che il
principe governasse \bsq{cristianamente}, aderendo appunto al modello
del libro. Era sostanzialmente precettistica morale e trattava delle
virtù e i vizi del governante. Il manuale di Machiavelli, invece
stabilisce l'autonomia della politica dalla morale religiosa; la
politica ha regole proprie, e la virtù del principe non è morale ma
\emph{tecnica}, l'abilità del principe di cooptare, mantenere ed
ampliare il potere. Questo, naturalmente, inserisce il
\textit{Principe} nell'indice dei libri proibiti, e soprattutto era
proibito durante la controriforma. Si è agli esorti del realismo
politico, e bisogna pensarla non per quello che si vorrebbe che
l'uomo fosse o com'è descritto nell'ideale religioso, ma si deve
riflettere da un'antropologia realistica, l'uomo così com'è. Lui pensa
che questi valori abbiano un valore universale, e ci arriva dopo aver
pensato agli autori classici, soprattutto Tito Livio.

% Citazione, dove io non mi vergogno parlare con loro e domandare, e
% quelli per loro umanità rispondono... E non sento alcuna noia, non
% tempo la povertà... Machiavelli... La cosa del lupo

Lui riflette anche sulla fortuna, il caso, l'imprevedibile, in un
senso estraneo alla provvidenza cristiana.
% Citazione. Machiavelli parla di Giulio II.
Questo, dal punto di vista del realismo politico è una novità, che
tuttavia è basata su un esame meticoloso delle opere antiche.
Machiavelli pensava che la politica avesse regole universali. Tutto
questo cambia quando poi già i politici sono fatti e subentra un
atteggiamento rassegnato. La repubblica non c'è più, e diventa una
politica cortigiana, di adulazione al principe. E uno che registra
questo è Guicciardini, per cui invece non ci sono regole, mentre invece
ci sono eccezioni per la varietà delle circostanze, che non si possono
misurare nella stessa maniera, e queste cose non si trovano scritte
nei libri, ma bisogna avere la discrezione, che è l nuova virtù del
mondo cortigiano, saper di volta in volta qual'è la scelta più
vantaggiosa senza affidarsi a regole generali, e comunque bisogna
avere \emph{buona} fortuna (che in Machiavelli era invece sostituibile
col caso).



L'ETÀ DELLA RIFORMA 1517-1559

% Cartina confessionale dell'Europa.

La rottura dell'unità d'Europa viene nell'epoca dell'imperatore Carlo
V. Quello che c'è da capire è che la comunità dei fedeli avvertiva una
totale inadeguatezza della capacità pastorale del clero, e soprattutto
della curia. Si pensi ad Alessandro VI o Giulio II, troppo immischiati
nelle vicende politiche per occuparsi della salvezza dei fedeli, e il
malessere non si limitava ai vertici della chiesa, ma anche
persino ai parrochi. C'era da tantissimo tempo soprattutto in Italia e
in Germania un sentimento anticlericale, anche se non arrivò ad
elaborare un'alternativa, ma si trattava della riforma della chiesa. E
la discussione era se si doveva cominciare dalla testa o da giù, e
tutti quelli avevano nostalgia della semplicità delle comunità dei
primi cristiani, immaginando che all'epoca gli insegnamenti fossero
incarnati in quella semplicità, e poi c'è stata una degenerazione. Tra
le tante voci autorevoli perché la chiesa fosse riformata, la prima è
quella di Erasmo da Rotterdam, grande intellettuale ed umanista. È lui
che fece un'opera di traduzione del vangelo nuova, per emmendarla da
tutti gli errori di quella di San Girolamo, è un vangelo scritto con
un latino elegante, umanistico, e poi l'insegnamento suo, quello di
lasciare da parte questioni liturgiche e così via, concentrandosi sul
nucleo dell'insegnamento di cristo, cioè precetti morali. Erasmo è
all'origine di una corrente importante all'interno del cristianesimo,
critico, ma non favorevole a creare una nuova chiesa.

Il personaggio che invece produsse una rottura è Lutero, che era una
persona completamente diversa. Lutero aveva studiato teologia a
Wittenberg, ed era un uomo medievale, anche se le conseguenze delle
sue scelte avrebbero innescato dinamiche di trasformazione non solo
del culto ma anche della società. Ora, la questione della salvezza:
come ci si guadagna la vita futura? Non era convinto che ci si potesse
salvare facendo delle opere buone perché tutto questo entrava in
profonda contraddizione con l'idea che aveva di Dio, che era
incommensurabilità. L'uomo è il peccatore assoluto, e che meriti
potrebbe avere rispetto Dio? Lui si appigliò alla dichiarazione di San
Paolo: delego alla fede la questione della salvezza, della promessa
cristiana di una vita dopo la morte. Non è possibile cambiare le
scelte e le buone opere non sono un merito. Il cristiano non è libero,
ma sono il frutto della fede, e queste sono le meditazioni di Lutero
che hanno esploso lo scandalo.

Per quanto riguarda invece le premesse politiche, si pensi ad Alberto
di Hohenzollern, che voleva essere vescovo di Magonza per essere
principe elettore. Massimiliano d'Asburgo era alla fine dei suoi
giorni, e presto ci starebbe stata la dieta per eleggere il nuovo
imperatore. L'elezione di Carlo d'Asburgo che era già re di Spagna,
abbia costato un sacco... La corruzione era una cosa sempre presente,
e anche Alberto che riuscirà a diventare vescovo, potrà arricchirsi in
quel passaggio. Gli Asburgo già prima di Carlo V avevano rapporto con
delle banche importanti, e anche avevano concesso ad essi lo
sfruttamento delle miniere del Tirolo, principale zona mineraria prima
delle scoperte in Perù. Alberto, dunque, vuole per forza diventare
arcivescovo di Magonza, e va dal papa, ma il papa Leone X, figlio di
Lorenzo di Magnifico ha un sacco di spesse -- sta facendo la basilica
di San Pietro. IL papa lo indusse a fare un prestito e li fa una
patente per vendere le indulgenze, ossia sconti di pene nel
purgatorio. Il papa poteva esonerare i fedeli da anni del purgatorio
-- dottrina del purgatorio che viene elaborata nel medioevo --. I
santi avevano fatto tanto bene che eccedevano quello necessario al
regno dei cieli, quindi che fare con tutti questi meriti in eccesso?
Rimassero a disposizione della chiesa di Roma, e a disposizione dei
fedeli. La chiesa era dunque una sorta di banca dei meriti dei
cristiani, e poteva rimetterli in ciclo tramite le indulgenze.

Questo, nel clima di una società non completamente monetizzata, veniva
dato alla comunità e così via. Poi l'economia monetizza, e in questo
modo papa Leone pensava che Alberto potesse rientrare nel debito. John
Eck o qualcosa, vendeva le indulgenze in piazza. Uno si può immaginare
la reazione di una testa quadrata ingarbugliata nella questione della
salvezza come quella di Lutero che assiste a questo spettacolo. Scrive
le famose 95 tesi, che circolarono tra i teologi piuttosto che fra il
\bsq{popolo tedesco}, e naturalmente era una condanna senza appello
non solo all'indulgenza, ma anche alla dottrina cattolica della
salvezza che mette al centro le buone opere. Ma è precisamente questo
che Lutero non riusciva a mandar giù. Quando si diffondono le tesi
Lutero si guadagna subito la scomunica, ma vista da Roma non sembrava
che avesse quelle conseguenze. Inoltre a quel tempo Massimiliano era
morto, e la dieta elegge Carlo V. E con Carlo imperatore, che è già re
di Spagna, che ha possesi di oltreoceano, si crea una superpotenza
politica che ambirà a ristorare l'impero com'era nell'antichità.

Superata la questione dell'elezione dell'imperatore, la critica di
Lutero era caduta in un contesto politico e sociale in cui la Germania
era scontenta per vari motivi. Era scontenta con Roma perché era
fondamentalmente la parte fondamentale della cristianità che pagava --
dal momento che i re di Francia e di Spagna avevano fatto accordi per
diminuire l'aggravio fiscale sui loro paesi -- ma poi questo
malcontento nei confronti del fisco ecclesiastico e della poca
credibilità della chiesa si legava con l'insoddisfazione di una parte
della società tedesca che era la vecchia cavalleria medievale.
L'impero, come s'è detto, era un'entità composita di cui facevano
parte stati regionali piuttosto grandi come la Sassonia, il
Palatinato, ecc., principati ecclesiastici, e città libere e
mercantili molto ricche come Francoforte o Hamburg -- che diversamente
alle città italiane non avevano un territorio. E poi c'era la piccola
nobiltà feudale che viveva dall'arte della guerra, ma era sta
marginalizzata dal rafforzamento dei principi territoriali, e quindi
si davano spesso ad attività di brigantaggio, e così via, spesso a
spese della chiesa. Perciò i primi che aderivano all'idea di una
riforma tedesca della chiesa furono esponenti di questo mondo
nobiliare che erano artefici di quella che si chiamava l'\bsq{anarchia
germanica}, e i capi di questo movimento erano da una parte Franz Von
Sickingen e Hulrich Hutten, un umanista. Poi furono sconfitti in un
assalto a Colonia (?). Nello stesso tempo Lutero si sta dando da fare
pubblicando opere assai provocatorie. \textit{Alla nobiltà cristiana
di nazione tedesca sulla riforma della società cristiana} Lutero pensa
che sia la nobiltà il motore politico, e poi vedremo che la riforma si
attuerà quando farà accordi con i signori territoriali. Siamo a inizio
Cinquecento e l'arte della stampa si sta diffondendo, e questa riforma
senza il libro stampato non ci sarebbe stata, è anche la conseguenza
di un cambio della comunicazione sociale.

% Altre opere del 1520 di Lutero.

Lutero è scomunicato e Carlo V promette...(?) Però Lutero ha anche un
protettore. Lutero doveva ritrattare tutto quello che ha detto, sia
riguardo le indulgenze, il valore delle opere, equiparare il papa
all'anticristo, e però Lutero ha la testa dura e si aspettava che
fosse placato davanti a Carlo V, e invece di fronte a lui dice,
% Citazione power point
c'è un'autorità più alta che è la coscienza. Alcuni hanno visto in
questo appellarsi alla coscienza un elemento di modernità da parte id
Lutero, e l'inizio di una religiosità non più comunitaria ma
individualistica.






\end{document}
