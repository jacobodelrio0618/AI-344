% !TEX root = ..\..\main.tex
\documentclass[../../main.tex]{subfiles}%
\ifSubfilesClassLoaded{\addbibresource{../../bibliography.bib}}{}

\begin{document}

\subsection{Heading}

% Francois Guizot.

% Regime di storicità, linea verso il meglio anziché linea verso la
% catastrofe del mondo cristiano medievale. Incisione mele catacombe,
% alpha e omega. Inizio e la fine di tutto e nel mezzo Cristo che
% porta un messaggio di salvezza e novità.

% Storia sacra aveva delineato tutto ciò da dire della presenza
% dell'uomo nel mondo. Rimediare il peccato originale che ha gettato
% sulle spalle dell'umanità un debito infinito.

% Nel medioevo si scrivono cronache, e quello che accade va ricondotto
% al quadro concettuale già stabilito della storia della salvezza, che
% è il senso della storia.

% Sentimento di debito con Dio, essere condotto al peggio, alla
% corruzione si ripercuote nella cosmologia. 

% L'uomo immaginava d'essere al centro dell'universo. Cosmologia di
% Aristotele. Impatto delle esplorazioni geografiche. Movimento è una
% mancanza di perfezione.

% Gli uomini non avevano la percezione di poter dominare queste forze
% esterne.

% Regime di storicità implicito nella storia sacra, espresso in
% maniera esplicita in uno dei due libri storici della bibbia, quelli
% di Daniele e l'apocalisse di Giovanni, dove si mostra come tutto è
% iniziato e come finirà. La fine dei tempi inizia con la predicazione
% dell'anticristo. Poi fine mondo. Dio colpisce gli uomini e si apre
% il tempo del giudizio in cui tutta l'umanità risusciterà con le
% proprie carni. Dopodiché la divisione tra i giusti e gli altri. In
% questa costruzione della storia umanità era inclusa anche la storia
% precedente alla venuta di Cristo, ed era riletta finalisticamente
% entro le traiettorie escatologiche cristiane, inclusa la storia
% delle grandi civiltà prima del cristianesimo. Cronache più antiche,
% immagine, cronologia sinottica (Eusebio di Cesarea, cronaca
% risistemata da san Girolamo).

Succedono cose che vanno incontro alle cose della storia sacra.
Accadono le esplorazioni geografiche, in primo luogo. C'era un
continente intero fra l'Europa e l'Asia, e tutto questo è
sconcertante perché non c'è nella bibbia. Una delle tante risposte era
di dare una giustificazione alla presenza di queste popolazioni con il
racconto dell'antico testamento. 12esima tribù di Israele.
% TODO Cfr. I segni del tempo.
La Bibbia dà indicazioni precise sull'epoca della creazione. 4000 anni
della storia dell'umanità.
James Ussher datazione della creazione, teologo anglicano, vissuto
il regno di Cromwell, contemporaneo di Galileo.
Comunque, fino ad allora, nessuno si era occupato di computare il
tempo della creazione, e questo è già un segno del cambiamento di
mentalità.

Nuovo modo di guardare al mondo, e quindi anche alla \emph{storia del
mondo}, il progresso scientifico si fa anche dai guardiani della fede.
Età che comincia a porsi dei dubbi, farsi domande e progettarsi
risposte. Matto di Kircher. Gesuiti, più importante ordine dell'età
moderna. Prospettiva di dotarsi di una milizia intellettuale per
confutare i riformatori. Intellettuali organici del papato. Insegnava
al collegio romano. \textit{Arca Noè}. Come aveva fatto Noè
praticamente. \textit{Discorso sulla storia universale}, Bossuet,
inizia la narrazione nel 4004. 1660s-70, Luigi XIV, piano di egemonia
su tutto il continente, l'età classica, storia francese. Anche il
periodo in cui non solo la Francia sul piano politico si impone, ma
anche sul piano culturale, e per quello Bossuet è importante. Nel
Settecento la lingua di scambio internazionale diventa il francese.
Frontespizio, antiporte. Spirito che ispira l'autore della storia
universale, avvenimenti nella finestra. Lo storico quando parla degli
avvenimenti del mondo non guarda alla finestra, ma se li fa raccontare
dall'angelo. La storia va scritta ispirata religiosamente, altrimenti
non si capisce. Allegoria comune della tradizione iconografica, con
riferimento in particolar modo agli evangelisti.
% San Matteo, Caravaggio, Matteo il tramite umano per fare arrivare il
% messaggio.
Il più antico degli storici è Mosè.

Spinoza, \textit{Tractatus}. Si mettono in discussione le verità
ricevute, su cui ci sono state già divisioni, epoca della riforma,
ricezione della verità. Propone una nuova interpretazione della
verità, guerra dei trent'anni. Dividersi per l'interpretazione
dell'eucaristia. 

Guerra dei trent'anni nasce come una guerra di religione, ma si chiude
in una guerra tra potenze squisitamente politica legata
all'equilibrio di poteri in Europa. Una guerra che inizia per la
verità di fede finisce per dar luogo a un panorama politico
completamente diverso. Col trattato di Westfalia inizia la politica
dell'equilibrio di potenze. A inizio Settecento, nella crisi della
coscienza europea si affacciano i cosiddetti \emph{libertini}, fra cui
gli spinozisti. Nel \textit{Trattato} c'è una cosa corrosiva. Egli
legge la Bibbia e si accorge che è piena di contraddizioni e arriva
alla conclusione che è stata scritta da un uomo e non ispirata.
Arriva a stabilire attraverso l'analisi e la critica, filologia sulla
Bibbia, arriva a dire che erano due libri diversi... 

Richard Simon, contestazione di Spinoza, ma non si appella alla fede,
ma vuole dimostrare anche lui con la filologia che Spinoza ha
sbagliato. Si mette sullo stesso terreno di Spinoza. La cosa
significativa è che c'è un cedimento, che il terreno del dibattito è
quello della critica. Intervenne anche Bossuet che critica Simon.
Queste sono verità di fede che non vanno discusse. Acta Sanctorum,
filologia diffusa negli ambienti cattolici.

Pierre Bayle. Dibattiti per via epistolare, redattore del giornale.
\textit{Pensieri diversi sulla cometa}. Sono gli anni in cui in Europa
passa la cometa di Halley. Smontare le superstizioni sulle apparizioni
di astri. Comete considerate come segni divini oppure funesti. Lo
scandalo se l'ateo può essere virtuoso. Fontanelle, \textit{Dialoghi
sulla pluralità dei mondi}.
% Saltano i brunisti.

Newton. Rivoluzione sconcertante, costanza delle leggi, dei fenomeni
della natura comincia a far maturare nella testa degli scienziati
l'idea che non solo si possono scoprire i segreti della natura, ma
anche prevedere e così via, e allora se siamo arrivati a questo
risultato, gli europei che sono vissuti nella venerazione del
principio di autorità, il tema della storicità, del peccato
originale, ecc. Riabilitazione degli antichi in un discorso cristiano
umanesimo, più vicini della fonte della verità. Non più nani sulle
spalle di giganti. Siamo qualcosa di diverso dagli antichi, siamo noi
diventati giganti. Si scatena un dibattito fra i letterati su questo.
Fontanelle sente bisogno di prendere posizione di questo dibattito, e
si schiera con i moderni.
% Citazione inizio libro Fontanelle. Le leggi della natura sono le
% stesse per tutte le epoche, non cambiano. Provocazione degli alberi,
% se gli antichi fossero stati più intelligenti di noi...



\end{document}
