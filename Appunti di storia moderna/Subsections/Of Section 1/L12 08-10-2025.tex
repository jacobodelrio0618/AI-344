% !TEX root = ..\..\main.tex
\documentclass[../../main.tex]{subfiles}%
\ifSubfilesClassLoaded{\addbibresource{../../bibliography.bib}}{}

\begin{document}

% Munster 1535.

Per altro, senza attendere l'avvento del regno dei cieli e la fine del
mondo, ma dare realizzazione concreta, qui ed ora, della Gerusalemme
celeste. Uno sperimento ben significativo, ma con conseguenze tragiche
è Munster, che una volta era un piccolo principato ecclesiastico, e
che fu riformata da predicatori anabattisti. Molti predicatori
provenienti dai paesi bassi, la loro ispirazione erano gli atti degli
apostoli, dove potevano leggere che i cristiani vivevano tra fratelli,
dividevano i loro beni. Quella di Munster esperimento di riforma
comunistiche, teocratiche, un comunismo fondato sul precetto
religiose, che arrivò a teorizzare la comunione delle donne, l'amore
libero (?). Molti aderirono alla riforma anabattista, poi in seguito
prese l'aspetto di una rivolta violenta e nel 1535 quando comincia a
cambiare il vento -- si è vista la guerra svizzera, che si fanno
alleanze per schiacciare i ribelli anabattisti, e subirono una fine
terribile. Si chiudeva così l'esperienza più radicale della riforma
religiosa, che nel frattempo però si andava espandendo in altri angoli
d'Europa alla periferia, sia in Inghilterra con una soluzione molto
popolare, sia nei paesi scandinavi, che erano riuniti dal
quattordicesimo secolo in una confederazione sotto la guida della
corona danese. La sostanza è che questi regni periferici e poco
popolati al confine dell'Europa stanno attraversando lo stesso
processo che interessava ai grandi regni al centro del continente,
specialmente la Francia e la Spagna, un processo politico di
accentramento del potere nella figura del monarca, che ottiene una
predominanza politica sugli altri soggetti, sulla parte guerriera dei
piccoli nobili, che era sempre stata una spinta contro l'accentramento
del potere, ed erano le città e le borghesie a fornire le capacità
economiche e sociali per ciò, e questo è un dato comune alla
formazione delle monarchie moderne in Europa, e acquistano tanto più
potere quando più mettono in conflitto il mondo urbano e le
aristocrazie rurali, è in questa dialettica che si rafforzano le
monarchie. Negli anni Quaranta la predicazione luterana arrivò in
Svezia tramite (?), e questi monarchi avevano vantaggi rispetto agli
altri d'Europa perché (?) risorse, sostenere le riforme che andassero
in vantaggio dei ceti sociali che potevano contrastare il dominio
aristocratico.

La Bibbia viene tradotta in svedese nel 1544, il luteranesimo quindi
is diffonde e si radica in tutta quest'area settentrionale mantenendo
l'unità dei suoi fondamenti.

Diverso è il caso inglese, dove nasce la riforma per una questione
politica. È una riforma che si tratta più che altro di uno scisma, una
divisione da Roma, senza però introdurre novità dal punto di vista
della dottrina e della fede. Le circostanze che portano Enrico VIII a
promulgare il cosiddetto atto di supremazia nel 1534 sono legate a
problemi dinastici, di sopravvivenza della dinastia, anche se ci sono
altre motivazioni che spingono a questo. La questione dinastica è però
fondamentale. Enrico aveva sposato Caterina d'Aragona, che era la zia
dell'Imperatore, però lei non sembrava in grado di dargli un erede
maschio che Enrico voleva a tutti i costi. E quindi, volendo impalmare
alla sua nuova passione, Ana Bolena, non poteva che chiedere lo
scioglimento, però il papa, che normalmente non facevano nessuna
difficoltà ad accordare lo scioglimento specialmente quando venivano
dai re e così via, ma in questo caso Clemente VII non può perché è
praticamente ostaggio di Carlo V, ed è lui che si oppone a che Enrico
VIII si divida da Caterina d'Aragona per mantenere l'influenza sul
trono inglese. Era talmente importante questa cosa della continuità
dinastica che Enrico VIII non ebbe esitazione, e scelse la strada
dello scisma. La monarchia si assume il compito di governare la chiesa
d'Inghilterra. Non interferisce nelle questioni spirituali -- il capo
spirituale è l'arcivescovo di Canterbury -- ma si tratta di scisma
senza eresia, la dottrina e la liturgia rimangono. L'unica differenza
è che il capo della chiesa non è il papa ma Enrico VIII. Ma ci sono
altre conseguenze a livello pratico: anche qui si arriva allo
scioglimento degli ordini monastici, e anche questo diventa un punto
di forza della monarchia inglese perché Enrico VIII potrà impadronire
del patrimonio delle ordini, e rendersi in qualche modo più
indipendente dalle aristocrazie e le città, ha un margine di manovra
maggiore e ciò avrà ripercussione anche sulle tasse ai sudditi.

Nel frattempo Carlo V è costretto a rivolgere la sua attenzione
all'area mediterranea e a fronteggiare le offensive dell'Impero
ottomano, e si ricorda, per altro, che l'Impero ottomano ha siglieta
nel 1526... a preoccupare in questa fase sono soprattutto
preoccupanti le attività dei corsari, soprattutto Barbarossa. Carlo si
impegna dunque in una campagna navale per ripulire il mediterraneo da
questi corsari e per riconquistare Tunisi, appunto conquistata da
Barbarossa. L'Impero ottomano all'epoca aveva avamposti dove
l'autorità del sultano non veniva gestita con la stessa forza, ma
c'era sempre un rappresentante. Il sultano era piuttosto un'autorità
nominata. I fratelli di Barbarossa erano pirati molto feroci e si sono
stabiliti prima del 1519 in quella che era diventata la città di (?),
minacciando in particolare il traffico di merci, ed entrarono in
contrasto particolarmente con la Spagna. Temendo di essere travolti
dall'offensiva spagnola Barbarossa chiese aiuto al sultano, che gli
mandò delle navi, ed è in questo modo che si creano delle provincie
che non sono direttamente sotto l'autorità del sultano, ma sono legate
all'Impero. Il reclutamento dei corsari, tutti in origine cristiani,
istituto dell'ordinamento turco. La cosa è stupefacente per gli
europei, era un'altra forma di civilizzazione basata su principi per
molti aspetti diversi e provocatori. Nel mondo ottomano, ad esempio,
il figlio di un contadino albanese poteva diventare braccio destro del
sultano, e la gran parte di questi ragazzi veniva mandata direttamente
in penisola anatolica ad istruirsi nella legge islamica e fare il
contadino; quello più dotati prendevano la via del palazzo di Topkapi,
per diventare dei funzionari del sultano. Tra l'altro si faceva
carriera esclusivamente sull'anzianità, quindi tutti potevano prima o
poi ricoprire cariche importanti, quindi un modello sociale in qualche
modo opposto a quello europeo, dove non c'era praticamente mobilità
sociale -- e c'era sul presupposto, per dispotismo, tutti potevano far
carriera per meriti, ecc., però erano considerati allo stesso modo
schiavi del sultano. Era un sistema sociale strano agli aocchi dei
cristiani, e l'altra cosa è che questo impero musulmano in realtà era
un impero multireligioso, e anche se c'era la predominanza dell'isalm
su tutte le altre religioni, c'erano considerevoli quantità di sudditi
del sultano che erano ebrei, cristiani, ecc., il cui unico obbligo era
quello di pagare una tassa al sultano. C'era dunque la tolleranza di
altri culti, cosa impossibile nel mondo cristiano, che erano anche
soggetti a questa cosa della raccolta dei bambini. Tanto per
cominciare erano le famiglie povere e numerose che davano il figlio al
sultano, e poi per loro era una fortuna, un modo per fare uscire il
figlio dalla miseria, eppoi chissà, magari diventa grande. In ogni
caso Carlo V riconquista Tunisi e ritorna in Europa in un viaggio
trionfale attraverso in Italia, dove da lì a poco...

Alla morte dell'ultimo duca di Milano entrava direttamente su
controllo della corona spagnola. Perciò alla morte di Francesco, per
sventare questa possibilità, l'altro Francesco di Francia attacca, ed
è una campagna militare che non porta grandi cambiamenti e nel 1538 si
raggiunge una tregua armante mediata dal nuovo pontefice, Paolo III,
anche investito del compito di promuovere, come aveva desiderato
inizialmente Carlo V, questo dialogo con i protestanti, per curare
questo scisma. Il pontefice però è molto sospettoso, e c'è il
concilio.

Nuova campagna di Carlo V nel mediterraneo diretta ad Algeri,
fallimento, non riesce a sbarcare ed è in questo momento che Algeri
guadagnerà la fama di città invincibile. Dall'altro lato ci sono gli
sforzi per arrivare a una ricomposizione della fede. Colloqui di
Ratisbona 1541, ultimo tentativo dove il papa manda al cardinale più
conciliante che abbia sotto mano, Contadini, che si incontra con
Melantone e con gli altri capi della riforma luterana per discutere
sempre la stessa cosa, la dottrina della salvezza e la preponderanza
delle opere. Bisognava trovare una conciliazione o mediazione tra due
istanze che non sono in realtà conciliabili, e di fatto il colloqui
di Ratisbona non arriva a nessun risultato, ed è ormai chiaro che il
confronto sarà violento e militare, e ci si attrezza in tutte le sedi.
È qui che viene fondata la compagnia di Gesù, uno dei grandi organi
della controriforma, nasce prima del concilio di Trento, ma già lo
spirito combattivo si vedeva. Il fondatore Ignazio di Loyola era un
militare che ebbe una conversione religiosa durante la convalescenza
per una ferita che aveva riportato. Andò a studiare teologia alla
Sorbona di Parigi, e lì si trovava una cappellina dove facevano
riunioni, lì si fonda il primo gruppo della compagnia, che diventerà
una potenza, e lo scopo è militare. Gli strumenti della compagnia sono
la predicazione, però è un ordine religioso particolare perché ha
caratteristiche particolari. Come in qualsiasi ordine si pronunciano i
voti, ma la compagnia è anche una struttura gerarchica, in qualche
modo anche con qualcosa di esoterico perché all'ultimo grado si
pronuncia il quarto voto, che è quello di obbedienza al papa fino alla
morte. Milizia compatta e gerarchica che si pone come fine la
riconquista dei popoli caduti in preda dell'eresia o di popoli che non
hanno ancora riconosciuto il messaggio di salvezza di cristo. Quindi
le missioni saranno sia in terra tedesca, e anche nelle nuove terre
che venivano scoperte e in India e Cina. Ma mentre da una parte ci si
attrezza per lo scontro, dall'altra si sta parlando dei calvinisti.
Giovanni Calvino studia anche lui alla Sorbona, una delle più
importanti e le migliori per studiare teologia, dove aveva studiato
Tommaso, e paradossalmente è un compagno di studi di Loyola. Il padre
era il capo della cattedrale e sperava che il suo figlio lo
sostituisse, che fosse l'avvocato, e l'aveva mandato perché studiasse
giurisprudenza, se non che aveva un interesse particolare per le
questioni teologiche, e quindi il padre lo lasciò fare la facoltà di
teologia, perché poi la possibilità di inserire il figlio come
avvocato era sfumata. La Francia in questo periodo sembra non fosse
ancora toccata dalla predicazione della riforma, ma questo anche
perché il re di Francia ha tutto l'interesse a tenere un basso profilo
e un atteggiamento tollerante. Non può però dimenticare che nella
sfida con Carlo V i luterani sono i suoi alleati, com'è suo alleato il
sultano. Quindi arriva a mettere come rettore della Sorbona un
fanatico cattolico, perché temeva che infiammasse la discussione, se
non che quest'atteggiamento di tolleranza ipocrita non fece altro che
favorire la diffusione delle nuove dottrine, anzi, il sostituto del
rettore fece un discorso che fece raggirare gli ortodossi per
l'apertura che faceva per la dottrina della grazia e tutto ciò.
L'affare che portò in piena luce quanto i protestanti si fossero
infiltrati in Francia erano quelli di (?). Il re aveva programmato un
incontro con il pontefici, quando a Parigi cominciano ad apparire
placard, manifesti contro il papa. La cosa più importante è che non
comparirono sui quartieri popolari, ma nella porta della camera da
letto del re, il cui indica che ci sono proprio dentro, comincia una
cacciata e Calvino si rifugia in Svizzera, aveva un amico (?), e
spinse per allineare la chiesa di Svizzera alla chiesa evangelica, e
chiede una mano a Calvino. Il primo tentativo va a vuoto, Francesco ha
la meglio, e però al secondo tentativo l'arcivescovo fu cacciato e la
chiesa riformata secondo i dettami di Calvino. \textit{Istituzione
della religione cristiana}, tradotta in tutte le lingue, grande
diffusione delle dottrine calvinistiche. Ginevra diventa un centro di
irradiazione della nuova dottrina.

Che differenza c'è tra Calvino e Lutero? Le opere non servono per
salvarsi, e se uno le compie è perché è già salvo. L'uomo non può fare
alcunché per decidere il suo destino. Dare libertà all'uomo significa
annichilire l'onniscienza di Dio, sono principi contraddittori, ma è
un aporia che ci si porta dietro dai padri della chiesa. Agostino
contro Pelagio, Lutero agostiniano. Teorizza la predestinazione degli
uomini Calvino. L'uomo non condiziona la salvezza, Dio non può
ricredersi e il suo giudizio è perfetto, Dio ha già predisposto tutto
dall'inizio dei tempi e sa il nostro destino, ed ha già diviso
l'umanità tra il gruppo degli eletti e i condannati. C'è la tentazione
di un atteggiamento fatalistico, ma non è così, piuttosto si traduce
in un attivismo incredibile, non solo nella pratica religiosa, il che
non avrebbe combinato grandi cambiamenti, ma nella vita civile, che
viene santificata e coincide con la pratica religiosa. Perché? Calvino
insegna che siamo predestinati, e di questa predestinazione l'uomo in
realtà non sa niente, ma è una predestinazione di cui può rilevare
segni nella sua vita terrena. Se Dio ha deciso che sei un eletto il
peccato non ti mortificherà, anzi, tu sei integralmente eletto, e
quindi se riesci nelle cose che fai, nel tuo mestiere, nelle tue
relazioni, se hai successo questo è un segno di appartenenza a chi è
stato eletto. È anche un modo per spingere i cristiani a sforzarsi
nella cose civile. Max Weber ha letto la cultura protestante e lo
spirito del capitalismo. Egli mise in relazione la morale, e in
effetti c'è una certa correlazione fra le aree calvinistiche e quelle
che si sono schierate in nuove forme economiche basate sul capitale.
Questo però non significa che le cause dell'evoluzione economica
siano morali. Weber voleva dire, più molestamente che quest'etica
protestante è confacente a uno sviluppo le cui cause sono materiali ed
economiche. Trasforma la città in una sorta di grande monastero, una
specie di campus ecclesiastico, passando attraverso un concetto,
quello di \emph{vocazione} che diventa strategico. Era
tradizionalmente la chiamata alla vita religiosa, a lasciare tutto per
Dio, la vita ascetica, l'abbandono per la vita materiale. Anche per
Calvino ci sono le vocazioni e non ce n'è una sola, ma piuttosto una
distinta per ogni uomo e per ogni ruolo sociale, e ogni ruolo sociale
è santificato in questo, nella misura in cui uno compier il suo
dovere. In questo modo si finisce per legittimare e anche congelare
la divisione del lavoro e delle gerarchie sociali, ma c'è uno sprone
all'impegno e all'etica del lavoro, che è questo sì veramente
confacente allo sviluppo dell'economia capitalistica. Queste vocazioni
civili hanno la stessa caratteristica di quella religiosa perché non
significa darsi e basta, non significa occuparsi anche di cose
materiali, il commercio, non significa abbracciare il mondo, maledetto
anche da Calvino, ma di rispondere alla volontà di Dio.
L'atteggiamento disvalore rimane, nello stesso modo in cui c'è la
figura dell'asceta. Questo è un superamento del mondo entrando nel
mondo. Weber dice ascesi intramondana, rimane l'adesione alla vita
spirituale. Uno dei paesi dove trionfa è l'Olanda, una delle potenze
mercantili del tempo. Uno storico ha scritto un libro sul secolo d'oro
in Olanda, il \textit{Disagio della ricchezza}, erano ricchi,
calvinistiche, e tutte queste ricchezze erano in contraddizione con le
finalità spirituale in cui loro si impegnavano.

Qui si entra anche nel tema del rapporto col potere politico. In prima
istanza è con Lutero per cui il potere politico non si discute poiché
legittimato da Dio, però di fronte al tiranno, a colui che governa
contro il precetto biblico bisogna fare resistenza, e anche si può
tradurre in atti politici nella misura in cui la chiesa come insieme
politico disconosce l'autorità. Lui trova indispensabile che siano i
corpi medi a prendere voce in nome di tutta la comunità contro i
tiranni. E anche all'interno del calvinismo. In ogni caso questo è una
presa di posizione anche solo nel limite della difesa passiva è
qualcosa di sconcertante... La realtà del governo di Ginevra prende le
vesti di un governo teocratico e tirannico (?), vengono messi ai
vertici pastori, che non sono ecclesiastici, ma vigilano sulla
corretta interpretazione della parola di Dio, e sulla morale pubblica,
svolgono la funzione di censura.

Una cosa che non si pensa è il diverso atteggiamento dei colonizzatori
dell'uno e dell'altro cristianesimo. E abbiamo visto la differenza tra
domenicani e gesuiti, e anche i calvinisti si sono preoccupati delle
colonie, però la missione evangelica era meno forte. Siccome Dio ha
già deciso chi si salva e chi no questi sono già condannati, e nessuno
si sente di doversi impegnare alla loro conversione. 

Michelle Serveto, 1533, a Ginevra finiscono per essere tutti
perseguitati dalla chiesa romana, tutti gli eterodossi possibili e
immaginabili, immaginando di essere bene accolti nella città di
Calvino, però essa non era una città della tolleranza, ma era una
città dogmatica, e il primo di tutti era questo, un antitrinitario, un
eretico, e per questo era fuggito dalla Spagna, e le stesse accuse
messe dall'inquisizione spagnole gli vennero messe a Ginevra e finì
subito sul rogo. È proprio in seguito alla sua sorte che si apre un
dibattito sulla legittimità della persecuzione dell'eterodosso. C'è un
libro che si intitola \textit{Se gli eretici devono essere
perseguitati} di Castiglione (?).

Ultimo conflitto, Carlo V fa alleanza con Enrico VIII contro Francesco
I che era a sua volta alleato con gli scozzesi. L'unica vera
conseguenza di questa guerra è stata quella di ritardare l'apertura
del concilio di Trento, che, come si sa, è un passaggio fondamentale
per la storia del mondo latino, e si riconduce alla categoria -- fino
ad adesso si è parlato di riforma -- di controriforma, una reazione a
quello che era avvenuto fin'ora in Europa. Il concilio, inizialmente
che doveva trovare una mediazione, vista la poca disponibilità dei
protestanti, si versa a enfatizzare lo scisma. Trento appartiene
all'Impero, e nello stesso tempo è una città italiana. In realtà a
prendere parte al concilio saranno soltanto italiani e spagnoli. Anche
i francesi eviteranno di prenderne parte. Mentre si apre il concilio
Carlo V si decide di liquidare il problema protestante e sconfigge a
Muhlberg nel 1547 alla lega... rittrato di Tiziano.

Nel 1547 muore anche l'antagonista di Carlo V Francesco I, Enrico II
lo succede, e Carlo V si rende conto che la riformazione dell'impero
non si può. Ha sconfitto i protestanti nel campo di battaglia, però il
protestantesimo è ormai un fatto cultural. Poi Successione di Carlo V,
Filippo II, e finalmente si decide di sigillare la pace col mondo
protestante, pace di Augusta 1555. Accordo su due punti, uno riguarda
le persone, l'altro le cose. (1) Viene riconosciuta autonomia della
chiesa luterana... Libertà di religione dei principi, l'obiettivo è
quello di fare comunità compatte per evitare le guerre civili. (2) Si
stabilisce che fino a un anno normale stabilito 1552, tutto quello che
è stato acquisito dai protestanti rimane nel loro possesso, tutto
quello che è stato presso dopo viene ridato alla chiesa.

\end{document}
