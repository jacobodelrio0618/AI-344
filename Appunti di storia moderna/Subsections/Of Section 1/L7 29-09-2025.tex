% !TEX root = ..\..\main.tex
\documentclass[../../main.tex]{subfiles}%
\ifSubfilesClassLoaded{\addbibresource{../../bibliography.bib}}{}

\begin{document}

% Juntas di Burgos, si è arrivati in ritardo.

Quando si fa la proposta si dice che bisogna che questi positivamente,
o anche con gli armi, dimostrino di non voler ricevere la salvezza,
dunque bisogna prima informarli di come stanno le cose al mondo e dove
sta la verità, ed ecco che viene elaborato questo \emph{riquerimento},
una richiesta che si deve fare ogni volta che si entra in contatto con
i pagani.

% Citazione in spagnolo.

Ci doveva essere un predicatore, un notaio, ma ben presto si passò
oltre tutte queste formalità. Contestualmente le nuove leggi, che
erano volte a organizzare il governo civile e militare delle nuove
terre conquistate, nuove leggi attraverso delle quali si produce il
sistema economico di cui si è parlato, i governatorati e anche il
sistema giudiziario. Come si è già accennato, si conosce tutta questa
storia e tutte queste vicende quasi esclusivamente attraverso le
fonti storiche che ci hanno lasciato i conquistatori. Non sappiamo
quasi nulla di quelle che furono le reazioni dei popoli conquistato,
perché la conquista fu un processo profondissimo di deculturazione,
oltre che massacro di queste popolazioni che furono decimate, sia
dalle guerre che dalla violenza, e in seguito per la trasmissione di
agenti patogeni dagli europei agli amerindi.

% Esempio di malattia.

E però questa malattia giunta nelle Americhe fece stragi. Distruzione
fisica, collasso demografico, cancellazione della cultura di questi
popoli. Fonte che in qualche misura rende conto del grande dramma
delle popolazioni amerindie è rimasta sconosciuta fino al secolo
scorso, finché non è stata trovata da un bibliotecario olandese (?),
libro che raccontava la conquista non proprio dal punto di vista d'un
indio, ma di uno di origini inche, ed ha lasciato questo libro pieno
di disegni, commenti, una delle pochissime fonti a cui si può fare
affidamento per intendere la profondità del dramma di queste
popolazioni.

% Las Leyes de Indias (1513)

Con Carlo V nel 1519, nuovo re, imperatore tedesco, impronta
fenomenale nella storia del Cinquecento, tentato di restaurare l'unità
del mondo cristiano sotto l'autorità imperiale, nel 1519 la questione
si riapre e ancora le posizioni si contrappongono intorno alla natura
di queste popolazioni. Hernan de Quevedo, vescovo, altro,
assoggetamento delle popolazioni va data per scontata perché sono
schiavi per natura, e Bartolome de las Casas.

% Citazione De las Casas

Riconosce che tutte le popolazioni, gli inca, hanno sviluppaot vita
pubblica, hanno le loro istituzioni e così via. Un altra importante
voce che interviene in questo dibattito è quella di un domenicano che
insegna diritto all'università di Salamanca, principale università di
spagna, che ha rinnovato la scolastica, una nuova scolastica, ci sono
profondi studiosi del diritto romano, teologi, e uno di questi è
Francisco de ditoria, che fa interventi sul diritto, le
\textit{Relectiones} avevano per argomento le indie e il diritto di
usare la forza, \textit{De indis e de iure belli}. Anche lui smentisce
che si possa impossessare di queste terre disconoscendo la natura
delle organizzazioni sociali della popolazione, però lui sostiene che
è lecito fare la guerra alle popolazione nel momento in cui
impediscono la circolazione delle predicazione. C'è un diritto
originario e naturale al commercio fra tutti gli uomini che non può
essere ostacolato senza fare un torto profondo che può essere
sanzionato con la guerra. È interessante notare che qui si parla
ancora di diritto bellico, e si fa anche oggi, riflettendo sulle cause
della guerra. SE le cause sono moralmente giuste o ingiuste.
Naturalmente l'impostazione su queste basi presuppone che ci si possa
allontanare dal contesto bellico e dare spazio alla mente, tutti
quelli che comattono sono convinti della giustezza della propria
causa. Invece fondare il dirotto bellico sull'etica era una cosa che
non funzionava, ma si continuava a ragionare in questi termini, avendo
come bussola naturalmente la religione cristiana. Anche Francisco de
Vittoria, per quanto fosse un profondo giurista, parte da queste basi.
Da lì a qualche anno, con l'esplosione delle guerre di religione si
vedrà che il diritto bellico dovrà essere rinnovato: la guerra è
giusta se il nemico è giusto. Non si considerano i momenti soggettivi,
ma le caratteristiche formali dei nemici (?). Giuristi del Seicento,
giurista spagonolo, operava nei paesi bassi, cuore dello scontro
confessionale tra cattolici e protestanti, e lui dice che la guerra
non dipende dalle cause ma dipende. Sono le caratteristiche del
belligerante, non il movente interiore. Se è un potere sovrano... la
guerra mossa da chi non ha la sovranità, la titolarità giurudica, è
illegittima, e quindi la guerra la possono fare solo i nuvoi soggetti
che si stanno formando, gli stati. Delegittimare i ribelli protestanti
che si sono sotratti a Filippo II e stanno facendo una rivoluzione, e
sicdcome non è una guerra giusta, il sovrano può fare quello ceh
vuole, si tratta di una cosa 'poliziale'. Qui invece si è ancora in un
dibattito fortemente ancorato alla morale religiosa.

I pontefici prendono naturalmente posizione, e lo fanno da un lato e
dall'altro. Clemente VII legittima l'impego della guerra e della
violenza degli indi,
% Citazione di una bolla
Differentemente Paolo III, bolla \textlatin{\textit{Veritas Ipsa}}.
\textit{Brevisima relacion de la destruccion de las indias} (1542).
Essa si rivolge direttamente al pontefice, riconosce la legittimità
della bola inter caetera, ma poi racconta tutto quello che è accaduto
nelle indie da che arrivarono gli spagnoli.

Idea che la natura americana è debole, non ha la forza della natura
europea. Tutte le specie animali che trovano corrispettivo europeo
sono depotenziate. Questa posizione di minorità, debolezza,
inferiorità nel continente a tutto livello. Questo è un aspetto
interessantissimo che arriva a Hegel nell'Ottocento.

Gli spagnoli si spartissero, tiranno supremo chiamato governatore...

Eppure essendo uomini per lo più idioti (cioè illiterati), mandare
maschi nelle mineire e le donne... 

Un atto di condanna molto forte. Piantagioni per produrre coltivi
molto ricercati in Europa, il principale dei quali era lo zucchero, ma
anche il cacao, il caffè -- che è d'origine asiatico che viene portato
ed acclimatato in America, però non si era mai smesso di cercare anche
i metalli preziosi, la leggenda del Dorado, e in effetti poi alla fine
si trova l'oro, perché nel 1545 nel Perù furono scoperti i giaccimenti
del Potosì, che è la montagna che ha alimentato il sistema monetario
Europeo per parecchio tempo, per diversi decenni in maniera clamorosa,
ma poi anche in seguito, montagna tutta d'argento, e quindi ci fu uno
sfruttamento anche con le miniere, furono create delle encomiende
specificamente minerarie. Continuavano comunque le denuncie, de las
casas era infaticabile, e con lui anche altri delle ordini dei
predicatori. Carlo V alla fine si convince e nel 1543 emana nuove
leggi che vanno a modificare le leggi del 1512, con l'intenzione di
mettere un limiti ai proprietari di schiavi, incomendentes (?), gli
indios possono essere fatti schiavi solo con giusot motivo, i
lavoratori debbono essere per lo più liberi e retribuiti, ma la cosa
più evversiva che contiene la legge è la non trasmissibilità della
encomienda, il cui significa che alla morte dell'encomenderos la terra
torna alla corona e non può essere trasmessa ai figli, e questo la
rese insopportabile, insieme alle altre impostazioni per gli indios.

Scopia una rivolta che la Spagna non riesce a sedare, e il re è
costretto a ritirare le leggi nuove. Si riaccende ancora una volta il
dibattito sulla natura degli indios, e sembra ad un certo punto che
prevalgano non solo nei fatti, come si è visto, ma anche nel dibattito
culturale i sostenitori dell'origine naturale dello stato di schivitù.
Juan GInes Sepulveda, contradittore di De las casas, anche qui il
dibattito sulla giustizia della guerra prende in considerazione anche
la condizione servile degli amerindi. È un trattato in forma di
dialogo che si ispira al modello platonico, però qui si prende
nettamente una posizione di ispirazione naturalmente della Politica di
Aristotele. Passaggio emblematico, giustificazione delle gerarchie
sociali su basi naturali. I barbari e i selvaggi vita condotti a vita
più umana, guerra giusta se si rifiutano, la guerra nasce dalla
natura, posto che una parte di essa coincide con l'arte della caccia,
che covniene usare non solo contro le bestie, ma con quelli uomini
servile, e questo lo disse Aristotele, e con lui conviene Agostino.
Passo già citato di Luca, costringeteli a entrare se gli invitati non
vogliono venire, e qui si mette anche Agostino. Gli spagnoli superioi
ai barbari come l'uomo è superiore al bambino e com'è al di sopra
della scimia. Naturalmente questi scritti non rimangono senza
risposta, e sono in tanti quelli che respingono la schiavitù naturale,
e anche discepoli di Bartolomé.

1550 giunta per considerare la questione.

Peccati contro natura

Servitù naturale

Necessità di evangelizzaizone che richeideva una sottomissione
violenta

Sacrifici umani e altre pratiche contrarie al diritto nauturale e al
diritto delle genti (su questo si ritornetà).

Anche questi popoli che stavano fuori, bisognava riconoscere diritti,
che poi coincidevano con quelli del diritto di natura, ma poi si
tornerà su questo. Qeusti sono gli argomenti di Sepulveda, che danno
raffigurazione degli indios opposta a quella di Bartolomé, secondo cui
sono modelli di cristianitas, umili, disposti al confronto
all'apprendere, mentre l'altro dice sodomia, sacrifici umani,
idolatria, e naturalmente c'erano dei costumi e anche pratiche che
agli occhi degli europei erano animali. C'erano anche costumi che
turbavano la sensibilità europea, ma c'era un'incapacità di capire le
motivazioni profonde di taluni atteggiamenti. Insomma, quest'attacco è
un'avversario temibile per Bartolomé, e nel mezzo delle dispute
chiarisce le sue cose e nega la concezione della schiavitù per natura.
Non come conquistatore, ma combattere errore con le armi della
persuasione e della pace.

Basta dire che di forte alla crisi demografica e dell'economia, si
doveva ricorrere alla schiavitù africana e non agli indio, diceva
Bartolomeo, e comunque mantenevano la superiorità degli europei in
ogni campo, si sentivano portatori della veirtà, della salvezza
religiosa, e se disputavano tra loro era perché non si trovavano
d'accordo sugli strumenti. Per Sepulveda erano bestie, e li potevi far
migliroare, ma non appartenevano allo stesso genere dell'umanità, per
Bartolomé invece sono uomini, fratelli in cristo, ma sono minori, non
esattamente come noi che siamo stati illuminati dalla rivelazione.
Perciò l'atteggiamento di Bartolomé lo si può definire come
paternalistico; per l'uni bestie irragionevoli, per l'altro bambini da
crescere e condurre sulla strada giusta, con la condizione che fossero
allievi ben inclinati. Ci sono dei precursori di quest'idea
dell'uguaglianza tra culture molto diverse. C'è la storia di una suora
francese nel Seicento. Ne parla in un libro famoso una storica
americana, Donne ai margini, tesi secondo cui le donne che sono sempre
state discriminate, ecc., stanno ai margini della cultura, degli
apparati di potere, potevano maturare un punto di vista eccentrico
rispetto alla cultura, e lei fa tre esempi piuttosto illuminati, una
cattolica, una protestante, una ebrea che avevano personalità
piuttosto consonante con quelle che sono le nostre convinzioni
attuali.

Bisognava amturare un punto di vista nuovo, sempre poroblematico, il
relativismo culturale, a proposito della cui nascita si fa riferimento
ai \textit{Saggi} che si astenne dalla condanna nei confronti dei
cannibali, e scrisse nel 1580.

% Citazione di Montaigne, e lui scriveva nel bel mezzo delle guerre
% religiose e delle persecuzioni.


\end{document}
