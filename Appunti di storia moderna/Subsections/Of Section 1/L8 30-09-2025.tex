% !TEX root = ..\..\main.tex
\documentclass[../../main.tex]{subfiles}%
\ifSubfilesClassLoaded{\addbibresource{../../bibliography.bib}}{}

\begin{document}


%%%%%%%%%%%

Rivolta dei baroni del sud d'italia, i baroni si riguggiano alla corte
del re di Francia e fa sì che venga in Italia e ristabilisca la
situazione. A questo punto poi si arriva alla vigilia delle guerre.
Oltre alla situazione di tensione, lorenzo il magnifico ed innocenzio
ottavo, due grandi personaggi cercano di mantenere equilibrio in
italia e non fare degenerare la situazione. Ad aggiungersi c'è la
situazione che Ludovico il moro, che ha ambizioni di diventare duca di
milano e teme la prioria vita, spinge per l'intervensione del re di
francia per ottenere il regno di napoli. Nel 1494 Carlo VIII passa il
confine e la cosa che fece grande impresssione, tanto che machiavelli
dirà che la campagna è la guerra del gesso, praticamente dal settembre
riesce rapidamente a scendere in Italia, ma trova aperte le porte, o
conquista facilmente. Invece di entrare in combattimento si retirano
ad ogni passo, e l'avanzata dei francesi è spedita. La rivolta di Pisa
è contestuale all'arrivo a Firenze di Carlo VIII, aprono le porte
della toscana, e questo porta alla ribellione di Pisa.

Lui in quattro mesi è arrivato oltre Roma, il pontefice anch'egli apre
le porte, e questo però non seda la rivolta dei baroni, che
chiaramente erano pronti a quella situazione, ingresso trionfali a
Napoli, non si trattiene lì, il suo grande sponsor, lidovico il moro a
questo punto è passato con gli altri stati, si credeva che il re
arrivasse e ristabilisse certe situaizoni, ma l'avanzata è stata
fulminante e spinge agli italiani a creare una lega contro il re di
francia, che poi torna rapidamente in Francia e viene intercettato
dopo. Lascia una guarigione militare nel sud. Durante la ritirata c'è
un incontro in cui si sostiene che non abbia cisto nessuno, Fornovo.
Nel mentre quello che è importante è che non tanto nel nord, quanto
nel sud, il re si trova ad affrontare una dura guerra di quella che
oggi si chiamerebbe guerrilla con seminara, in calabria, qualcuno di
Cordoba, vittoria strategica spagnola, le forze francesi non riescono
a distruggere gli spagonli che non hanno basi lontane ma sono
riforniti, e quindi non hanno la meglio sulle forze di spedizione
spagnole. e così si conclude la campagna francese in Italia. Si
credeva tradizionalmente che il grande successo della spedizione
francese fosse \emph{tecnologico}. Quando scende in italia arriva con
artiglieria che oggi si chiamerebbe moderna. In precedenza erano
immaginati come grandi tubi appoggiati su quella che chiamavano (?)
che avevano problemi tecninic per sparare. Quelli francesi sono molto
simili all'immagine che abbiamo oggi dei cannoni, che sono
ipotrainati, sono relativamente leggeri, possono essere trasportati
con velocità e numero esiguo di cavalli, e farlo prima ci metteva
persino 24 buoi. Palle di ferro e cannone di bronzo, diplice
vantaggio, non esplodono come si vede nei film, ma semplicemente
colpiscono.

Soprattutto il vantaggio che avrebbero dato è la \emph{velocità} e la
facilità di trasporto. Era più facile disporli in batteria, pronti per
il combattimento. Spesso, più banalmente la vista dei cannoni spingeva
ai difensori ad arrendersi presto, poiché l'artiglieria comportava la
distruzione delle mura. Questo comporta anche conseguenze per il
saccheggio. Se veniva pressa una città per assalto il saccheggio era
libero, se invece si arrendeva era controllato o non veniva fatto.
Quindi in pratica l'impatto, o la percezione di esso era devastante.
La maggior parte degli assedi viene tuttavia condotto con assalto
tradizionale, però comunque il vantaggio è avere una batteria
immediatamente disponibile al combattimento. Quindi queste sono.

La campagna di Carlo VIII aveva comunque ridefinito la situazione in
Italia, ma da lì a poco, di nuovo un esercito francese entra in
Italia. In 1498 muore Carlo VIII senza erede e sale Luigi XII, il suo
cugino. Lui aveva l'ambizione di diventare duca di Milano, perché i
francese, imparentati con gli Sforza, hanno pretese legittime per il
ducato di Milano, quindi nel 99, con l'intento anche di vendicarsi di
Luca il Moro. Per tanto il re di Francia stipola un accordo nel 99 con
la repubblica di Venezia, il noto trattato di Blois, campagna
congiunta nel nord d'Italia. Questo accordo, importante mente, è
sostenuto anche dal pontefice. In questo accordo rientra lo stato
pontificio, al tempo è salito Alessandro VI Borgia, che ha l'mabizione
per ritagliare per il figlio Cesare un dominio territoriale in Italia,
nella romagna, nei domini di quella che è Caterina Sforza,
sostanzialmente. Dopo aver raggiunto l'accordo si prepara ad invadere
l'Italia. Il nervo dell'essercito francese sono i mercenari svizzeri,
diventati famosi alla fine della second ametà del Quattrocento.
Chiaramente la Svizzera sono quelle regioni che vengono definite
militari, c'è grande pressione demografica, diffusa povertà, zone
montagne, e quindi sono regioni buone per fornire i mercenari agli
stati europei. La Svizzera diventa fra quattrocento e cinquecento una
terra di rinomati mercenari, perché creano un mdoo di combattere
particolarmente effettivo. Riescono a riportare. UNità che si
scheirano a quadrato, rige una grande disciplina, e sono pensate per
sostanzialmente impattare contro (?) push, spinta con le picche, e
premendo contro un'unità la travolgono. Una massa di contadini armati
sconfigge l'esercito di... Svizzera divisa in cantoni, alleanze con le
potenze per fornire soldati e si facevano pagare. L'esercito di LUigi
XII ha una componente di mercenari svizzeri perché rivoluzionerà la
battaglia campale, e mette in crisi l'unità di cavalleria pesante.
L'impero germano invece svilupperà i propri mercenari svizzeri, che
sono la coppia di quelli svizzeri alleati con la Francia.

Il fronte principale di tre è quello del nord d'italia, con al
reoubblica di venezia e il re di francia. In Agosto inizia
l'invasione, in settembre i francesi si impossessano di milano,
ludovico il moro è fuggito, e inizia la campagna militare in romagna
guidata da cesare borgia, a prendere i domini di Caterina sforza, e
tagliare i propri domini territoriali. Poi Lodovico sforza riconquista
Milano. Trattato di Trento che è importante perché è un accordo che
segna sostanzialmente l'investitura di Luigi XII come duca (?). Quello
che interessa per i cambimenti tecnologici è il fronte meridionale. A
questo punto assicurato il fronte setterntironale si presenta nel
regno di napoili, arriva con grande esercito, sta portantdo avanti la
guerra con cordoba che combatte in nome del regno di Aragona, e poi
c'è una serie di combattimenti ormai noti, soprattutto in Puglia e
nella Calabria.

Battaglia di Cerignola, sbaragliato l'esercito francese, poi
armistizio di Lione che crea la spartizione in aree d'influenza
l'Italia, sud al regno di argano, la parte settentrionale sotto
influenza francese. È però importante la campagna militare nel sud
d'Italia perché si stabilisce per la prima volta che si dimostreranno
efficaci le armi da fuoco sul campo di battaglia. Nell'epoca
medievale si parla di una sorta di tubo che sparava. Polvere da sparo,
arma che bucasse l'armatura dei cavalieri, verso la fine del
Quattrocento si investe perché si cerca di avere un armamento per
penetrare l'armatura della cavalieria. Però il problema è il tempo di
ricarica, però è importante perché durante in quella battaglia si
domostrano importante, ma perché, cos'è cambiato? Perché schierano i
soldati in una posizione preparata, una sorta di trincea. Riescono a
bloccare l'avanzata dei francesi. Prima volta che le armi da fuoco
sono riuscite ad incidere sulla fanteria. Il terzo fronte in realtà è
in conflitto che dura un quindicennio e non lo si può collocare
specificamente all'interno delle varie discese francesi.

La ribolta di Pisa, si dichiara indipendente rispetto a Firenze nel
1494, e diventa un fronte importante per la geopolitica europea. Non è
soltanto uno scontro tra Pisa e Firenze, ma anche la Francia,
l'impero, Venezia, però è importante perché durante l'assedio di Pisa
che dirà quasi quindici anni, è un momento di grande elaborazione e
trasformazione dell'architettura militare, e anche se la maggior parte
delle innovazioni si svilupperà posteriormente è comunque considerata
un momento importante. Oltre i cannoni, si trasforma anche la'antomia
delle difese, si passa dai vertical wall a quello che viene definito
il combattimento orizzontale, per cui la fortezza diventa un arma
offensiva. L'introduzione dell'artiglieria fa presente come le mure
medievali erano vulnerabili al fuoco dei cannoni, e quindi si trattò
di rendere sempre più spese le mure per assorbire il fuoco, ma
potevano essere anche utili per alloggiare l'artiglieria. Si permette
alla fortezza non soltanto di assorbire, ma anche di fare fuoco contro
il nemico schierato all'esterno della fortezza, e un'altra innovazione
e trasformazione della guerra sono le piatadorme (?), dare protezione
a 360 gradi delle mure. Perché l'assedio di Pisa è così importante?
Siccome la costruzione di terrapieni e mure più spese, si fa la
consuetudine di scavare sotto le mure così da farle crollare, ed era
molto efficace, però in occasione nell'assedio c'è la doppia ritirata
pisana. In corrispondenza della caduta del muro si relizzava un nuovo
muro all'interno della fortificazione. Alla fine i fiorentini hanno la
meglio.

Dopo la morte di Alessandro VI, viene eletto Giulio II, che ha una
visione abbastanza aggressiva della funzione del pontefice. Forse
l'unico papa che scende in armatura. Determinato a riacquistare tutti
i possesi che la chiesa aveva perso nel Quattrocento. Aveva seduto a
famiglie il controllo di città. Quindi la prima parte del suo
pontificato è dedicata alla rconquista dei territori all'intenro dei
confini dello stato pontificio, ma non se ne accontenta, ma vuole
prendersi anche quello ceha aveva perso dalla Venezia, e quindi inizia
a creare una grande alleanza internazionale, ed è quella che è la lega
di Cambrai, e un aspetto interessante della lega non è soltanto per
comabttere fra venezia, ma che si spartiscano i territori veneziani
fra id loro. Il casus belli è che Venezia (?). Nel 1509 inizia la
campagna militare, battaglia di agnadello, sconfitta venezi , e c'è
temore che l'esercito della lega arrivi a Venezia.

% SUdditi a difendere venezia

RIesce a stipolare accordo con Giulio secondo, sottomissione formale a
roma, ma perché in realtà a questo punto Giuluo II si rende conto di
un nuovo pericolo in Italia -- lui voleva avere la politica italiana
-- ed è la francia. Nel 1511 coalizza contro la francia, lega santa
che all'inizio non va molto bene, perdono a ravenna, ma
contestualmente le forze spagnole pongono fine alla repubblica di
firenze, restaurazione dei medici a firenze. SItuazione comunque
incerta e in quel momento venezia abbandona la lega satna e si schiera
con la francia, dopodiché gli eventi proseguono con battaglie
importanti, però il momento culminate sono 2, nel 1512, dopo la morte
di luigi xii sale al trono, e nello stesso anno riprende milano.
Massimiliano d'asburgo poco prima di morire tenterà di riconquistare
milano e la fuerra si concluderà col trattato di NOyon, e si
riconferma l'accordo precedente. MIlano alla francia, il sud alla
spagna.

\end{document}
