% !TEX root = ..\..\main.tex
\documentclass[../../main.tex]{subfiles}%
\ifSubfilesClassLoaded{\addbibresource{../../bibliography.bib}}{}

\begin{document}

Paul Hazard. Principi di autorità vengono criticati e messi in
discussione, e inizia un processo di corrosione della tradizione che è
alla base del moderno. La crisi della coscienza europea, tutto
inquadrato dentro il discorso religioso. Le esperienza che si fanno
dicono che il mondo e l'uomo è diverso, l'uomo è moderno. Si è detto
che gli antichi avevano un livello di perfezione irraggiungibile, ma
noi siamo migliori degli antichi. Cambiamento di atteggiamento,
ottimistico verso il futuro, laddove c'era la convinzione che si
sarebbe andati verso il peggio. Adesso si comincia a pensare di poter
migliorare e staccarsi da quella tradizione. Nel medioevo quando si
parlava di novità scattava subito una valutazione negativa, qualcosa
che si discostava da quello che avevano insegnato gli antichi, e
invece qua si ricerca la novità. C'è un apprezzamento positivo per
tutto ciò che è nuovo, e tra le cose nuove c'è naturalmente la fisica
newtoniana.

Tutto questo nasce dalla cosiddetta rivoluzione scientifica che ha
l'indirizzi esperimentali induttivo sulla scorta di Galileo, e il
metodo cartesiano, \textit{Discorso sul metodo}, nuovo percorso. È
possibile quantificare il mondo. C'è un approccio completamente
diverso da quello scolastico che apprezzava soltanto le differenze
qualitative. Le regole metodologiche di Cartesio.

% Citazione di Cartesio.

A fine Seicento si inizia la statistica, proprio per fini politici.
Ottimismo di fondo che si trova anche in Fontanelle.

% Citazione di Fontanelle. 

Confronto in tutte le forme di espressione artistiche che si basano
sull'immaginazione, che non richiedono un accumulo di conoscenza.
Omero può restare insuperato, ma non è così nelle scienze. Le scienze
sono composte da un numero infinito di idee, e dipendono dal
ragionamento che si perfeziona lentamente, ma anche incessantemente.
Scoperte casuali, tutto ciò non ha fini e gli ultimi fisici e
matematici saranno più abili. È l'idea del \emph{progresso}. 

E tuttavia queste cose non penetravano nella società la cui cultura
era ancora basata nella tradizione, era anche pericoloso avere a che
fare con queste idee nuove. 

% Cronologia Torino 1730.

La cronologia della storia del mondo è ancora la cronologia classica.
Con l'età dell'illuminismo si ampliano ancora di più le conoscenze
delle parti del mondo. Viaggi di James Cook, esistenza dell'Oceania,
esplorazioni degli anni Sessanta e Settanta, e tutto ciò doveva anche
suscitare una grandissima curiosità nei confronti delle altre parti
del mondo, si vendono moltissimi libri di viaggiatori, alcuni di
grande ispirazione anche romantica.

Dopo la seconda rivoluzione inglese, libertà di stampa nell'impero,
cominciano a essere giornali, industria editoriale non legata
all'ambito accademico. Non c'era letteratura di largo consumo, e
invece tra sei-settecento nasce un pubblico di lettori più ampio, e
nuovi generi letterari come il romanzo, e questo dà una scossa perché
comincia a nascere un'\emph{opinione pubblica}. Quando alla fine del
seicento c'è un pubblico pronto per recepire ciò che prima era stato
soggetto degli epistolari sorge l'illuminismo. La generazione
precedente ha demolito la tradizione dal punto di vista intellettuale,
andare oltre i dogmi della tradizione. Gli illuministi non si
accontentano di capire il mondo ma lo vogliono cambiare, questo
programma di miglioramento che era stato indicato. Politica sociale.
Tant'è vero che l'opera manifesto dell'illuminismo è un'opera teorica
ma anche pragmatica, mette le conoscenze perché possano essere
applicate, l'\textit{Encyclopédie}. Le acquisizioni devono trovare
applicazioni pratiche nella vita. E nell'illuminismo c'è questa
crescente curiosità per il resto del mondo che a poco a poco viene
svuotato della sua aurea mitica e leggendaria.

È a questo punto che compare probabilmente uno dei primi libri di
storia che non solo sfugge alla metodica classica iniziale di
inquadrare la narrazione umana nel contesto della storia sacra, di
Voltaire. Tutti i libertini erano accusati di immoralità,
\textit{Saggio sui costumi e lo spirito delle nazioni}. A differenza
di Bossuet, che parte dalla creazione, Voltaire fa la storia cercando
di capire la logica interna di sviluppo della storia, e lo fa in
maniera comparata, perché quello che gli interessa sono le nazioni, e
non solo l'Europa, ma quello che si conosce del mondo. Ha scritto
anche la vita di Luigi XIV, libro nuovo dove si illustra la politica e
versa sulle vicende politiche ma non solo, e anche una galleria di
personaggi che hanno illustrato la Francia di quest'epoca per mostrare
i progressi. Questo però era il più rivoluzionario perché scardina la
storia che non è fatta dal punto di vista religioso. All'epoca la
conoscenza della Cina si erano evolute. A partire dal Cinquecento è
una destinazione delle missioni evangeliche gesuite. Per quanto
riguarda lo scambio culturale questo aggancio fu fondamentale.

Missioni dei gesuiti controllati dalla corona francese, anche
operavano come diplomatici del re di Francia. Nel Settecento c'era una
corrispondenza continua tra il centro della compagnia a Roma e poi
Parigi e uno scambio di informazione intenso. Nel Settecento, quando
ci sarà grande interesse dell'opinione pubblica per la Cina, vengono
pubblicati in diversi volumi le corrispondenze dei missionari gesuiti,
è precisamente la forma da cui si è servito Voltaire per fare la sua
opera. I gesuiti avevano capito che per controllare il mondo che si
stava secolarizzando bisognava controllare le coscienze. Dopo la
riforma il pontefice non ce la fa ad avere un rapporto da pari in pari
con le corone ed avere le influenze che aveva al tempo. Il cardinale
Bellarmino, teorizza il potere indiretto della chiesa, controllare le
coscienze dei popoli e specialmente le classi dirigenti: la stessa
strategia dei gesuiti in Cina. Di ciò si occupa l'istruzione, e si
crea dunque un collegio, e si cerca la concorrenza. Tutti andavano a
studiare dai gesuiti nei paesi cattolici e non solo, era l'equivalente
del liceo, dove si prendeva la licenza in filosofia. Nel Settecento
parte la campagna contro i gesuiti. Clemente XIV sopprime la compagnia
di Gesù che sarà ristabilita dopo la restaurazione. Era
un'organizzazione sovranazionale che faceva riferimento direttamente a
Roma e che sembrava essere in contrasto con i poteri di rafforzamento
statale.

Spiegare il vangelo in Amazonia, trasformare la gente in contadini,
insegnare l'agricoltura, creare comunità. In Cina si aveva a che fare
con gente che aveva da insegnare, erano loro i maestri su tanti campi.
Però i cinesi erano curiosissimi a imparare le matematiche. I cinesi
aprivano le porte ai gesuiti perché volevano confrontarsi non sul
vangelo, ma sulla scienza. Matteo Ricci capisce subito, e una delle
prime cose che traduce al cinese sono gli \textit{Elementi} di
Euclide. La scienza in questo caso viene usata per la penetrazione
religiosa, e questo significa anche che i padri gesuiti tengano
scambio epistolare non solo con le parti superiori dell'ordine, ma
anche con gli scienziati per essere aggirati sulle questioni
scientifiche. Come mai i cinesi conoscevano la metallurgia da molto
tempo fa, e non si capiva quando abbiano raggiunto l'uso del ferro.

% Citazione argomento del ferro.

% Digressione sui calvinisti.

Le cronache cinesi, appariva che conoscevano il ferro da tempo
immemore, e le loro cronologie rinviavano a tempi ancora prima della
creazione del mondo, il cui metteva ulteriori tensioni con la storia
sacra.sui calvinisti.

Le cronache cinesi, appariva che conoscevano il ferro da tempo
immemore, e le loro cronologie rinviavano a tempi ancora prima della
creazione del mondo, il cui metteva ulteriori tensioni con la storia
sacra. La tradizione è screditata dal punto di vista scientifico. Con
Voltaire si comincia a scrivere la storia in modo simile a come si fa
adesso, anche se il principio direttivo era l'idea di progresso
evocato da Cartesio e Fontanelle. Nel Novecento invece la cosa è
diversa.

Un ultimo esempio è la geologia che ha cominciato a rendere conto di
certi fenomeni che venivano interpretati sempre alla luce del racconto
biblico. Questi fossili sopra le montagne venivano spiegati con il
diluvio universale, con la generazione spontanea, rifiuti della
natura, ecc. E comunque, il tema del diluvio alludeva a una storia
di cataclismi, in tempi corti. Oggi invece sappiamo che i tempi
geologici sono molto lenti, e la cosa importante è quello che si è già
incontrato nel discorso di Fontanelle, le leggi di natura sono
costanti. Ci portano a dedurre un'antichità lunghissima della Terra.
Questi tre secoli hanno comportato una svolta con tutto ciò che si
pensava prima, almeno per la storia occidentale, dal punto di vista
della coscienza europea. Chiave di volta, punto di passaggio nel quale
gli europei hanno creduto nel liberarsi dell'uomo nuovo, destinato a
un progresso illimitato e padrone del suo futuro. Ottimismo procede
per tutto l'Ottocento. Ora si è entrati alla messa in processo del
periodo moderno, del progresso. Si vive ormai in un'epoca
\bsq{postmoderna}.



\end{document}
