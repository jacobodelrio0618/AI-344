%Document Properties
\documentclass[a4paper, 14pt, twoside, notitlepage]{extarticle}
\title{
    Diderot e Goethe \\
    \smaller{\textit{Dalla storia naturale al progresso storico}}
}
\author{
    Jacobo del Río Molano \\%
    \textit{Teoria e storia dei sistemi filosofici} \\%
    4 dicembre 2025
}
\date{}

\usepackage{preamble}

% Bibliography Packages and management commands (to be put here in order for the
% bibliographic autocomplete to work)

\usepackage[
    backend=biber, 
    style=philosophy-verbose, 
    latinemph = true,
    language=italian, 
    sorting=nyt, 
    autopunct=false
]{biblatex}

\usepackage[
    style=italian, 
    autopunct=false
]{csquotes}

\DefineBibliographyStrings{italian}{
    loccit = {ibid\adddot}, 
    pages = {pp\adddot}, 
    opcit = {\unskip\addcomma\addspace cit\adddot}, 
} 

\addbibresource{bibliography.bib}

% Command to add a comma before "cit.,".
\renewcommand{\mkcitation}[1]{#1}
\newenvironment*{smallquote}{\quote\smaller}{\endquote}
\SetBlockEnvironment{smallquote} \AtBeginEnvironment{quote}{\smaller}

\begin{document}
\maketitle

\section{L'uomo e la Natura}

\subsection{Dichiarazioni programmatiche} 

\numberedparagraph % 1
\textsc{Giovane}, %
\textit{%
    prendi e leggi. Se riuscirai a leggere quest'opera
    fino alla fine, non sarai incapace di di comprenderne una migliore
    [\dots]. Persone più abili di me t'insegneranno a conoscere le
    forze della Natura; a me sarà sufficiente aver messo alla prova le
    tue. Addio.%    
}

\textsc{P.S.} %
\emph{
    Ancora una parola e poi ti lascio. Tieni sempre in mente che la
    \emph{Natura} non è \emph{Dio}, che un \emph{uomo} non è una
    \emph{macchina}, che un'\emph{ipotesi} non è un \emph{fatto}, e
    sta certo che non mi avrai ben compreso ogni volta che crederai di
    individuare qualche cosa di contrario a questi
    principi\footnotemark.
}
%
\footnotetext{\Cite[405]{diderotPensieriSullinterpretazioneNatura2019}.
Cfr. \textit{Ibid.}, pp. 439-441; \cite[29, 78-80]{goetheTeoriaNatura}.}

\subsection{L'uomo interiore si indaga nell'interpretazione naturale}

\numberedparagraph % 2
Scriverò sulla Natura. Lascerò che i pensieri si succedano sotto la
mia penna secondo lo stesso ordine in cui si offrono alla mia
riflessione, perché potranno meglio rappresentare i movimenti i
movimenti e l'andamento del mio spirito %
\autocite[407]{diderotPensieriSullinterpretazioneNatura2019}.%

\vspace{1\baselineskip}
\numberedparagraph % 3
Ora confesso che il grande compito, che sembra tanto importante,
espresso dalla massima \textquote{conosci te stesso}, mi ha sempre
ispirato qualche sospetto, quasi fosse un'astuzia di preti
segretamente alleati, che volevano confondere l'uomo ponendogli
esigenze irraggiungibili e distoglierlo dall'attività verso il mondo
esterno, inducendolo a una malintesa contemplazione del mondo interno.
L'uomo conosce se stesso soltanto in quanto conosce il mondo, di cui
si rende conto soltanto in se stesso, proprio come nel mondo prende
coscienza di sé. Ogni nuovo oggetto, se ben contemplato, dischiude in
noi un nuovo organo\footnotemark.
%
\footnotetext{\Cite[92-93]{goetheTeoriaNatura}. Cfr. \textit{Ibid.} pp. 146-147.}

\numberedparagraph % 4
L'interesse della verità richiederebbe che chi riflette si degnasse
finalmente di associarsi a quelli che si danno da fare in modo che il
teorizzatore fosse dispensato dal muoversi; che il filosofo manovriero
avesse uno scopo nei movimenti infiniti che compie; che tutti i nostri
sforzi si trovassero riuniti e diretti allo stesso tempo contro la
resistenza della Natura; e che, in questa specie di associazione
filosofica, ciascuno assumesse il ruolo che gli conviene %
\autocite[407]{diderotPensieriSullinterpretazioneNatura2019}.%

\numberedparagraph % 5
Anche nelle scienze, a dire il vero, non si sa niente, bisogna sempre
fare. [\dots] Ogni vero \textit{aper\c{c}u} viene da una serie e
comporta una serie. È l'anello intermedio di una grande catena che
aumenta in modo produttivo. [\dots]  La scienza ci aiuta soprattutto
perché, in qualche modo, diminuisce lo stupore cui per natura
incliniamo; ma anche perché risveglia nella vita, resa sempre più
intensa, nuove più elevate capacità, che permettono di eliminare ciò
che è dannoso e introdurre ciò che è utile %
\autocite[148]{goetheTeoriaNatura}.%

\subsection{Il materialismo \bsq{spinozista}}

\numberedparagraph % 6

Sembra che la Natura si sia compiaciuta a variare lo stesso meccanismo
in un'infinità di modi differenti. Essa abbandona un genere di
produzioni solo dopo averne moltiplicato gli individui sotto tutti gli
aspetti possibili. Quando si considera il regno animale, e ci si
accorge che tra i quadrupedi, non ce n'è nessuno che abbia le funzioni
e le parti, soprattutto quelle interiori, completamente simili a un
altro quadrupede, non si crederebbe volentieri che ci sia stato un
primo animale prototipo da cui derivano tutti gli animali, a cui la
Natura non ha fatto altro che allungare, accorciare, trasformare,
moltiplicare, annullare certi organi? Immaginate le dita di una mano
riunite, e la materia delle unghie, così abbondante che estendendosi e
gonfiandosi, avvolga e copra tutto; al posto di della mano di un uomo,
avrete il piede di un cavallo %
\autocite[415]{diderotPensieriSullinterpretazioneNatura2019}.%

\numberedparagraph % 7

L'autore [Maupertuis] comincia esponendo rapidamente i sentimenti di
coloro che l'hanno preceduto, e l'insufficienza dei loro principi per
lo sviluppo generale dei fenomeni. Alcuni non chiesero che
l'\emph{estensione} e il \emph{movimento}. Altri credettero di
aggiungere all'estensione l'\emph{impenetrabilità}, la \emph{mobilità}
e l'\emph{inerzia} [\dots]. Le operazioni più semplici della chimica,
o la fisica elementare dei piccoli corpi ha fatto ricorrere a delle
attrazioni che seguono da altre leggi, e l'impossibilità di esplicare
la formazione di una pianta o di un animale con le \emph{attrazioni},
l'inerzia, la mobilità, l'impenetrabilità, il movimento, la materia o
l'estensione ha condotto il filosofo Baumann [pseudonimo di
Maupertuis] a supporre altre proprietà ancora nella natura [\dots].
Quest'essere è il corporeo, queste modificazioni sono il
\emph{desiderio}, l'\emph{avversione}, la \emph{memoria} e
l'\emph{intelligenza}; in una parola tutte le qualità che si
riconoscono negli animali, che gli Antichi comprendevano sotto il nome
di \emph{anima sensibile}, e che il dottor Baumann ammette, in maniera
proporzionata alle forme e alle masse, nella particella più piccola di
materia come nel più grosso animale [\dots]. Cosa impedirà alle parti
elementari intelligenti e sensibili di allargare all'infinito l'ordine
che costituisce la specie? Da questo deriva un'infinità di specie
animali emerse da un primo animale, un'infinità di esseri emanati da
un primo essere, un solo atto nella natura %
\autocite[449-451]{diderotPensieriSullinterpretazioneNatura2019}.%

\numberedparagraph % 8

Saremo così giunti a poter affermare senza esitazione che tutti gli
esseri organici più complessi fra cui, i pesci, gli anfibi, gli
uccelli, i mammiferi e, al sommo di questi, l'uomo, sono formati
secondo un archetipo che solo varia in misura maggiore o minore nelle
sue parti più costanti e che, inoltre, si sviluppa e si trasforma
incessantemente [\dots]. Ma sebbene, concordano ormai sulle linee
generali, sembrasse di lavorare tutti ad un unico fine, era
inevitabile che in singoli aspetti sorgessero numerose confusioni.
Infatti, per quanto gli animali possano, nell'insieme, essere simili,
certe singole parti presentano forme profondamente diverse nelle
differenti creature: doveva quindi accadere fin troppo spesso che si
scambiasse una parte per l'altra, oppure che la si cercasse nel punto
sbagliato e che, non trovandola, se ne negasse l'esistenza [\dots].
Afferrata però l'idea di questo Tipo, ci si renderà subito conto che è
impossibile elevare a cànone un singolo genere. Il singolo non può
essere il modello dell'insieme; non dobbiamo quindi cercare nel
singolo un modello universale. Le classi, gli ordini, i generi e gli
individui si comportano come casi nei confronti della legge; sono in
essa contenuti, a non la contengono e neppure la fondano %
\autocite[57-58]{goetheMetamorfosiDeliAnimali1986}.%

\numberedparagraph % 9

Ma ciascun elemento perderà, accumulandosi e combinandosi, il suo
piccolo grado di sentimento e di percezione? Per niente, dice il
dottor Baumann. Queste qualità gli sono essenziali. Cosa accadrà
dunque? Questo: di queste percezioni di elementi messi insieme e
combinati, ne risulterà una percezione unica, proporzionata alla massa
e alla disposizione; e questo sistema di percezione nel quale ciascun
elemento avrà perduto memoria del \emph{sé} e concorrerà a formare la
coscienza del \emph{tutto}, sarà l'anima dell'animale [\dots]. È
adesso che bisogna praticare il nostro metodo all'esame dei suoi
principi. Gli chiederò dunque se l'universo o la collezione generale
di tutte le molecole sensibili e pensanti forma un tutto o no [\dots].
Se si conviene che è un tutto in cui gli elementi non sono meno
ordinati, che le parti, o realmente distinte, o solamente
intelligibili, lo sono in un elemento, e gli elementi in un animale,
bisognerà che ammetta che in conseguenza di questa copulazione
universale, il mondo simile a un grande animale a un'anima; e che
potendo il mondo essere infinito, quest'anima del mondo, non dico è,
ma può essere un sistema infinito di percezioni, e che il mondo può
essere Dio %
\autocite[451-453]{diderotPensieriSullinterpretazioneNatura2019}.%

\numberedparagraph % 10

L'infinito però, o l'esistenza perfetta, non può essere pensato da
noi. Noi possiamo pensare soltanto cose che sono o già limitate o
vengono limitate dal nostro spirito. Perciò abbiamo un concetto
dell'infinito soltanto in quanto possiamo pensare che vi sia
un'esistenza perfetta al di fuori della capacità di comprensione di
uno spirito limitato. Non si può dire che l'infinito abbia delle
parti. Tutte le esistenze limitate sono nell'infinito, ma non sono
parti dell'infinito, esse piuttosto partecipano della infinità %
\autocite[19]{goetheMetamorfosiDeliAnimali1986}.%

\numberedparagraph % 11

Se d'altra parte definiamo soltanto anatomica la nostra impresa,
questa, per essere fruttuosa e perfino, nel nostro caso, per essere
soltanto possibile, dovrebbe venir sempre condotta in stretto rapporto
alla fisiologia. Non si tratta dunque di limitarsi a considerare le
parti nel loro coesistere una accanto all'altra, ma di individuarne il
reciproco vivente influsso, la mutua dipendenza e azione %
\autocite[60]{goetheTeoriaNatura}.%

\needspace{10\baselineskip}
\section{Progresso della natura; progresso nella storia}

\subsection{Lo \bsq{spinozismo vitalista}}

\numberedparagraph % 12

Tuttavia l'incoronamento che manca a questo articolo è l'intuizione
dei due grandi impulsi di tutta la natura: il concetto della
\emph{polarità} e dell'\emph{accrescimento graduale}, quella propria
della materia in quanto la pensiamo materiale, questo invece in quanto
la pensiamo spirituale; quella consiste in un continuo attrarre e
respingere, questo aspira continuamente verso l'alto. Poiché però la
materia non esiste né può mai essere efficace senza lo spirito né lo
spirito senza la materia, anche la materia è suscettibile di
gradazione, così come lo spirito non tralascia di attrarre e
respingere; proprio come è capace di pensare soltanto colui che ha
diviso a sufficienza per collegare e collegato a sufficienza per poter
nuovamente dividere %
\autocite[113]{goetheTeoriaNatura}.%

\numberedparagraph % 13

\textsc{Abbozzo} di fisica sperimentale. La fisica sperimentale si
occupa in generale dell'\emph{esistenza}, delle \emph{qualità}, e
dell'\emph{impiego} [degli esseri]. L'\textsc{esistenza} comprende la
\emph{storia}, la \emph{descrizione}, la \emph{generazione}, la
\emph{conservazione} e la \emph{distruzione}. La \emph{storia} è
storia dei luoghi, dell'importazione e dell'esportazione, del prezzo,
dei pregiudizi, ecc. La \emph{descrizione}, dell'interno e
dell'esterno, attraverso tutte le qualità sensibili. La
\emph{generazione}, considerata dalla prima origine fino allo stato di
perfezione. La \emph{conservazione}, di tutti i mezzi di fissarla in
questo stato. La \emph{distruzione}, presa dallo stato di perfezione
fino all'ultimo grado conosciuto di \emph{decomposizione} o di
\emph{deperimento}, di \emph{dissoluzione} o di \emph{scissione} %
\autocite[421]{diderotPensieriSullinterpretazioneNatura2019}.%

\subsection{L'indagine sulla natura}

\numberedparagraph % 14

Una delle verità che sono state annunciate nei nostri giorni con il
più grande coraggio e forza, che un buon fisico non perderà di vista e
che avrà certamente le conseguenze più vantaggiose, è che la regione
dei matematici è un mondo intellettuale, dove quelle che vengono
considerate verità rigorose perdono completamente questo vantaggio
quando vengono portate sulla nostra terra. Da questo si è dedotto che
era la filosofia esperimentale a rettificare i calcoli della
geometria, e questa conclusione è stata riconosciuta persino dai
geometri. Tuttavia, a cosa serve correggere il calcolo geometrico con
l'esperienza? Non è più semplice attenersi al risultato di
quest'ultima? È evidente che le matematiche, quelle trascendenti
soprattutto, non conducono a niente di preciso senza l'esperienza; che
è una specie di metafisica generale in cui i corpi sono spogliati
delle loro qualità individuali, e che gli resterà almeno da fare una
grande opera che si potrebbe intitolare l'\textit{Applicazione
dell'esperienza alla geometria}, o \textit{Trattato dell'aberrazione
delle misure} %
\autocite[407]{diderotPensieriSullinterpretazioneNatura2019}.%

\numberedparagraph % 15

Un grande compito sarebbe bandire le teorie filosofiche e matematiche
da quelle parti della fisica in cui invece di aiutare sono di
ostacolo, e in cui l'elaborazione matematica, per l'unilateralità
della moderna istruzione scientifica, ha trovato una applicazione così
sbagliata. [\dots]. Bisognerebbe esporre quale è la vera via
dell'indagine della natura, come essa si basi sul più semplice
progredire dell'osservazione, come l'osservazione debba essere elevata
a esperimento e questo infine conduca al risultato %
\autocite[138-139]{goetheTeoriaNatura}.%

\numberedparagraph % 16

Quando i geometri hanno screditato i metafisici erano ben lontani dal
pensare che tutta la loro scienza non fosse altro che una metafisica.
Ci si chiese un giorno: \textquote{Chi è un metafisico?}. Un geometra
rispose. \textquote{È un uomo che non sa niente}. I chimici, i fisici,
i naturalisti, e tutti quelli che si dedicano all'arte esperimentali,
altrettanto estremi nel loro giudizio, mi sembrano sul punto di
vendicare la metafisica e di applicare la stessa definizione al
geometra %
\autocite[409]{diderotPensieriSullinterpretazioneNatura2019}.%

\subsection{Il progresso delle scienze}

\numberedparagraph % 17

Un'altra accusa, che si deve fare contro il procedimento delle
scienze, è che di tanto in tanto talune discipline conquistano una
specie di predominio nella scienza e soltanto il tempo può ristabilire
l'equilibrio. Ciò che è nuovo, la conoscenza di recente acquista,
stimola la partecipazione degli uomini. Uomini, che si sono segnalati
occupandosi in modo egregio di queste discipline, le sviluppano
accuratamente, si fanno degli scolari, collaboratori e seguaci, sicché
una parte determinata di ciò che è intero si gonfia fino a diventare
il punto principale, mentre le altre sono già retrocesse nei loro
confini come partecipi di una totalità.

\numberedparagraph % 18

Le scienze dovrebbero agire sul mondo esterno soltanto con una prassi
elevata, giacché sono propriamente tutte esoteriche, e soltanto se
migliorano una qualche azione pratica possono diventare essoteriche.
Ogni altra partecipazione non conduce a niente [\dots]. Un fatto
notevole, un \textit{aper\c{c}u} geniale occupa un grandissimo numero
di uomini, prima solo per conoscerlo, poi per comprenderlo, quindi per
elaborarlo e portarlo avanti. [\dots] A ogni nuovo fenomeno la massa
chiede a che cosa serve e non ha torto, giacché può afferrare il
valore di una cosa soltanto per la sua utilità %
\autocite[165]{goetheTeoriaNatura}.%

\needspace{4\baselineskip}
\numberedparagraph % 19

Quando una scienza comincia a svilupparsi, la straordinaria
considerazione che si ha in società per gli inventori, il desiderio di
conoscere personalmente una cosa che fa molto scalpore, la speranza di
diventare celebri grazie a qualche scoperta [\dots] spingono agli
uomini in quella direzione [\dots]. Tanti sforzi riuniti portano
rapidamente la scienza fino a dove essa può arrivare. Tuttavia, man
mano che si comprendono i suoi limiti, si restringono quelli della
considerazione [\dots]. Allora la folla diminuisce [\dots]. Io non
faccio eccezione nemmeno per la storia della natura %
\autocite[411]{diderotPensieriSullinterpretazioneNatura2019}.%

\numberedparagraph % 20

Fintanto che le cose restano solo nel nostro intelletto, sono nostre
opinioni; sono delle nozioni che possono essere vere o false,
accordate o contraddette. Esse non prendono consistenza che legandosi
agli esseri esterni. Questo legame si crea attraverso una catena
ininterrotta di ragionamenti, il cui filo si regge da un lato
sull'osservazione, e dall'altro sull'esperienza; o remiate una catena
di esperienze disperse, intramezzate da ragionamenti [\dots] %
\autocite[411]{diderotPensieriSullinterpretazioneNatura2019}.%

\numberedparagraph % 21

C'è solo un mezzo per rendere la filosofia raccomandabile agli occhi
del volgo, è di mostrarla accompagnata dall'utilità. Il volgo domanda
sempre: \textit{\textquote{a cosa serve?}} e non bisogna mai trovarsi
nella situazione di rispondere \textit{\textquote{a niente}}: non sa
che ciò che illumina il filosofo è ciò che se[r]ve al volgo stesso
sono due cose molto diverse, poiché l'intelletto del filosofo è spesso
illuminato da ciò che nuoce, e oscurato da ciò che serve %
\autocite[419]{diderotPensieriSullinterpretazioneNatura2019}.%

\needspace{4\baselineskip}
\numberedparagraph % 22

D'altra parte l'utile circoscrive tutto. Sarà l'utile che tra qualche
secolo fisserà dei limiti alla fisica sperimentale, come è sul punto
di porne alla geometria. Concedo alcuni secoli a questo studio, poiché
la sfera della sua utilità è infinitamente più estesa di quella di
qualsiasi scienza astratta, ed essa è, senza alcun dubbio, alla base
delle nostre vere conoscenze %
\autocite[411]{diderotPensieriSullinterpretazioneNatura2019}.%

\subsection{Il linguaggio e i termini riproduttivi}

\numberedparagraph % 23

Con le parole non esprimiamo pienamente né gli oggetti né noi stessi.
Con il linguaggio sorge quasi un mondo nuovo, fatto di cose necessarie
e casuali. \textlatin{\textit{Verba valent sicut nummi}}. Ma vi è
differenza tra le varie monete. Ce ne sono d'oro, d'argento, di rame e
vi è anche la carta moneta. Nelle prime vi è più o meno realtà,
nell'ultima soltanto la convenzione %
\autocite[38]{goetheTeoriaNatura}.%


\numberedparagraph % 24

Nel primo brano da noi tradotto, d'Alembert paragona una serie di
proposizioni geometriche derivate l'una dall'altra ad una specie di
traduzione da un idioma in un altro che si sarebbe sviluppato dal
prim, in questa concatenazione però sarebbe contenuta soltanto la
prima proposizione, anche se resa più accessibile all'uso: posto e non
concesso che in una impresa, già dubbia di suo, si sia riusciti a
mantenere una estrema stabilità %
\autocite[106]{goetheTeoriaNatura}.%

\numberedparagraph % 25

La filosofia esperimentale potrebbe lavorare per secoli e secoli, ma i
materiali che essa accumulerebbe sarebbero ancora molto lontani da
un'enumerazione esatta. Quanti volumi sarebbero necessari solo per
contenere i termini dei quali ci serviamo per designare gli insiemi
distinti dei fenomeni, se conoscessimo i fenomeni? Quando sarà
completa la lingua filosofica? E quand'anche fosse completa, chi tra
tutti gli uomini potrebbe conoscerla? Se l'Eterno per manifestare la
sua onnipotenza in modo ancora più evidente che attraverso le
meraviglie della Natura si fosse degnato di sviluppare il meccanismo
universale su dei fogli tracciati di sua mano; crediamo forse che
questo grande libro sarebbe più comprensibile per noi che l'universo
stesso? %
\autocite[411]{diderotPensieriSullinterpretazioneNatura2019}.%

\subsection{I diversi modelli di scienza}

\numberedparagraph % 26

Cuvier lavora senza posa come analista, come meticoloso descrittore, e
conquista un sicuro dominio su un territorio immenso. Geoffroy de
Saint-Hilaire, invece, si occupa in silenzio delle analogie fra gli
esseri viventi e delle loro misteriose affinità: quegli va dal singolo
al tutto, che è bensì considerato presupposto, ma come giammai
riconoscibile; questi persegue la conoscenza del tutto nel suo
significato interno, e vive nell'incrollabile certezza che il singolo
potrà essere poco a poco enucleato [\dots]. Sono qui in gioco due
diversi modi di pensare, che nella specie umana si trovano per lo più
divisi e distribuiti in maniera tale che, nella scienza come dovunque,
difficilmente s'incontrino appaiati, ed essendo distinti, non si
lasciano agevolmente ricongiungere %
\autocite[77-78]{goetheMetamorfosiDeliAnimali1986}%

\numberedparagraph % 27

Cuvier si attiene decisamente e in modo, e in modo sistematicamente
ordinatore, al singolo; diciamo sistematicamente ordinatore, perché
una visione complessiva superiore gli suggerisce, anzi gli impone, un
\emph{metodo} di presentazione ed esposizione dei fatti. Conformemente
al suo modo di pensare in termini complessivi, Geoffroy cerca invece
di penetrare nel Tutto, visto però, diversamente da Buffon, non nel
presente, nell'esistente, nel già formato, ma nell'agente, nel
diveniente, in ciò che si evolve [\dots] %
\autocite[89-90]{goetheMetamorfosiDeliAnimali1986}.%

\numberedparagraph % 28

Possa ciascuno di noi, a proposito di questa controversia, dire che
scindere e collegare sono due atti di vita inseparabili. O meglio,
forse, che è indispensabile, lo si voglia o no, andare dal Tutto al
singolo e dal singolo al Tutto, e che quanto più vitalmente queste due
funzioni dello stesso spirito, come l'espirare e l'inspirare,
procedano unite, tanto meglio si opera per il bene della scienza e dei
loro cultori %
\autocite[91]{goetheMetamorfosiDeliAnimali1986}.%

\numberedparagraph % 29

Se ci si limita a vedere soltanto ciò che risponde alla norma, si
finisce per credere che così necessariamente debba essere, che così
esso sia stato determinato \textlatin{\textit{ad ovo}}, e che sia,
quindi, immutabile. Se invece si osservano le deviazioni della norma,
le malformazioni, le mostruosità, si è costretti a riconoscere che la
regola è bensì fissa ed eterna, ma nello stesso tempo viva; che nel
suo ambito, se non per diretta derivazione, gli esseri possono bensì
degenerare nell'informe, ma devono ogni volta, come trattenuti da
briglie, piegarsi all'assoluta sovranità della legge %
\autocite[92]{goetheMetamorfosiDeliAnimali1986}.%

\numberedparagraph % 30

La stretta parentela fra scimmia e uomo costrinse i naturalisti a
minuziose riflessioni, e l'insigne Camper ritenne di aver scoperto la
differenza fra scimmia ed uomo nel fatto che quella è dotata di un
osso intermedio del mascellare inferiore e a questo, invece, esso
manca. Non posso che esprimere il mio dolore nel trovarmi in aperto
contrasto con lui verso il quale avevo contratto un debito così forte
[\dots] %
\autocite[94-95]{goetheMetamorfosiDeliAnimali1986}.%

\numberedparagraph % 31

Ora Geoffroy ha giustamente individuato e formulato l'importante
principio -- dal quale non si deve mai prescindere nell'osteologia --
secondo cui ogni particolare osso che sembri celarsi alla nostra
indagine può sempre essere scoperto con la massima sicurezza nella
regione in cui è situata la sua sede. Egli inoltre è convinto di
un'altra fondamentale verità, direttamente rapportabile al tema che
andiamo svolgendo; vale a dire che la natura, ubbidendo a un principio
di economia, si è fissato un bilancio preventivo e, pur riservandosi
per i singoli capitoli la più assoluta libertà, ad esso si attiene
fedelmente nella loro somma totale, e se in una parte ha speso troppo,
nell'altra risparmia, ristabilendo così nel modo più scrupoloso
l'equilibrio %
\autocite[103]{goetheMetamorfosiDeliAnimali1986}.%

\numberedparagraph % 32

La \emph{meraviglia} sorge spesso dal fatto che presupponiamo numerosi
prodigi laddove ce n'è uno solo; dal fatto che immaginiamo nella
Natura tanti atti particolari quanti sono i fenomeni, quando forse
essa ha prodotto un solo atto. Sembra anche che, se avesse avuto la
necessità di produrne numerosi, i differenti risultati di questi atti
sarebbero isolati; ci sarebbero delle collezioni di fenomeni
indipendenti le une dalle altre, e questa catena generale di cui la
filosofia presuppone la continuità, si romperebbe in numerosi pezzi.
L'indipendenza assoluta di un solo fatto è incompatibile con l'idea
del tutto; e senza l'idea dtotalitàel tutto, non c'è più filosofia %
\autocite[413]{diderotPensieriSullinterpretazioneNatura2019}.%

\needspace{20\baselineskip}
\pagestyle{plain}
\printbibliography
\section*{Contenuti}
\tableofcontents

% \section{\textquote{Metamorfosi deli animali}}
%
% \subsection{La scienza e il tutto}
%
% Ma come non potremmo sentirci spronati a soddisfare questi desideri,
% queste speranze dei naturalisti, noi che, se resteremo fedeli alla
% visione del Tutto, potremo attenderci innumerevoli soddisfazioni e,
% insieme, molteplici vantaggi per la scienza? %
% \autocite[53]{goetheMetamorfosiDeliAnimali1986}%
%
% \subsection{L'archetipo}
%
% Saremmo così giunti a poter affermare senza esitazione che tutti gli
% esseri organici più complessi, fra cui i pesci, gli anfibi, gli
% uccelli, i mammiferi e, al sommo di questi, l'uomo, sono formati
% secondo un archetipo che solo varia in misura maggiore o minore nelle
% sue parti più costanti e che, inoltre, si sviluppa e si trasforma
% incessantemente %
% \autocite[57]{goetheMetamorfosiDeliAnimali1986}.%
%
% \vspace{1\baselineskip}
%
% Ma allora, è realmente impossibile, avendo riconosciuto che l'energia
% creativa produce secondo uno schema generale le nature organiche più
% complete, presentare questo archetipo se non ai sensi almeno allo
% spirito, utilizzandolo come norma per completare le nostre descrizioni
% e, dopo averlo dedotto dalla forma dei diversi animali, ricondurre
% nuovamente ad esso le forme più diverse? %
% \autocite[58]{goetheMetamorfosiDeliAnimali1986}%
%
% \subsection{Il cambiamento \emph{impercettibile} della natura}
%
% L'armonia di un tutto organico è infatti resa possibile proprio dal
% suo essere composto da parti identiche che si modificano attraverso
% mutamenti quasi impercettibili %
% \autocite[68]{goetheMetamorfosiDeliAnimali1986}.%


\end{document}
