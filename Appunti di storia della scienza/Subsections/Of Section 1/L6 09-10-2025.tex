% !TEX root = ..\..\main.tex
\documentclass[../../main.tex]{subfiles}%
\ifSubfilesClassLoaded{\addbibresource{../../bibliography.bib}}{}

\begin{document}

Manoscritto azteco, un codice che è presente a Firenze, e fa parte di
tutt'una serie di codici riguardanti la cultura azteca che vengono
riprodotte dai conquistadores. Questo perché fondamentalmente il
concetto principale di Bosket è l'idea che quella che noi chiamiamo
rivoluzione scientifica e scienza moderna non sia un affare europeo.
Con Shapin abbiamo visto che si può rifiutare la categoria stessa di
rivoluzione scientifica, ma piuttosto si può dire che non esiste un
concetto centrale, ma si ha tutt'una serie di cambiamenti che si
intrecciano con elementi di continuità. Bosket parte dal presupposto
che effettivamente c'è qualcosa di nuovo nel Cinquecento, ma non si dà
solo in Europa. Si ha anche la Spagna e il Portogallo che partecipano
alle invenzioni con scoperte geografiche, e che sono, secondo questo,
aiutano a decentralizzare le cause della nascita della scienza
moderna. Questo movimento non è perché c'erano geni europei a lavoro,
ma perché essi incontrano altri saperi, astronomi africani, i cinesi,
gli aztechi. Gli europei hanno un nuovo sguardo, e quindi si mette in
crisi appunto l'autorità degli antichi. Quindi di fatto la nascita
della rivoluzione scientifica va compresa come uno scambio globale,
tenendo conto delle relazioni di sfruttamento che fanno parte e hanno
in qualche modo ripercussione non solo in quel contesto. Uno che si
chiama \textit{Miles of Empire} che dettaglia una serie di innovazioni
nella medicina come la fisiologia, il metabolismo, anche l'aria --
quanta aria serve per funzionare -- che non è lo spirito vitale di
Galeno, quanti metri cubici d'aria servono per vivere? E questo serve
alle navi e alle prigioni. E come si fa? Si mettono lì nelle navi con
delle caratteristiche e si vede quanti arrivano vivi. È chiaro che
l'idea di fare sperimenti di qualsiasi tipo dipende dalla possibilità
di farli. In molti casi, il fatto di poter mettere mani su cose nuove,
nuove piante e animali permette fare osservazioni diverse, di prima
mano, non sono gli sperimenti ideali sulle cause di Aristotele. Sono
favorite e sono richieste da queste istituzioni che sono coloniali ed
imperialisti.

Ora, questa roba qui, prima di provarla sui bambini europei, si fa
sugli schiavi e soldati. Uno schiavo vaccinato lo vendo meglio, un
esercito che non è suscettibile a determinate malattie batte meglio.
Molte pratiche mediche si diffondono ad esempio nell'esercito
americano nelle guerre contro gli inglesi, e tutto fa parte di un
dominio coloniale di migliorare le tecnologie dell'epoca, e quali
erano le tecnologie delle piantagioni del sud degli Stati Uniti? Gli
schiavi. Si creano le condizioni e in alcuni casi vengono causate
direttamente. In particolare Bosket chiede di allargare lo sguardo e
il sapere che circola. Quell'immagine lì è d'un geografo in cui il
nord è sotto. Non aveva particolari problemi ideologici, ma
semplicemente mettere il nord verso l'alto non è sempre stato
scontato. 

\textit{Historia...} (?) sede a costa spagnola, compagnia di Gesù, va
in Messico e racconta quello che accade, e non è il primo, è un'opera
del tardo Cinquecento. Opera enorme, racconta un sacco di cose, anche
di storia naturale, scritto da uno che effettivamente va nel nuovo
mondo e chiacchiera con tutti. La prima cosa che nota è che, da un
lato, molte di quelle cose che arrivano nel Nuovo Mondo, (?), l'altra
cosa è che c'è un problema. Aristotele ha detto che attorno
all'equatore fa troppo caldo e non si può vivere, invece in questo
mondo ci si può trovare ad avere freddo nell'equatore. Aristotele
ragionava da un punto di vista mediterraneo, e ci ragiona in termini
della sua cosmogonia, della sua filosofia, e però ad un certo punto
questi nuovi esploratori, com'è che la regione che chiamiamo torrida è
così. Lui da subito si rende conto che certe cose di Aristotele vanno
prese con le pinze. Ieri si è visto appunto che certe cose della
filosofia scolastica e della fisica fossero estremamente connesse alla
religione; non ci si può avvicinare al copernicanesimo senza
questionare questo, e qui abbiamo un gesuita che mette in dubbio
queste cose, ed è un gesuita che dice: andiamo a vedere le cose e
riportiamole con le nostre esperienze. Lui invece va là, impara la
lingua dei locali e la fa traslitterare nell'alfabeto, e da lì lui ed
altri interagiscono con loro più o meno come faranno i gesuiti nel
Seicento con i cinesi. Si impara la conoscenza, si cerca di mediare un
po' di fare da tramite, e concetti che vengono da lì vengono tradotti
in concetti che possiamo capire, prendono aspetti della medicina
cinese che assomigliano a quella galenica. Comunque si immergono
dentro il sapere locale, ed è un sapere completamente nuovo per loro,
gli aztechi possiedono medici professionali, e non sono stregoni, e
così via, danno nomi alle piante, alle malattie, tutto il contrario di
una società primitiva e inferiore a quella occidentale. Il primo
impatto che si dà è qualcosa di simile a noi, qualcosa meritevole di
rispetto che ci può insegnare un sacco di cose, e se impariamo la loro
lingua le insegnano e lo fanno meglio. C'è un orto botanico quando
arrivano a Tenochtitlan, città grandissima in confronto con quelle
europee, e c'è uno shock della scoperta che rivela un sacco di cose
non presenti nei libri degli antichi, piante, ecc., che portano i
naturalisti europei a farsi le domande, e non è solo andiamo a
esplorare, però andare a chiedere le persone della loro cultura, ed è
la stessa esistenza di quelle culture che diventa meravigliosa.

Arrivare in questo mondo completamente nuovo e diverso, con persone
importanti, nobili, strutture sociali importanti e ci si riconosce
anche dal punto di vista antropologico e sociale, ci si riconosce tra
i nobili spagnoli e i nobili aztechi. Montezuma sposa le sue figlie ai
nobili spagnoli, o forse meglio viceversa: gli Spagnoli gli chiedono
di avere le sue figlie, e non lo chiedono a chiunque, lo chiedono
all'imperatore, così come succede nella nobiltà europea, cosa che è
sorprendente considerando che spesso i gesuiti e i cattolici si
domandano se hanno l'anima come noi, se vale la pena convertirli, o se
sono come degli animali e così via. Sono domande che nascono con
queste nuove scoperte, che producono un nuovo sapere perché fanno
queste nuove domande, e in generale si mette l'enfasi su
quell'esperienza e osservazione che tipicamente si associa alla
rivoluzione scientifica in Europa, Bacon, l'induzione, la riforma
protestante, però abbiamo poi questi gesuiti che dicono di non
raccontare delle indie se non ci si è stati, lo facciamo noi. L'enfasi
sull'osservazione non nasce solo da Bacon, ma anche con questi
gesuiti che vanno lì, e lo fanno in tanti anche per mostrare che c'è
un impero ben organizzato e ricco che sa un sacco di cose che ci
possono tornare utili. Si spera un equilibrio della natura, il
guayabo, ci si rende conto della sifilide, e andiamo a trovare il
farmaco, che Dio ha fatto e nella sua bontà e che si trova nello
stesso posto. Poi nel Seicento arriva la corteccia di... Contro la
febbre... (?). 

Si mette in crisi l'autorità e allo stesso tempo c'è un'enfasi
sull'osservazione. Si ritrova questo orto botanico di Montezuma, e per
capire tutta questa roba ci dev'essere qualcuno che traduce,
osservando prima, riportando tutto ciò. Colonna sinistra spagnolo,
colonna destra in Nahuatl, con immagini fatte da aztechi in cui c'è di
tutto, la prima descrizione è quella del colibrì, che è una cosa
incredibile (?). Questo ce lo raccontano gli aztechi, come si facevano
le tortillas con il mais. L'altra cosa fondamentale è che la tortilla
di mais è fatta secondo un metodo particolare e peculiare, la farina
di mais viene messa in un liquido molto basso che permette di
rilasciare un chimico, che poi previene la malaria, e invece quando si
porta in Europa, i contadini prendono la malaria perché hanno un
deficit alimentare, e queste cose si scoprono a inizio Novecento:
perché non si fa questo come la fanno loro? Anche perché si impone un
sistema in cui i contadini nel nord Italia mangiano solo polenta, ma
comunque guardando lì si vede come si fa il mais. Ovviamente in queste
pagine sono descritte una serie di cose. Quando arriva in Italia nel
Cinquecento, se fosse medicinale i medici non sarebbero tutti
d'accordo.

Cambia medicina, idea del corpo, e anche cosa mangiare. Gli usi medici
del pomodoro comunque vengono riportati lì. Quando si parla di
rivoluzione scientifica si va a cancellare queste cose. Poi ovviamente
gli spagnoli da un lato hanno devastato l'impero, dall'altro si
parlava dei sacrifici umani, ma in templi bellissimi, su un calendario
avanzato, ecc. E si creano istituzioni anche in cui si produce sapere,
in cui queste cose fermentano, si producono tutt'una seire di cose, il
Real Colegio de Santa Cruz, alle periferie di Tenochtitlan in cui i
francescani vanno lì, fanno studiare anche i locali -- tentando ci
cristianizzare l'impero, di farli imparare il latino --, e però
conosciamo le loro cose, la loro lingue, ci affidiamo a loro per
imparare la loro cultura e vediamo cos'hanno d'utile per noi. E questi
sforzi danno origine a nuove istituzioni anche in Europa. A un certo
punto diventa un farmaco diffusissimo in Europa. Poi i primi orti
botanici, il primo universitario è quello pisano. Francisco Hernandez
torna dal Messico, dà l'idea di guardare le cose di prima mano, senza
affidarsi ai racconti di qualcuno, nuova idea di che cos'è il mondo,
l'armadillo che ha un nome azteco, ed era un racconto in Europa, si
pensi ai quattro fiumi di Bernini e nel Rio de la Plata c'è anche
questo. Attanasius Kircher, collegio romano, ha visto la corazza
dell'armadillo, ma poi che cosa fosse realmente Bernini non lo sa. 

La disponibilità di tutti questi nuovi materiali contribuisce all'idea
dello sfruttamento, nuovo mondo, materie prime abbondanti nuove. Gli
uomini muoiono presto, nei primi decenni l'impero azteca è devastato
dalle epidemie, non solo quelle venute dagli occidentali, ma anche una
che non conosciamo, che sicuramente va a innescarsi in una popolazione
debilitata dalla conquista spagnola, che in due grosse epidemie spazza
via una grande parte della popolazione azteca. Spedizione di gaffetta
(?), navigatore argentino, sono un po' rozzi, ma alla fine fanno le
stesse cose che facevano i medici in Europa.

Si hanno poi le varie tribù d'Israele che si spargono per il mondo, ma
questi chi sono? Cosa sono? Sono umani come noi? Vale la pena
convertirli? Vale la pena rispettarli? Hanno più o meno diritto di
noi? E qui è anche una questione antropologica, è una domanda che
costringe agli europei a fare domanda, nuova cultura sulla
fisliologia e cose dell'ambiente, e lo fanno rimanendo alla fisiologia
galenica ed aristotelica, oppure vedendo che c'è un'influenza
dell'ambiente sulla fisiologia, e questa sarà una questione
fondamentale per i colonialisti: rischiamo di diventare come loro?
Appunto l'idea dell'influenza del clima, dell'ambiente generale. Se
siamo natu uguali nella creazione, perché siamo così diversi nel
tempo? Queste domande verranno affrontate sempre di più dal punto di
vista scientifico. Altre cose è iniziare a vedere nel Messico e come
alcune cose dell'eredità si misurano in maniere diverse. Se è figlio
di un europeo e d'un indio, o d'un africano e così via, se ha sangue
spagnola ma è nato lì si ha un altro nome, un criollo. È un nuovo modo
di guardare agli umani a seconda del loro sangue, che non sempre ha
fatto parte della nostra idea di nobiltà, il sangue non è sempre stato
importantissimo, il concetto di famiglia per i romani era molto più
sfumato, mentre invece in questa nuova società chiaramente c'è una
mescolanza molto più grande, creata subito con prerogative sociali,
appunto il figlio ci Cortez spossa il figlio dell'imperatore, poi
questa mescolanza un po' si perde, questa nobiltà percepita. Si
comincia a parlare di \emph{eredità biologica}, mentre l'eredità erano
le terre e i soldi, chi eredita cosa? Pian piano si associa all'idea
di sangue.

I portoghesi e spagnoli hanno un modello Reale di carta che
conoscono solo certe persone. Prima carta post-magallanes.

% Citazione si Scienze e nazioni. Parla di chi è a casa e aspetta
% notizie dei viaggiatori. Rivoluzione copernicana, cosa accade a una
% disciplina nel momento in cui diventa cumulativa ed entra nel
% cammino della scienza, sono le cose per dare alla mente... cose come
% Copernico, e i cartografi diventano coloro cui le cose girano
% attorno, sta parlando della sale nautiche dell'impero francese.
% Invece di essere dominati nella natura, fanno sale nautiche dove
% hanno le aree geografiche, dimensione della terra non più grande di
% un atlante, il cartografo domina quel mondo che a sua volta domina i
% navigatori, la cartografia diventa scienza... L'Europa al centro. 



\end{document}
