% !TEX root = ..\..\main.tex
\documentclass[../../main.tex]{subfiles}%
\ifSubfilesClassLoaded{\addbibresource{../../bibliography.bib}}{}

\begin{document}

L10, 23-10-2025

L'idea che fra la pubblicazione del \textit{De revolutionibus} e l'a
bubblicazione dei \textit{Principia} di Newton nasca una cosa come la
scienza moderna. Questa categoria storiografica sembra che se la sia
inventata Alexandre Koyré. Invece Secord ci dice che in realtà la
rivoluzione scientifica resta come momento fondatore della modernità
che invece verrà qualcosa di cattivo nel postmoderno (?) e si vedrà
perché. La tesi di questo articolo però è che la tesi non se la sono
inventati storici della scienza europei, ma durante in New Deal gruppi
di autori di libri di divulgazione scientifica -- e che quindi hanno
molta più influenza di questi storici, che li leggono solo gli
accademici. Ecco perché Secord è preoccupato, tant'è vero che alla
fine dell'articolo parla del motivo per cui in qualche modo bisogna
vedere l'idea di rivoluzione scientifica, e che è un fenomeno
europeo. La tesi di Secord è in fondo questa: il motivo per cui è così
arrampicata, il mito di Galileo, il pendolo, il Newton e la mela,
quelli restano nella mente della gente a prescindere, e quindi a
partire da quest'idea che non può essere stato uno storico della
scienza a far spargere questo. Non può essere stato un lavoro di alta
cultura intellettuale, ma che è entrato più profondamente
nell'opinione pubblica, e quindi rintraccia l'idea di scienza e
progresso all'interno dei gruppo di intellettuali che seguono il New
Deal di Roosevelt, che è precisamente la risposta che
l'amministrazione Roosevelt darà alla crisi del 29 con un sistema
enorme di lavoro sociale. Ora, in questo periodo di estrema diffusione
di idee progressive e democratiche, Ornstein pubblica una tesi di
dottorato in cui comincia a usare quest'espressione, pubblica questo
libro, \textit{Mind in the making} il quale diffonde queste idee, che
viene stampato in milioni di copie, e si ha quest'immagine che quando
sbarcano in Normandia, Tom Hanks ha una copia di questo libro, e così
si diffonde quest'idea, che la scienza sia fondamentale per la
rivoluzione industriale e subito dopo quelle politiche, è un'idea
chiave e progressista che poi ha un aspetto molto importante nella
ricostruzione postbellica della decolonizzazione.

Il punto è che questo gruppo di intellettuali, uno dei principali dei
quali è Dewey, propongono un'idea della scienza come motore di
progresso che poi sarà fondamentale nel mondo della decolonizzazione,
e un altro personaggio, Joseph Needam, nel momento in cui la storia
della scienza non è una disciplina accademica, ma coltivata
soprattutto da scienziati amateur, e non in facoltà di storia della
filosofia della scienza, che nasce nel Novecento come disciplina
autonoma. Lui è un biologo il quale si trova nella missione UNESCO e
si rende conto dell'influenza della cultura cinese, e pubblica una
storia della scienza in Cina, e però lui fa tutto questo non per pura
curiosità intellettuale, ma perché è marxista, appartiene a questo
gruppo di intellettuali britannici marxisti, e si pone la domanda di
perché è accaduta in una certa parte del mondo. Perché è avvenuta lì e
non in Cina? Questo non è per mostrare la superiorità occidentale, ma
recupera la tradizione del pensiero cinese, e questo perché nel
dopoguerra, nel momento in cui finiscono gli imperi coloniali, si
creano al contempo una serie di istituzioni, tra cui UNESCO i quali si
chiedono come fare uscire questi paesi dal sottosviluppo, e però
adesso, secondo Secord, la cosa si è volta contro quest'agenda
progressista, e cita alla fine uno storico attuale molto famoso,
Ferguson, il quale ha scritto un libro, \textit{Civilization, the West
and the rest}, da cui sono tratte una serie di video, e lì si trovano
le sue lezioni. C'è una superiorità dell'Occidente che si è costruita
a partire da 6 pilari, tra cui la scienza. Ecco perché la rivoluzione
scientifica è importante, e racconta che i turchi ottomani vengono
sconfitti, e però la superiorità della scienza occidentale si basa su
due cose, i \textit{Principia} di Newton, e il \textit{Trattato sul
governo civile} di Locke. Quando viene avvistata una cometa, i turchi
chiudono l'osservatorio astronomico, mentre in Occidente c'è Tycho
Brahe. Ecco perché c'è una ragione perché Secord e molti storici della
scienza sono molto diffidenti di questa categoria, perché racchiude
la superiorità dell'idea dell'occidente, e quando si parla di questa
rivoluzione, Descartes, Galileo, Bruno, tutta questa gente sono tutti
maschi, bianchi, relativamente giovani, occidentali, cristiani, e nel
corso della rivoluzione scientifica Boyle, Hobbes, tutti uomini, tutti
bianchi, Anne Conway, sono figure minori, e poi More?

Quello che ci spiega Shepard è che in realtà non c'è mai stata una
rivoluzione scientifica. L'inizio dice che non c'è un momento coerente
in cui si rivelano a un ancien regime aristotelico e Galielo fa un
metodo, Descartes fonda la filosofia moderna, problemi irrisolvibili
come mente e corpo, e però perché non solo non c'è un momento coerente
perché tutti questi personaggi discutono tra di loro, ma anche nella
rivoluzione russa, eppure continuiamo a usare l'espressione, però c'è
un altro aspetto, che quest'idea si è basata su figure chiavi tra gli
scienziati, e ci fa vedere che questi artigiani, questo sapere dei
mercanti, tutto questo viene lasciato fuori. Il grande storico della
scienza che scrive in un libretto degli anni Trenta critica che (???r)
l'attività scientifica è senza dubbi cumulativa, e quindi è scrivere
la biografia intellettuale dei grandi scienziati. Personaggi che non
hanno lasciato Galielo, la natura dei grandi. Alla fine un punto di
vista matematico, ecc., però appunto questo dice no, ma che è una
disciplina molto importante, perché attraverso questi grandi uomini e
le loro scoperte, si capisce che la storia della scienza è continua,
cumulativa, progresso. Naturalmente negli ultimi anni questo si è
rivisto (r), pecca di anacronismo, e questo finisce per non farci
comprendere molti aspetti del passato. Questo mondo invece lo stiamo
sottovalutando se consideriamo solo i grandi personaggi.

Chi inventa il cannocchiale? Fate qualcosa di utile, a cosa serve alla
società? Se non serve nessuno lo finanzia, e Galielo dice che scopre
questo cannocchiale che serve a motivi militari, e quando scoprono gli
storici che l'hanno inventato due artigiani olandesi, lo fanno
sostanzialmente come gadget, e si diffonde il possesso di oggetti che
come mappamondi, orologi, ecc., che sono simbolo di status sociali,
dice Venezia, questo non l'hai inventato tu. Su Descartes un po' meno,
su Newton si vedrà che Descartes ha bisogno di spiegare la legge di
rifrazione, che è fondamentale per costruire telescopi, la rifrazione
tra le vari lenti e quella atmosferica, e per fare questo si deve
appoggiare a un artigiano. Quindi è vero che questi aspetti sono stati
studiati, però quest'idea che siano stati maschi bianchi cristiani i
motori di qualcosa che la scienza moderna ha portato a un vantaggio
dell'occidente al resto del mondo radica nella nozione di rivoluzione
scientifica. Ora, ci dice soprattutto perché appunto se all'epoca del
New Deal questo aveva una visione progressista.

Giordano Bruno, si hanno tonnellate di documenti del processo, di
Menocchio si sa poco, e si sa grazie a i verbali dei processi
inquisitoriali, e però questi erano forme di conoscenza popolare molto
importanti. Se studiamo solo Galileo e Descartes si perdono cose
importanti della cultura popolare, però aveva letto qualcosa, ai
mercati si parlava, la gente discuteva, e Ginsburg trova nel Corano,
viaggi di Mandeville, ecc., tutti questi frammenti menocchio li mette
insieme, quindi i meccanismi di diffusione dell'idea sono vari.
Copernico aveva letto su libri di astronomia dell'epoca e si
trasmetteva tramite due fondamentali tipi di testi, la
\textit{Sphaera} di Sacrobosco, nel quale spiegava l'astronomia
elementare, ma non con precisione matematica, però all'inizio servono
come testi chiavi, e però non è un trattato, Sacrobosco spiega
argomenti di astronomia elementare, però la conoscenza. Ora, perché
non si trattava solo di un trattato? Se ne fanno tanti fino al
cinquecento sulla sphaera, si fanno riferimento a punti diversi. Uno
per esempio è un signore a Siena, Alessandro Piccolomini, che non è
un commento, ma una parafrasi in volgare per una donna che il latino
non lo masticava, e questo per dire che i meccanismi di trasmissione
sono indiretti, e anche noi adesso, quanti di noi abbiamo l'etica
protestante o lo spirito del capitalismo, però la tesi di Weber la
conosciamo. Ed è anche questo uno degli aspetti di questo articolo di
questo, che le idee circolano nell'opinione pubblica. È difficile
uscire dall'immagine di Newton come uno scienziato puro, i testi negli
anni Trenta in Newton, alchimia, numerologia, ecc., e i suoi interessi
principali erano lì. Poi ad un certo punto Newton, uomo molto
fanatico, viene nominato capo della Royal Society, e lui introduce
implicitamente il gold standard. Essendo l'oro il bene più stabile
dell'argento, questo stabilisce la stabilità della Gran Bretagna
permette l'impero mondiale degli inglesi.

Normalmente si pensa che sia un'invenzione di Koyrè, e ad un certo
punto si trova qualcosa citato che sembra strano che questa scuole a
New York... perché ci interessa in questo luogo che non caga nessuno?
E invece ci interessa perché è una scuola molto progressista che
implementa l'idea di Dewey e lì studiava Kuhn, e quindi lì ci dice
Secord, Kuhn assorbe. Perché è importante questo signore? È uno dei
lobri di filosofia della scienza più importanti del secolo. Il
paradigma è il periodo storico in cui si lavorava ai rompicapi,
risolvere i problemi all'interno del sistema, poi arrvavano delle
contraddizioni interne, si trovavano errori, non si riusciva a
soluzionare, e si arrivava al cambio di paradigma, ma suddividere
tutto in tre momenti è difficile. Questo spiega molti fenomeni,
matematica complicata, anomalie, sti pianeti si comportano in maniera
strana, non stanno dove dovrebbero essere osservati, e si cercava di
risolvere questo dentro il paradigma di Tolomeo. Se si pensa di avere
il centro dell'universo, la cosa non funziona, e invece questo
pianeti. Ma se mettiamo invece il sole e i pianeti intorno al sole? E
in questo caso possiamo interpretare questo. Copernico fa una
rivoluzione, ma poi sarà una cosa molto complessa, e fa una serie di
innovazioni, ma perché a questo punto non si considera questo modello
come quello del progresso della scienza, comincia a leggere Koyré in
francese, e attraverso questo arrivò a pensare: perché la scienza non
potrebbe essere tutta la ricerca scientifica? Crea problemi nello
establishment culturale del New Deal in cui questo era nato. La scuola
progressista di Kuhn, e nel 95, tiene una lunga intervista a tre
studiosi greci nel quale racconta tutto questo, si sentiva un po' a
disagio in questo ambiente troppo di sinistra, però a un certo punto
questo signore che è uno dei grandi esponenti di questa cultura
liberale che crea la categoria di rivoluzione scientifica, ora, ci
racconta appunto Kuhn come arriva alla formazione delle sue idee,
immerso in questo ambiente culturale. Perché a un certo punto, essendo
uno studente di dottorato di fisica, e dovendo comunque sbarcare in un
area in qualche modo, si è messo a insegnare un corso sulle grandi
figure della scienza, e sono quelle che determinano il progresso
scientifico e portano avanti quel valore dell'internazionalismo
liberale-scientifico, che saranno valori che soprattutto una parte
degli scienziati teneterà di fare nella competizione dell'Unione
Sovietica, e questo ambiente in cui difendere il liberalismo e i
valori scientifico, a Kuhn mettono a fare questo corso, il quale,
James B. Conant, e arriva all'idea che bisognava formare anche gli
scienziati con questa visione progressista della scienza, incarica lo
studente di fare questo corso, e si mette a leggere Aristotele,
Newton, Descartes. Com'è possibile che una persona che ha scritto cose
come la Politica, l'Etica, ecc., però queste idee sono talmente
interessanti che tutt'oggi si studiano a filosofia politica, e com'è
possibile che con la filosofia naturale abbia scritto solo sciocchezze
(esplode perfetti)? Però Kuhn dice che il problema è che se noi
prendiamo l'insieme delle sue idee, il paradigma, si spiegano
moltissimi fenomeni naturali, per cui si afferma per più di mille
anni. E questo non piace a Conant, già l'uso della parola rivoluzione,
ecc., e Kuhn dà quest'idea, che tutta la scienza passa per paradigmi,
crisi e rivoluzioni, però c'è un altro aspetto, Structure di Kuhn, e
questa storia della scienza... è un personaggio strano, perché ci
parla della scienza come una serie di cambiamenti concettuali? Com'è
possibile che si parli nel 1957 di questo? Lo Sputnik, e prima nel 56
la repressione dell'Unione Sovietica, e l'altro la nazionalizzazione
del canale del Suez. I sovietici possono competere con noi, non è
tranquillizzante la cosa, e di questo parliamo. Ora, nel 1962 è anche
un anno importante, nei primi anni Sessanta c'è la crisi dei missili
di Cuba, l'uccissione di Kennedy, e questo ci parla che la scienza è
un pradigma... noo, la scienza è anche tecnologia militare, ideologia,
ecc. E soprattutto nel 61 c'è qualcuno che ha avvisato gli americani
che quest'unione tra scienza, industria e militari è pericolosa nel
paese. Arriva, fa un discorso, si pensa a quello che dovete fare, però
c'è un discorso del presidente in uscita, ma ci si presta poca
attenzione. Eisenhower fa un discorso in cui comincia a parlare di una
cosa che non se l'inventa, ma il senatore (?) che è il complesso
militare-industriale, e dice che il pericolo è la non-richiesta
influenza di esso. E non è che i militari stavano antipatici ad
Eisenhower, ma ci dice che una volta è il contratto federale quello
che ha sostituito la ricerca scientifica. -- Si occupava del reverse
engineering, e ci racconta in quest'intervista che questa parte della
bomba atomica, i radar non interessano più e si dedica alla storia
della scienza. Uno che sa com'è la ricerca scientifica dell'epoca e
com'è connessa con l'elemento militare, parla di paradigmi, ecc., e il
libro di Thomas Kuhn, una nota per le conseguenze della teoria della
relatività generale, ed è che la luce, passando vicino al campo
gravitazionale del sole non solo si piega, ma anche perde velocità.
Naturalmente Einstein non lo poteva provare, ma sarà provato al
Lincoln laboratori, che si hanno radar enormi per studiare i missili
Sovietici, e a questo punto riescono a provare tutto questo, quindi sa
perfettamente cos'è questo complesso militare-industriale. È vero che
Conant se la presse, ha trasformato la scienza in un'attività
meccanica di rompicapi, quando è realmente razionale, però al contempo
Kuhn dà un'idea di progresso scientifico che però scompare, però nella
prossima volta si parlerà di un altro aspetto. Serie di storici che
Secord cita, e l'avevano compresso anche perché questo complesos
l'hanno vissuto.

\end{document}
