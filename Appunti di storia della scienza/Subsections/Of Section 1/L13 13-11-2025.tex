% !TEX root = ..\..\main.tex
\documentclass[../../main.tex]{subfiles}%
\ifSubfilesClassLoaded{\addbibresource{../../bibliography.bib}}{}

\begin{document}

L13 13-11-2025

Problema a livello accademico nei paper sul global health, come la
conoscenza porta all'equità. Come la conoscenza deve lavorare, esiste
nella scienza biomedica, ma non nella salute globale. Marginalizza, il
ruolo dell'apprendimento sociale per l'equità. Cercare di csotruire
ordina vloriale diverso, principio di sussidiarietà, subsidiarity.
Utilizzato con il significato di aiuto, e che cosa consiste questo
principio. Riguarda le decisioni per ottenere il bene comune,
organizzazioni possibili, e prevede comunque una prossimità morale e
fisica, e per molti critici questo viene visto come un peso.
Riprendendo la distinzione da un economista prende due unità,
conoscenza salute globale, difensori dei diritti, ecc.

Bene o male i professori non tengono conto delle cose delgi
emancipators e plumbers, ma solo scambio su scala nazionale e globale.
4 regole della sussidiarietà. Posizione di questa figura e il loro
ruolo. 

Menziona all'articolo sulla giustizia epistemica.

L'organizzazione istituzionale poi porta a definire gerarchie
immediate e pratiche nella risoluzione di problemi. 

Relativismo e autodeterminazione.

--- \textit{The knowing world. A new global history of science}

Risorgere di un uso strumentale della storia scienza e del progresso
scientifico alimentato dalle narrazioni nazionalistiche. Caso cinese,
anche l'accademia cinese propone una narrazione di se come avanguardia
del mondo scientifico, e capace di scalzare l'egemonia occidentale nel
progresso scientifico. Dimostrazione della superiorità tecnica e
scientifica e territorio adatto. Propaganda partito nazionalista
indiano che tenta di rintracciare in passi della mitologia indiana un
sapere scientifico perso a causa dell'intervento occidentale. Due
interventi, vede nella teleologia nella conoscenza enciclopedica, il
modo di fare scienza che nell'Occidente si è incarnato e che si prende
come un unico modello.

Modello contestualista, tornare al sapere locale con un approccio
relativista. Modello che si distacca da entrambi, che sia globale, e
che si è caratterizzato a questi due approcci, pluralista, non
considerare il valore della singola scoperta e del singolo
personaggio, ma e figure storiche nella loro globalità, la figura di
Newton nella sua pluralità e presentista. L'uso deve servire non solo
per un fine storiografico, ma per analizzare le relazioni del
presente. Il suo approccio si struttura attraverso quattro passaggi:

Visione geopolitica della storia della scienza, non teleologica e
finalizzata, non estrapolata dal proprio ambiente socio-culturale.
Pluralismo, teleologismo, ma neanche contribuzionismo. I saperi
vengono pressi in considerazione, ma sempre nello sviluppo di una
scienza data, mantenendo rigorismo e attenzione alle dinamiche del
presente, come l'uso propagandistico dello sviluppo delle scienze e
delle nuove invenzioni. Pone attenzione non alla scoperta o
invenzione, ma a come questi saperi siano stati trasmessi, e come
hanno sviluppato l'avanzamento della scienza, e la visione e la
narrazione di una comunità. La trasmissione del sapere antico-romano
poi traslata nel mondo arabo e recepita al mondo ottomano si è
trasmessa e recepita dal mondo occidentale con caratteristiche
orientalistiche. Propone un ex cursus sempre fondato su un aspetto
geopolitico, e per lui, l'aspetto geografico è importante per
l'aspetto presentista della storia della scienza. Vivacità nel sapere
malgrado legati a contesti militari.

Problematiche del paper. Differenza tra accademismo e la divulgazione
e ricezione di massa. La proposta è efficace nell'impianto, rimane in
ambito molto accademico, costruire una storia globale condivisa che
nell'accezione divulgativa troverebbe molte difficoltà. La seconda
cosa che tiene in poca considerazione è la necessità di una narrazione
del sé delle comunità, il problema è considerare la narrazione della
storia dal punto di vista di quelle comunità che hanno vissuto la
narrazione occidentale dal punto di vista occidentale. Pericolo del
retro-colonialismo: la nuova educazione di una storia della scienza,
da un punto di vista occidentale, elimina il punto di vista
particolare delle comunità che vogliono riconoscersi in quella
storia. 

\end{document}
