% !TEX root = ..\..\main.tex
\documentclass[../../main.tex]{subfiles}%
\ifSubfilesClassLoaded{\addbibresource{../../bibliography.bib}}{}

\begin{document}

% Mandare cinque articoli in ordine di preferenza. Settimana prossima
% mandare questa lista. 25 minuti a seminario.

Si parte di nuovo da un'immagine, strumenti chirurgici, viene da uno
dei testi più famosi della storia della medicina, di Vesalio (1543),
illustrato. L'altra cosa che succede nel 1543 si pubblica il
\textit{De revolutionibus} che appunto, se da un lato rappresenta la
rivoluzione scientifica in ambito astronomico, il grande passaggio dal
sistema aristotelico-copernicano all'eliocentrismo, si ha la
pubblicazione di quest'opera di Vesalio. Studia prima in Olanda, poi
si sposta a Parigi dove non è soddisfatto e va a Padova dove pubblica
le sue opere principali e dove rivoluzione l'insegnamento della
medicina con l'importanza dell'anatomia, 1544, non lo prendono
professore a Pisa, e diventerà medico imperiale, poi fa un viaggio in
terra santa per espiare un crimine, si ammala nella via di ritorno e
muore nelle isole ione. È fondamentale nella storia della medicina,
rilegge i classici, introduce tutt'una serie di novità da cui
partiamo. In particolare proprio nella produzione e circolazione del
sapere.

Immagine di Vesalio con la barba che fa da lettore ed anatomista, apre
il cadavere, e sappiamo che lui faceva autopsie per insegnare, aveva
permessi speciali per accedere ai cadaveri, e fa lezione indicando le
parti anatomiche direttamente sui cadaveri. Quel libro è estremamente
bene illustrato, e viene dopo un'altra opera di qualche anno prima,
sei tavole anatomiche che costituivano già il suo programma, e vede in
quest'immagine del medico laureato che si sporca le mani con i
cadaveri, effettivamente una rivoluzione da questo punto di vista. A
confronto, si vede l'immagine che viene da un testo stampato nel 1494,
immagine colorata del 91. Vesalio aveva presenta che cos'era il suo
oppositore, lui aveva sperimentato a Parigi, un professore sulla
cattedra che leggeva i testi dell'autorità, un dimostratore che ha una
bacchetta in mano, e un settore che è quello che apre il cadavere e fa
vedere. È una scena che si fa non al 490, ma al quattordicesimo
secolo, \textit{Fasciculim medicinae}, che riunisce tutti i testi di
osservanza galenica, medicina molto tradizionale, diversi argomenti al
suo interno. È uno dei testi che viene stampato di più all'epoca. È un
testo quasi presente in ogni casa colta, ci sono anche molte soluzioni
pratiche. Ora, fra questo e Vesalio intercorre mezzo secolo, e la
differenza appunto è nella pratica del medico stesso, quello che deve
guardare il corpo umano, dobbiamo guardare con i nostri occhi non
quelli del libro. Però lo studente poteva guardare con gli occhi, il
cadavere c'è e l'anatomia c'è. Vesalio fa qualcosa di completamente
nuovo perché ha a disposizioni pratiche nuove, ha una città più grande
a sua disposizione, Padova, ed ha connessioni politiche che gli
consentono di avere i cadaveri che dovevano essere freschi. Quindi
doveva avere appunto i cadaveri dei condannati a morte, naturalmente
il miglior materiale per queste pratiche. Questa è un'innovazione
pratica non legata teoricamente, anche se l'interesse per l'anatomia e
il confronto con le autorità del passato c'erano già. Comunque le
autorità del passato sono importanti, e lui usa ancora Galeno come
guida. Un'altra importante novità che si ha è la produzione
dell'immagine e della stampa, che non era a disposizione mezzo secolo
prima. Il libro di Vesalio è costosissimo per l'epoca, ma viene
comunque prodotto in diversi esemplari, e ancora è piuttosto comune.
Le illustrazioni sono fondamentali perché hanno il valore di
testimonianza ed esperimento, uguali per tutti e riproducibili, cambia
il modo in cui si fa scienza, cambia il modo in cui si può accedere ad
essa che diventa molto più diffusa, a Oxford, Parigi e Padova guardano
tutti la stessa immagine, e ciò è connesso appunto a queste capacità
tecniche nella riproduzione e diffusione del sapere. Anche nella
botanica si passa dai codici e nei disegni del passato, Teofrasto e
(?) Dioscolide, \textit{De materia medica}, le piante con cui si
possono fare i farmaci e le loro proprietà, e ci sono poche
illustrazioni, Plinio non le usava ne Teofrasto e questo, tranne in
casi molto particolari, anche se in queste illustrazioni si vede la
prospettiva in cui si guarda, la mandragola, immagini antropomorfiche.

Queste pratiche consentono appunto di accedere a una visione della
realtà più oggettiva a un sacco di persone, diventa orgoglio mostrare
la realtà com'è senza l'appello ai significati allegorici, che
comunque permangono, e ci sono spesso delle teorie metafisiche dietro.
Riprodurre un'immagine realistica, una dimostrazione della realtà si
avvicina a un esperimento. Si può migliorare la conoscenza del passato
grazie a quest'immagine. Non necessariamente la lettura della realtà è
diversa dal passato, non si mettono in dubbio i poteri della
mandragola, ma magari altre cose vanno messe in dubbio, però certe
qualità rimarranno. Nonostante tutto le immagini non escludono i
miracoli, le cose strane l'esotismo, l'esistenza di fenomeni molto
peculiari. 
% Immagine del Seicento che segue il racconto di un ambasciatore
% tedesco a Mosca, esistenza di questo agnello che cresce come pianta.
Con nuovi mezzi cambiano le pratiche di legittimazione, e con la
stampa certe cose non solo non scompaiono, ma anzi arrivano a un
pubblico maggiore, quindi certe leggende, storie dei draghi e serpenti
si diffondono, ed è lui stesso a far disegnare queste cose, e racconta
che ha visto lui questi animali direttamente, e non ti puoi non fidare
di lui. Il valore della testimonianza unito a queste nuove pratiche
sta cambiando la produzione del sapere, ci sono cambiamenti importanti
nell'economia della conoscenza, la circolazione della conoscenza come
merce, che non è solo valore d'uso ma anche valore di mercato. La
produzione delle idee viene domandata, e viene fatta circolare
attraverso questo nuovo mezzo della stampa, che ha un valore nel
mercato della conoscenza a seconda delle decisioni che prende questo
mercato su cos'è importante e cosa non lo è, più o meno
consapevolmente. Questi cambiamenti si vedono piuttosto in questi
scambi che nelle idee stesse. Non si parla di una rottura completa, ma
di cambiamenti che passano sia da visioni filosofiche, ma anche di
cambiamenti pratici che permettono alla conoscenza circolare in modi
nuovi, e dando cambiamenti al mercato della conoscenza. Vesalio è uno
dei medici che cerca di guardare ai corpi così come sono, dà e
diffonde l'importanza dell'anatomia. Galeno, la grande autorità della
medicina i cui testi erano studiati come manuali fino al Settecento,
aveva fatto lo stesso con Ippocrate, aveva rifondato la sua conoscenza
sulle sue conoscenze anatomiche, però per una questione culturale
dell'epoca non aveva guardato il corpo umano. Il suo programma di
ricerca era piuttosto andare a guardare animali simile all'uomo in un
quadro in cui forme e funzione vanno sempre insieme. Vesalio innova la
pratica dell'insegnamento, spinge in là il programma di Galeno, anche
se ovviamente c'era stato tanto in mezzo negli anni trascorsi. I testi
di Galeno non erano illustrati e le dispute erano \emph{filologiche},
e Vesalio comunque is rifà a Galeno, e fa vedere anche tutt'una serie
di errori nei testi galenici, e non tutti i moderni saranno disposti a
prendere per buone le critiche di Vesalio. Una delle critiche che fa a
Galeno è che i nervi siano cavi perché devono fluire degli spiriti,
ecc. Vesalio apre i corpi e vede che non c'è questo spazio nei nervi,
e Cartesio è comunque convinto delle particelle che devono passare nei
nervi per dare mosso al corpo nella sua visione corpuscolare e
meccanicista. Vesalio invece dice di no nel Cinquecento, eppure non
mette in dubbio il sistema generale di Galeno, così come non lo fanno
molti grandi innovatori che la storia ha ricordato sia per la medicina
come per la scienza in generale. La rivoluzione scientifica viene dopo
l'umanesimo, che è quando si rivalutano appunto gli antichi, si
riscoprono i testi del passato, si va a rileggere finalmente gli
originali di Ippocrate, di Aristotele e di Galeno, e questi vengono
continuamente riferiti con grande autorità. Dobbiamo tornare agli
antichi, e se da un lato ci sentiamo meglio rispetto al passato,
dall'altro c'è un profondo sentimento di nostalgia con il passato. Lo
stesso fanno molti di questi moderni, periodo di corruzione morale e
del sapere e quindi torniamo agli antichi. Nani sulla spalla di
giganti, essi esistono e sono gli antichi. Si riconosce anche la
grande tecnologia degli antichi, il pantheon, gli acquedotti,
tecnologie perse per cui non c'è necessità di tutto ciò.

La crisi di Roma ha fatto perdere tutt'una serie di acquisizioni non
solo intellettuali ma anche materiali. Sono entrate in decadimento le
grandi città, non c'era un impero con bisogno di strade, e invece alla
fine del medioevo c'era bisogno di commercio ed altre cose, quelle
strutture diventano parte integrante della società. Le infrastrutture
stabili sono sia intellettuali che pratiche, le strade mantenute
libere tutto l'anno richiedono anche istituzioni statali. C'è un nuovo
modo di vivere insieme che cambia anche la necessità del sapere. Di
nuovo, economia della conoscenza, perché produciamo certi tipi di
sapere e non altri. Bisogni sociali che vengono soddisfati anche
attraverso questo nuovo sapere. Pressioni economiche, intellettuali,
di prestigio, come appunto la cupola del duomo di Firenze, che
dev'essere meglio di quella del pantheon di Firenze, perché siamo
meglio di loro, o almeno meglio di quello che c'è stato in mezzo.

Dal punto di vista storico vediamo invece la continuità. Copernico,
William Harvey (circolazione del sangue, 1628), ecc. Galeno dice che
il sangue viene continuamente prodotto nel fegato. Ora Harvey è
considerato uno dei campioni della nuova mentalità scientifica
moderna, dimostra con una visione idraulica e matematica che la
produzione continua del sangue non funziona. Prende il cuore di un
cane, lo misura, calcola quanto sangue viene espulso ad ogni battito,
e calcola che non è possibile che ciò che mangia produca tutto il
sangue che viene espulso, e quindi la sua idea è quella di una
circolazione, ed era già scoperta e c'era nella teoria galenica la
circolazione dai polmoni al cuore, il sangue circola in continuazione
e lo si dimostra con un esperimento meccanico, bloccare certi vasi e
vedere dove si gonfia e così via, e in base a questa meccanica e
questo flusso produce l'idea di circolazione del sangue. Modernità, si
cambia il modo di vedere il corpo. Un'altra cosa che fa Harvey è che
ci sia una continuità dei viventi attraverso un origine, un uovo. È
considerata la prima confutazione della generazione spontanea, che
l'organico possa nascere dall'inorganico perché c'è circolazione di
vita ovunque. Da un lato si cercano delle leggi regolari di queste
forze vitali, dall'altro lato aspettiamo miracoli da queste forze.
Harvey in qualche modo si allontana da queste idee, ma non perché sia
un materialista che se ne frega della metafisica, ma anzi è molto
classica e antica e sempre presente in questi autori, e la troviamo in
particolare nei medici.

% Walter Pagel, William Harvey's biological ideas, p. 13. Citazione.

Non è uno scienziato e basta, ma è appunto il figlio di un'età
pre-razionalista e un devoto della filosofia naturale. È un filosofo
che guarda il mondo con le sue speculazioni cosmologiche in cui il
ciclo è importante, e quindi la vita dipende da quella circolazione e
si trasmette attraverso il ciclo delle generazioni, un ciclo di lunga
durata di cui si deve trovare il collegamento; la generazione
spontanea esce da questo ciclo, e quindi non va bene. Qua si vede
comunque il bias della rivoluzione scientifiche. Poi tacciarlo di
pre-razionale naturalmente è antistorico, però si deve tuttavia
riconoscere che tutti questi sono figli del loro tempo e queste cose
agiscono dentro la scienza, soprattutto nella medicina, che ancora non
riesce ad affermarsi come scienza come potrebbe essere la biologia e
così via. I medici aspirano sempre di più a diventare intellettuali
che si confondono con i filosofi, e in medicina questo qui è
importante, cercare l'autorità del passato è una continuità piuttosto
importante per il passato. I grande Paracelso, lui davvero prende i
libri di Aristotele e di Galeno e li brucia in piazza: non finisce
benissimo, si costruisce la sua aura di eretico, di medico popolare, a
differenza di questi che parlano di metafisica in latino e sono
incomprensibili, e che da questa metafisica traggono la loro
conoscenza, ma anche l'utilizzo di questa metafisica dà luogo
all'autorità, alla medicina tradizionale. Poi così come i filosofi
discutono delle strutture, delle interpretazioni, ecc., ma i medici
sono quasi sempre galenici, e ancora si utilizza il giuramento di
Ippocrate, è un gioco di ruolo che si fa per dare autorità: il mio
maestro è il mio padre, gli altri medici sono i miei fratelli. Tutti
condividono il fatto che certe autorità vanno rispettati e non si può
essere troppo rivoluzionari perché si rischia di non essere pressi sul
serio. Si è in un mercato della conoscenza che dev'essere conquistato.
Questo fa sì che le ricostruzioni storiche delle discipline siano un
po' diverse. La medicina, ad esempio ha da sempre utilizzato la
storia, le altre scienze meno. Ha utilizzato la storia molto spesso a
fini didattici: gli antichi erano un po' ingenui, e poi costruiamo la
conoscenza, come si fa in geometria a partire da Euclide e Pitagora,
anche se questi andarono molto più avanti, però il punto è che si
utilizza la storai dele discipline come parallelo per l'avanzamento
didattico, così le lezioni seguono un andamento cronologico più facile
da imparare. Certe cose appunto rimangono nei grandi moderni, la
metafisica è sempre presente nei grandi innovatori moderni, non si
escludevano fenomeni miracolosi, anche se alcuni vogliono spiegazioni
meccaniche per essi, o parlano di cause occulte e diventano sinonimi
di misteri, altri negano proprio l'esistenza di miracoli come
Mersenne, è mettere in crisi la loro idea dell'universo e di
difendersi contro l'occultismo, e fondamentalmente il fenomeno del
magnetismo, la cosa più evidente dell'azione a distanza la cui causa
conosciamo. 

% Citazione libro sulla scienza.

Le proprietà magnetiche coincidono per Boyle con l'astrusità della
natura, e queste sono tesi metafisiche su come le leggi di natura
possono essere disattesi, o sono molto più complesse di quello che
immaginiamo, e ha a che fare con un'interpretazione molto ampia del
libro della natura. In tutto il Seicento si trova nei manuali
illustrati l'uomo astrologico che è la rappresentazione dell'umano
con i vari segni zodiacali attaccati alle parti del corpo. Ogni parte
anatomica del corpo ha a che fare con le costellazioni e quindi si
lavora in conseguenza. Harvey lavora con i meccanici, non riescono ad
allontanarsi da queste idee: quando quindi siamo diventati moderni?
Bruno Latour, famosamente scrive un libro con questo titolo. Siamo
convinti che gli scienziati siano gli unici a parlare della natura,
quando ci siamo convinti di ciò, se lo siamo stati affatto? Che c'è
realmente la modernità?

\end{document}
