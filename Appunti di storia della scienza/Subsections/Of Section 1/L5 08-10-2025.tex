% !TEX root = ..\..\main.tex
\documentclass[../../main.tex]{subfiles}%
\ifSubfilesClassLoaded{\addbibresource{../../bibliography.bib}}{}

\begin{document}

La rivoluzione scientifica viene vista sia dal punto di vista della
storia che quello della filosofia, ma in punti di vista diversi.
\textit{Leviathan and the pump} (?). Critica la rivoluzione della
scienza come una rottura, una rivoluzione violenta in cui c'è un
momento determinato: la pressa della Bastiglia e la pressa del Palazzo
d'inverno. Ma c'è una differenza: all'epoca della rivoluzione francese
sapevano di stare facendo una rivoluzione, e chiamavano le cose come
l'\textit{ancien regime}, che ha una data di morte, il 4 agosto dell'89.
F. Furet, \textit{Critica della rivoluzione francese}. Invece i nostri
protagonisti della rivoluzione scientifica, che secondo un articolo
recente di un altro storico, James Secord nella rivista ISIS, sostiene
che quest'idea di rivoluzione scientifica, come si leggerà nel libro
di James, è una delle cose che portano al progresso dell'umanità.
Prima una rivoluzione scientifica che è il presupposto necessario per
quella industriale (non è vero, naturalmente), e poi quelle
politiche. Questa visione semplicistica, secondo Secord, è stata
inventata non da Corré e altri storici nel periodo successivo alla
prima guerra mondiale, ma durante l'epoca del New Deal negli stati
uniti. Questa è un'epoca estremamente progressista in cui si vale
appunto trasmettere un'idea di progresso, quindi sono una serie di
storici della scienza e divulgatori che escono con questo, che parla,
per fare una battuta, della rivoluzione contro l'\textit{ancien
regime} aristotelico. Poi però, se andiamo a vedere il lavoro di
questi protagonisti, è vero che è un critico feroce contro Aristotele,
nel ripetere a memoria i commenti di Aristotele, già all'epoca di
Descartes si studiava sui manuali. Poi nel Cinquecento quando gli
europei cominciano i viaggi interatlantici scoprono che un sacco di
cose sono incerte, e Montaigne scrive nei suoi saggi questa
conversazione con un indio, che parla attraverso il suo interprete, e
capiamo perfettamente che il senso del saggio di Montaigne è criticare
l'insegnamento in Francia, non conoscere il mondo dei selvaggi.
Entrano in crisi varie assunzioni.

Tornando a Descartes, ne parla come un incubo dell'insegnamento
aristotelico. E si studiavano le opere di Aristotele perché grazie a
secoli di lotta, di Tommaso, Alberto Magno, l'aristotelismo ha vinto
sul platonismo (eh non precisamente). E Descartes viene mandato a una
scuola dei gesuiti, una delle migliori dell'epoca, però era non solo
basato sui manuali di Aristotele, ma un'idea che era di Aristotele per
cui ha dato anche un ordinamento delle discipline. Nell'\textit{Etica
nichomachea}, ogni tecnica è arte, attività manuale, e ogni metodo o
attività che segue le regole ha un proprio oggetto, un proprio bene.
Per la stessa ragione Aristotele ci dice che non deliberiamo su una
dimostrazione geometrica, come la somma degli angoli interni è di due
angoli retti. Aristotele implicitamente ci spiega che ci sono scienze
più fondamentali delle altre, ma comunque se ogni scienza tende a un
bene diversa, c'è n'è una che è l'architettonica dei fini, ed è la
politica... E c'è una gerarchia tra le discipline, ed è logica a
seconda dei principi delle varie scienza. La fisica deriva i suoi
principi dalla metafisica, e la geometria dalla fisica -- cosa contro
intuitiva perché a partire del Seicento si considerano le matematiche
come la base della fisica. Invece Aristotele deriva il principio del
continuo dalla fisica. Per prolungare una retta all'infinito, in
Euclide, dev'esserci uno spazio, continuo, omogeneo e infinito. Ora,
tornando all'epoca della rivoluzione scientifica, per Aristotele
l'idea del continuo la prende dalla fisica, ed ecco perché c'è una
gerarchia tra le scienze.

I gesuiti le matematiche applicate, l'ottica, ecc., non le insegnano.
Ma c'è una disciplina trasversale, l'analitica che poi sarà la logica.
Questa gerarchia delle discipline è implicitamente anche una gerarchia
dell'insegnamento e a sua volta una gerarchia sociale. Il filosofo
diventerà anche teologo, ed è una delle questioni che discuterà
Galileo. Il teologo-naturalista conta molto nelle università, è molto
apprezzato. Ora, in quest'epoca Galileo critica Aristotele. Cominciamo
a pensare a qualcosa di certo, e quindi si immette col dubbio, e prima
di tutto c'erano i sensi, ma chi mi assicura che sono sveglio? Questo
non è dunque un fondamento saldo, che sono le matematiche, e quando
dormiamo 7 + 5 fa sempre 12. Dio potrebbe ingannarmi, no? E questo
chiuderebbe il discorso, e perché esce col genio maligno? Qui esce
fuori il famoso ... Caterus dirà che non bisogna stabilire la causa di
un'idea, però se ci ho un'idea in testa so che penso. Poi tira fuori
la materia e così via, però poi Descartes, quando decide di scrivere
questo libro, perché decide di dimostrare l'esistenza di Dio? Perché
sta sviluppando una nuova fisica, ed è una fisica nella quale
l'ipotesi copernicana è centrale. Descartes scriveva tantissime
lettere, ma particolarmente a Mersenne, che corrispondeva con tutti
gli uomini di cultura. Era un po' paranoico, scrive le lettere
falsificando dove stava, e scrive tramite Mersenne. Descartes nel
1633, quando gli arriva notizia in ollanda della condanna di Galileo,
scrive fin dove si può arrivare, perché finché non sia dichiarata
eretica si può insegnare, più o meno perché già Galileo era stato
ammonito a non insegnare copernico da Bellarmino. E dice a Mersenne
che sta scrivendo in Latino, il \textit{Mondo} che comparirà dopo in
francese, e che dopo la condanna di Galileo lo vuole bruciare e
scriverà ciò a Mersenne. Dice però che Galileo ha sbagliato in un
punto chiave, ha preteso, come ha scritto nella lettera Cristina di
Lorena -- lettera che circola e che viene poi stampata e diventa
manifesto pubblico, storicamente importante alla base di nostra
posizione rispetto alla posizione di scienza e religione -- la
religione salva le anime, la scienza non ci dice nulla di tutto ciò.
Ecco Descartes dice che Galileo si è sbagliato su questo. Nel
\textit{Dialogo sui due massimi sistemi}, che è un dialogo tra tre
personaggi, a Roma non lo prendono bene. Qualche anno dopo, si
pubblica questo grande romanzo che è l'\textit{Avventuroso
simplicissumus}, e poi \textit{Courage} da cui Brecht trarrà il
coraggio di qualcosa. Processo a Galileo e viene dichiarato eretico il
movimento della terra, ora lui scrive Mersenne che ha sbagliato perché
pensava che si potesse separare la scienza della religione. A la
principessa ha scritto questo, invece al suo discepolo ha scritto
un'altra cosa. Ora, perché la chiesa ce l'aveva tanto col movimento
della terra? Ci sono passi della Bibbia in cui Dio ferma il sole per
permettere alla tribù di vincere, e scrive a Benedetto Castelli.. Ma
questi sono iognoranti, perché se Dio avesse creato il mondo secondo
Aristotele doveva non fermare il sole, incastonato nella sfera, ma il
primo motore. Il problema è che questa lettera circola anche se non
doveva, e Descartes dice proprio questo: Galileo non ha compreso che
il punto è che non si può attaccare la fisica aristotelica senza
attaccare la teologia cattolica, e quindi si decide di stare zitto.
Poi comincia a ripensarci perché molta gente a Parigi dice Mersenne
che vuole leggere questa fisica. Pubblica il famoso discorso del
metodo, ma non i saggi successivi che sono tre (?). In realtà sono due
discorsi, uno parla della diottrica, che sarebbe la parte dell'ottica
che si occupa della rifrazione, della legge di rifrazione, e un altro
discorso sulle meteore, in cui ci parla dei fenomeni meteorologici, la
formazione dell'arcobaleno, applica le sue idee di ottica. È un
trattato, un altro, che è la famosa geometria di Descartes.

Nel discorso sul metodo ha detto qualcosa di importante. Ho sviluppato
una nuova fisica, ma non ve la espongo perché l'ipotesi copernicana è
centrale, però la gente continua a chiedere a Mersenne perché non
pubblica questo. Il discorso sul metodo viene pubblicato in Francese,
perché vuole sia letto anche dagli artigiani che non masticano il
latino. Vuole che la diottrica serva a quelli artigiani che dovrebbero
fare i telescopi. Descartes ha scritto il Discorso, ma poi nella
Geometrie dice che la dimostrazione non la dà, ma è data al lettore
perché se no è noioso. Ma come, ci hai detto che l'hai scritta in
francese per fargliela leggere agli artigiani.. Naturalmente se voleva
scrivere agli scienziati scriveva in Latino, che però era già
cominciato a perdersi nelle corti. La scolastica si era allenata per
secoli a tutte le possibili obiezioni di un mondo infinito, ma è uno
spazio di libertà perché lì sì che si sviluppavano cose eretiche.
Keplero si è convinto proprio quando il suo maestro lo insegnava, a
Tubinga, seminario teologico dei protestanti. Ma in qualche punto
nelle dispute, prendeva le difese del sistema copernicano. Tornando al
nostro Descartes, ad un certo punto se ne esce con queste meditazioni
metafisiche, e perché? Perché a Mersenne, dice che deve far girare la
terra attorno al sole, ma non si deve muovere, ed ecco i vortici (R),
anche se alcuni sostengono che è una conseguenza naturale della sua
fisica, anche se qua si propende che è per sfuggire a questo problema.
Però c'è un altro problema, ed è vincere le ragioni più forti degli
scettici, e smette le cose di fisica per scrivere il \textit{Discorso}
che, contrariamente agli interpreti, non lo scrive contro lo
scetticismo in Francia, ma Mersenne scrive \textit{La vérité des
sciences} dove dice con molta erudizione ciò che si conosce con la
scienza. Ma sembra strano che passati 16 anni si fosse preoccupato
degli scettici, e invece perché deve vincere contro gli scettici? Per
un motivo: dopo galileo non si può attaccare la cosa di Aristotele
senza attaccare la teologia, e quindi sto zitto... Ma se invece
facessimo un altra cosa? Se dimostriamo che la mia fisica può essere
un fondamento ugualmente forte e stabile per la teologia scolastica, a
quel punto potrei poi pubblicarla, ed ecco perché se ne esce col
\textit{Discorso sul metodo}. Costantin Huygens, obiezioni Mersenne,
fai obiezioni e io ci metto le risposte. Il primo lo fa questo teologo
Caterus, poi il grande Arnauld, poi Gassendi, e poi Mersenne come
seconda obiezione dice a Descartes che ha raccolto alcune tra teologi
a Parigi, ma non dice che quasi tutte le fa Mersenne stesso. E
Mersenne gli dice che c'è un inglese, e Descartes gli chiederà chi è,
ed è Hobbes. Ad un certo punto gli dice il \textit{De cive}, e
Descartes capisce che è lo stesso che gli ha fatto le obiezioni. Se ci
si legge questo dialogo, se la piglia questa gente, discutono con la
stessa cosa che discutono gli aristotelici, e Hobbes è più vicino ad
Aristotele che a Descartes, sebbene Hobbes ha un'impostazione
meccanicistica più forte, si nega l'anima e si va direttamente
all'ateismo. E su questo, quando Descartes di troverà a Discutere con
il famoso matematico Fermat. E però la dimostrazione di Fermat è così
lunga che non si può stampare, ed è stata fatta 30 anni fa, col
computer, ecc., e quando discutono Descartes e Fermat, stai
confondendo fra movimenti violenti e movimenti naturali, e non è che
Fermat sia aristotelico, ma quello è il lessico dell'epoca.

E come si può parlare di rivoluzione nel senso di rivoluzione francese
tenendo questo? 

\end{document}
