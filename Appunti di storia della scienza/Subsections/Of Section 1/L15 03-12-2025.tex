% !TEX root = ..\..\main.tex
\documentclass[../../main.tex]{subfiles}%
\ifSubfilesClassLoaded{\addbibresource{../../bibliography.bib}}{}

\begin{document}

L15 03-11-2025

La scienza inizia a non essere una prerogativa degli scienziati, ma
diventa accessibile agli storici. Kuhn che mette in crisi l'idea di
progresso, la scienza procede con rivoluzioni scientifiche. La scienza
avviene più a salti che in maniera lineare; la scienza va studiata con
approccio storico, mostrando ciò che gli scienziati fanno e non quello
che dicono di fare. Da questo punto di vista, quest'idea incontra
anche delle problematiche, ad esempio quella di definire il proprio
oggetto. Poi c'è il problema di un'epistemologia differente,
l'inclusione di nozioni metafisiche nella scienza, che oggi non si
fanno, e poi il problema che i campioni della rivoluzioni scientifica
avevano interessi che si collocavano in posizioni antiscientifiche per
i nostri standard. Quindi già una prima difficoltà è quella di
definire l'oggetto specifico della storia della scienza. Ci si trova
ad avere a che fare con notevoli divergenze rispetto al modo di
approcciare la scienza. Poi c'è il problema del contesto globale, la
decolonizzazione, creazione di un mondo moltipolare, e si inizia a
dare maggiore rilevanza a contributi di altre parti del mondo. Si
mette in discussione la narrazione eurocentrica, si mostra che tanti
luoghi che sono stati esclusi dalla storia sono stati esclusi. Daston,
non si può più parlare di storia della scienza come si è fatto ora:
quale diventa l'oggetto di questo ripercorrere del sapere? Storia
della conoscenza, segna il campo della scienza, ma tiene conto di
queste problematiche, di integrare sempre un numero maggiore di
contributi. Una tendenza è quella di includere a pieno titolo molte
forme di conoscenza che venivano dal basso, e che in effetti avevano
contribuito all'accrescimento della conoscenza; oppure si potrebbero
includere i frutti dell'erudizione umanistica come la filologia,
prodotti di una elite culturale. In tanto ci si riesce di liberare di
aggettivi come 'moderno' e 'occidentali' che diventano superflui.
Vengono a meno i limiti tra le scienze, permette le pratiche
strettamente. Quando è che sono nati i modi di pensiero che ci sono
portati a questo punto? Parlare di nascita della modernità è un modo
di rispondere a questo. Parlare di scienza sembra essere parlare di
qualcosa di più prestigioso di quanto non sarebbe la conoscenza. 

C'è uno spazio per la storia della conoscenza? Si deve parlare in
tanto di analisi e raffinazione concettuale. La storia della scienza
presenta un campo più vasto, campi di studio, esempi per supportare le
proprie tesi, mentre la storia della conoscenza dovrebbe partire da
zero. Un esempio è che studiare la conoscenza può portare a nuovi modi
per declinare le relazioni tra le varie discipline. Più o meno in ogni
fase della storia si individua una disciplina che prende l'avvento più
delle altre.

Dall'altro lato c'è un altro elemento rilevante, quello della
classificazione della conoscenza, e oggi abbiamo un sistema abbastanza
rigido e che si può vedere anche nella struttura nei campus
universitari. Di base si ha una caratterizzazione che deriva da una
narrazione classica di storia della scienza, ma non è l'unica
possibile, ma si potrebbero fare studi comparati fra le varie
discipline, ponendo attenzione alle diverse pratiche che fanno i
ricercatori. Il metodo storico non è poi così dissimile alle pratiche
del biologo, e il filosofo può essere assimilato a un fisico teorico.
Si potrebbe fare attenzione a diversi aspetti per fornire una nuova
visione della classificazione della conoscenza. Poi, parlare di storia
della conoscenza significa includere un nuovo metodo di ricerca. Non è
in effetti qualcosa di possibile per un singolo studioso, come si fa
ora, uno storico scrive un articolo. Il passaggio a un'attività di
ricerca andrebbe a stimolare un lavoro collettivo, dialogo
discussione, non chiude l'isolamento ma tiene conto che è impossibile
all'individuo impadroneggiarsi.

Siamo nel miglior modo della storia della scienza. Dal momento in cui
la storia della scienza si è ritagliata uno spazio e un rispetto come
disciplina accademica, perché_parlare di storia della conoscenza? Le
motivazioni sono due: 1) la narrazione classica della storia della
scienza è sbagliata, è incredibile --shapin-- dire che non c'è stata
rivoluzione scientifica in un libro che si chiama rivoluzione
scientifica; 2) India e così via, molto tempo esclusi, cercano
legittimazione volendosi inserire nella storia della scienza in
maniera classica, proprio secondo quell'idea e prospettiva
eurocentrica, che dà origine alla modernità e al dominio
dell'occidente sul mondo.

--- Secondo seminario

\textit{I filosofi e le macchine}


\end{document}
