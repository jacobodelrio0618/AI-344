% !TEX root = ..\..\main.tex
\documentclass[../../main.tex]{subfiles}%
\ifSubfilesClassLoaded{\addbibresource{../../bibliography.bib}}{}

\begin{document}

Articolo di Shepin, fa riferimento al pragmatismo, la verità come
qualcosa di collettivo. La conoscenza non è mai isolata a un
individuo.

Il moral bond è quello della fiducia. Il valore della testimonianza e
lo story-telling. Assolve anche una funzione di mantenimento
dell'ordine sociale. La conoscenza della natura è in stretta relazione
con una conoscenza sociale così strutturata. Concetto di \emph{free
action}. L'idea è quella che nel contesto da analizzare è che il
gentiluomo sia l'attore per eccellenza, e a lui vengono ascritte
caratteristiche di gruppi che non sono diffuse in vari gruppi sociali.
L'idea che avessero una cura per la propria reputazione, competenza
percettiva, disinteresse, centralità delle condizioni economiche e
discendenza. Analizza principalmente la figura di Boyle. Shepin è
molto attento sul tema della discendenza, l'eredità della disposizione
ad agire in maniera virtuosa, permettono di comprendere la struttura
della conoscenza sociale, e poi la conoscenza scientifica. Le mensogne
sembrano emergere da circostanze ignobili, o nel caso delle donne
da mancanze di capacità di raziocinio. È interessante che una delle
prime preoccupazioni, l'essere dello storytelling ci costringe a dire
sempre la verità? È possibile omettere certe cose per tutelare il
decoro e il proprio onore. L'attenzione alla bugia rispetto alle
epistemologie delle menzogne, oltre le condizioni dell'enunciato
falso, o dell'essere intenzionato a mentire c'è la condizione che
riguarda l'interlocutore. In quel contesto poteva essere considerata
menzogna se l'interlocutore non merita la verità. Una delle tematiche
fondamentali è la questione della testimonianza, e quali sono gli
scontri concettuali nel modo in cui viene storicamente considerato lo
statuto epistemico della \emph{testimonianza}, e ci sono scontri fra
l'affidabilità epistemiche della testimonianza, e la necessità di
basarci su essa, spesso in persone non particolarmente affidabili.
Tensione tra necessità di testimonianza e sfiducia costitutiva nel
rifarsi alla tradizione, in questo contesto emergono criteri di
valutazione della testimonianza, fa riferimento a Locke, e così via,
tuttavia il criterio più importante è l'ultimo, la valutazione non
tanto del contenuto, la plausibilità, coerenza, ecc., ma uno dei pesi
maggiori, che viene utilizzato ad esempio dalla Royal Society, è
l'integrità morale, l'assenza di motivi per mentire di chi proferisce
una testimonianza. Questo si svolge in un'opposizione tra scetticismo
e credulità. Si creano delle reti di garanzia, spesso non per motivi
logici o epistemici, ma per moral bond in determinati contesti
sociali.

Problematiche di Robert Boyle, analisi della sua figura,
caratteristiche che contraddistinguevano anche Boyle, viene
considerato un master of credibility, il padre era uno degli uomini
più ricchi di Inghilterra, assenza di costretti materiali, fonti della
menzogna e non-attendibilità. Caso interessante, quello di Denis
Papin, uno dei collaboratori di Boyle, e ci sono parti delle opere,
soprattutto quando Boyle è malato, il suo nome rimane nell'ombra, e
questo, secondo Shepin, conferma la sua tesi. La verità scientifica è
correlata da queste virtù morali nel contesto storico. Proprio perché
sono caratteristiche politiche e morali, assumendo il suo presupposto
della conoscenza collettiva, acquisisce un ruolo epistemico
importante.

Kitchin Big Data e una nuova rivoluzione scientifica.

Se i bid data hanno portato a un quarto paradigma della scienza e
della conoscenza? Definizione di big data. Big bang data. 

Obsolescenza del metodo scientifico, la correlazione sostituisce la
causalità. I dati sono liberi di bias. I dati possono essere
interpretati da chiunque -- quattro fallacie. Computational social
sciences.

\end{document}
