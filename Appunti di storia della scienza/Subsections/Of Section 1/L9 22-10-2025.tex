% !TEX root = ..\..\main.tex
\documentclass[../../main.tex]{subfiles}%
\ifSubfilesClassLoaded{\addbibresource{../../bibliography.bib}}{}

\begin{document}

% Appunti della lezione.

1. «In questi secoli si parla continuamente, con un'insistenza che ha
del monotono, di una logica dell'invenzione, concepita come
\textlatin{\textit{venatio}}, come caccia, come sforzo di penetrazione
in territori prima sconosciuti. Questa logica dell'invenzione viene
concepita come arte e come strumento; appare paragonabile, e di fatto
viene spesso paragonata, agli arnesi. Essa appare scarsamente
interessata all'analisi dei termini del discorso e ha, quasi sempre,
un tono di rozzezza e ingenuità se confrontata alle sottili disunioni
della tarda Scolastica, ma, a differenza di quest'ultima, appare
soprattutto preoccupata di progettare metodi nuovi, di estendere le
possibilità di dominio dell'uomo sugli altri uomini e sulla natura. Il
prodigioso allargamento dei confini del mondo celeste e del mondo
terrestre che si verificò nel secolo XVI, non fu senza risonanza né
sulle opere dei filosofi e dei logici, né su quelle degli artigiani
superiori e dei tecnici».

2. «Il libro di Vannoccio Biringuccio sulla \textit{Pirotechnia} fu
pubblicato a Venezia nel 1540: si trattava del primo libro a stampa
dedicato alla metallurgia e, come scrive il Farrington, Biringuccio
era "consapevole della sua originalità e si vantava di essere il solo
ad aver pubblicato un'opera che non era fondata su altre opere, ma
sulla diretta esperienza della natura"».

3. «Il \textit{De re MetallicA}, pubblicato nel 1556, un anno dopo la
morte del suo autore, restò per due secoli fondamentale e insuperata
di tecnica mineraria. Il libro era apparso negli stessi anni in cui le
miniere del Centro e  del Sud America stavano raggiungendo uno
sviluppo prodigioso [\dots]. L'atteggiamento assunto da Agricola ha
una parentela assai stretta con quello che, una ventina d'anni prima,
era stato assunto dal Vesalio in un differente campo di ricerca.
Nell'uno e nell'altro autore troviamo presente la convinzione che la
situazione di un determinato campo del sapere richiede, per essere
migliorata e modificata, una vasta opera di conservazione e di
descrizione dei dati di fatto. Tale descrizione dev'essere
sistematica, analitica, meticolosa. Essa richiede speciali tecniche
illustrative il cui scopo fondamentale è quello di tradurre i
risultati dell'osservazione in immagini grafiche il più possibile
chiare e comprensibili. Questo desiderio di chiarezza, questa precisa
volontà di evitare gli equivoci, di allontanarsi coscientemente dal
fiabesco è ciò che accomuna un'opera come il \textit{De Fabrica} di
Vesalio al De re metallica di Agricola. [\dots]. Secondo Agricola
l'arte dei metalli è stata finora assai poco coltivata. Fra i
pochissimi autori che hanno trattato seriamente delle caratteristiche
dei vari metalli, della struttura dei terreni metalliferi, dei
procedimenti necessari per procedere all'estrazione dei metalli dal
sottosuolo egli ricorda Biringuccio [\dots]. Contro quest'arbitrarietà
[degli alchimisti] Agricola protesta in nome di un sapere che sia
comunicabile e il cui linguaggio abbia i caratteri della precisione e
della intersoggettività [\dots]. Alcuni caratteri della ricerca
scientifica emergono negativamente da un'insistenza su quegli aspetti
di precisione di comunicabilità che sono indispensabili al sapere
tecnico [\dots]. Su questi motivi si insisterà del resto, in pieno
Rinascimento, anche da parte di chi, come Cornelio Agrippa, non era
estraneo ai problemi tecnici e all'"invenzione di macchine"».

4. «Difesa di Agricola, sempre nel \textit{De re metallica} dell'arte
dei metalli di essere indegna e vile in confronto con le arti
liberali [\dots]. Il lavoro del tecnico non può dunque in alcun modo
andar disgiunto da quello degli scienziati».

5. «These various positions show how Adam Ries was, for several
decades, at the exact position where arithmetic was implemented in
mining. Raw ore was extracted, then smelted and refined to give pure
silver, as Georgius Agricola would later explain and beautifully
depict in his \textit{De re metallica}».

6. Tema della 'finestra d'opportunità' che svanisce a inizio
cinquecento, collegamento con le scoperte geografiche e cronologiche,
il nuovo modo di generalizzare che anche si collega al cercare un
nuovo organo per trarre gli assiomi a partire dalle opere degli
artigiani. Agricola, Bacone, e poi i Newtoniani. Cioè, quando comincia
a venire a meno la distinzione fra l'arte e la natura.

% Note a lezione

Possiamo usare le fonti dei testi matematiche come fonti della storia
della matematica? E nel caso come si dovrebbero leggere queste fonti?
Considerare non solo i grandi scienziati delle università che
scrivevano in latino e così via, ma è necessario anche andare a
considerare il contributo di un insieme di autori molto più ampio. Per
appuntare il telescopio sulla luna Galileo ha bisogno di sostegni
della Repubblica di Venezia e così via, e poi un insieme di conoscenze
tecnologiche e manifatturiere che permette di produrre lenti molto
trasparenti. Rossi, I filosofi e le macchine. Rivoluzione scientifica
mette insieme la filosofia naturale e la macchina, un certo tipo di
meccanicismo. Trading zones, contesti che permettono forme diverse di
conoscenza di incontrarsi e parlarsi.

% Bruno Latour, studio sul cancro del laboratorio, andare a vedere
% effettivamente come la scienza viene fatta.

Cristallizzazione, processo per cui si forma a partire da una materia
liquida che poi si propaga in determinate direzione. C'è ripetizione
di cose tradizionali, però c'è uno spazio dove si vede come novità
emergono in stretto contatto con il contesto sociale. Come i testi
reagiscono a ciò che li circonda.

La prova non sta nel teorema, ma sta nel prodotto. Più corroborare che
verificare le cose, dimostrare le cose. Testi come \emph{strumenti},
usati nelle scuole e servivano come dei prontuari che usavano a
seconda delle esigenze e mischiavano alla pratica pedagogica, in cui
il ruolo fondamentale comunque era dato alla trasmissione orale.

7. Vasoli, l'elloquenza e la facoltà della memoria come organizzazione
della conoscenza, Pietro Ramo e Melantone, Giorgio Agricola.

Esplosione di testi di matematica pratica in Germania, libri di
calcolo.

Domanda: 

1. Conoscenza corroborata attraverso le opere, matematiche non
dimostrate. Cristallizzazione.

2. Appello alle opere nella sistematizzazione della conoscenza:
Agricola, Melantone, Ramo, Agrippa, Nuovo metodo, logica
dell'invenzione, enciclopedia.

3. Nella sistematizzazione, comunque si perde l'appello alle
matematiche, Bacone come esempio di spicco, come mai?

\end{document}
