% !TEX root = ..\..\main.tex
\documentclass[../../main.tex]{subfiles}%
\ifSubfilesClassLoaded{\addbibresource{../../bibliography.bib}}{}

\begin{document}

Marx, Focault, Benjamin. L'importanza della storia nelle filosofie e
la trascuratezza sia destra che sinistra del suo ruolo in questi
autori. Leavitt.

Il mito dell'eterno ritorno. Dedica un capitolo chiamato il terrore
della storia. Dopo che l'idea di storia è stata separata dalla
religione, dall'apocalittica, ecc., la storia laica è una cosa che
suscita terrore. Come fa l'umanità a reggere una storia che non ha un
punto di arrivo, e allora lui mette a punto la ciclicità, che ha
affascinato via via moltissimi filosofi, da Machiavelli a Vico, e
continua a riproporsi in modo forte. Il problema è che non esiste un
tempo ciclico, ma nel tempo ciclico c'è la linearità e viceversa. Gli
storici questo l'hanno chiarito molto bene. Organizzazione della
successione degli imperi, cosa importante per la tradizione
giudeo-cristiana, tema che attraversa quasi tutto il sapere
occidentale. È una teoria che viene assunta da Agostino e ripresa dal
suo allievo Rossio, e arriva fino a Melantone che riprende al profeta
Daniele, ed è interessane dal punto di vista politico, perché colui
che si oppone alla teoria dei quattro imperi è già moderno. Qual'è il
punto della storia? Lezioni di Hegel, la traduzione troppo forte della
Storia universale -- o storia del mondo. Modello che lui trae da
questo tipo di tradizioni che reinterpretava letteralmente la storia.
Si interpretava Carlo Magno come il continuatore della tradizione
romana, un occidente che avanza secondo la tradizione romana, un
occidente che avanza. Melantone riprende Daniele per fare quello che
sarà proposto poi da Hegel, che la Germania è il punto finale
dell'occidente, che è grazie alla tradizione celtica che assorbe
l'impero. La storia universale è una dove ogni popolo impersona la
storia di tutti. E siamo nella crisi di questo modello della
tradizione occidentale, una svolta che va contro il modello hegeliano
perché l'oriente prende mano, Cina ed India. La teoria della
successione dei tempi è piuttosto complessa ma anche duttile, perché
dentro questa teoria del tempo freccia si sono innescate diverse
teorie che sono giustificazionista come il progresso, successioni
quasi senza rotture verso il benessere. Condorcet scrive un abbozzo di
un quadro storico del progresso. La cosa interessante è che lui divide
il mondo in 10 epoche, ma per la storia l'aveva inventata Bossuet,
epoche, filosofico nella tradizione accademica dello scetticismo, poi
astrologico, e Bossuet lo traduce in termini storici. È un momento
storico -- abbastanza paradossale, una sospensione -- le grandi epoche
che segnano l'umanità. Distingue tra età ed epoca, collegare la storia
effettiva umana alla tradizione biblica, ed è sempre stato fatto come
un gioco di corrispondenza. Cristo è il punto di mediazione tra Terra
e Cielo, e devono essere corrispondenze simboliche con le epoche della
storia con quelle che hanno senso dal punto di vista religioso. Un
capolavoro come quello di Dante è costruito con la corrispondenza
Cielo e Terra. Però quello che è un grandissimo critico letterario del
Novecento, Eric Fauerbach parla della \emph{interpretazione figurale}
personaggi storici, letti e interpretati all'interno dell'altro mondo,
che è quello di Dio, ed è un problema cultural , dare senso alle
vicende del mondo secondo una credenza. Il grande problema della
storia è che nella tradizione giudeo cristiana il tempo che prevale è
quello lineare e non ciclico. Nella tradizione orientale questo non
c'è, e in parte deriva anche dal mondo greco. Questi modelli prendono
egemonia, anche se alcuni si oppongono a questo. Il paradigma più
potente che ebbe un influsso influenti è Comte (?) che sosteneva
queste cose, era un modello molto forte. Poi c'erano le teorie che
cercavano in questi flussi le rotture e i mutamenti, Marx cercava
questo, e l'ultimo è stato Focault, che vede la storia come passaggi
di rottura epistemologica collegandosi a una tradizione francese.

Vuoto-vacuo nella tradizione orientale, differenza col vuoto
occidentale, Spengler parla di cose di questo genere. Ci sono degli
aspetti che permeano in questo. Ultimo Focault, quando i movimenti
della liberazione diventano despotici, che cosa succede? Movimenti di
liberazione del secondo dopoguerra in Africa e così via. Che cosa
accade? Ferocia e violenza in nome della democrazia e della libertà.
Noi non sopportiamo il tempo vuoto, l'ansia diffusa, il panico diffuso
o problema legato a questo tempo. Non siamo abituati a lasciarci
attraversare dal tempo ma a controllarlo e questo crea grande disagio.
Non c'è dubbio che c'è un problema di fondo che è storico. Le neurosi
che si soffrono oggi sono più vicine alla psicosi, per fare
psicanalisi. La patologia è sempre un ingrandimento del normale, e
questi sono i nostri paradigmi, il nostro paradigma è il
tempo-freccia, si deve crescere ed accelerare, e se non si fa, si
pensa che si è un reazionario. Poi ovviamente ci sono varianti come lo
sviluppo sostenibile, si cerca di tamponare qualcosa che non si può
tamponare: ogni pubblicità ora è ecologica. Il sistema capitalistico
non può per sua essenza gestire una sostenibilità. Fino a poco tempo
fa si pensava che una battaglia ecologica poteva abbattere il
capitalismo, ma pensare che sia compatibile con esso è un'illusione.

Morte del dialogo, dialogo, l'ascoltare l'altro nella diversità è
sparito nel nome della stessa democrazia, e questo non viene visto. Ma
perché Spengler è siginfiicativo? È un reazionario che non usa il
tempo freccia, è un hihilista. LUi pensa che l'unica cosa che può
esserci è la lotta, e lì vince chi è più forte. Lui non pensa affatto
che l'occidente sia il punto finale di una storia unica, il tramonto
dell'occidente è perché lui pensa che sia appunto alla fine, ed è un
concetto che si diffonde moltissimo in quelli anni, ed è un concetto
anche molto forte in Heidegger, anche se non dichiarato
esplicitamente, il secondo Heidegger, l'essere che non puoi cogliuere,
mondo greco e poesia come unica risposta possibile è la nostalgia
dell'essere legato a questa parola che ha affascinato i tedeschi che è
l'elemento e il punto originario, il pinto in cui si parte, dove
mythos e logos stanno ancora insieme. Poi si scindono e quindi si vive
con questo tipo di nostalgia, o con il senso della scissione
hegeliana, questo è un tema che esplode a inizio Ottocento,
dell'idealismo. È proprio la nostalgia dell'uno, non la teoria del
progresso, non si va sempre verso il meglio ma al contraio, il mondo
tende a scendere a cedere, e tutto ciò che prevvede questo è la
tecnica, che rovina il mondo e rovina la poesia. Però c'è una
sommiglianza, Heidegger conosce benissimo SPengler, ma questa cosa
storia e filofociamente va collegata alla rottura che ha collegato
Nietzche, non c'è per forza in lui una nostalgia per l'uno, ma
chairamnete spezza quella che un filosofo chiamava l'illusione del
progresso, e l'andata verso la concetualizzazione, l'astrattezza, in
cosa consiste la deduzione del concetto? Che singifica astrattezza,
disumanità e aridità. Tutto ciò che è matematico è arido, è una
perdita della vita. Ciò spiega perchè nel primo dopoguerra la vita
diventa importante, al vitalismo, ai processi di conoscenza, e questo
lo fa Spengler, sono quelle che chiamiamo intuizioni, emozioni, ecc.

% Parentesi, oggi si parla di emozione, non di passione, e Spinoza e
% Cartesio ne parlavano, e oggi è una parola che è sparita. Nella
% ricerca di mercato la passione non c'è, c'è l'emozione immediata.
% Uno come Heidegger è ancora vicino alla passione, si richiama al
% pathos. Oggi si intendono le emozioni in chiave debole, democratica
% rispetto al secolo scorso dove invece questi temi dell'emozioni
% erano colegati alla vita, al vitalismo, a qualcosa che si rischia di
% perdere. Cosa perde quell'uno nella scissione di inizio della
% storia? La vita, si perde la dimensione dell'immediatezza della
% vita, era questa la sensazione che circolava, anche se si sta
% naturalmente schematizzando.

Spengler si pone questi problemi, queste cose non si possono esprimere
in concetti, dimostrazioni, ecc., esse vanno sentite, vissute,
intuite, fra vivere e conoscere, fra la certezza immediata che va
all'intuizione -- illuminazione... Fantasia sensibile di sarta --
risultati dalla conoscenza intellettualistica e la tecnica esiste una
differenza raramente fatta. Separa il mondo vitale dell'intuizione e
quello della conoscenza messa astratta, sono due modi che dal punto di
vista suo non si conciliano. Il problema è che funziona bene quando
una delle due cose non distrugge all'altra; decade invece quando
l'astratto prevale sulla vita. Questa è una variante di quello che
aveva fatto Nietzche fra apollineo o dionisiaco, e il dionisiaco
sparisce, e infatti Spengler non distingue franto fra apollineo e
dionisiaco, ma fra apollineo e faustiano, l'uomo faustiano chi è?
Chiamerà apollinea l'anima della civiltà antica, che scelse come tipo
ideale dell'esteso i singoli corpi presenti. Dopo Nietzche... ad essa
oppongo l'anima faustiana, il cui corpo è la civiltà occidentale,
nata nelle pianure nordica stile romanico, ecc., è un'interpretazione
molto tedesca ed ingenua. Ma chi è l'uomo faustiano? E qui viene fuori
una peculiarità interessante perché l'uomo faustiano in fondo è quello
che noi possiamo considerare come l'uomo moderno, colui che scopre la
tecnica, e soprattutto colui che scopre la strada. Ecco qui dà
un'interpretazione abbastanza originale della questione. Per lui
l'uomo faustiano è l'uomo prospettico, e chi è l'uomo prospettico? È
l'uomo che si forma nel rinascimento chiaramente, anche se lui
l'anticipa, si forma attraverso un'idea astratta fondamentale di
profondità ed infinito, e perché? Qui bisogna riflettere.

Profondità ed infinito. Perché hanno a che fare con l'uomo prospettico
e quello faustiano? L'uomo prospettico inventa la prospettiva lineare,
che conoscevano i greci e i romani. Saggio del 1435, \textit{De
pictura} di alberi. E perché è importante per la modernità? A
differenza di quella classica, applica la \emph{matematica}, non è
empirica. Introduce l'elemento matematico della rappresentazione e
nella prospettiva succede che è necessario violare un principio di
Euclide, che le rette non si incontrano, e invece qua c'è il punto (?)
che dà il senso di profondità. Questa è l'illusione della terza
dimensione. Questa era la testa di trucco di Platone,
\textit{Repubblica} libro X, la coppia della coppia, oltre la verità,
che cosa fai? Che cosa aveva in testa Platone? Il modello prospettico.
Scrittura delle ombre, usava questi termini perché quello che gli dava
fastidio era proprio l'inganno all'occhio, il fatto che uno poteva
confondere congenitamente il letto dipinto con quello reale costruito
dall'artigiano. Non è che ce l'avesse con la pittura in general , ma
con quella che passava dalla rappresentazione alla coppia, era questa
la ragione, e quindi è un grande problema filosofico, e questo libro è
stato una dannazione per secoli. È quasi tutti gli artisti del
rinascimento sono plotiniani in questo senso, anche uno come Marsilio
Ficino, perché Plotino esce dalla condanna, rovescia le cose e apre un
discorso sulla bellezza. Il problema di Platone è che non accettava il
fatto che sparisse la differenza, tant'è vero che nel terzo libro,
quando condanna il teatro, ma come fa se scrive scene teatrali? Quello
che non accetta è lo stesso della pittura ma attraverso le parole. Ci
vogliono le virgolette, quelle segnali che stanno parlando loro,
confondi la prima coppia con la seconda coppia. 

Ad un certo punto Fauerbach si rifà al decimo libro della
\textit{Repubblica}, cosa rappresentiamo e quando? La miglior
rappresentazione possibile è quella più fedele? Davvero siamo sicuri
che l'imitazione migliore è quella che diventa una coppia? Se è una
coppia non è più conoscenza. Racconto delle mille e una notti, non è
soddisfato dal racconto geografico finché fanno una duplicazione del
territorio, ma a questo punto non c'è più conoscenza. La
rappresentazione ha senso se non c'è differenza fra la mappa e il
territorio, galleria di Escer dove i quadri diventano i passaggi, René
Magrit, fa i disegni della pipa, della mela del formaggio, il più
reale possibile, e poi mette, questo non è una pipa, una mela, ecc.,
ed è epistemologicamente rilevante perché al massimo livello di
somigliane , quella è un'asserzione sul mondo e non è reale. Tu devi
cogliere il visibile nel visibile, l'arte non ti fa cogliere qualcosa
che non c'è ma che vedi, e quello ha a che fare con la profondità, e
chiaramente il modello prospettico, com'è noto, associa la profondità
con la distanza inevitabilmente? Ma siamo sicuri? Per i greci non era
così, e non è così per la pittura contemporanea, ma il modello che
prevale è l'uomo faustiano, colui che applica la matematica alla
riproduzione del mondo. Il problema qual'è? Come si riproduce questo
attraverso la geometria? Un dispositivo, il sistema degli assi
cartesiani anticipaot, un retticolo, e in questo modo riproducono in
modo tecnicamente forte l'immagine, e quindi è una tecnica, e Spengler
è un matematico, e quindi questo cosa comporta, che nella pittura
attraverso la tecninca, profondità di distanza, e si costruisce
un'idea di infinito basato su questo tipo di cosa, si crea l'uomo
moderno, due linee che si incrocciano, un punto fermo che apre a un
infinito. Non è l'apeiron di Anassimandro, non è nel senso greco. Qui
c'è una differenza. Così la profondità greca non è associata a una
distanza. 

Inizialmente è ciò che contiene in basso l'acqua per i greci, ed è dal
punto di vista del contadino, il vaso, e ci sono più tecniche in greco
per il concetto di profondità. L'idea che sia distanza, un \emph{punto
di fuga}, questo inaugura un nuovo rapporto soggetto-oggetto, e
Spengler coglie la questione. Ecco perché per noi l'uomo faustiano è
l'uomo scientifico, la scoperta del sapere scientifico attraverso
l'astratto. Questa è la cosa che lo caratterizzano rispetto alla
tradizione della \emph{modernità}, stiamo parlando di questo. Bacone,
Cartesio, questa è la modernità, magari con il rinascimento in mezzo,
perché c'è la tradizione della storia della filosofia, però le cose
sono più complesse, e dove queste cose sono paradigmatiche, discorso
sul metodo doveva introdurre tre libri, uno dei quali era l'ottica,
perché lui conoscenza questo e anche Galileo. Venivano usati gli
strumenti, e questo è solo un aspetto della questione ma ci sono
altri, strumetni che applicano la matematica concretamente per
riprodurre il modno, ed ha un'esplosione internazionale questa cosa,
ma è un modello che si lega anche all'invensione del paesaggio,
invensione moderna, focalizzare dentro una cornice un mondo, e oggi
c'è tutt'una discussione sul paessaggio. Il quadro apre una porta alla
portabilità, un quadro si sposta, un affresco no. Cambia la modalità
della rappresentazione, e uno come Cartesio fa sì una rivoluzione, ma
ci arriva, ci sono una serie di elementi che lui ha messo genialmente
insieme.

\end{document}
