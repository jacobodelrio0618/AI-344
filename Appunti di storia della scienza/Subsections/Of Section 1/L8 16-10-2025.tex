% !TEX root = ..\..\main.tex
\documentclass[../../main.tex]{subfiles}%
\ifSubfilesClassLoaded{\addbibresource{../../bibliography.bib}}{}

\begin{document}

% Immagine di Blumenbach. De genirs umanae

Fondatore del razzismo scientifico, fa classificazione della razza
umana. È un personaggio estremamente influente nella storia naturale
in Germania e Francia. È una delle influenze anche di Kant, e anche i
filosofi hanno la loro parte nell'origine del razzismo.

Si riprende con la genetica, e però quello che vediamo è che ci sono
influenze locali che danno forma alla scienza e che di fatto
modificano lo sviluppo scientifico di questa disciplina fondata sul
modello della doppia elica. Già a inizio Novecento viene creato il
termine, e quindi da allora si inizia a parlare concretamente di
genetica, la riscoperta di Mendel nel 1900, e parte tutto un nuovo
filone di ricerca riguardo la genetica. Questa storia è una nella
quale le scoperte vengono soprattutto in Inghilterra e Stati Uniti. In
Inghilterra si ha un altro filone, la scoperta della doppia elica, gli
svizzeri che scoprono la molecola della nucleina, però appunto questa
storia che avviene nei primi decenni è narrata dal punto di vista
europeo-statunitense, e si è visto che c'è una serie di tradizioni
sviluppati all'interno della genetica, in particolare quella umana.
Utilizzare i concetti della biologia alla specie umana, in particolare
si parte, di nuovo, dal Messico, dove questo genetista, Alfonso Leon
de Garay studia la generica degli atleti, che a loro volta protestano
contro il razzismo, e ci si ricorda il pugno alzato, e senza scarpe.
Soprattutto, De Garay cerca di studiare le caratteristiche biologiche
degli atleti, con questionari e domande topologiche per scoprire
caratteristiche biologiche speciali, caratteri ereditari selezionabili
nella classe di atleti. Qualcosa che riesce a mettere in atto nel 68
con aiuto del governo. Si pensa che l'antropologia, attraverso la
costruzione di tipi di umani ideali, riuscisse a migliorare la
popolazione. In generale, la genetica umana ha un ruolo fondamentale
per creare l'identità nazionale. Questo approccio di mettere insieme
genetica e antropologia attraverso gli atleti era già stato fatto
nelle olimpiade di Roma del 60, da Luigi Gedda. È un medico
genetista, studia i gemelli, estremamente cattolico, molto vicino agli
ambienti cattolici della democrazia cristiana, e fonda i comitati
civici che spingono il voto ai cattolici nel 1948, dove si sconfigge
il fronte popolare dei socialisti e comunisti. In realtà da genetista
era convinto che si potessero andare a cercare caratteristiche che non
fossero selezionate. Scrive che gli atleti sono selezionati come i
migliori tipi umani, e questo dà l'idea che si può fare anche nella
società, al di fuori dello sport stesso: eugenetica. Ha messo su poi
il principale studio della genetica in Italia, istituto Mendel a Roma.
Questo a dimostrare che c'è influenza che si sposta in maniera varia.
Un po' di storia.

Selezione artificiale degli umani, è un approccio che può essere
diviso in due, eugenetica positiva e negativa, Francis Galton, cugino
di Darwin, scopritore dell'unicità dell'impronta digitale. Sostenitore
delle teorie del cugino, vuole andare a cercare i tratti ereditari
negli umani, disciplina che tenta di misurare tutto all'interno della
specie umana, il cranio quanto è largo, ecc., misurare tutto per poi
cercare variazioni statistiche, e se questi tratti possono essere
selezionati e fare in modo che si possano riprodurre di più tratti
desiderabili: la ricchezza, l'intelligenza (casualmente gli operai non
vanno mai in università). Positiva e negativa, prendiamo quelli che
hanno frequentato l'università e facciamo riprodurli, al contrario si
può prevenire la produzione dei peggiori, e poi nella versione
nazista. Un'idea che queste caratteristiche fossero biologiche, così
come i disabili, che vengono uccisi dal regime nazista, che coinvolge
i medici, le famiglie, peso per la società e rischiano di inquinare la
razza. Questo si vede nella Germania nazista, ma è presente ovunque,
anche prima negli anni Venti. In Stati Uniti molti stati avevano leggi
di sterilizzazione forzata, che incoraggiano o scoraggiano la
riproduzione, ed è anche una cosa che si trova fino agli anni Settanta
nei paesi scandinavi, sterilizzazione forzata, groenlandesi peso per
la società danese. Ma è qualcosa che avviene anche in Norvegia e
Svezia per l'alcolismo, che è considerato un problema per la società,
avrebbero contribuito con i loro tratti negativi all'eredità della
popolazione. Ovviamente il problema fondamentale che troviamo qui è la
definizione di caratteri ereditari: alcune cose sono semplici, le
malattie e così via, probabilità che può essere prevista all'interno
delle generazioni, colore della pelle, degli occhi, altre cose sono
più complesse, l'intelligenza, e che cos'è? Saper suonare il violino?
Sapersi rapportare con le persone? Tutte queste tre cose insieme? I
talenti sono complessi da misurare e delimitare. C'è un problema nel
capire cosa sono queste cosa, se sono ereditari, e quanto influisce
la cultura. Il tema dell'alcolismo e del criminalised partono
dall'idea che fossero caratteri biologici non influenziate dalla
cultura. Questa genetica cade ufficialmente in disgrazia dopo la
seconda guerra mondiale, processo di Norimberga, e da allora parlare
di eugenetica fa sì che si parli di eugenetica nazista. In realtà
passa all'interno dei genetisti, come De Garay, Gedda, ma non solo,
tutt'una serie di caratteri eugenetici che rimangono nella cultura
scientifica. Alcune cose vengono rifiutate pubblicamente. Gedda fa dei
questionari profondi, ampi, intimi che vuole portare agli atleti: sei
un amante passionale oppure freddo? Per esempio la federazione
britannica dice: non rispondete. Quindi i dati di Gedda sono in realtà
non molto funzionali. Il campione non è assolutamente significativo.

Negli anni Cinquanta e Sessanta sono sotterrate nella comunità
scientifica, e non se ne parla nella centralità della comunità
scientifica, ma in luoghi periferici. Italia, Messico, India, ecc.,
Gedda è uno che diceva di non aver nulla contro i neri, però i metici
da cui l'Italia si è riempita, meglio che rimangano tra di loro,
sarebbero stressati se vivessero fra i bianchi. Facciamo dunque scuole
apposite per loro. In questi luoghi non centrali queste tradizioni
rimangono. In Messico c'è la volontà di vedere il mix etnico come
positivo, invece per Gedda la purezza razziale è importante --
qualsiasi cosa essa significhi. Al centro, invece, si spinge sempre di
più per rifiutare il razzismo e l'eugenetica. Negli anni Cinquanta
l'UNESCO dichiara che il razzismo è scientificamente infondato, alcuni
ascoltano, altri no in particolare laddove si è un po' alla periferia,
e in qualche modo queste tradizioni sotterranee hanno qualcosa da
fare. De Gara è in contatto con Gedda, e sono in contatto con gli
ultimi genetisti inglesi, dove esisteva prima la società di
eugenetica, ecc. Ci hanno messo un po' ma se ne sono liberati, e in
Italia molti non hanno accettato. In India, nell'India di Modi che si
sta configurando come stato etnocentrico, fondamentalismo induista
relativamente nuovo che mira a raccontare l'India come stato
etnicamente uniforme. Idea di una razza indiana diversa dai musulmani
pakistani, veri abitanti millenari, gli altri diverse minoranze
arrivate dopo. La biologia, il sangue sono un fattore identitario
fortissimo. La genetica come scienza si pone sempre di più come
scienza dell'identità umana, e questa retorica può essere adattata o
contrastata a seconda del contesto culturale. Ci sono tradizioni dove
ci si identifica nativo americano. Nella Roma antica non si pensava
all'eredità come tema di sangue. Credenze diverse, pratiche
scientifiche diverse.

Ovviamente alcuni di questi aspetti hanno tratti positivi. Se da un
lato c'è un rifiuto pubblico dell'eugenetica, molto spesso si pensa a
Hitler, e questo è un discorso che la bioetica cattolica fa nella
selezione di impianto nella fecondazione artificiale. Andare a vede
evidenza di patologie genetiche che conosciamo, geni che espongono
con certezza a una malattia, che embrioni impiantare nell'utero. La
bioetica cattolica si rifiuta dicendo che è eugenetica, eliminare
individui sulla base di preferenze: individui malati non devono
nascere. E nei dibattiti pubblici, l'arma è il nazismo, non
distinguendo naturalmente tra un embrio in una piastra e un uomo già
sviluppato. Incredibilmente troviamo che questo approccio ha trovato
un enorme favore attraverso quelle che sono state le principali
vittime dell'eugenesi, gli ebrei. In essi ci sono malattie che non ci
sono in altre comunità -- si riproducono tra di loro comunemente --
hanno riconosciuto ed affrontato questo problema, prima con consulenza
genetica, consigliare cosa fare, avete possibilità di produrre una
prole malata, e quindi c'è il rabbino di turno che consiglia.

L'idea era individuare i portatori e dire di stare attenti quando vi
riproducete. Strumento di consulenza genetica pre matrimoniale. Si
sviluppa moltissimo all'interno delle comunità ebraiche, e la si
ritrova all'interno dello stato di Israele. Si trova anche un mito di
origine anche nelle comunità ebraiche, si tenta di seguire la
genetica, l'idea che l'endogamia sia stata così forte così da
preservare l'identità biologica della comunità ebraica. Si cerca di
ricostruire il sangue originario degli ebrei sparsi, si deve trovare
anche a livello genetico. Per anche ricostruire l'idea di popolo e di
nazione. Non diciamo che lo stato di Israele è stato un esempio di
costruzione della nazione sul tavolino, che gli ebrei siano un popolo,
che il popolo è fatto dal sangue. I popoli si devono autodeterminare,
e il sangue conta nella creazione dello stato. Così anche lo stato di
Israele cerca il sangue delle nazioni. Per farlo utilizzano le nuove
tecniche, c'è un forte studio dietro questo, e ci sono forti
avanzamenti che vengono prodotti dai genetisti israeliani, tecniche
per stabilire la genealogie di diverse persone, le parentele, e
nascono soprattutto a inizio Novecento, che si scoprono i gruppi
sanguini. Si stabilisce anche subito come prove di paternità, e anche
nella medicina legale. Tecniche di immunogenetica, fondamentalmente
qui si trova questa lunga storia:

Una tensione fra globale e locale, veniva indebolita dalle istituzioni
globali, Nazioni Unite, UNESCO, ecc., la razza non esiste, ma siamo
tutti un'unica specie, anche se siamo diversi all'interno delle
popolazioni. Non possiamo basare su queste caratteristiche l'idea di
razza o di gruppo etnico. A livello locale, invece, si cerca
l'identità etnica in modi diversi a seconda del contesto globale, in
Messico, Italia, Israele. Molto spesso queste distinzioni si basano
asiatici, neri, polinesiani, ecc. Non ci sono sfumature, non ci sono
differenze. Hutus e Tutsi, si creano differenze biologhce che non
c'erano prima, anagrafe, colonialimso, si decide che un gruppo può far
parte delle amministraizonie gli altri fuori, imparità di ricchezza,
contribuendo a divisioni culturali. Storia del genocidio in Rwanda,
passa diversamente alla storia del razzismo classico. Disugualgianze
che vengono biologizzate.

Sequenziamento di tutto il DNA umano. Si conclude nel 2001. Quel
genoma che hanno sequenziato nel corso di circa 25 anni di lavoro era
di 6 individui, quindi assolutamente non rappresentativo, e anzi
proveniva di un individuo maschio, bianco, nordamericano, e questo fa
sì che diventasse il riferimento del genoma umano, però ha fatto sì
che fosse la sequenza del genoma bianco americano, e si parla il
concetto di genoma cinese, genoma indiano, ecc. Anche in Russia c'è
questo. Ognuno utilizza queste pratiche in modo diverso, anche se c'è
l'idea di trovare l'identità di un individuo in una popolazione nel
DNA, che ovviamente è una narrazione costruita dagli scienziati
stessi. È una narrazione che ovviamente il genetista ha portato
avanti, le caratteristiche umane, il sangue di questa biologia ultima
invisibile, in essa si trova l'identità, che non è data da dove si è
studiato, dal contesto famigliare, ecc., ma da questo DNA che si
eredita, che è immutato, e che può essere manipolato con la selezione
e altre cose. Sempre nella condizione che certe cose vadano studiate e
operate solo a livello del DNA. Se trovo, come si sta trovando, dei
geni che aumentano il rischio di malattia, si può modificare fino a
ridurre il rischio. Questo fa sì che possa fumare tutta la vita con
minor rischio, e le aziende del tabacco sarebbero interessantissime a
una popolazione del genere. Qualcosa di culturale -- il fumo -- si
sposta alla biologica. Alla fine è genetica, non educazione o società.
È un discorso di quanto si vuole utilizzare il discorso scientifico
alle narrazioni politiche che costruiamo, e di quanto la politica er
la società voglia assorbire il discorso scientifico, anche seguendo le
possibilità di accesso a certe tecnologie piuttosto che altre, se uno
stato può manipolare geneticamente la popolazione o no, se ha i soldi.
Le scienze come la genetica sono influenzati dalla circolazione di
conoscenza da contesti globali a contesti sociali, e si parla della
genetica agraria e quella umana, vanno a studiare in Inghilterra,
stati Uniti o Europa. Anche il libro di Bosket è basato solo su
scritti in inglese, di tanta letteratura secondaria dei suoi colleghi,
e bisogna stare attenti a queste narrazioni e quanto sia difficile
uscire da queste prospettive influenziate dal contesto culturale.

\end{document}
