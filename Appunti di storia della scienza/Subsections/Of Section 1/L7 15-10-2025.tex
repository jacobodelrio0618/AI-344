% !TEX root = ..\..\main.tex
\documentclass[../../main.tex]{subfiles}%
\ifSubfilesClassLoaded{\addbibresource{../../bibliography.bib}}{}

\begin{document}

% 27-11 seminario Fuller.

Oggi si riprende quello che è più o meno l'ultimo capitolo del libro
di Bosket, sulla parte della genetica. Non vediamo un concetto forte
come la rivoluzione scientifica, ma come si decentralizza nel
contemporaneo, e vedremo come una delle scienze come la genetica
abbiano una mappa di produzione molto diversa dal \bsq{nord globale}
ma sua qualcosa di policentrico, che va più sulla circolazione della
conoscenza. Quindi una narrazione centro-periferia non è adeguata a
cogliere quello che sta succedendo nella scienza, ma le conoscenze
arrivano alle culture, circolano, e vengono riprodotte in nuove forme,
con nuove domande, concetti e nuove epistemologie a seconda della
società. Messico, Cina, India, Israele, sono paesi che hanno
largamente contribuito allo sviluppo di questi temi. In tanto, cosa
succede soprattutto nella genetica nei primi decenni del Novecento?
L'utilizzo della radiazione, che si conosce dagli anni Venti, e Muller
vince il nobel nel 1946, e lavora soprattutto su moscerini.

A partire degli anni Dieci Morgan comincia a utilizzare moscerini
perché sono facili a riprodursi, e hanno cromosomi facili da vedere
accanto ad altre caratteristiche. Si possono vedere tutt'una serie di
proprietà che possono essere modificati geneticamente, si scopre che
c'è una ragione fisica dei geni, e che sono situate sui cromosomi.
Facendo tutt'una serie di incroci possono vedere dove sono posizionati
queste cose. In particolare, con l'utilizzo delle radiazioni sappiamo
in che dimensione stiamo, urto fisico di queste praticelli che urtano
il materiale genetico e lo modificano, e Muller è capo di questo,
effetto si radiazione ionizzante sul genoma. Poi nella seconda guerra
mondiale, a genetica molecolare esplode grazie alla bomba atomica.
C'è un interesse di ricerca di base per le modificazioni genetiche. A
partire delle bombe c'è grande interesse generale per studiare gli
effetti umani di queste radiazioni, e non è una cosa nuova: la
radiazione si usava già contro i tumori, e si sapeva che troppa poteva
essere cattiva, ma ora acquisisce un interesse militare. D'altro canto
c'è anche un aspetto dell'inizio della guerra fredda, e gli stati
uniti si preoccupano di mantenere buone relazioni con stati come il
Giappone. Quindi per motivi anche poliotici gli stati uniti iniziano a
studiare gli effetti della radiazione sui sopravviventi di Hiroshima e
Nagasaki, si crea l'Atomic Bomb Causality Commission, e attrae anche
scienziati giapponesi, e dei principali sono Masao Tsuzuki e Masuo
Kodani.

Da un lato è onnipresente il rischio di una guerra nucleare quanfo i
sovietici riescono a produrre una deterrenza nucleare, ma in generale
perché c'è la diffusione dell'uso pacifico del nucleare, e quindi se
ne studiano i rischi. Buzzati-Traverso, va in Germania, torna in
italia dopo la guerra, diventa un genetista noto, e crea uno studio a
Napoli che si chiama (?). Lo studio delle radiazione era
particolarmente diffuso civile e militare, la cassaccia a pascati, vicino a Roma, poi ENEA, utilizza le radiazioni ver vedere gli effetti sulle piante, il campo gamma, palo con fonte di radiazione, piante intorno che vnegono irradiate, e poi si va a vedere cosa segue con queste radiazioni, e in questo modo si crea una serie di varietà di grani molto utili, tra cui il grano plesmo (?), le cui varietà si usano in tutti i grandi d'Italia, ibridi nati appunto con le radiazioni alla casaccia. In questo contesto di guerra fredda, l'uso pacifico del nucleare viene usato come ferramenta diplomatica con una serie di stati. Poi col decolonialimso c'è il problema di alimentare la popolazione, che inoltre sta crescendo con il baby boom. È la cosidetta rivoluzione verde, enorme cambiamento nella produzione che nasce con Norman Boularg, che addirittura vince il nobel di pace. Fertilizzanti, varietà di maggior rendimento, soprattutto dove c'è più bisogno. Cina, India, Messico, Brasile, che esplodono con popolazione.

La supremazia teconologica diventa un modo in cui ongi blocco tentat
di portare vicini a sé i diversi governi. È qualcosa che veniva già
dal passato, si pensi alla Rockefeller FOundation che aveva fondato
cose pubbliche in tutto il mondo per evitare che paesi cadessero al
socialismo, progetti di sanità pubblica, malaria, ecc. La Rockefeller,
negli anni Trenta, lavora nei paesi destinatari. A partire dagli anni
Quaranta, dopo la guerra loro come la OMS che viene fondata e la Ford
Foundation applicano uno schema più diverso. Si danno borse agli
scienziate per andare a studiare nei buoni laboratori statunitensi, e
poi tornano a casa ad applicare quello che hanno imparato, e in questo
modo si toglie potere ai governi locali, e pensano che questo è più
efficiente nell'evitare il comunismo. Questi che mettono i soldi si
scontrano talvolta con le comunità locali. Creare un istituto a Roma
col governo fascista, non si occupa solo di Malaria ma di altre cose
per necessità diplomatiche. Rivoluzione verdwe, maggior rendimento,
perdita di biodiversità. Si inizia a coltivare lo stesso tipo di grano
di alto rendimento, rese adatte con fertilizzanti a certi terreni,
inquinamento.

In particolare il concetto stesso di rivoluzione verde nasce per
contrapposizione alla rivoluzione rossa, la principale preoccupazione
degli stati uniti che si imbattono nelle guerra fredda. Il programma
agricolo messicano nasce dalla Rockefeller. Messico ha una tradizione
socialista che spaventa gli Stati Uniti. Teniamo presente anche che in
questo periodo filosofi della scienza, ecc., vedono essa come un
motore di benessere, caratteristiche della nuova scienza, liberalismo,
ecc., proprio per far fronte alle perversioni della scienza che
c'erano a quel tempo nei regimi totalitari, prima con i nazisti, poi
con la genetica sovietica, e cose che, dice Thomas Bacon che dice che
ciò non può accadere nelle democrazie liberali. La buona scienza è
utile per lo sviluppo della nazione, e dovranno creare delle
istituzioni scientifiche, e passo a passo andranno in democrazia
liberale. Scienza moderna, idea di razionalità unica, rivoluzione
scientifica, nuovo modo di guardare il mondo nato in Europa.

Conoscenze indigene, E. Hernandez Xolcotzi, produrre varietà di alto
rendimento che produrrano aumento della produzione alimentare in
messico. Conoscenza viene negli stati uniti, viene complementata con
conoscneze locali e si arricchisce, e grazie all'enorme biodiversità
del messico, più varietà a disposizione. Diventa quindi anche un
modello per le altre nazioni, e dal Messico parlano altri genetiste
che tentatno di replicare queste istituzioni altrove. Battaglia del
grano fascista passa attraverso la formazione di genetisti. Libro, tre
contesti totalitari europei, il controllo del governo sulla scienza è
stato estremamente pervasivo, nonostante ci fosse una retorica
ruralista, conservativa e passatista, c'era una retorica che dà
importanza a queste cose come la genetica, e sono cose che danno
origine a queste cose che si usano oggi coem il grano attuale.

Anche in India c'è questo aspetto di ricerca alimentare, uso retorico
e non solo, fiducia della sciena come motore di progresso e di
indipendenza, anche miltare in un contesto particolare. Smiling
Buddha, primo test nucleare indiano, l'india diventa potenza atomica.
Il governo Nehru sviluppa moltissimo la scienza, anche la genetica per
lo sviluppo agrario. Carestia degli anni quaranta, produzione agricola
indiana verso altre parti dell'Impero, e affamano l'India. Hanno
questo riccordo neto della carestia in testa, e quindi è una
preoccupazione del governo, e anche dimostrare di poter farcela da
soli, senza altri forme di colonialimso. Istituto nato anche sotto
l'impero britannico, che crea istituti scientifici nelle colonie, in
India, nella Palestina, poi questi istituti sono fondamentali per la
formazione delle nazioni indipendenti.
Con ideologia dietro, che è appunto l'indipendenza, ed è il risultato
di queste dimensioni globali e locali, che sono fondamentali per
sviluppare questo tipo di conoscenza. Da qui nasce una grande scuola
di genetica cfr. foto, si forma in maniera cosmopolita, girano il
modno, tornano in India, e grazie al supporto governativo lavorano
notevolemnte e riescono a raggiungere risultati importanti. Iniziano a
pubblicare in riviste internazionali, e questi di paesi poveri
cominciano ad accedere aglis apzi internazionali della scienza,
Nature, Science, ecc.

Problema sudeste asiatico, possibilità militare, necessità mediche
legate alla guerra del Vietnam, agente arancio. Storia di Rodrigues.
Anche qui l'aspetto dell'influenza di fattori globali, guerra fredda,
indipendenza indiana, scontri interni all'India, questi grandissimi
eventi storici favoriscono lo sviluppo scientifico. Non è un caso che
c'è una zona in via di sviluppo e ora si hanno due potenze nucleari.
Per questo punto di vista replicano modelli globali di deterrenza
nucleare, e su questo della deterrenza parlano scienziati, logici,
ecc., è meglio colpire per primi o aspettare? Nasce una serie di
teorie di logica sociale.

In cina si ha un percorso simile, anche lì c'è un storia di
collaborazione internazionale, dal 49 c'è la rivoluzione di Mao, la
sicurezza alimentare viene declinata in maniera diversa, prima per i
contati con Mosca. L'economia centrale pianificata busenko, agronomo
sovietico degli anni trenta, decide su esperimenti falsi o
interpretati male che le piante acquistano proprietà dell'ambiente. Se
riesco a far fiorire grano in freddo, i semi che ne escono
cresceranno nel freddo. Nasce quindi una tradizione agraria sovietica
che non è adatta; parte una riorganizzaizone politica completa delle
istituzioni agrarire dell'agronomia e la genetica sovietiche.
Contemporaneamente, nel mondo occidentale si va in direzione opposta.
Rivoluzione dareiniana, non prevvede eredità di cose acquisite, e qui
la cosa darwininana si unisce alla genetica di Morgan e Muller. SUlla
base di gisenko, la genteica occidentale viene detta contro il
socialismo, quindi tutta una generazione di genetiste viene mandata ai
gulag o fuori dell'unione sovietica. Il passaggio è che le idee sono
così forti da cambiare la natura, uomo socialista con nuove
caratteristiche anche biologiche, e lo possiamo fare come i semi.
Quindi questa nuova visione della genetica si adatta e supporta
l'ideologia. Risultato: carestia. Mortalità nel regimi comporta a
mentire nelle statistiche, la popolazione muore. In Cina si afferma
questa genetica dopo la rivoluzione comunista, gli agronomi vanno in
Unione Sovietica, poi i rapporti non sono buoni, e un uivo colpo alla
genetica cinese viene nellla lotta contro i quattro mali, le zanzare,
le mosche, gli uccelli che mangiano i grani e i topi che si trovano
nei granai. Tra il 59 e il 61 c'è un'enorme carestia, millioni di
morti. Questo perché l'utilizzo di pesticidi delle zanzare uccide
altre cose utili, uccelli esterminati, popolazione di insetti nocivi
diventa enormi. Errori di calcolo fatti sul campo dagli agronomi
cinesi, che fanno sì che questo non funzioni.

Sino soviet split, valorizzazione della conoscneza locale, libertà
della ricerca si entiifica, si riesce a tornare a lavorare anche con
la genetica occidentale, e si arriva finalmente a creare qualcosa che
negli anni sessanta è impensabile, il riso ibrido. Yuan Lingping. Uno
di quelli che va nei campi a parlare con i contadini, storia locali ,
trova piante che hanno slo organi femminili nel fiore, il che vuol
dier eche si possono incrocciare con altre varietà e ha rivoluzionato
la produzione del riso nel mondo, non solo in Cina. Poi la rivoluzioen
culturale, gli intellettuali servono a poco, intellettuali mandati a
scavare carbone,  finché si accorgano a cosa serve, segue il corso
della rivoluzione verde, la cina si sta riavviando verso le
organizzaizoni internazionali, imporne vivioni rispetto ai programmi
di sanità pubblica. Cambiano i programmi della OMS che non sono top
down, ma servizzi di base sviluppati secondo la necessità di ogni
paese. Servizi id base e non cose sofisticate, cosa particolarmente
padroneggiata dal socialismo cinese. Parte con un programma di
modrnizzaione, novità teconoligca servizi di base in cui si fa
accenno. In tanto diamo un minimo di salute a tutti e da lì partiamo
come base per la sanità della popolazione. D'altro canto, i rusultati
cinesi influscono sulla rivolzuionei verde per il tema del riso, che
fa sì che il sudest asiatico raggiunga indipendenza aliemtnare.
Passano da paesi che subiscono carestia a paesi esportatori. Stessa
cosa in Israele, in cui questo si nnesca non solo nella creazione
dlela nazione nella guerra fredda ma anche nel colonialismo locale
dove nasce l'israele. C'è influenza coloniale, itsittuto profotto del
mandato britannico, l'agricoltura diventa la modernità che gli
occidentali, in particoalri gli ebrei stanno portanto in quel posto
dove c'erano pocchi abitanti incapaci di sfruttare, e anche questo
influisce nel colonialismo. Fa parte della narrazione iniziale del
colonialimso. In realtà c'erano sfruttamenti delle risorse, sebbene
diversi dei canoni occidentali. L'agricoltura è fondamentale, produrre
cibo in una nazione circondate di nemici. Grande impegno governativo,
grande diplomazia scientifica, grande riconoscimento dell'agricolutura
israeliana. Pomodori nati da cose gentiche fatte da un'azionda
israeliana, pomodori di pacchino.

Nelle interpretazioni di Bosket, tutti questi eventi fanno capire che
le scienze non sono occidentali. Ma poi c'è la critica che questa
gente studia in occidente. Si può parlare di colonialismo culturale?
Impero britannico che fonda istituti che poi vengono mantenuti. Centro
Pasteur, ecc., emananzaione coloniale francese senza essere francese,
porta la scienza francese, è un'emanazione coloniale. Istituti che
godono dell'immunità nella scienza in quanto dovrebbe essere oggettiva
e immune a perversioni ideologiche. Istituto Pasteur in Senegal, viene
rispettato, ci hanno portato i vaccini e altri farmaci. In Cambogia
tutte le strade cambiano nomi tranné la via Pasteur e la via nominata
dal suo cooperatore. In realtà sappiamo che vengono veicolati valori
attraverso l'impresa scientifica. Da un lato c'è una diplomazia
scientifica centralizzata che fanno sia gli stati uniti che l'unione
sovietica. D'altro canto sono queste conoscenza che permettono di
sviluppare certe conoscneze in loco in base a certe necessità. La
narrazione generale, comunque, è che c'è una scienza occidentale
oggettiva che è la base dello sviluppo.


\end{document}
