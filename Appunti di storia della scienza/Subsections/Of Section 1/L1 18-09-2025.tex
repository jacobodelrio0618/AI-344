% !TEX root = ..\..\main.tex
\documentclass[../../main.tex]{subfiles}%
\ifSubfilesClassLoaded{\addbibresource{../../bibliography.bib}}{}

\begin{document}

\subsection{Heading}

Menzione a Koyré. Portare la storia delle idee dentro la storia della
scienza. Filosofia come parte fondamentale della storia della scienza.
Differenza fra come si faceva al tempo e come si fa al corso. K.
Tratta di storia del pensiero scientifico. Biografie degli scienziati,
metà Novecento, ricostruzione filologica. Richiamo nel fascismo
dell'importanza della scienza. L'idealismo imperante nell'accademia di
quel periodo la storia della scienza viene messa da parte. Gli
umanisti non si occupano delle scienze, riforma Gentile. Non si può
eliminare un Galileo dalla storia della filosofia. In Koyré si trova
ad esempio la discussione di Cusano.

Copernico, rivoluzione scientifica. 1687 i Principia di Newton.
L'istituzione del cosmo. Mondo chiuso e ordinato, gerarchia.
Cosmologia di Aristotele. Koyré, il cosmo viene distrutto, viene
aperto. Non c'è il limite delle sfere fisso, ma sono sparse in giro
per lo spazio che diventa infinito. Le leggi celesti sono le stesse di
quelle terrestri. Le sfumature della matematizzazione del mondo e
dell'universo in generale. Tutto l'universo funziona allo stesso modo
nello spazio infinito, quindi è uno spazio che diventa uniforme e lo
si può descrivere con pochi principi. È quello che secondo Koyré è una
rivoluzione, ed è importante. Anche c'è Butterfield (1965). Ora,
Butterfield è uno storico vero e proprio, a differenza di Koyré, che
si era occupato anche di filosofia della scienza. È interessante
andare a vedere perché permette di vedere (1) gli episodi fondamentali
della rivoluzione scientifica. Pierre Duhem, alcune cose che si
dicevano dal Settecento non erano nuove, per esempio la rivolta contro
Aristotele. Non è vero invece che dall'interno della chiesa non ci
fossero critiche ad Aristotele. È una lettura ingenua abbattere
Aristotele e pensare che c'è una rivoluzione scientifica. Altra cosa
che fa Butterflied e che va tenuta presente nel corso è collegare
questa rivoluzione scientifica a tutta la storia dell'occidente (p.
210 Butterfield). Conflitto fra l'Europa e l'Asia, la più grande
minacce invasioni dal cuore dell'Asia. Indipendenza e supremazia
europea, certe cose che possono solo nascere in occidente, ed è in
questo contesto in cui si mette la rivoluzione scientifica. Libro
pubblicato in piena guerra fredda, Russi europei ma dominati dagli
asiatici per tanto tempo. Modernità che dimostra la supremazia
dell'Europa, e poteva accadere solo lì. La Cina non va nemmeno presa
in considerazione. Quindi quando si usa il termine \bsq{rivoluzione
scientifica} va inquadrata in questo contesto. La modernità non è
semplicemente una tecnologia ma una mentalità diversa che ci rende
moderni. Che ci fa passare dall'universo chiuso allo spazio
dell'infinito. Dal per lo più e dal verosimile alle cose precise e
misurate fino a quantità infinitessimali. Geometrizzare il mondo.
Questa mentalità è connessa intrinsecamente alla cultura Europea.

La storia e la Rivoluzione scientifica stanno lì, ignorando una serie
di cose e di strutture sociali, influssi tecnologici da altre culture,
o certi concetti introdotti perché erano necessari per certe strutture
di potere, e il risultato di lotte fra poteri diversi, il progresso
della cultura europea è a volte legato a lotte di potere che hanno
avuto quel particolare sviluppo. Basterebbe allargare un po' l'ottica.

C'è stato sicuramente un grande sviluppo della scienza e la tecnologia
in Europa. In Cina hanno investito in cose diverse per ragioni
precise. Quando arrivano i Gesuiti trovano una società molto avanzata
in grado di capire ciò che riportano i gesuiti. Riportano anche una
nuova astronomia. Sono pronti a ricevere questa \bsq{modernità}. Poi
c'è sicuramente un periodo da grande divergenze, in cui date delle
condizioni materiali simili l'occidente, in particolare
nell'Inghilterra c'è stata una rivoluzione industriale e non in Cina.
Allo stesso modo più recentemente ci sono stati storici che hanno
parlato di un'altra rivoluzione, quella matematica dei greci, che era
arrivata a una sofisticazione simile alla matematica del Seicento.
Anche i greci, nonostante appunto credessero ad Aristotele, erano
molto avanti, e mettevano le capacità in modo diverso. È chiaro che
l'emergere della scienza moderna è strettamente connesso al
capitalismo. L'idea di rivoluzione scientifica è già presente
nell'\textit{Encyclopédie}. Tutte queste persone sono il canone della
rivoluzione scientifiche e sono moderni, non nel senso cronologico, ma
un momento di rinnovamento rispetto a un passato brutto e cattivo.
Anche Butterfield cercano una genealogia, precursori, ecc. e tuttavia
non disconoscono la grandezza di Aristotele, ma di quelli che hanno
lavorato con Aristotele. La storia della scienza nasce anche per
trovare genealogie che sono d'autorità. Le storie delle discipline
scientifiche nascono proprio da questo. Si mette l'accento su rutture
nette, ad un certo punto dopo la caduta dell'impero romano non si fa
più nulla... Fibonacci scrive per i mercanti, è una cosa che serve per
fare meglio i crediti e così via. I primi libri di aritmetica sono
libri per queste persone, e non hanno nulla a che fare con Aristotele,
e si copiano dove stanno nascendo le nuove borghesie, che non hanno
nulla a che fare con le università o con Aristotele. Le cose si
trovano in fonti diversi, ed è una cosa che si sta facendo nella
storia della scienza, dai libri ai commercianti ai libri dei segreti,
che a differenze delle opere di Newton o di Galileo non circolano,
perché appunto è la garanzia di dominare il mercato. E anche la
circolazione di queste cose cambiano le cose, dai commercianti a
Newton. Il grande monumento della mentalità moderna, nasce in Newton a
Roma, o nasce in India con la loro notazione diversa? I problemi che
affronta la storiografia della scienza possono essere molto diversi.
La rivoluzione scientifica porta a casi molto interessanti di
separazione delle fonti, i testi nascosti di Newton che sono venuti
fuori. Il grande eroe della rivoluzione scientifica che però pensava
come un alchimista del Quattrocento. Quando si guardano le macchie
solari devono essere interpretate, l'interpretazione è necessaria.
Innanzitutto mi devo fidare del cannocchiale. Cosa c'è nel sole?
Galileo decide di mostrare che sono macchie solari, e che sono punti
di imperfezione nel regno celeste, ma non aveva i dati esperimentali
in questo momento, lui sta interpretando. Esistono esperimenti
cruciali, ma vanno interpretati. Questa è una cosa fondamentale. In
questo periodo c'è una spinta che aumenta il ritmo della conoscenza,
che però dipende dal fatto di questi cambiamenti mentali. Le scoperte
geografiche, la diffusione della stampa, nuove malattie, ecc., e
quindi forse nei libri degli antichi non c'è tutta la sapienza, e
quindi forse il campo d'interpretazione va cambiato. Ancora fino a
tutto il Seicento la testimonianza è importantissima, così come dare
affidabilità alle fonti antiche. Il racconto della natura per tutto il
Seicento si basa su queste cose, sul rapporto e l'autorevolezza della
testimonianza. Si crede all'esistenza del mostro perché la natura è
onnipotente, si producono forme strane, mostri in qualsiasi momenti, i
fossili che si trovano nelle pietre, gli scherzi di natura. E questo
ce lo raccontano naturalisti rispettabili fino al Settecento, e
cercano di dare delle spiegazioni il più possibile legittime
all'interno della loro scienza, in cui esistono forze generative
sconosciute, che in qualche modo influiscono in Newton quando prova a
ragionare sulla causa della gravitazione. Lui crede nell'azione a
distanza, anche se prova a cercare una spiegazione meccanica di queste
leggi. Cercano di meccanizzare il mondo ma non sanno cosa c'è dietro,
e fanno ipotesi assurde, non molto diverse dall'occultismo e
l'alchimia. Quindi è una modernità molto più sfumata di quanto si
pensa. Camere della meraviglie nel Cinquecento. Alla fine questa
macchina del mondo è come quelle che facciamo noi. Non c'è più
distinzione fra artificiale e naturale, non c'è più imitazione della
natura, ma l'uomo ne diventa ministro e interprete. In questo aspetto,
con il venire a meno di questa distinzione fra naturale e artificiale,
ci si mette di tutto. Disincantiamo il mondo pensando che il mondo è
una macchina, ma continuiamo a \emph{meravigliarci} di com'è fatta,
delle bizzarrie, delle forme dei fossili, ecc., è una lettura in cui
la modernità è molto sfumata, la si vede insieme a qualcosa che noi
non diremmo assolutamente essere moderno. Una delle critiche che viene
fatta a Newton è l'introduzione dell'azione a distanza, una cosa
occultista, ed è la critica che gli fanno i gesuiti, i più razionali.
Ma cosa c'è fra i due corpi? Non si sa. È possibile il  vuoto? Chi è
l'irrazionale fra i due? Newton con la forza a distanza o i gesuiti
perplessi? La modernità che passa attraverso la rivoluzione
scientifica e che trascura le sue sfumature è una postura ideologica e
finalistica. Il grande passaggio dalla storia del pensiero scientifico
alla storia della scienza -- da una branca della filosofia a una
branca della storia -- e che si hanno gli strumenti per andare a
vedere cosa c'è fuori dai libri stessi. Le \emph{reti di sapere}, il
contesto culturale, la circolazione dei testi, vanno a complemento e
aiutano a comprendere meglio il pensiero scientifico nella realtà
storica.

\end{document}
