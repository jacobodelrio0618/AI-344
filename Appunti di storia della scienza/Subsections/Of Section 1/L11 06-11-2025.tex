% !TEX root = ..\..\main.tex
\documentclass[../../main.tex]{subfiles}%
\ifSubfilesClassLoaded{\addbibresource{../../bibliography.bib}}{}

\begin{document}

\emph{Black Metallurgists and the Making of the Industrial Revolution}

Furto di innovazione e furto epistemico. Metallurgia mediazione fra il
mondo sacro e quello profano. In Giamaica fu trapiantato questo
patrimonio umano, tecnico e simbolico. Ferro come elemento di potere e
non di lusso, Iron Act (1750), si proibisce la produzione di ferro
nelle colonie, il cui porta alla clandestinità. La metallurgia diventa
una scienza della resistenza, la libertà, ed è qua che ci sono i neri
nella metallurgia. Ciò che sostiene l'autore è che svilupparono questo
metodo basandosi sugli stessi strumenti con cui lavoravano la canna di
zucchero. Three Finger Jack, metallurgista Maroon battezzato John
Reeder. La cosa procede con Henry Cort, uomo di finanza all'orlo della
bancarotta, e riceve fondi, e con attraverso strumenti trapiantati,
disse di aver scoperto questo processo con cui si potevano produrre
queste cose di alta qualità, e depositò i brevetti con i processi
metallurgici dei neri.

L'articolo è particolarmente controverso, e forse si forzano le fonti
in questo riguardo. Quello che Bulstrode vuole emergere è una lettura
decoloniale della scienza, e in particolare prende anche uno dei
contrubuti più importanti. Le 10 tesi di Edgerton.

Decentralizzare l'innovazione dall'Inghilterra, e costruire un
contesto decolinale (?). 

--- The origin of fossil capital

Quello che si cerca di fare riguardano problemi attuali,
l'inquinamento dato dall'infusione di diossido di carbonio, e si
rintraccia l'origine nella rivoluzione industruale, quando cominciò a
essere usato il carbone per alimentare le macchine nei processi
produttivi. Veniva anche usato in altri contesti come la metallurgia,
e per la prima volta viene usato in un motore che permetteva di
infilare il cottone, in questo caso, e ci si chiede com'è avvenuto
questo passaggio, che può essere visto come scontato e invece ci sono
delle criticità. Innanzitutto, da un punto di vista storico emergono
punti interrogativi importanti. 1712 nace la pompa Newcomen, Motore
Watt 1769, 1771..... Motore James Watt, si studia la rivoluzione
industriale.

Perché il passaggio dall'acqua al vapore? David Ricardo e Thomas
Malthus, Ricardo sosteneva che la crescita di un'economia fosse
disponibile attraverso risorse, ecc., e che dovesse fermarsi nella
crescita demografica, il modello malthusiano, limitazione nelle
nascite, ecc., che le economie dovessero destabilirsi attraverso
momenti catastrofici, e quindi molti affermano che il carbone cominciò
a risolversi in una crisi delle risorse. Disponibilità di terreni,
ecc.

C'erano tantissimi punti, invece in cui si vede che questo non è così,
e attorno a 1830, questa produzione, abbastanza conveniente, il vapore
comincia a essere produttivo come l'acqua nel 1870. Quindi perché
motori a vapore nelle fabbriche? Si afferma che è stato il modello che
fonda l'economia fossile, e sono un'economia che ha sicuramente delle
svolte ambientali imporranti. Se si scopre che le macchine a vapore
non erano così efficienti, e che prima venivano utilizzate forme meno
inquinanti, qual'è il fattore che ha scattato l'economia fossile? Una
possibile soluzione è nella teoria sociologica di Marx, cambiamenti
sociologiic innescati dai proprietari dei mezzi di produzione, il
carbone è molto più controllabile dell'acqua, e c'era bisogno di
controllo perché cominciavano a diffondersi leggi che tutelavano i
diritti dei lavoratori, che erano quelle della riduzione della
giornata lavorativa. Poi l'acqua non può essere controllabile,
appunto, e comincia a diventare problematico, che la produzione
dell'acqua sia a pieno regime. Invece il carbone permette tenere sotto
controllo il processo produttivo. Quindi, banalmente, cominciano a
essere costruite fabbriche vicine alle città. Mentre invece il Carbone
permette avvicinamento alle urbi, avvicinare il processo di
produzione, e appunto il carbone diventa conveniente, seppure non
produttivo, perché permette di controllare un fenomeno naturale che è
quello della combustione sufficiente, utilizzare carbone, scaldare le
macchine, e si può decidere quando azionarle e quando spegnerle.
Questo tipo di ragionamento ha presso piede fino alla contemporaneità,
l'inquinamento sistematico, cominciano a esserci problemi per il
possesso di giacimento dei combustibili, e soprattutto si comincia a
diffondere l'ideale capitalistico, la spazio-temporalità
capitalistica, che consente all'essere umano di allontanarsi dalla sua
dimensione naturale, e si pensi a Marx e alla scuola di Francoforte.
Ci sono possibilità in questo contesto per le energie rinnovabili? La
risposta è abbastanza pessimistica, e sembrano essere lasciate un po'
in disparte, e afferma che si potrebbero costruire pannelli nei
deserti, dighe, piante eolici, ma rimane un'energia difficilmente
trasportabile, e il problema dell'industria, ancora predominante dei
fossili.

\end{document}
