% !TEX root = ..\..\main.tex
\documentclass[../../main.tex]{subfiles}%
\ifSubfilesClassLoaded{\addbibresource{../../bibliography.bib}}{}

\begin{document}

Lezione 10

Miranda Fricker

\emph{Matthew effect}, frase del vangelo di Matteo, sostiene che nella
distribuzione delle ricchezze a seguito delle scoperte, o gli
strumenti per fare scienze, vengono distribuiti in maniera
proporzionale alle ricchezze che si hanno già.

\emph{White Bull Effect}. Simile ma in ambito più ridotto, accademico,
ricercatori senior si attribuiscono il lavoro di quelli junior
giovani.

\emph{Razzismo editoriale}. Impatto della ricchezza nei paesi in
quanto, a seconda di chi produce una ricerca, la pubblicazione e
lettura sarà in correlazione al luogo di pubblicazione. C'è un
importante ruolo nella lettura di articoli e non solo nella scrittura,
ci sono soggetti ritenuti più importanti, e sono quelli dei paesi di
alto reddito, ma comunque in generale il pubblico occidentale viene
maggiormente presso come universale, e questo comporta una concezione
di scienza pensata per quel pubblico e una nozione di scienza
costruita su questo frammento del mondo. Le riviste elite spingono su
studi molto generali, analizzano situazioni molto baste, e questo
comporta la marginalizzazione di casi specifici, che, per quanto
importanti, sono sembrano abbastanza degni di studio. Un altro
problema è l'autorità epistemica, il rifarsi solo alle fonti
accademiche, che tende a settorializzare ulteriormente il campione
della ricerca, si restringono i casi a quelli che sono già stati
trattati. La conseguenza è che la diffusione che queste riviste hanno
nei paesi di medio-basso reddito è scarta, perché non venendo
raccontati eventi che riguardano quei territori c'è uno scarso
interesse, e c'è una sfiducia e una scarsa volontà di fare scienza,
riconoscendo l'egemonia dell'occidente.

Un esempio importante potrebbero essere le malattie invisibili, ad
esempio quelle delle donne, il dolore femminile veniva ritenuto come
qualcosa di esagerato, o attribuibile a qualcos'altro. In questo caso
si può trovare sia un'ingiustizia testimoniale, c'è uno screditare il
dolore, ma anche un'ingiustizia epistemica, perché venendo le donne
presse per pazze prima che effettivamente doloranti -- si pensi al
caso dell'isteria -- non sono riuscite ad affermare effettivamente il
proprio dolore.

--- Tra intelligenza artificiale e pseudoscienze

Eticità e robustezza epistemica, i problemi del machine learning.

Fisiognomica e machine learning. DNNs, deep learning neural networks:
malattie mentali, orientamento sessuale, ecc., e favoriscono siti e
compagnie che cercano di usare la fisiognomica in chiave mercantile.
Sito Faception, "facial personality analysis". Utilità commerciale di
questi studi.

Problemaitiche nell'utilizzo machine learning.

Costruzione di inferenze induttive, quanto corrette sono queste
predizioni?

Motivazione nel machine learning, e peer review. Articoli sfuggiti al
processo del peer review.

1) Il machine learning è theory-laden.
2) Regolarità artificiale, anziché regolarità obbiettiva o naturale,
le due si mescolano. Spesso i valori utilizzati per stabilire
regolarità naturali e così via nasce da principi ideologici pressi
come obiettivi. Qual'è lo standard facciale? L'aspetto per stabilire
quale sarà l'anormale.
3) Fattori multipli. Certi dati possono essere influenziate
dall'ombreggiatura.

Problema della base empirica e i dati del machine learning.

\end{document}
