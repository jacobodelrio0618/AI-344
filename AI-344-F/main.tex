%Document Properties
\documentclass[a4paper, 14pt, twoside, notitlepage]{extarticle}
\title{
    Magia e scienza nell'età moderna \\
    \smaller{Spiriti, effluvi e fenomeni occulti. 
    }
}
\author{
    Silvia Parigi, Carocci, Roma 2022 \\%
    \textit{Recensione di Jacobo del Riio Molano}%
}
\date{}

\usepackage{preamble}

% Bibliography Packages and management commands (to be put here in order for the
% bibliographic autocomplete to work)

\usepackage[
    backend=biber, 
    style=philosophy-verbose, 
    latinemph = true,
    language=italian, 
    sorting=nyt, 
    autopunct=false
]{biblatex}

\usepackage[
    style=italian, 
    autopunct=false
]{csquotes}

\DefineBibliographyStrings{italian}{
    loccit = {ibid\adddot}, 
    pages = {pp\adddot}, 
    opcit = {\unskip\addcomma\addspace cit\adddot}, 
} 

\addbibresource{bibliography.bib}

% Command to add a comma before "cit.,".
\renewcommand{\mkcitation}[1]{#1}
\newenvironment*{smallquote}{\quote\smaller}{\endquote}
\SetBlockEnvironment{smallquote} \AtBeginEnvironment{quote}{\smaller}

\begin{document}
\maketitle

\blockquote[{\autocite[51-52]{voltaireDalleLettereFilosofiche1962}}]%
{Un francese che càpiti a Londra trova che le cose sono molto cambiate
nella filosofia come in tutto il resto. Ha lasciato il mondo pieno, e
lo trova vuoto. A Parigi, l'universo lo si vede composto di vortici di
materia sottile; a Londra, nulla si vede di tutto questo [\dots].
Presso i cartesiani, tutto avviene per effetto d'un impulso
incomprensibile; per Newton, invece, in forza di un'attrazione di cui
non si conosce meglio la causa}. %
%
Voltaire -- diffusore del newtonianesimo in Francia -- ci dà un
ammonimento sommario del cambiamento di mentalità che attraversava
allora la filosofia, lasciando vedere le due vie rimaste dopo il
crollo dell'edificio scolastico: vano, disputatorio e inutile, insulso
al gusto \bsq{moderno}; cambiamento profondo o retorica illuminista,
sarebbe a lungo rimasto l'interpretazione canonica del pensiero della
modernità e della sua ancella, la scienza.

\textquote{Ha lasciato il mondo pieno e lo trova vuoto}. Silvia
Parigi, invece, ci addentrerà, nella sua monografia, ad un mondo
\emph{magnetico} -- popolato cioè da \emph{fenomeni occulti}:
apparizioni attestabili le cui ragioni sono celate o latenti. Sarà
attorno a queste curiosità epistemiche che si svilupperanno gli
strumenti \bsq{scientifici} a noi consueti, dai vortici cartesiani
alla forza occulta che muove l'universo newtoniano (o anzi, gli
universi dei newtoniani, tutt'altro che vuoti)
%
\footnote{Si veda, ad esempio, come i gesuiti Gaspar Schott e
Francesco Lana Terzi interpretano gli esperimenti di Boyle e Hooke
sulla pompa a partire dagli effluvi, e lo stesso Boyle non sarà
estraneo a queste teorie.
(\cite[76]{parigiMagiaScienzaNelleta2022}).}. 
%
Risulta fondamentale circoscrivere queste acquisizioni, senza dubbio
innovative, dai i loro contesti storici particolari, e ne risulteranno
delle continuità con le tradizioni che sono più utili a delimitare gli
elementi veramente nuovi che le dichiarazioni programmatiche dei
filosofi. 

% Risulterà non trascurabile la dimensione teleologica ed escatologica,
% in continuità con la tradizione.

La \emph{specie} è l'elemento al centro dell'ilozoismo aristotelico,
la parte concettuale dell'insieme ontologico materia-forma e quella
che ne fornisce attualità. Parigi mostra efficacemente come esso è
ripreso dai nostri moderni dalle tradizioni medievali e rinascimentali
con una miriade di accostamenti, da Plotino a Galeno, in cui la
\emph{forma} slitta nel dominio di una \textlatin{\textit{virtus}} a
cavallo fra spirituale e materiale che vivifica gli oggetti. La
distinzione rischia di apparire innocua, ma è invece al seno della
concezione \bsq{moderna} della forma, tutta incentrata sulla figura
come datrice di consistenza alla materia (a cui si dà la prerogativa
ontologica), evitando così ogni concessione al realismo delle specie
aristoteliche. Alle leggi del moto (proprio della materia, accanto
all'estensione) e la causa di esso, gli elementi mistici e magici che
si aggiungono lasciano un quadro ben lontano dal proverbiale
scetticismo verso l'antico che non regge il confronto con le fonti
riportate dall'autrice.

L'\textlatin{\textit{anima mundis}} di Ficino e More permea le
spiegazioni causali del magnetismo e la gravità (che anzi è talvolta
spiegata come un tipo di magnetismo), fenomeni che restano
\emph{occulti} persino negli schemi causali moderati di Boyle e
Gassendi.
% TODO Corroborare se il magnetismo e la gravità sono fenomeni occulti
% in Boyle e Gassendi.
Così come -- nelle concezioni più eterodosse --  lo Spirito non è che
il principio di emanazione di Dio o anima del mondo, ogni essere ha
una sua sfera di influenza rappresentata dalla materia più rarefatta
(o l'elemento spirituale più denso, valga l'ossimoro) entro cui può
influire su altri oggetti a distanza o tramite contatto, a seconda
dell'autore. Essa è talvolta definita in termini di \emph{effluvi} di
spiriti irradiati dai corpi, il tutto assai vicino all'analogia
plotiniana dell'emanazione solare. Dal lato dell'ortodossia, cattolica
o luterana che sia, le stesse distinzioni si useranno per
\emph{caratterizzare formalmente le apparizioni angeliche e demoniche}
e, spiega la Parigi, si scriveranno cospicui elenchi di entità occulte
che popolano il cielo dei moderni: la parsimonia ontologica, valore
epistemico di moda all'epoca, coesiste con una molteplicità di effetti
occulti che si naturalizzano in uno \emph{spirito di sistema} prodotto
dal bisogno dottrinario della cosiddetta età confessionale. Melantone
definirà uomini pii e diabolici con spiriti sottilissimi che si
innescano nel corpo; altrettanto faranno gesuiti come Schott e
Kircher. L'istituzionalizzazione della religione e
l'innovazione filosofica vanno di pari passo %
\autocite[Cfr.][81]{parigiMagiaScienzaNelleta2022}.%

La bibliografia che tratta della distinzione fra \emph{qualità
primarie/secondarie} è infinita; Parigi l'integra con quella delle
qualità occulte, che la magia naturale rovescia sulla simpatia e
antipatia di una materia non inerte ma dotata di vita (Gassendi).
Teoria aperta a spiegazioni corpuscolari -- moderne, ma che non
manchino i richiami a Lucrezio --, raggiunge compiutezza in Boyle, che
usa gli effluvi per conciliare l'infinita divisibilità della materia,
cara a cartesiani e aristotelici, con le filosofie atomistiche
risuscitate dal Rinascimento. Nemmeno la riduzione delle qualità
secondarie in quelle primarie operata, com'è noto, da Descartes,
influisce nell'adozione o meno di teorie degli effluvi: se questi non
parla di distinzione qualitativa dei corpuscoli, li intende tuttavia
come materia sottilissima, diversificata da principi quantitativi come
la grandezza e la figura (si tratterebbero dell'etere), sebbene, a
questo punto, emancipati dal dominio delle qualità occulte. In essi,
tuttavia, 
%
\blockquote[{\autocite[50]{parigiMagiaScienzaNelleta2022}}]%
{continua e sopravvive la magia naturale del Rinascimento}, o almeno
la sua eredità.
%
Non sarebbe inusitato rinchiudersi nella strada cartesiana lasciando
alla magia naturale magari una nota a piè di pagina nella storia della
scienza, ma l'autrice sceglie piuttosto di rovesciare lo schema,
mostrando come sia proprio l'atteggiamento cartesiano a essere
un'originalità rispetto agli autori della cosiddetta \bsq{rivoluzione
scientifica}, fra cui Newton e Boyle:
%
\blockquote[{\autocite[53]{parigiMagiaScienzaNelleta2022}}]%
{il dominio dell'occulto, pur riducendosi significativamente, non si
esaurisce, ma si sposta dalla ricerca delle \emph{cause dei fenomeni}
\textquote{sottili} all'indagine sulle \emph{cause delle leggi} che
spiegano quei fenomeni}.
%
Un altro aspetto che la storia della Rivoluzione scientifica tende a
trascurare è la \emph{storia naturale}. La Parigi nota come il suo
sviluppo sia intrecciato con quella dei fenomeni occulti. Se è la
\textlatin{\textit{fascinatio}} a motivare i gabinetti di curiosità di
Kircher e Ferrante Imperato, lungi dal trattarsi di un mero ozio o
morbosità per oggetti mostruosi, si tratta di ricostruire la
dimensione temporale degli excursus naturali in vista della
spiegazione causale dell'universo e le sue bizzarrie. Punto di svola
(anche in confronti dei gabinetti di curiosità) è per l'autrice
Francis Bacon che, scostandosi dalla storia come catalogo  -- si pensi
alla celebre mappa borgesiana %
\autocite[Cfr.][81]{parigiMagiaScienzaNelleta2022} --, %
l'antepone come premessa al raggiungimento del filo conduttore
attraverso cui si discosta lo \emph{schema} causale; filo fatto da
simpatie campanelliane e dal moto di spiriti ed effluvi che causano i
fenomeni, pur quelli mostruosi
%
\footnote{Quelli cioè mostruosi \emph{secondo la specie}, variazioni
di cui ci sono numerosi casi; quelli mostruosi \emph{secondo
l'individuo} non sono spiegabili nella storia naturale così
interpretata, ma vanno corretti dalla tecnica: i \emph{miracoli delle
arti} subentrano a quelli della natura
(\cite[100]{parigiMagiaScienzaNelleta2022}).}.
%
Alla dimensione causale, il \bsq{capostipite della scienza moderna}
aggiunge quella \emph{simbolica}. Non si tratta più di una storia che
coincida con la spiegazione causale dello sviluppo universale, ma che
sia utile per trovare, via \bsq{vera induzione}, le leggi che formano
le cose e che sono, esse sì, equivalenti a \emph{causa}: l'assioma non
è che simbolo. Questa dimensione, però, prende notevolmente un altro
carattere fra i seguaci della \textlatin{\textit{mathesis
universalis}} (Galielo, Descartes, Newton, ecc.), che guardano con
disprezzo alla storia naturale: una spiegazione causale non è
immediatamente traducibile in una razionalizzazione matematica,
direbbero i primi.

La meraviglia è in Aristotele ciò che muove l'uomo alla filosofia.
Nell'Italia meridionale moderna le fascinazioni d'amore e odio
capitano non solo nei mercati cittadini, ma anche nelle corti dei
signorotti o nei palazzi vescovili. Non è l'inizio del cammino alla
virtù e la felicità, ma invece porta spesso alla rovina fisica e
morale; si trasmette attraverso raggi emessi dagli occhi del
\emph{jettatore}, oppure tramite gli effluvi che emana il suo corpo e
che si innescano negli occhi della vittima (qua si distingue a seconda
del modello ottico a cui sottoscriva ciascun autore) %
\autocite[Cfr.][123]{parigiMagiaScienzaNelleta2022}. %
Nel cielo -- lo stesso verso cui Galielo punta il telescopio -- ci
sono vapori e sostanze che trasmettono messaggi occulti, malattie ed
epidemie, oppure voglie nate da immaginazioni inquiete ed energumeni.
\textquote{A Parigi, l'universo lo si vede composto di vortici di
materia sottile, a Londra, nulla si vede di tutto questo}; a Napoli,
l'universo è talvolta corpuscolare, talvolta pervaso di luce
universale, talvolta cartesiano o gassendista, talvolta demonico e
meraviglioso: il lume della fredda ragione fa i conti un vitalismo
che, non essendo mai scomparso con l'età dei lumi né in Francia né in
Inghilterra -- si pensi a Berkeley, che riprende la gravità newtoniana
in termini di simpatia e antipatia, o al mesmerismo che non a caso
ebbe grande influenza proprio a Napoli --, qua rimase al centro della
cultura in maniera particolarmente evidente. La Parigi inserisce una
realtà sociale concreta e del resto ben nota, in dialogo con un
illuminismo su cui si gettò un velo storiografico pernicioso. Solo
studi contemporanei, di cui questo è un degnissimo esempio, stanno
riuscendo a fornire al lettore uno sguardo più ricco del pubblico
patrimonio culturale.


% NOTICE. THE TEXT IS EDITED UP TO THIS POINT.

\end{document}
