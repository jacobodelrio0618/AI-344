% !TEX root = ..\..\main.tex
\documentclass[../../main.tex]{subfiles}%
\ifSubfilesClassLoaded{\addbibresource{../../bibliography.bib}}{}

\begin{document}

\section{Storia del \emph{LCC}}

\subsection{La medicina tra scienza e pratica. Il modello di Parigi}

\begin{frame}{Il primo connubio tra la medicina clinica e quella
    scientifica}
    \begin{block}{\emph{Like Comparison Criterion}}%
        La base storica della medicina scientifica è --pure prima
        dello standard aureo-- l'attestamento degli effetti di un
        intervento sperimentale con gruppi di individui ritenuti
        \emph{simili}. 
    \end{block}
    \vspace{10pt}
    Fin qui si ha la parte \bsq{scientifica}, però, se lo scopo è
    \emph{terapeutico}, essa serve a \emph{determinare il trattamento
    più adatto} che sarà poi \emph{somministrato}.
\end{frame}

\begin{frame}
    \begin{figure}
        \centering
        \includegraphics[width=0.65\textwidth]{./Images/Vue_de_la_Salpêtrière_sous_Louis_XIV-CISB0147.jpg}
    \caption{\textit{Vue de la Salpêtrière sous Louis XIV}, 
    Biblioteca Interuniversitaria di Sanità. Dominio Pubblico.}%
    \end{figure}
\end{frame}

\begin{frame}{Il \emph{Like Comparison Criterion} come demarcazione
    \emph{storica}} È negli ospedali parigini, nell'Ottocento, dove si
    comincia a \bsq{sistematizzare} la medicina in stretto
    collegamento con le \emph{pratiche cliniche}. \\
    \vspace{5pt} %
    Negli Stati Uniti gli aderenti a questa nuova prassi medica
    faranno appello al \emph{Like Comparison Criterion} per
    differenziarsi dagli omeopati, altrettanto diffusi.
\end{frame}

\begin{frame}{Metodi usati dai medici}
% TODO Inserire con Tixz spiegazioni grafiche dei parametri; riportare
% citazione.
    \begin{scriptsize}
    \begin{table}[h!]
        \centering
        \setlength{\tabcolsep}{3pt}
        \begin{tabular}{| c c | c c | c c | c c | c c | c c | c c | c c
            | c c |}
            \hline %
            % First Row
            \multicolumn{2}{|c|}{1} & %
            \multicolumn{2}{c|}{2} & %
            \multicolumn{2}{c|}{3} & %
            \multicolumn{2}{c|}{4} & %
            \multicolumn{2}{c|}{5} & %
            \multicolumn{2}{c|}{6} & %
            \multicolumn{2}{c|}{7} & %
            \multicolumn{2}{c|}{8} & %
            \multicolumn{2}{c|}{9} \\ %
            \hline %
            % Later Rows
            10 & 3 & 7 & 3 & 19 & 3 & 19 & 3 & 28 & 2 & 13 & 1 & 24 & 
            2 & 19 & 2 & 35 & 1 \\
            %
            12 & 2 & 10 & 2 & 29 & 3 & 12 & 2 & 17 & 3 & 16 & 2 & 12 &
            4 & 12 & 1 & 11 & 2 \\
            %
            14 & 2 & 12 & 2 & 20 & 2 & 15 & 2 & 40 & 2 & 23 & 3 & 19 &
            2 & 18 & 1 & 17 & 2 \\
            %
            ~ & ~ & ~ & ~ & 20 & ~ & 22 & 4 & 13 & 2 & 35 & 5 & 18 & 2
            & 20 & 3 & 30 & 3 \\
            %
            ~ & ~ & ~ & ~ & 16 & 3 & 12 & 4 & 21 & 2 & 17 & 2 & 15 & 2
            & 13 & 2 & ~ & ~ \\
            %
            ~ & ~ & ~ & ~ & 17 & 4 & 21 & 2 & 13 & 2 & ~ & ~ & 27 & 2
            & 21 & 2 & ~ & ~ \\
            %
            ~ & ~ & ~ & ~ & ~ & ~ & 25 & 3 & ~ & ~ & ~ & ~ & ~ & ~
            & ~ & ~ & ~ & ~ \\
            %
            ~ & ~ & ~ & ~ & ~ & ~ & 28 & 4 & ~ & ~ & ~ & ~ & ~ & ~
            & ~ & ~ & ~ & ~ \\
            %
            ~ & ~ & ~ & ~ & ~ & ~ & 40 & 2 & ~ & ~ & ~ & ~ & ~ & ~
            & ~ & ~ & ~ & ~ \\
            %
            ~ & ~ & ~ & ~ & ~ & ~ & 16 & 2 & ~ & ~ & ~ & ~ & ~ & ~
            & ~ & ~ & ~ & ~ \\
            %
            ~ & ~ & ~ & ~ & ~ & ~ & 12 & 4 & ~ & ~ & ~ & ~ & ~ & ~
            & ~ & ~ & ~ & ~ \\
            \hline
        \end{tabular}
        \caption{Relazione tra il salasso e la durata della pneumonia.
        Cfr. 
        \citereset \cite[180]{fullerDemarcatingScientificMedicine2024}.}
        \label{tabella louis}
    \end{table}
    \end{scriptsize}
\end{frame}

\begin{frame}{Metodi usati dagli omeopati}
    \begin{block}{\cite[75-76]{hahnemannOrganonMedicalArt1996}.}
        In all careful experiments, pure experience (the only
        infallible oracle of the medical art) teaches us the
        following: A medicine which, in its impingement on healthy
        human bodies, has proven that it is able to engender the
        greatest number of symptoms \emph{similar} to those found in
        the case of disease to be cured, does also (in properly
        petentized and diminished doses) rapidly, thoroughly and
        permanently lift the totality of symptoms of this disease
        state.
    \end{block}
\end{frame}

\begin{frame}{Metodi usati dagli omeopati}
    \begin{block}{\Cite[n. 189d, \pno~261]{hahnemannOrganonMedicalArt1996}.}
        A robust ten year-old country lad, due to a minor
        indisposition, was treated one morning by a so-called
        stroke-stress who repeatedly passed both tips of her thumbs
        very powerfully from the pit of his stomach out under his
        ribs, and he fell at once, deathly pale, into such an
        unconsciousness and immobility that, with all due exertion,
        one could not awaken him and he was almost taken for dead. I
        then had his older brother give him as rapid a negative pass
        as possible from the top of his head down over his feet, and
        he immediately regained consciousness and became lively and
        well.  
    \end{block}
\end{frame}

\begin{frame}{Prima demarcazione storica}
    Fino al 1848 è stato pubblicato \emph{soltanto uno} studio
    comparativo sugli effetti dei trattamenti omeopatici.
    \begin{block}{Conclusione}
        Il \emph{Like Comparison Criterion} è stato
        \emph{storicamente} utilizzato, nella prima metà
        dell'Ottocento, come marca che identifica la medicina
        scientifica quando vuole affermarsi nella società.
    \end{block}
\end{frame}

\subsection{Medicina di laboratorio e test randomizzati}

\begin{frame}{Medicina di laboratorio} Nella seconda metà
    dell'Ottocento, distanziandosi dai medici della scuola di Parigi,
    si apre strada un altro modello di medicina scientifica che si
    cura di indagare \emph{teoreticamente} le ragioni meccanicistiche
    (etologiche e fisiologiche) che causano le malattie. \\%
    \vspace{10pt}%
    Questi risultati di laboratorio hanno lo scopo di essere integrati
    nella pratica terapeutica.
\end{frame}

\begin{frame}
    \begin{figure}
        \centering
        \includegraphics[width=0.65\textwidth]{./Images/Claude-Bernard-Tavola-alfabetica-ed-analitica.png}
        \caption{C. Bernard, \textit{Tavola alfabetica ed analitica}, in Storia
        della Medicina, disponibile all'indirizzo:
        https://www.storiadellamedicina.net (consultato il 18 novembre 2025).}
    \end{figure}
\end{frame}

% TODO Forse qualche approfondimento sui postulati di Koch ci
% starebbero se sono reperibili.

\begin{frame}{Medicina epidemiologica}
    \begin{itemize}
        \item \emph{Evidence-Based Medicine (EBM)}. La branca della
            medicina contemporanea che si occupa di interpretare i
            risultati degli studi (che prendono la forma di RCTs).
        \item \emph{Randomized Control Tests (RCTs)}. Studi in cui
            l'assegnamento ai gruppi (di controllo e sperimentale)
            segue criteri aleatori per eliminare (o quanto meno
            ridurre) il rischio di parzialità (\emph{bias}).
    \end{itemize}
\end{frame}

\begin{frame}{Il \emph{Like Comparison Criterion} nei diversi modelli
    di medicina}
    \begin{itemize}
        \item \emph{Scuola di Parigi}. Si usano (come nel caso
            dello studio di \cref{tabella louis}) gruppi formati
            \bsq{naturalmente}. Il \emph{like comparison criterion}, a
            livello metodologico, punta a accostare fattori \emph{causali}
            simili quale l'età e le abitudini.
        \item \emph{Medicina di laboratorio}. Si cerca ugualmente di
            allocare gli individui a seconda di condizioni causali
            simili isolando l'intervento esperimentale.
        \item \emph{EBM}. Si usa il \emph{caso} per evitare:
            \begin{itemize}
                \item \emph{Parzialità di collocazione}. Quando ci sono
                    condizioni causali distribuite in maniera parziale fra i gruppi.
                \item \emph{Parzialità di selezione}. Quando si
                    scelgono soggetti con caratteristiche tali che
                    la collocazione sia, sin dall'inizio, parziale.
            \end{itemize}
    \end{itemize}
\end{frame}

\begin{frame}{Metodi contemporanei per limitare la parzialità}
    \begin{itemize}
        \item \emph{Alternanza}. Si prende una mostra e si assegnano
            gli individui mettendo il primo di essi in un gruppo, il
            secondo nell'altro, il terzo al primo e così via. Il
            metodo divenne popolare negli anni Venti nel Novecento e
            si pose l'obiettivo esplicito di eliminare la parzialità.
        \item \emph{Metodi aleatori}. Si sono cominciati a utilizzare
            a metà secolo, ad esempio attraverso tavole numeriche e
            così via.
    \end{itemize}
    \begin{block}{La nascita dello \emph{standard aureo}}% 
        Austin Bradford Hill, fautore delle prime allocazioni causali,
        contribuì a introdurre le RCTs come lo \emph{standard aureo}
        della medicina. L'alternanza permette \emph{predire} a che
        gruppo ciascun individuo sarà assegnato; \emph{la
        randomizzazione invece cancella questo rischio di 
        parzialità}.
    \end{block}
\end{frame}

\begin{frame}{Le \emph{cause nascoste} e la parzialità}
    \begin{block}{Il ruolo dell'aleatorietà}% 
        Poniamo uno studio non-randomizzato la cui collocazione,
        riguardo le condizioni causali, sia \emph{perfettamente equa}.
        Poiché però, ci possono essere cause nascoste allo
        sperimentatore che influiscono nel risultato, \emph{non si può
        dire che lo studio è privo di parzialità}.
    \end{block}
    \vspace{5pt}%
    \emph{L'assegnazione aleatoria, d'altro canto, è il miglior
    criterio a disposizione per \emph{sperare} una distribuzione equa
    \emph{anche} della cause nascoste}.
\end{frame}

\begin{frame}{Funzione \emph{negativa} del \emph{Like Comparison
    Criterion} nella medicina epidemiologica contemporanea}%
    Poiché \emph{non è determinabile} una distribuzione causale del
    tutto equa, la funzione dell'aleatorietà è intesa \emph{in chiave
    negativa}. Non si assume che le condizioni causali siano uguali,
    ma si elimina o riduce che la loro allocazione sia il prodotto
    della
    parzialità. \\ %
    \vspace{10pt} %
    Qualsiasi differenza tra i gruppi si può dunque attribuire a:
    \begin{enumerate}
        \item Parzialità nelle altre tappe dello studio o la sua
            interpretazione.
        \item Vero e proprio caso.
        \item \emph{Effetto} dell'intervento esperimentale.
    \end{enumerate}
\end{frame}

\end{document}
