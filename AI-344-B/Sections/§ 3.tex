% !TEX root = ..\..\main.tex
\documentclass[../../main.tex]{subfiles}%
\ifSubfilesClassLoaded{\addbibresource{../../bibliography.bib}}{}

\begin{document}

\section{Riforma del \emph{LCC}}

\subsection{L'evidenza empirica in favore dell'omeopatia}

\begin{frame}{Si possono attestare certi effetti del trattamento
    omeopatico con lo standard aureo} %
    Ci sono RCTs che confermano effetti (discreti) del trattamento
    omeopatico \emph{oltre il placebo} detti \emph{\bsq{Mathie
    Reviews}}. \emph{Sarebbe dunque un atteggiamento
    \emph{scientifico}
    prescrivere trattamenti omeopatici?} %
    \vspace{5pt} %
    \begin{itemize}
        \item Si devono, appunto, prescrivere trattamenti omeopatici
            considerandoli scientifici.
        \item Si deve abbandonare il \emph{Like Comparison Criterion}
            come demarcazione fra la scienza e l'omeopatia.
        \item Si deve aggiornare il \emph{Like Comparison Criterion}
            dell'EBM contemporanea perché funzioni come demarcazione.
    \end{itemize}
\end{frame}

\begin{frame}{L'attestamento di parzialità di Cochrane}
    \begin{block}{\Cite[2]{higginsRevisedCochraneRiskofbias2019}}
        Empirical evidence of bias in a randomized trial comes from a
        field known as \textbf{meta-epidemiology} (2). A
        meta-epidemiological study analyses the results of a large
        collection of previous studies to understand how
        methodological characteristics of the studies are associated
        with their results. The first well-known meta-epidemiological
        study examined 33 meta-analyses containing 250 clinical trials
        (3).    
    \end{block}
\end{frame}

\begin{frame}{L'attestamento di parzialità di Cochrane}
    \begin{block}{\dots}
        Each trial was categorized on the basis of four
        characteristics: whether sequence generation was reported to
        have a random component; whether allocation was reported to be
        adequately concealed, whether the trial was described as
        double-blind, and whether the trial reported exclusion of
        participants from its analysis [\dots].
    \end{block}
\end{frame}

\begin{frame}{La parzialità delle \emph{Mathie Reviews} in
    comparazione con quella presente nell'EBM}
    \begin{itemize}
        \item Non c'è una \emph{Mathie Review} che presenti un rischio
            basso di parzialità.
        \item Criteri come quelli del passo sopra sono, tuttavia,
            \emph{insufficienti} a far sì che il \emph{like comparison
            criterion} (inteso come imparzialità) denoti una
            differenza della medicina con l'omeopatia.
        \item Infatti, anche parecchi studi della medicina assodata come
            scientifica non sono esenti di parzialità; \emph{ciò che
            è parziale non è automaticamente antiscientifico}.
    \end{itemize}
\end{frame}

\subsection{L'assenza di cause meccaniche nell'attestare il rischio di
parzialità}

\begin{frame}{La classifica GRADE per la medicina clinica}
    \begin{itemize}
        \item La EBM è stata ampiamente criticata, fra altre cose, per
            mancanza di criteri sistematici per le raccomandazioni
            cliniche.
        \item Un tentativo di risposta dall'interno del movimento
            dell'EBM si dà con la linea-guida del \emph{Grading of
            Reccomendations, Assessment, Development and Evaluation}
            (GRADE), che serve a misurare l'affidabilità dell'EBM
            \emph{in vista della pratica}.
        \item La GRADE viene utilizzata dalla OMS, NHS, CDC e la
            Cochrane.
    \end{itemize}
    \begin{block}{Riferimento bibliografico}
        \Cite{perillatClinicalRecommendationsRole2022}.
    \end{block}
\end{frame}

\begin{frame}{Lo statuto epistemico delle spiegazioni causali}
    \begin{block}{\Cite[2]{guyattGRADEGuidelines82011}}
        Another type of indirect evidence that we have not addressed
        relates to mechanism of action. The GRADE system does not rate
        evidence either up or down based on the mechanism or
        pathophysiological basis of a treatment. RCTs rypically begin
        with a reasonable expectation of success based, to some
        degree, on biological rationale. But judgements of exactly how
        strong is the rationale are easily open to dispute, and GRADE
        does not suggest using them directly as basis for rating
        evidence quality up or down.
    \end{block}
\end{frame}

\begin{frame}
    \begin{block}{\Cite[340]{turnerEvaluatingUKHouse2017}}
        \begin{enumerate}
            \item It is very implausible that there could be a
                mechanism by which the quasi-pharmacological component
                of homeopathic treatment is effective.
            \item If it is very implausible that there could be a
                mechanism for how some component of a treatment is
                effective, then it is very unlikely that component is
                effective.
            \item It is very unlikely that the quasi-pharmacological
                component of homeopathic treatment is effective.
        \end{enumerate}
    \end{block}
\end{frame}

\begin{frame}{La \emph{plausibilità meccanicistica} nel considerare la
    parzialità}
    \begin{itemize}
        \item Se ci sono buone ragione per credere \emph{implausibile}
            il funzionamento dei trattamenti omeopatici, non è più
            plausibile che l'effettività attestata nelle \emph{Mathie
            Reviews} sia dovuta a \emph{parzialità} (o persino al
            \emph{caso})? 
        \item La \emph{plausibilità meccanicistica} dovrebbe influire
            sulla probabilità di parzialità in una RCT.
        \item Inserire la plausibilità meccanicistica come criterio
            che può aggiungere o togliere al rischio di parzialità nel
            nuovo \emph{Like Comparison Criterion} serve a
            differenziare l'omeopatia dalla medicina scientifica.
    \end{itemize}
\end{frame}

\subsection{La medicina tra scienza e pratica. Demarcare l'omeopatia
come pseudo-scienza a partire dell'\emph{utilizzo} di evidenza
parziale}

\begin{frame}{Dalla \emph{distinzione} alla \emph{demarcazione}}
    \begin{itemize}
        \item Nel nuovo \emph{Like Comparison Criterion} l'omeopatia
            avrà sempre un rischio di parzialità maggiore alla
            medicina scientifica.
        \item Per quanto possiamo ora differenziarle, la
            \emph{parzialità} non è immediatamente un \emph{criterio
            di demarcazione} -- ricordiamo, non tutto ciò che è
            parziale è antiscientifico.
        \item La demarcazione sta nel \emph{ruolo pratico che si dà
            all'evidenza a seconda della parzialità}.
    \end{itemize}
\end{frame}

\begin{frame}{Le RCTs nel contesto della medicina clinica}
    \begin{block}{\Cite[3]{perillatClinicalRecommendationsRole2022}}
        The more controlled a RCT is, the more dissimilar it is real
        life circumstances, and the more unlikely it is that the
        results will translate to clinical practice. As Cartwright
        puts it, \textquote{evidence does not always travel}, and this,
        regardless of whether that evidence is derived from clinical
        trials or laboratory studies.
    \end{block}
\end{frame}

\begin{frame}{\emph{Demarcazione} della medicina scientifica
    dall'omeopatia}
    \begin{itemize}
        \item La \emph{medicina clinica} è scientifica perché
            \emph{riconosce} la probabilità di parzialità nel
            \emph{prescrivere} un determinato trattamento.
        \item L'omeopatia \emph{manca al criterio di demarcazione}
            perché non riconosce, \emph{nel prescrivere}, il rischio
            di parzialità dell'evidenza in base a cui prescrive
            determinato dal \emph{Like Comparison Criterion}.
    \end{itemize}
\end{frame}

\end{document}
