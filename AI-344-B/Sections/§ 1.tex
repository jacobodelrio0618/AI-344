% !TEX root = ..\..\main.tex
\documentclass[../main.tex]{subfiles}%
\ifSubfilesClassLoaded{\addbibresource{../bibliography.bib}}{}

\begin{document}

\section{Introduzione}

\subsection{Il problema della demarcazione}

\begin{frame}{Cosa rende un'asserzione \emph{scientifica}?}
    Tradizionalmente si è fatto appello a un \emph{criterio di
    demarcazione}, la cui soddisfazione dovrebbe determinare il
    carattere scientifico o metafisico di un'asserzione. 
    \vspace{10pt}
    \begin{block}{Riferimento bibliografico}
        \cite{oreskesPercheFidarsiScienza2021}.
    \end{block}
\end{frame}

\begin{frame}{I criteri di demarcazione astratti}
    \begin{itemize}
        \item \emph{Comte}. Il sapere è attendibile se è positivo.
            Segue un \emph{metodo} in opposizione a una
            \emph{dottrina}, e tale metodo è determinato
            dall'\emph{osservazione}.
        \item \emph{Circolo di Vienna}. Un'asserzione è scientifica
            sse ha significato, vale a dire, se è 
            \emph{verificabile}. La verificazione, inoltre, è
            \emph{probabilistica}.
        \item \emph{Popper}. Un'asserzione è scientifica sse è
            \emph{falsificabile}. La fiducia in una teoria sta nella
            sua capacità di superare esperimenti che mirano alla
            \emph{confutazione}, in cui caso si dice che una teoria è
            \emph{corroborata}.
    \end{itemize}
\end{frame}

\begin{frame}{I \emph{science studies} e il tramonto del metodo}
    \begin{itemize}
        \item \emph{Fleck}. La scienza non è praticata dallo
            scienziato isolato, ma da un \emph{collettivo di
            pensiero}, aprendo la strada a un ruolo epistemico dei
            fattori sociali nelle \emph{istituzioni scientifiche}.
        \item \emph{Sotto-determinazione}. Le teorie scientifiche
            isolate sono, a loro volta, \emph{theory-laden}, creando insiemi teorici di ampiezza maggiore; lo sperimento, dunque, è insufficiente a provare o confutare razionalmente
            qualsiasi teoria.
            \begin{itemize}
                \item \emph{Duhem}. L'attività sperimentale stessa guida i ricercatori a determinare che parti di una teoria sono affettate dall'evidenza empirica.
                \item \emph{Quine}. Non c'è maniera all'interno
                    dell'attività scientifica di determinare che parte
                    dell'insieme teorico viene toccato dai risultati
                    esperimentali.
            \end{itemize}
    \end{itemize}
\end{frame}

\begin{frame}{I \emph{Science studies} e il tramonto del metodo}
    \begin{itemize}
        \item \emph{Kuhn}. Gli scienziati lavorano in comunità entro
            una serie di caratteristiche epistemiche condivise, il
            \emph{paradigma}, e la loro attività e risolvere i
            problemi che sorgono dalle osservazioni entro quel
            paradigma.
            \begin{itemize}
                \item Quando un problema non è definitivamente
                    risolvibile entro un paradigma si ha una
                    \emph{rivoluzione scientifica}.
                \item Non si può dare la prerogativa epistemica a
                    nessun paradigma poiché sono
                    \emph{incommensurabili}; si sfuma l'idea di
                    progresso scientifico.
            \end{itemize}
        \item \emph{Sociologia della scienza}. Si studia la scienza in
            maniera scientifica, vale a dire, si osserva l'attività
            effettiva degli scienziati nella loro \emph{pluralità
            metodologica}.
    \end{itemize} 
\end{frame}

\begin{frame}
    \begin{itemize}
        \item \emph{Feyerabend}. \emph{Non esiste} un metodo
            scientifico. La scienza, con una miriade di mezzi, va da
            sola, e proporre un metodo è di fatto una costrizione.
        \item \emph{Harding e Longino}. L'\emph{oggettività} può
            essere riscattata come \textquote{\emph{traguardo
            sociale}}. La scienza si \emph{auto-corregge} e la
            diversità può mitigare i vizi epistemici che altrimenti
            rimangono latenti.
    \end{itemize}
\end{frame}

\subsection{Non basta essere empirici}

\begin{frame}{Lo \emph{standard aureo} della scienza medica}
    Si ha un \emph{gruppo di controllo}, a cui si somministra un
    trattamento consueto o un placebo, e un \emph{gruppo
    esperimentale}, a cui si somministra il trattamento da studiare.
    L'assegnazione è
    \emph{randomizzata} e può essere senza conoscenza dei partecipanti (cieco), degli scienziati, o di entrambi (doppio cieco). %  
    \vspace{10pt} %
    \begin{block}{Demarcazione della medicina scientifica} %
        \emph{Come distinguere la medicina scientifica dall'omeopatia
        se ci sono anche studi che costatano effetti terapeutici in
        quest'ultima?}
    \end{block}
\end{frame}

\begin{frame}{Il punto di vista epistemico} Che ci sia o meno
    l'omeopatia non è intrinsecamente una minaccia per la scienza
    medica come \emph{istituzione}. Il problema che si pone Fuller è
    nella cornice del \emph{revival} del problema della demarcazione.
    \begin{enumerate}
        \item \emph{Interesse storico}. Un criterio applicabile a
            tutta la medicina scientifica sarebbe una chimera della
            filosofia della scienza, e renderebbe impraticabile una
            distinzione netta fra l'omeopatia e la medicina
            scientifica.
        \item \emph{Interesse filosofico}. Un criterio troppo concreto
            rischia di limitare lo scopo a casi individuali, di fatto
            cancellando il problema della demarcazione.
    \end{enumerate}
    %
    % The challenge is that formulating standards too abstractly makes
    % them difficult to apply unambiguously to orthodox medicine and
    % homeopathy, while specifying them too concretely risks producing
    % criteria that work only for historically narrow instances of
    % scientific  medicine and homeopathy - p. 2.
    %
    % In demarcating scientific medicine from the unscientific
    % medicine of classical homeopathy, our solution need only apply
    % to these two medical traditions rather than offer universal
    % criteria that sort all instances of scientific medicine from all
    % instances of unscientific medicine - p. 3.
    %
\end{frame}

\begin{frame}{Demarcazioni recenti della medicina scientifica}
    \begin{enumerate}
        \item [a. ] \emph{Sistematicità}. La medicina scientifica, negli
            ultimi due secoli, ha attraversato un processo di
            istituzionalizzazione sempre maggiore; ma, a seconda di
            ciò che si intende con \bsq{sistematicità}, si può anche
            dire che l'omeopatia è sistematica.
        \item [b. ] \emph{Trascuranza degli standard epistemici}.
            L'omeopatia spesso pretende di essere \bsq{scientifica}
            quando invece trascura valori epistemici storicamente
            associati con la scienza.
    \end{enumerate}
    Questi criteri, sebbene riescano a individuare patroni comuni
    fra le scienze (e fra le pseudoscienze), sono \emph{troppo astratti} per essere applicati alla medicina scientifica.
\end{frame}

\begin{frame}{Concretizzare i criteri} %
    Si cerca di rendere questi criteri applicabili al caso concreto
    della demarcazione tra medicina come scienza e omeopatia come
    pseudoscienza.
    \begin{enumerate}
        \item [a1. ] \emph{\textquote{Marcatore di sistematicità}}. Il
            \textquote{\emph{like comparison criterion}} nel suo
            sviluppo storico e gli studi scientifici a cieco.
            \begin{itemize}
                \item Lo standard aureo non è proprio della medicina
                    fino alla seconda metà del Novecento; ciò non
                    toglie che la medicina anteriore ad esso sia
                    scientifica.
            \end{itemize}
        \item [a2. ] \textquote{\emph{Standard epistemico}}. Esso può ed
            è di fatto variato nel tempo, e nelle circostanze storiche e
            istituzionali.
    \end{enumerate}
\end{frame}

\end{document}
