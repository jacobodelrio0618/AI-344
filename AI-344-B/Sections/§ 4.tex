% !TEX root = ..\..\main.tex
\documentclass[../../main.tex]{subfiles}%
\ifSubfilesClassLoaded{\addbibresource{../../bibliography.bib}}{}

\begin{document}

\section{Conclusione}

\subsection{Le assunzioni \emph{metodologiche}}

\begin{frame}{Il \emph{Like Comparison Criterion} come marcatore di
    sistematicità}
    \begin{itemize}
        \item La comparazione degli effetti terapeutici tra
            \emph{simili} (nello sviluppo storico di ciò che si
            intende scientificamente per \emph{simile}) ha composto,
            anche a livello istituzionale, l'impalcatura sistematica
            della scienza.
            \begin{itemize}
                \item Nel modello di Parigi si parlava di
                    \emph{similitudine causale}.
                \item Nel modello della EBM si parla di similitudine
                    \emph{in proporzione al rischio di parzialità}.
                \item Nella versione aggiornata si inserisce la
                    \emph{plausibilità meccanicistica} come fattore
                    nell'attestamento del rischio di parzialità.
            \end{itemize}
    \end{itemize}
\end{frame}

\begin{frame}{Il \emph{Like Comparison Criterion} come standard
    epistemico}
    \begin{block}{\emph{Demarcazione vera e propria}}
        La medicina scientifica usa il \emph{Like Comparison
        Criterion} come standard per la prescrizione di trattamenti a
        livello clinico, e in questo si differenzia dall'omeopatia.
    \end{block}
\end{frame}

\begin{frame}{Dichiarazioni metodologiche}
    \begin{block}{\Cite[184]{fullerDemarcatingScientificMedicine2024}}
        As suggested earlier, my localist solution to the demarcation
        problem can be seen as specifying or de-abstracting more
        general solutions for a medical context [\dots]. Strictly
        speaking, the modified like comparison criterion should not be
        seen as a competitor to EBM because while the purpose of the
        criterion is to demarcate scientific from unscientific
        medicine, the purpose of EBM is to evaluate evidence for
        medical decisions.
    \end{block}
\end{frame}

\subsection{Dove si inserisce la storia del \emph{Like Comparison
Criterion} nella storia del problema della demarcazione?}

\begin{frame}{Confronti con la filosofia della scienza}
    \begin{itemize}
        \item \emph{Feyerabend}. 
            \begin{itemize}
                \item Il \emph{Like Comparison Criterion} non è un
                    \emph{metodo}, ma, appunto, un \emph{marcatore}
                    che ha uno sviluppo storico e una presenza
                    istituzionale concreta.
                \item D'altro canto, il suo utilizzo come
                    \emph{standard epistemico} può essere di fatto una
                    costrizione; essa tuttavia, è \emph{localizzata}
                    alla medicina scientifica. Una costrizione che la
                    scienza stessa si è imposta per differenziarsi
                    dall'omeopatia, e che è costantemente sottomessa a
                    critica.
                \item Il \emph{Like Comparison Criterion} mostra che
                    può esserci rilevanza attuale del problema della
                    demarcazione, facendo a meno del concetto di
                    \emph{metodo scientifico}.
            \end{itemize}
    \end{itemize}
\end{frame}

\begin{frame}{Confronti con la filosofia della scienza}
    \begin{itemize}
        \item \emph{Kuhn}. 
            \begin{itemize}
                \item Il \emph{Like Comparison Criterion} non può
                    essere visto come il \emph{paradigma} entro cui si
                    svolge la medicina scientifica: esso non ne
                    esaurisce il campo, ma viene usato come strumento
                    in vista di demarcare la medicina dall'omeopatia;
                    anzi, nemmeno si può dire che il il criterio serva
                    a demarcare l'EBM da \emph{tutte} le
                    pseudoscienze, ma si rimane al caso concreto
                    dell'omeopatia.
                \item Si può parlare di \emph{progresso}, nel senso
                    che gli studi della scuola di Parigi bocciano il
                    \emph{Like Comparison Criterion} della medicina
                    epidemiologica, e tuttavia, sono nel lato
                    scientifico della demarcazione.
            \end{itemize}
    \end{itemize}
\end{frame}

\begin{frame}{Confronti con la filosofia della scienza}
    \begin{itemize}
        \item \emph{La sottodeterminazione}.
            \begin{itemize}
                \item \emph{Modello di Parigi}. Condizioni causali
                    simili permettono di \emph{isolare} la causa
                    della diversità fra i due gruppi all'intervento
                    esperimentale.
                \item \emph{Medicina epidemiologica}. Si tiene
                    presente della sottodeterminazione dell'evidenza
                    empirica nella causa della differenza. Tuttavia,
                    si cerca di eliminare o ridurre, tra le
                    possibilità, la \emph{parzialità}.
            \end{itemize}
    \end{itemize}
\end{frame}

\begin{frame}{Inquadrare il \emph{Like Comparison Criterion}}
    \begin{block}{Il \emph{Like Comparison Criterion} come istanza del
        \bsq{\emph{razionalismo negativo}}} Possiamo inserire il
        \emph{Like Comparison Criterion} entro una \emph{filosofia
        della scienza}? Possiamo dire che è un'istanza di quella che
        cerca \emph{negativamente} la \emph{razionalità} mitigando,
        per quanto possibile, i pregiudizi e le contingenze inerenti
        alla ricerca scientifica?
    \end{block}
\end{frame}

\end{document}
